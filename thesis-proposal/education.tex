\section{Education}
Actively-typed programming systems are useful for researchers and domain experts because they allow them to develop and distribute new constructs modularly. This modularity is also useful to educators who wish to incrementally introduce advanced language features to students, as well as to integrate more helpful error messages and diagnostics that may be less useful to professional programmers and more advanced students.

Our proposed work in this area will center largely around using Ace in the classroom. As of this writing, Ace is best suited for teaching low-level programming because of the library-based implementations of C99 and OpenCL. At CMU, the introductory course on Principles of Imperative Programming teaches a language called C0, which is a significant simplification of the C language, augmented with some features to improve safety and allow users to explicit include pre-conditions and post-conditions on functions. This kind of development effort is precisely suited to an actively-typed language like Ace -- features can be implemented in libraries, which can be introduced incrementally, and new features can be added to complement existing libraries easily. Thus, we plan on evaluating the suitability of Ace for this course and may expand to similar efforts at other institutions.

Another course, Advanced Computer Graphics, teaches GPU programming to students using a combination of the CUDA and OpenCL languages. We have presented Ace as a guest lecture to this class in the past, and the feedback has been overwhelmingly positive. We are discussing incorporating Ace as a replacement for OpenCL, due largely to its simplicity from the end-user perspective, in future editions of this course. This opportunity will allow us to analyze the learning curve for Ace, and compare it to past editions of the course reasonably directly. We will also be able to instrument the compiler to gather detailed, anonymous data to complement the studies we discuss in Section 5.1.