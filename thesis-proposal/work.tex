\section{Work To Be Done}
\subsection{Completion and Evaluation of Ace}
The work done so far on the Ace language establishes the practicality of active typechecking and translation as a foundational language mechanism and begins to apply it to one realistic problem domain, high-performance scientific computing on GPUs and other coprocessors, by internalizing OpenCL. However, to more fully establish the practicality of this approach, it must be shown that Ace can support more sophisticated, higher-level language constructs as well, ideally from a variety of application domains. It must also be further shown that users who are not language design experts, and will not be utilizing the extension mechanism directly, can nevertheless use the language in practice. 

Thus, a portion of the resources allocated to this project will be dedicated to completing the implementation of Ace, documenting it and releasing it both to the research community and to practitioners in areas where a useful set of constructs have been developed, beginning with the members of the high-performance computing community. This will also support synergy with ongoing collaborations in our group with other problem domains, including front-end and back-end web development, security, functional programming and object-oriented programming.

With broader usage by the research community and in practice, we will be able to continue to conduct case studies and empirical evaluations examining the usability and usefulness of Ace and the active typing mechanisms it introduces. Indeed, because of the flexibility of active typing and our implementation strategy, we plan on instrumenting the Ace compiler to send data and direct feedback from willing users back to us for analysis. This will allow us to analyze questions like:
\begin{itemize}
\item How often are different constructs and mechanisms used in practice?
\item What kinds of errors appear most frequently in practice, and how long does it take users to resolve these errors?
\item How do users feel about the mechanisms, constructs and error messages that they encounter, as determined by direct feedback solicited at the time of occurrence.
\item How commonly do users utilize the active typing mechanism directly?
\end{itemize}

To our knowledge, no such detailed study of usage patterns of a programming language in the wild has been conducted previously, and we anticipate producing insights that are relevant to programming language design broadly.
\subsection{Active Type Theory}
The work on Ace is aimed at establishing a practical implementation of active typechecking and translation. However, because it uses a dynamically-typed language, Python, for the type-level language, it is not suitable for rigorous theoretical analysis, nor does it achieve safety in the strictest sense, due to high levels of dynamism exploitable in Python's object model. Moreover, it is a full language and so it does not necessarily distill the essential concepts that we wish to introduce to the research community in their simplest possible form or connect them to existing concepts in the theory of typed programming languages. For these reasons, we plan to develop a minimal type theoretic formulation, which we call $\lambda_{\text{A}}$.

\subsubsection{Type-Level Computation}

We have chosen to use the term \emph{type-level language}, rather than \emph{metalanguage} or, as it is called in the original work on active libraries, \emph{metacode} intentionally. This is because the concept of a type-level language is already well-established and includes a key feature necessary for active types -- the reification of types as compile-time entities that can be manipulated programmatically. We believe that this connection between type-level computation and active types will provide new insights into each mechanism individually, and provide a clean theoretical basis for active typechecking and translation as an extension to type-level computation.

\subsubsection{Active Types and Wadler's Expression Problem}
Another key insight is that in monolithic programming languages, the fixed set of base types can be written as an algebraic datatype. Indeed, in functional implementations of typecheckers, this is precisely what is done, for example for G\"{o}del's \textbf{T}:

\begin{verbatim}
    datatype Type = Arrow of Type * Type 
                  | Nat
\end{verbatim}

If we wish to allow users to introduce new base types and type families, it can then be seen that we face a variant of the expression problem, so named by Wadler \cite{wadler1998expression}. That is, we wish to add new constructors of kind \verb|Type|, but this conventional formulation of datatypes does not allow us to do so. There are competing approaches that aim to solve this problem. The most common is to use object-oriented inheritance, where subtypes of \verb|Type| represent new type families. This is the approach taken in Ace. The functional programming approach is to instead use \emph{open data types} \cite{conf/ppdp/LohH06}. We will thus aim to further explore this approach, drawing a clear connection between active types, the expression problem and type-level open data types.

\subsubsection{Rigorous Safety Proofs}
A benefit of a simple core theory for active typing is that it will allow us to formally state and prove key safety theorems, formalizing the somewhat informal notions described in Section 3.3. Based on early sketches of a theory that we have discussed, we have implemented some run-time checks into Ace that ensure some of these properties for compiled programs based on a notion borrowed from the area of verified compilation called \emph{type-directed compilation} \cite{tarditi+:til-OLD}. We would like to be able to ensure these properties for all possible programs, and a type theory will allow us to demonstrate the machinery needed to do this in a rigorous manner.

\subsubsection{Implementation With a Dependently-Typed Language}
Finally, we would like to implement a functional version of active type-checking and translation, both as a counterpoint to Ace and as a useful tool for theoretical researchers. We plan on doing so as an extension to the Coq theorem prover, which contains a dependently-typed functional programming language at its core, and is implemented in Ocaml. This will allow us to understand the challenges and opportunities of using a dependently-typed functional language at the core of an extensible programming system.
 
