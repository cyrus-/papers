\documentclass[10pt]{article}
\usepackage{stmaryrd}
\usepackage{mathpartir}
\usepackage{palatino}
\usepackage{listings}
\usepackage[T1]{fontenc}
\usepackage[usenames,dvipsnames,svgnames]{xcolor}
\definecolor{mauve}{rgb}{0.58,0,0.82}
\usepackage{mdwlist}
\usepackage{paralist}
\usepackage{todonotes}
\usepackage{xypic}
\usepackage{amsmath,amssymb}
\usepackage{url}
%\usepackage{callout}L
 \usepackage{graphicx}
%\usepackage{natbib}
\usepackage{array}
%\bibpunct();A{},
%\let\cite=\citep

\lefthyphenmin=64

%\newcommand{\code}[1]{\texttt{#1}}
%\newcommand{\TODO}[1]{\textbf{[TODO: #1]}}
\newcommand{\cut}[1]{}

    %
    \oddsidemargin	0.5in
    \evensidemargin	0in
    \textwidth      5.5in
    \headheight     0.0in
        \headsep        0.0in
    \topmargin      0.0in
    \textheight=9.0in
    %

% Different font in captions
\newcommand{\captionfonts}{\small\bf}

%\renewcommand{\topfraction}{0.99}
%\renewcommand{\textfraction}{0.01}
%\renewcommand\floatpagefraction{0.99}

\newcommand{\minisec}[1]{\vspace{2ex}\noindent\textbf{#1}}
\newcommand{\minisecnosp}[1]{\noindent\textbf{#1}}
\newcommand{\perm}[1]{\keyw{#1}}
\newcommand{\keyw}[1]{\texttt{\textbf{#1}}}

\lstset{tabsize=3, basicstyle=\ttfamily\small, commentstyle=\itshape\rmfamily, numbers=left, numberstyle=\tiny, language=java,moredelim=[il][\sffamily]{?},mathescape=true,showspaces=false,showstringspaces=false,columns=fullflexible,frame=single,xleftmargin=15pt,escapeinside={(@}{@)}, morekeywords=[1]{atomic,state,requires,unique,immutable,shared,pure,full,none,method,val,with,match,of}}
\lstloadlanguages{Java,VBScript,XML,HTML}




\newcommand{\code}[1]{\texttt{#1}}

\newcommand{\ruleT}[1] {\textsf{\sc #1}}

\usepackage{caption}
\usepackage{listings}
\lstset{
  language=Python,
%  showstringspaces=false,
  formfeed=\newpage,
  tabsize=2,
  commentstyle=\itshape,
  basicstyle=\ttfamily\fontsize{8pt}{1em}\selectfont,
  morekeywords={lambda, self, assert, as},
  numbers=none,
  captionpos=b,
%  numberstyle=\footnotesize\color{gray}\textsf,
%  xleftmargin=2em,
  stringstyle=\color{mauve},
  frame=none,
  columns=flexible
}
\lstdefinestyle{Bash}{
    language={}, 
    moredelim=**[is][\color{blue}\bf\ttfamily]{`}{`},
}
\lstdefinestyle{OpenCL}{
	language=C++,
	morekeywords={kernel, __kernel, global, __global, size_t, get_global_id, sin}
}

\definecolor{mauve}{rgb}{0.58,0,0.82}
\definecolor{light-gray}{gray}{0.75}

\usepackage{float}
\floatstyle{ruled}
\newfloat{codelisting}{tp}{lop}
\floatname{codelisting}{Listing}
%\usepackage[scaled]{beramono}
\renewcommand{\ttdefault}{txtt}

\usepackage{caption}
\usepackage{subcaption}
\captionsetup{compatibility=false}
\usepackage{mathrsfs}

\title{Type-Oriented Foundations for\\ Safely Extensible Programming Systems\\\vspace{10px}\large{{PhD Thesis Proposal}}}
\author{Cyrus Omar\\
 Computer Science Department\\
 Carnegie Mellon University\\
 comar@cs.cmu.edu}
\date{\today}                                           % Activate to display a given date or no date

\begin{document}
\maketitle

\begin{abstract}
\noindent
%Conventional \emph{monolithic} programming systems are built around a dichotomy between \emph{built-in features}, which enjoy support from the language and tools but must be designed and implemented in advance by the system's  designers, and \emph{user-defined constructs}, which are distributed within libraries but must creatively combine or repurpose the small set of built-in features to express all desired run-time, compile-time and edit-time behaviors. This often forces researchers and domain experts who want to significantly advance the state-of-the-art in programming systems, particularly in statically-typed systems where this dichotomy is most salient, to specify and develop new or derivative languages and tools. But this then couples innovations to a collection of unrelated design decisions, requiring more effort from the experts developing the feature as well as the targeted  developer community. It also leads to a fundamental interoperability problem when building applications consisting of components written in different languages. 

%Extensible programming systems promise to avoid this problem by ... but either limited expressiveness or safety issues have plagued previous mechanisms. 

We propose a thesis defending the following statement:
\begin{quote}
\emph{Active types allow providers to extend a programming system with new syntax, semantics and editor services from within libraries in a safe and expressive manner.
}
\end{quote}
\end{abstract}
%\newpage

\sloppy

\begin{center}
%\large \textbf{SHF:Small: Active Types}

%\vspace{0.5cm}

%\vspace{0.5cm}

\normalsize
\end{center}

% !TEX root = omar-thesis-proposal.tex
\vspace{-25pt}
\section{Introduction}\label{motivation}
% \begin{quote}\textit{The recent development of programming languages suggests that the simul\-taneous achievement of simplicity 
% and generality in language design is a serious unsolved 
% problem.} -- John Reynolds, 1970 \cite{Reynolds70}\end{quote}
%%One might exp ``generality''
Well-designed general-purpose programming languages are remarkably stable: they rarely need to be revised or forked into dialects, because programmers can express the constructs that they need just as well in orthogonal libraries. %Instead, . . %Indeed, a preponderence of new dialects is clear evidence against claims of generality.  %Not all useful abstractions can be  realized, when considered comprehensively, as libraries.% The problem of achieving both simplicity and generality simultaneously with a single system remains unsolved. 
Abstraction mechanisms informed by type theory, like  the ML module system \cite{MacQueen:1984:MSM:800055.802036}, have put this ideal within reach. 
Yet despite their expressive power, new {dialects} of languages like ML continue to arise. Why is this?

% Unlike libraries organized by a module system, language dialects are not trivially composable.%Constructing a software system using libraries written in different dialects is difficult because they cannot always be composed. %This suggests that library-based mechanisms are still not general enough to encompass many desirable modes of expression.   %Library implementations are not yet satisfying.
%Why would a well-informed programmer consider forming a new dialect? 
Perhaps the most common justification is that the \emph{syntactic cost} of a desirable construct is not  preserved when it is expressed using general-purpose mechanisms. %Syntactic cost is often assessed qualitatively \cite{green1996usability} (though quantitative metrics can also be defined). 
For example, the semantics of regular expressions can be expressed using abstract data types (which in ML are a mode of use of the module system), but as we will detail in Sec. \ref{sec:syntax}, this presents some syntactic difficulties. For programmers in problem domains  where regular expressions are common, e.g. bioinformatics, it is tempting to design a ``domain-specific'' dialect that differs only in that it includes derived syntax (i.e. ``syntactic sugar'') for regular expression patterns. There is no shortage of tools that aid in the construction of such syntactic dialects (e.g. Camlp4 \cite{ocaml-manual}, SugarJ \cite{erdweg2011sugarj}, SugarHaskell \cite{erdweg2012layout} and many others).%Put another way, such embeddings may not preserve the cognitive cost associated with a construct (qualitative criteria are generally used to establish this \cite{green1996usability}). 

Somewhat more rarely, dialects are motivated by the need to extend a general-purpose language with new derived semantic constructs, i.e. constructs that cannot be expressed as purely syntactic sugar, but where the general-purpose language is a suitable translation target. As a minimal example, consider System \textbf{F}, i.e. the polymorphic lambda calculus (perhaps the simplest general-purpose language) \cite{pfpl}. If we wanted to express record types in \textbf{F}, we could not do so simply with derived syntax. We could, however, develop a dialect that included the type and term operators relevant to records, specifying their static semantics directly but their dynamic semantics by translation to the polymorphic lambda calculus (using a standard Church-style encoding).\todo{couple more sentences} %Constructs specified by a type-directed translation like this can be considered \emph{derived semantic constructs}.


% express record types as syntactic sugar over the simply-typed lambda calculus with  binary product types.\footnote{Pairs can of course be expressed as syntactic sugar atop records, though one could argue that using binary products as the more primitive concept is simpler.} The static semantics need to be extended with new type and term operators. However, the simplest way to express the dynamic semantics of the newly introduced term operators is by translation to nested binary products, so we can leave the operational semantics alone. \todo{fill this out} %For example, there are dozens of constructs that go by the name of ``records'' in various languages, each defined by a slightly different collection of primitive operations. \todo{examples} %, encouraged  historically  by the availability of tools like compiler generators and,  more recently, language workbenches \cite{workbenches} and DSL frameworks \cite{dsl}. Unfortunately, taking this approach makes it substantially more difficult for clients to import high-level abstractions orthogonally. 

The problem with designing new language dialects in these situations is that a program written in one dialect cannot safely and idiomatically interface with libraries written in a different dialect. %At best, one can hope that the compilers for the two languages target a common intermediate language,  
To do so would require combining the two dialects into a single language, but there is no systematic method to do so, much less one that guarantees that important metatheoretic properties established about each dialect in isolation will be conserved. As a simple example, two dialects might be known to have an unambiguous concrete syntax in isolation, but if the syntax is na\"ively combined, ambiguities could easily arise. %the ideas housed in one dialect are often only available to programmers willing to do without ideas housed in other dialects. 
%It is thus infeasible to simply allow different contributors to a software system to choose their own favorite dialect for each component they are responsible for. %A complex software system written in multiple distinct language dialects is far too unwieldy for it to be viable. 
Without strong modular reasoning principles to rely upon, it is difficult to construct large software systems and ecosystems. 
%This means that it is difficult to construct large software systems and ecosystems from components written in different dialects. 
The only reasonable approach for software engineers who must program ``in the large'' is to eschew constructs housed in dialects and settle on a single common language, where they can  rely on the guarantees provided by its type and module system. %The designers of these languages may slowly adopt the most generally useful new ideas. 
This has left many useful constructs for programming ``in the small'' obscure.% It is safe to say that in contemporary times, language dialects should be considered harmful. % It is little wonder that many programmers eschew esoteric language dialects in favor of languages that stopped evolving decades ago.

\section{Contributions}
Our aim in this thesis is to make dialect formation less frequently necessary by designing language mechanisms that give library providers the ability to express derived syntactic and semantic constructs from within the language, in a manner that ensures that these constructs can be reasoned about separately and composed arbitrarily. 

%Such mechanisms are colloquially known as \emph{extension mechanisms} because the sorts of constructs they are used to express are those that would otherwise require extensions to the language itself. %Defining a new construct then corresponds to the usual notion of language extension. We will make this intuition more precise later.


%in languages without such mechanisms, expressing such constructs would require extensions to the language itself (a better qualifier might thus be \emph{extension-mitigating}). %We will make these notions more precise as we continue. %Thus, these mechanisms precisely capture the notions of language extension and language composition.



%The mechanisms we propose are designed to decrease the number of constructs that need to be built in \emph{a priori} by a language designer. However, there are still some basic syntactic and semantic decisions that must unavoidably be made. 

To provide a coherent vehicle for these mechanisms, we will introduce a new general-purpose language called Wyvern. Let us briefly summarize its organization. Syntactically, Wyvern has a layout-sensitive textual concrete syntax (i.e. indentation is meaningful). This choice is not fundamental to our proposed mechanisms, but it will be useful for a class of examples that we will discuss later.  Semantically, Wyvern consists of a \emph{module language} and a \emph{core language}. The module language is based directly on the Standard ML module language, with which we assume familiarity for the purposes of this proposal \cite{harper1997programming}. The core language (i.e. the language of types and expressions) is split into an \emph{external language} (EL) and an \emph{internal language} (IL). The semantics of EL constructs are specified by a type-directed translation to the much simpler IL. Notionally, this can be thought of as simply a shifting of the first stage of a type-directed compiler into the language specification \cite{tarditi+:til-OLD}. The internal language we will use is the polymorphic lambda calculus with binary product and sum types and recursive types (features like state and exceptions could also be included, but for simplicity, we will omit them) \cite{pfpl}. %Again, we will make these statements more precise as we continue.%The external language provides local type inference \cite{Pierce:2000:LTI:345099.345100}. % (for the purposes of this work).

We will begin by describing Wyvern's mechanisms for expressing new forms of derived concrete syntax, assuming for the moment that the EL is otherwise specified in a standard way, in Section \ref{sec:syntax}. We then turn in Section \ref{sec:semantics} to a mechanism for introducing new semantic constructs -- in particular, new base types and operators -- into the EL. Constructs that are built in to languages like ML, like $n$-ary tuples/records, labeled sums and others, can in Wyvern be realized as modes of use of this mechanism. In both cases, the mechanisms can be broadly understood as novel forms of \emph{static metaprogramming}. Users define each new syntactic and semantic construct by writing code that generates terms and types. This code is  evaluated statically, in the course of typechecking EL terms.


%Languages that are organized around mechanisms like these are colloquially known as \emph{extensible languages}. 


For each mechanism that we introduce, we will first demonstrate its expressive power by showing how constructs that are, or would need to be, built directly in to contemporary languages can in Wyvern be expressed in libraries. To demonstrate that these mechanisms are theoretically sound, we will then develop a formal specification and proofs of various important metatheoretic properties. The most interesting of these will be various \emph{modular reasoning principles} that guarantee that user-defined syntactic and semantic constructs can be reasoned about separately and then used together in any combination. These justify our classification of Wyvern as a \emph{modularly extensible programming language}.



%Designing a modularly extensible language is a challenge because giving away too much control can easily make reasoning about program behavior difficult and lead to problems that arise only when extensions are used at the same time. For example, a library provider given free reign over syntax and semantics could introduce parsing conflicts, violate {type safety}, and  interfere with other extensions. We aim to design mechanisms that eliminate these possibilities, while providing a degree of expressive power sufficient to express a variety of constructs that are, or would need to be, built directly in to today's languages. %ll constructs that can be defined using a mechanism we will introduce can be reasoned about separately and composed arbitrarily.% With these mechanisms, we anticipate dialect formation being far less common. %Indeed, all of these problems have plagued previous  designs.



% Common dialects of ML that suffice for \cite{harper1997programming}. %We will . 

   %Our challenge will be to design mechanisms that achieve these goals while also being powerful enough to express features that today require dialects.%Our challenge will be to design such extension mechanisms, while still retaining significant expressive power -- we wish to permit library-based expression of a range of features that today are, or would need to be, built into a system by its designers.%It can also make it more difficult to define new sorts of tools for working with the language because they cannot rely on there being a fixed abstract syntax (the canonical \emph{expression problem}). For programmers, 
%It can make it difficult to understand and reason about the type of an unfamiliar term, a critical facet of program comprehension (the \emph{typing discipline problem}).

%So our goal will be to design extension mechanisms that handle these metatheoretic and composition-related issues while still  permitting library-based expression of a range of features that today are, or would need to be, built into a system by its designers. %More specifically, we will show how to delegate control, in a methodical manner, to \emph{static functions}, i.e. user-defined functions written in a \emph{static language}. By layering these mechanisms over a fixed typed internal language, constraining the static language appropriately and validating the code it is used to generate, we will retain key  metatheoretic properties and arrive at powerful modular reasoning principles. %By associating these static functions with type constructors, forming \emph{active type constructors},  we will further show how they can enjoy a usage profile essentially identical to that of built in features. % while maintaining key metatheoretic properties and modular reasoning principles. % By keeping the abstract syntax fixed and layering these mechanisms atop a fixed typed internal language (essentially lifting the first stages of compilation into the language), we avoid most aspects of the expression problem. %By delegating control to user-defined functions provided alongside user-defined type constructors (forming \emph{active type constructors}), we can give library providers substantially more control over these features of the system. 
%We structure each mechanism that we introduce as a \emph{bidirectional type system} combined with an  \emph{elaboration semantics} targeting a fixed typed internal language. 
%By constraining these functions and enforcing critical abstraction barriers between extensions, we will ensure key metatheoretic properties of the language and system components and guarantee that extensions can be reasoned about modularly. %We call user-defined types that introduce new features into the system in this way \emph{active types}.  % also highly expressive.
%But taking a \emph{language-internal approach} to implementing a feature is the most practical. If a feature can be realized by creatively using existing language constructs and distributed as a library, clients face fewer barriers to adoption because it is easy to integrate library-based features into existing projects gradually and granularly and they leverage well-understood and well-developed mechanisms.
% But taking this approach is often \emph{not} possible today %We call designs \emph{monolithic programming systems}.

%To realize a new abstraction or system behavior, such experts can consider either a \emph{language-internal approach}, where they work within an existing language and distribute their solutions as libraries, or a \emph{language-external approach}, where they create a new, distinct programming system (often centered around what has come to be called a new \emph{domain-specific language} \cite{dsl}) or extend an existing system by some mechanism that is not part of the language itself, such as an extension mechanism supported by a {particular} compiler, editor or other tool.

%The result of this work will be the design of a \emph{modularly extensible programming language} called Wyvern.   and end with a brief foray into editor services in Sec. \ref{sec:editor-services}.\todo{fix sections}

\section{Modular Derived Syntax}\label{sec:syntax}
Most contemporary computer programming languages specify a textual concrete syntax. Because it serves as the language's human-facing user interface, it is common practice to include  derived syntactic forms (colloquially, \emph{syntactic sugar}) that capture common idioms more concisely or naturally. % (i.e. considering cognitive dimensions \cite{green1996usability}). 
For example,  list syntax is built in to most dialects of ML, so that instead of having to write \lstinline{Cons(1, Cons(2, Cons(3, Nil)))}, a programmer can equivalently write \lstinline{[1, 2, 3]}. Many languages go further, building in syntax associated with other types of values, like vectors (SML), arrays (Ocaml), commands (Haskell) and syntax trees (Scala).  %The desugaring from the latter to the former is specified by the language itself. %Typically, the language designer controls what forms of derived syntax are built in to the language.

Rather than privileging  particular data structures, Wyvern instead exposes mechanisms that allow library providers to introduce new derived syntactic constructs on their own. %Lists need no special consideration from the language specification.
%The purpose of this section is to motivate and then introduce Wyvern's syntax extension mechanisms. %For forms with a clear analogy to a form in Standard ML, we will assume the  semantics are analagous without providing details.
To motivate our desire for this level of syntactic control, we begin in Sec. \ref{sec:examples} with the example of regular expression patterns expressed using abstract types, showing how the usual workaround of using strings to introduce patterns is not ideal. We then survey existing syntax extension mechanisms in Sec. \ref{sec:syntax-existing}, finding that they involve an unacceptable loss of modularity and other undesirable trade-offs. In Sec. \ref{sec:syntax-contributions}, we outline our proposed mechanisms and discuss how they resolve these issues. We conclude in Sec. \ref{sec:syntax-timeline} with a timeline for remaining work.


\subsection{Motivating Example: Regular Expression Syntax}\label{sec:examples}
Let us begin by taking the perspective of a \emph{regular expression} library provider. Recall that regular expressions are a common way to capture patterns in strings \cite{Thompson:1968:PTR:363347.363387}. We will assume that the abstract syntax of {patterns}, $p$, over strings, $s$, is specified as below:\[p ::= \textbf{empty} ~|~ \textbf{str}(s) ~|~ \textbf{seq}(p; p) ~|~ \textbf{or}(p; p) ~|~ \textbf{star}(p) ~|~ \textbf{group}(p)\]
The most direct way to express this abstract syntax is by defining a recursive sum type \cite{pfpl}. Wyvern supports these as case types, which are analagous to datatypes in ML:

\begin{lstlisting}[numbers=none]
casetype Pattern
  Empty
  Str of string
  Seq of Pattern * Pattern
  Or of Pattern * Pattern
  Star of Pattern
  Group of Pattern
\end{lstlisting}

However, there are some reasons not to expose this representation of patterns directly to clients. First, patterns are usually identified up to congruence. For example, $\textbf{seq}(\textbf{empty}, p)$ is congruent to $p$. It would be useful if congruent patterns were indistinguishable from the perspective of client code. Second, it can be useful for performance reasons to maintain additional data alongside patterns (e.g. a corresponding finite automata here) without exposing this ``implementation detail'' to clients. Indeed, there are many ways to implement regular expression patterns, each with different performance trade-offs. For these reasons, a better approach is to define the following \emph{module type} (a.k.a. \emph{signature} in SML), where the type of patterns, \lstinline{t}, is held abstract. The client of any module \lstinline{P : PATTERN} can then identify patterns as terms of type \verb|P.t|. 
Notice that it exposes an interface otherwise isomorphic to the one available using a case type:

\begin{lstlisting}[deletekeywords={case},numbers=none]
module type PATTERN
  type t
  val Empty : t
  val Str : string -> t
  val Seq : t * t -> t
  val Or : t * t -> t
  val Star : t -> t
  val Group : t -> t
  val case : (
    'a -> 
    (string -> 'a) ->
    (t * t -> 'a) ->
    (t * t -> 'a) ->
    (t -> 'a) ->
    (t -> 'a) -> 
    'a)
\end{lstlisting}

\paragraph{Concrete Syntax} The abstract syntax of patterns is too verbose to be used directly in all but the most trivial examples, so patterns are conventionally written using a more concise concrete syntax. For example, the concrete syntax \lstinline{A|T|G|C} corresponds to abstract syntax with the following encoding:
\begin{lstlisting}[numbers=none,mathescape=|]
P.Or(P.Str "SSTRAESTR", P.Or(P.Str "SSTRTESTR", P.Or(P.Str "SSTRGESTR", P.Str "SSTRCESTR")))
\end{lstlisting} 



% \begin{figure}
% \begin{center}
% \includegraphics[scale=.55]{regex-palette.pdf}\\
% $\Downarrow$ \text{(pressing \textbf{Enter})}\\
% \includegraphics[scale=.55]{regex-code-generated.pdf}
% \end{center}
% \vspace{-20px}
% \caption{An example of type-specific editor services, here shown for Java.}
% \label{fig:regex-palette}
% %\vspace{-10px}
% \end{figure}

%In a conventional \emph{monolithic} programming system, support for each of these features would need to be built into the language and tools. 

To encode the concrete syntax of patterns, regular expression library providers usually provide a utility function that converts strings to patterns. Because there may be many pattern implementations, the usual approach is to define a parameterized module (a.k.a. \emph{functor} in SML) defining such utility functions generically, like this:

\begin{lstlisting}[numbers=none]
module PatternUtil(P : PATTERN)
  fun parse(s : string) : P.t
    (* ... pattern parser here ... *)
\end{lstlisting}
This allows the client of any module \lstinline{P : PATTERN} to use the following definitions:
\begin{lstlisting}[numbers=none]
module PU = PatternUtil(P)
let pattern = PU.parse
\end{lstlisting}
to construct patterns like this:
\begin{lstlisting}[numbers=none]
pattern "SSTRA|T|G|CESTR"
\end{lstlisting}
 %Again, none of our goals are comprehensively achieved.
This approach is imperfect for several reasons:
\begin{enumerate} 
\item Our first problem is syntactic (or, one might say, aesthetic): string escape sequences conflict syntactically with pattern escape sequences. For example, the following will not be well-formed:
\begin{lstlisting}[numbers=none,mathescape=|]
let ssn = pattern "SSTR\d\d\d-\d\d-\d\d\d\dESTR"
\end{lstlisting}
In fact, when compiling an analagous term using SML of New Jersey (SML/NJ), we encounter the rather puzzling error message \verb|Error: unclosed string|. Using \verb|javac|, we encounter \verb|error: illegal escape character|. In a small lab study, we observed that this error was common, and nearly always initially misinterpreted by even experienced programmers who hadn't used regular expressions recently \cite{Omar:2012:ACC:2337223.2337324}. The workaround is to  double backslashes:
\begin{lstlisting}[numbers=none]
let ssn = pattern "SSTR\\d\\d\\d-\\d\\d-\\d\\d\\d\\dESTR"
\end{lstlisting}


\item Our second problem has an impact on both correctness and performance: pattern parsing does not occur until the pattern is evaluated. For example, the following malformed pattern will only trigger an error when this term is evaluated during the full moon: %Achieving this goal is an explicit goal of this proposal, so we are obviously not happy with this.

\begin{lstlisting}[numbers=none]
case moon_phase
  Full => pattern "SSTR(GCESTR" (* malformedness not statically detected *)
  _ => (* ... *)
\end{lstlisting}
Such issues can be found via testing, but empirical data gathered from open source projects suggests that there are many malformed regular expression patterns that are not detected by a project's test suite ``in the wild'' \cite{spishak2012type}. 

Parsing patterns at run-time also incurs a performance penalty. To avoid incurring it every time the pattern is encountered during evaluation, an appropriately tuned caching strategy must be introduced, increasing client complexity.% Regular expressions are often used across large datasets in scientific applications, so the absolute peformance penalty can be non-trivial.

\item The final problem is that using strings to introduce patterns makes it more likely that programmers will use string concatenation to construct patterns derived from other patterns or from user input. For example, consider the following function:
\begin{lstlisting}[numbers=none,escapechar=|]
fun example_rx(name : string)
  pattern (name ^ "SSTR: \\d\\d\\d-\\d\\d-\\d\\d\\d\\dESTR")
\end{lstlisting}
The (unstated) intent here is to treat \lstinline{name} as a pattern matching only itself, but this is not the observed behavior when \lstinline{name} contains special characters that are meaningful in patterns. The correct code is more verbose, again resembling abstract syntax: 
\begin{lstlisting}[numbers=none]
fun example_fixed(name : string)
  P.Seq(P.Str(name), P.Seq(pattern "SSTR: ESTR", ssn)) (* ssn as above *)
\end{lstlisting}

The mistake was the result of a programmer using a flawed heuristic, so it could be avoided with sufficient developer discipline. 
The problem is that it is difficult to enforce this discipline mechanically. Both functions above have the same type and behave identically at many inputs, particularly those that would be expected during typical executions of the program (i.e. alphabetic names). In applications that query sensitive data, failures of this variety are observed to be both common and catastrophic from the perspective of security \cite{owasp2013}. Ideally, our library would be able to make it more difficult to inadvertently introduce subtle security bugs like this.
\end{enumerate}
% \begin{enumerate}
% \item \textbf{Conventional Concrete Syntax.} Patterns must be written in exactly the conventional manner, shown in blue below. Curly braces serve as an ``unquote'' form to splice one pattern into another:
% \begin{lstlisting}[numbers=none]
% let N : Pattern = /SURLA|T|G|CEURL/
% let BisI : Pattern = /SURLGC{EURLNSURL}GCEURL/
% let twoDigits : Pattern = /SURL\d\dEURL/\end{lstlisting}

% %The choice of delimiter, i.e. matching forward slashes here, is not important.
% %The cognitive load of reading and writing patterns is low and patterns are parsed once at compile-time. 
% \item \textbf{Compile-Time Parsing.} Pattern parsing must occur at compile-time, i.e. it must not incur run-time cost and malformed patterns, e.g. \lstinline{/SURLGC(EURL/}, must result in {compile-time} errors no matter where they appear in the program. 
% \item \textbf{Security By Default.} Strings must not be treated as patterns, nor used idiomatically to construct patterns. For example, the function below must be ill-typed because \lstinline{name} is not of type \lstinline{Pattern}:
% \begin{lstlisting}[numbers=none]
% fun example(name : string) : Pattern
%   let twoDigits : Pattern = /SURL\d\dEURL/
%   /SURL{EURLnameSURL}: {EURLtwoDigitsSURL}EURL/
% \end{lstlisting}
%     It is also acceptable for \lstinline{example("SSTR[a-z]ESTR")} to be equivalent to \lstinline{/SURL\[a-z\]: \d\d/}, i.e. strings are treated as patterns matching only themselves. 
% It is \emph{not} acceptable for \lstinline{example("SSTR[a-z]ESTR")} to be equivalent to \lstinline{/SURL[a-z]: \d\d/}. 

%     This is a security issue because if \lstinline{name} is derived from user input, such confusion could lead to \emph{injection attacks}, which are both common and catastrophic when using regular expression libraries that fail to achieve these goals, discussed below \cite{owasp2013}. 
%   %\item A more advanced semantics might ensure that out-of-bounds backreferences to a captured group do not occur \cite{spishak2012type}. For example, we would want \verb|BisI|, above, to have the more precise type \lstinline{Pattern(1)}, i.e. a pattern with one captured group. 
% % \end{enumerate}
% %When an error is found, an intelligible error message is provided.
% %\item An \textbf{implementation} that partially or fully compiles known regular expressions into the efficient internal representation that will be used by the regular expression matching engine (e.g., a finite automata \cite{Thompson:1968:PTR:363347.363387}) ahead of time. In most languages, this compilation step occurs at run-time, even if the pattern is fully known at compile-time, thereby introducing performance overhead into programs. If the developer is not careful to cache compiled representations, regular expressions used repeatedly in a program might be needlessly re-compiled on each use. %By performing this step ahead-of-time, these dangers can be avoided.
% %\item A range of \textbf{editor services} that support syntax highlighting for patterns, retrieval of relevant documentation, interactive testing, and pattern extraction from example strings have been shown to be helpful for programmers when working with complex regular expressions \cite{ACC_VLHCC}. An example of some of these is shown in Figure \ref{fig:regex-palette}: the editor service shown helps programmers generate correct code, here via the Java standard library's implementation of regular expressions (discussed below).
% \end{enumerate}
% Let us consider the ways that providers of regular expression libraries commonly choose to define the type \verb|Pattern|: as synonymous to \lstinline{string}, as a recursive sum type, or as an abstract type.

% \vspace{-5px}
% \paragraph{Patterns as Strings} The most na\"ive choice is to define \lstinline{Pattern} as synonymous to \lstinline{string}:

% \begin{lstlisting}[numbers=none]
% type Pattern = string
% \end{lstlisting}
% The only reason to consider this definition is that it allows us to approximate pattern syntax using standard string literal syntax:

% \begin{lstlisting}[numbers=none]
% let N : Pattern = "SSTRA|T|G|CESTR"
% \end{lstlisting}
% To splice one pattern into another, we can perform string concatenation:
% \begin{lstlisting}[numbers=none,escapechar=|]
% let BisI : Pattern = "SSTRGCESTR" ^ N ^ "SSTRGCESTR"
% \end{lstlisting}
% or use string interpolation (as in Scala or SML/NJ using quotation/antiquotation \cite{SML/Quote}):
% \begin{lstlisting}[numbers=none,mathescape=|]
% let BisI : Pattern = s"SSTRGC${ESTRNSSTR}GCESTR"
% \end{lstlisting}

% Though this comes close, goal 1 is not truly achieved because the syntax of strings can conflict with the syntax for patterns. 

% Goal 2 is not achieved because parsing occurs only at run-time when we match a string against a pattern (worse, every time we do so unless the implementation uses an effective caching strategy).

% Goal 3 is also not achieved because the following code typechecks:
% \begin{lstlisting}[numbers=none,mathescape=|]
% fun example_bad_1(name : string) : Pattern
%   let twoDigits : Pattern = "SSTR\\d\\dESTR"
%   s"SSTR${ESTRnameSSTR}-${ESTRtwoDigitsSSTR}"ESTR
% \end{lstlisting}
% but \lstinline{example_bad_1("SSTR[a-z]ESTR")} evaluates to \lstinline{"SSTR[a-z]-\\d\\dESTR"}.



% \paragraph{Patterns as Recursive Sums} A more semantically justifiable strategy is to define \lstinline{Pattern} as a recursive sum type  \cite{pfpl}. Wyvern supports these in the form of case types, which are analagous to datatypes in ML:

% \begin{lstlisting}[numbers=none]
% casetype Pattern
%   Empty
%   Str of string
%   Seq of Pattern * Pattern
%   Or of Pattern * Pattern
%   Star of Pattern  
% \end{lstlisting}

% More generally, we can hold the representation of patterns abstract while exposing an isomorphic interface to clients by defining an abstract type in a {module} opaquely ascribed the following {module type} (a.k.a. \emph{signature} in SML): % exposes the constructors and case analysis as functions:

% \begin{lstlisting}[deletekeywords={case},numbers=none]
% module type PATTERN
%   type t
%   val Empty : t
%   val Str : string -> t
%   val Seq : t * t -> t
%   val Or : t * t -> t
%   val Star : t -> t
%   val case : (
%     'a -> 
%     (string -> 'a) ->
%     (t * t -> 'a) ->
%     (t * t -> 'a) ->
%     (t -> 'a) ->
%     'a)
% \end{lstlisting}
% The client of any module \lstinline{P : PATTERN} can define \verb|Pattern| as synonymous to \verb|P.t|: 
% \begin{lstlisting}[numbers=none]
% type Pattern = P.t
% \end{lstlisting}



%The first one might not be found incorrect without rigorous testing with unexpected inputs. Goal 3 expresses the ideal of designing a language where security does not require considering such subtle issues extralinguistically.%This is precisely why injection attacks are so difficult to detect.

% The dialect of Standard ML implemented by SML/NJ, recognizing that directly using constructors is often less than ideal, provides a facility for quotation/antiquotation \cite{SML/Quote}, which if introduced into Wyvern could be used as follows:
% \begin{lstlisting}[numbers=none,escapechar=|,mathescape=|]
% fun example(p : string)
%   let twoDigits = P.qparse `SQT\d\dEQT`
%   P.qparse `SQT^(EQTP.Str(p)SQT)-^(EQTtwoDigitsSQT)EQT`
% \end{lstlisting}

% Here, backticks are used to indicate a quoted term within which carets serve as an antiquote operator. This still represents a  compromise against goal 1. The most serious problem is that, once again, parsing does not occur until run-time. There are also  conflicts between quotation syntax and pattern syntax, i.e. \verb|^| conventionally means character set negation inside patterns, which conflicts with antiquotation here. This mechanism also does not support antiquotation at more than one type, so we could not define the antiquote operator to insert the \verb|Str| constructor automatically, as suggested in goal 3. %For syntax that necessary involves antiquotation at multiple types (e.g. concrete syntax for a programming language with multiple syntactic sorts, HTML which might include Javascript and CSS, and others), this 

\subsection{Existing Approaches}\label{sec:syntax-existing} These problems with string-oriented approaches to concrete syntax arise in many similar scenarios \cite{TSLs}, motivating  research on more direct syntax extension mechanisms \cite{Bravenboer:2007:PIA:1289971.1289975}.

The simplest such mechanisms are those where each new syntactic form is described by a single equation. For example, the surface language of Coq (called Gallina) includes such a mechanism \cite{Coq:manual}. A theoretical account of such mechanisms has been developed by Griffin \cite{5134}. Unfortunately, these mechanisms are not able to express pattern syntax in the conventional manner for various reasons. For example, sequences of characters should not be parsed as identifiers.


Syntax extension mechanisms based on context-free grammars like  Sugar* \cite{erdweg2013framework}, Camlp4 \cite{ocaml-manual} and many others are more expressive, and would allow us to directly introduce pattern syntax into our base language's grammar. However, this is perilous because none of the mechanisms described thusfar guarantee \emph{syntactic composability}, i.e. as stated in the Coq manual, ``mixing different symbolic notations in [the] same text may cause serious parsing ambiguity''. If another library provider used similar syntax for a different variant of regular expressions, or for an entirely unrelated abstraction, then a client could not simultaneously use both libraries in the same scope. % Resolving such parsing amibiguities is left to each client of the library. 

In response to this problem, Schwerdfeger and Van Wyk have developed a modular analysis that accepts only syntax extensions that use a unique starting token and satisfy some subtle conditions on follow sets of base language non-terminals \cite{conf/pldi/SchwerdfegerW09}. However, simple starting tokens like \lstinline{pattern} cannot be guaranteed to be globally unique, so we would need to use a more verbose token like \lstinline{edu_cmu_wyvern_rx_pattern}. There is no simple, principled way to define scoped abbreviations for starting tokens because this mechanism is language-external.

In any case, we must now decide how our newly introduced derived forms desugar to forms in our base language, which in this case requires determining which module the constructors the desugaring uses will come from. Clearly, simply assuming that a module named \lstinline{P} satisying \lstinline{PATTERN} is in scope is a brittle solution. Indeed, we should expect that the extension mechanism actively prevents such capture of specific variable names to ensure that variables (including module variables) can be freely renamed. However, we note that such \emph{hygiene mechanisms} are only well-understood when performing term-to-term rewriting (as in Lisp-style \emph{macro systems}\todo{citation}) or simple equational systems like those found in Coq. For more flexible mechanisms, the issue is a topic of ongoing research (none of the grammar-based mechanisms described above enforce hygiene).

Putting aside the question of hygiene, we can address the problem by requiring that the client explicitly identify the module the desugaring should use:
\begin{lstlisting}[numbers=none]
let N = edu_cmu_wyvern_rx_pattern[P] /A|T|G|C/
\end{lstlisting}
Syntactically, this is the best we can do using existing approaches.  

These approaches further suffer from a paucity of direct reasoning principles, i.e. the program can only be reasoned about post-desugaring. Given an unfamiliar piece of syntax, there is no simple method for determining what type it will have, or even for identifying which extension determines its desugaring, causing difficulties for both humans (related to code comprehension) and tools. 

% \begin{enumerate}
% \item Given an unfamiliar piece of syntax, there is no simple method for determining what type it has, or to identify which extension determines its desugaring, hindering code comprehension.
% \item It requires that the syntax be expressed as a grammar that can be consumed by a particular parser generator. Not all syntax  is most naturally expressed in this way, but there is no way to use a ``hand-rolled'' parser.
% \end{enumerate}

\subsection{Contributions}\label{sec:syntax-contributions}
We propose designing a  syntax extension mechanism that addresses all of the problems just described. It consists of two constituent concepts:
\begin{itemize}
\item The concept that the mechanism is oriented around is the \emph{typed syntax macro} (TSM). %A TSM is applied to a \emph{delimited form}, which can contain  arbitrary syntax in its \emph{body}, and zero or more module or type parameters. Based on the body of the delimited form, the TSM statically computes an elaboration to a {base form}. 
For example, consider the following concrete term:
\begin{lstlisting}[numbers=none]
pattern[P] /SURLA|T|G|CEURL/
\end{lstlisting}

The TSM \lstinline{pattern} is being applied to a module parameter, \lstinline{P}, and a \emph{delimited form}, \lstinline{/SURLA|T|G|CEURL/}. This term elaborates \emph{statically} to the following:

% import cs.cmu.edu/~wyvern/regex as R
% let syntax pattern = R.pattern
% let module P = R.P
% The first line imports our top-level regular expression module (identified uniquely by a URI) and binds a shorter name, \lstinline{R}, to it. The second line binds a local name to the  TSM named \lstinline{pattern} that \lstinline{R} exports. The third line gives a shorter name, \lstinline{P}, to the module \lstinline{R.P : R.PATTERN}, as defined previously. The fourth line then applies \lstinline{pattern} to this module parameter and a form delimited by forward slashes. This causes it to elaborate \emph{statically} to the following base form:
\begin{lstlisting}[numbers=none]
P.Or(P.Str "SSTRAESTR", P.Or(P.Str "SSTRTESTR", P.Or(P.Str "SSTRGESTR", P.Str "SSTRCESTR")))
\end{lstlisting}

The TSM is defined as shown below:
\begin{lstlisting}[numbers=none]
syntax pattern[P : PATTERN] for P.t
  fn (ps : ParseStream) => (* pattern parser here *)
\end{lstlisting}
We declare a module parameter (calling it \lstinline{P} here is unimportant, i.e. the TSM can be used with \emph{any} module satisfying module type \lstinline{PATTERN}), and the type clause \lstinline{for P.t}, which guarantees that any use of this TSM will either be ill-typed or elaborate to a term of type \lstinline{P.t}. The elaboration is computed by the compile-time action of the parse function defined in the indented block above. This function must be of type \lstinline{ParseStream -> Exp}, where the type \lstinline{ParseStream} gives the function access to the \emph{body} of the delimited form (in blue above) and the type \lstinline{Exp}  encodes the abstract syntax of the base language. Both types are defined in the Wyvern prelude, which is a set of definitions available ambiently.

When the type of the term is known, e.g. due to a type annotation on the let binding, the module parameter \lstinline{P} can be inferred:
\begin{lstlisting}[numbers=none]
let N : P.t = pattern /SURLA|T|G|CEURL/
\end{lstlisting}

TSMs can be abbreviated using a let-binding style mechanism:
\begin{lstlisting}[numbers=none]
let syntax pat = pattern[P]
pat /SURLA|T|G|CEURL/
\end{lstlisting}

TSMs can parse portions of the body of the delimited form as a base language term, to support splicing syntax:
\begin{lstlisting}[numbers=none]
let N = pat /SURLA|T|G|CEURL/
let BisI = pat /SURLGC({EURLNSURL})GCEURL/
\end{lstlisting}
A hygiene mechanism ensures that only those portions of the elaboration derived from such spliced terms can refer to variables in the surrounding scope.

Strings are never involved, so the attendant security issues described previously are easily avoidable. That is, the following 

\item To support an even more concise usage profile, Wyvern also  supports \emph{type-specific languages} (TSLs), which allow library providers to associate a TSM directly with a declared type. For example, the module \lstinline{P} above can associate \lstinline{pattern} with \lstinline{P.t}. Local type inference then determines which TSM is applied implicitly to handle a form not prefixed by a TSM name. For example, the following is equivalent to the above:
\begin{lstlisting}[numbers=none]
let N : P.t = /SURLA|T|G|CEURL/
\end{lstlisting}
\end{itemize}
\subsubsection{Typed Syntax Macros (TSMs)}\label{sec:tsms}
For simplicitly, let us begin with a typed syntax macro providing pattern syntax using the case type encoding above:
\begin{lstlisting}[numbers=none]
syntax pattern => Pattern = e_parser
\end{lstlisting}
The term \lstinline{e_parser}, elided here, must have type \lstinline{ParseStream -> Exp}, where \lstinline{ParseStream} classifies a sequence of characters and \lstinline{Exp} classifies a reified Wyvern expression. 

\subsubsection{Type-Specific Languages (TSLs)}\label{sec:tsls}

\subsection{Timeline}\label{sec:syntax-timeline}
SAC, ECOOP.

TODO: module-parameterized TSMs

\section{Extensible Semantics}\label{sec:extensible-semantics}


\subsubsection{Example 2: Regular Strings}\label{sec:rstr}
The example above deals with how regular expression patterns are  encoded, introduced and reasoned about. Going further, programmers may also benefit from a semantics where they can statically constrain strings to be within a particular regular language \cite{sanitation-psp14}. For example, a programmer might want to ensure that the arguments to a function that creates a database connection given a username and password are alphanumeric strings, ideally including this specification directly in the type of the  function:
\begin{lstlisting}[numbers=none]
type alphanumeric = rstring /SURL[A-Za-z0-9]+EURL/
val connect : (alphanumeric * alphanumeric) -> DBConnection
\end{lstlisting}

This example requires that the language include a {type constructor}, \verb|rstring|, indexed by a statically known regular expression, written using the syntax described above. Ideally, we would be able to use standard string literal syntax for such regular strings as well, e.g.
\begin{lstlisting}[numbers=none]
let connection = connect("admin", "password")
\end{lstlisting}

To be useful, the language would also need a static semantics for standard operations on regular strings. For example, concatenating two alphanumeric strings  should result in an  alphanumeric string. This should not introduce additional run-time cost as compared to the corresponding operations on standard strings. Coercions that can be determined statically to be valid in all cases due to a language inclusion relationship should have trivial cost, e.g. coercions from alphabetic to alphanumeric strings.

Regular strings might go beyond simply \emph{refining} standard string types (i.e. specifying verification conditions atop an existing semantics \cite{Freeman91}). For example, they might define an operation like \emph{captured group projection} that has no analog over standard strings:
\begin{lstlisting}[numbers=none]
let example : rstring /SURL(\d\d\d)-(\d\d\d\d)EURL/ = "555-5555"
let group0 (* : rstring /\d\d\d/ *) = example#0
\end{lstlisting}
It should be possible to give this operation a cost of $\mathcal{O}(1)$. %From a language implementation perspective, this requires that different regular string types  have different representations, also not possible when using a refinement system over strings. %This is not possible if regular strings are defined using a refinement system atop a language with only strings. %This example requires introducing new operators into the system that did not already exist, and using an underlying represention that differs for different type indices. 

\subsubsection{Example 3: Labeled Products with Functional Update Operators}\label{sec:lprod}
The simplest way to specify the semantics of product types is to specify only nullary and binary products \cite{pfpl}. However, in practice, many more general but also more complex variations on product types are built in to various dialects of ML and other languages, e.g. $n$-ary tuples, labeled tuples, 
records (identified up to reordering), 
records with width and depth coercions \cite{Cardelli:1984:SMI:1096.1098} and records with functional update operators (also called \emph{extensible records}) \cite{ocaml-manual}. {The Haskell wiki notes that ''extensible records  are not implemented in GHC. The problem is that the record design space is large, and seems to lack local optima. [...] As a result, nothing much happens.'' \cite{GHCFAQ}}%, 
 %mutable fields \cite{ocaml-manual}, 
%field delegation \cite{atlang-gpce14} \todo{gpce submission} 
%and 
% ``methods'' (i.e. pure objects) \cite{TSLs}


We would ideally like to avoid needing the language designer to decide \emph{a priori} on just one privileged point in this large design space. Instead, the language designer might define only nullary and binary products, and include an extension mechanism that makes it possible for these other variations on products to be defined as libraries and  used together without conflict. For example, we would like to define the semantics of labeled products, which have labeled rows like records but maintain a row ordering like tuples, in a library. An example of a labeled product type (constructed by the type constructor \lstinline{lprod}) classifying conference papers might be:
\begin{lstlisting}[numbers=none]
type Paper = lprod {
  title : rstring /SURL.+EURL/,
   conf : rstring /SURL[A-Z]+ \d\d\d\dEURL/
}
\end{lstlisting}

The row ordering should make it possible to introduce values of this type either with or without explicit labels, e.g.
\begin{lstlisting}[numbers=none]
fun make_paper(t : rstring /SURL.+EURL/) : Paper = {title=t, conf="EXMPL 2015"}
\end{lstlisting}
should be equivalent to
\begin{lstlisting}[numbers=none]
fun make_paper(t : rstring /SURL.+EURL/) : Paper = (t, "EXMPL 2015")
\end{lstlisting}
(note that we aim to be able to unambiguously re-use standard record and tuple syntax.)

We should then be able to project out rows by providing a positional index or a label:
\begin{lstlisting}[numbers=none]
let test_paper = make_paper "Test Paper"
test_paper#0
test_paper#conf
\end{lstlisting}

We might also want our labeled tuples to support functional update operators. For example, an operator that dropped a row might be used like this:
\begin{lstlisting}[numbers=none]
let title_only (* : lprod {title : rstring /.+/} *) = test_paper.drop[conf]
\end{lstlisting}
An operation that added a row, or updated an existing row, might be used like this:
\begin{lstlisting}[numbers=none]
fun with_author(p : Paper, a : rstring /SURL.+EURL/) = p.ext(author=a)
\end{lstlisting}
% \subsection{Language-External Approaches}\label{external-approaches}


% In situations like these where a library-based approach is not satisfactory, providers must today take a \emph{language-external approach}, specifying a new system dialect, reasoning about its metatheory monolithically and implementing it using a {compiler generator} \cite{brooker1963compiler}, {language workbench} \cite{erdweg2013state}, {DSL framework} \cite{fowler2010domain}, an extension mechanism for a {particular} compiler\footnote{Compilers that modify, or allow modification of, the semantics of their base language, rather than simply permitting semantics-preserving optimizations, should be considered a pernicious means for creating dialects. 
% %That is, some programs that purport to be written in C, Haskell or Standard ML are actually written in compiler-specific dialects of these languages.
% }, editor or other tool, or by forking an existing system implementation. 

% For example, a researcher interested in providing the regular expression related features just described (let us refer to these collectively as \texttt{R}) might design a new system with built-in support for them, perhaps basing it on an existing system containing standard general-purpose features, \texttt{G} (e.g. the features of Standard ML). A different researcher developing a new language-integrated parallel programming abstraction, \texttt{P}, might  take the same approach (e.g. Manticore is such a system based on SML \cite{conf/popl/FluetRRSX07}). A third researcher, developing a type system for reasoning about units of measure (\texttt{Q}) might again do the same \cite{conf/cefp/Kennedy09}. This results in a collection of  systems, as diagrammed in Figure \ref{approaches}a. 

% Although these systems are valid ``proofs of concept'', this is a rather weak notion because no abstraction will be used entirely in isolation. Such an approach gives us no rigorously justifiable  reason to believe that it can safely and naturally be merged with others specified or implemented in the same way. That is, one must either use the system defining features \texttt{G+R}, \texttt{G+P} or \texttt{G+Q}. There is no system defining \texttt{G}, \texttt{R}, \texttt{P} and \texttt{Q} in other combinations. % and there is no formal, general mechanism for merging these systems.
% We argue that, due to these problems of composition, taking a language-external approach to realizing a new feature should be considered harmful and avoided whenever possible. 

% \begin{figure}
% \begin{center}
% \includegraphics[scale=.48]{approaches.pdf}
% \end{center}
% \vspace{-20px}
% \caption{\small (a) When taking a language-external approach, new features are packaged together into separate languages and tools. (b) We propose an approach where there is one extensible host language and the compile-time and edit-time logic governing new constructs is expressed within safely composable libraries.}
% \label{approaches}
% %\vspace{-10px}
% \end{figure}
%  %
% % More specifically, there is no guarantee of \textbf{orthogonality} or even \textbf{interoperability}.% This has limited the broad adoption of these kinds of innovations.%This latter method couples the semantics of the feature to the implementation details of a particular tool. Because the use of one implementation entails a different semantics for the feature than another, the extended tool acts, \emph{de facto}, as a distinct system for our purposes. 

% %\paragraph{Orthogonality} Features implemented by language-external means like these cannot be adopted individually, but instead are only available coupled to a fixed collection of other features. This makes adoption more costly when these incidental features are  not desirable or are insufficiently developed (``toy languages''), or when the features bundled with a different language or tool are simultaneously desirable. 

% % Recent evidence indicates that this is one of the major barriers preventing research  from being driven into practice. For example, developers prefer high-level language-integrated parallel programming abstractions that provide  stronger semantic guarantees  \cite{cave2010comparing}, but library-based approximations are far more widely adopted because the ``parallel programming languages'' privilege only a few  specialized abstractions at the language level. In contrast, it is widely acknowledged that  different specialized abstractions are more appropriate in different situations \cite{Tasharofi:2013rc}. Moreover,  parallel programming is rarely the only relevant specialized concern. Support for regular expressions as above would be simultaneously desirable for processing large amounts of genomic data in parallel.% but using these features together in the same compilation unit would be difficult or impossible if implemented using language-external means. Indeed, switching to a ``parallel programming language'' would likely make it \emph{more} difficult to use regular expressions, as these are likely to be less well-developed in a specialized language than in an established general-purpose language.% This intuition was perhaps most succinctly expressed by a participant in a recent study by Basili et al. \cite{basili2008understanding}:  ``I hate MPI, I hate C++. [But] if I had to choose again, I would probably choose the same.'' %Similarly, a language and tools designed primarily to support regular expressions might make an interesting research project, but it would not be a suitable tool for writing large applications with more varied needs.

% %\item Developing a new language and its associated tools places a significant development burden on providers who may wish only to promote a few core innovations, although tools like compiler generators, language workbenches and easy-to-extend tools can decrease this burden. 
% %\item 

% %Clients seem to prioritize the ability to choose different features for different portions of an application. 
% %If calling between languages were safe and easy, then using a variety of specialized languages and associated tools might be less problematic. In fact, s
% %Recognizing the limitations of relying on monolithic collection of primitives, some researchers have advocated instead for a model where multiple languages used within a single application, calling it the \emph{language-oriented approach} to software development \cite{languageoriented}. 

% % \paragraph{Interoperability} Even in cases where, for each component of a software system, a programming system considered entirely satisfactory by its developers is available (e.g. a team goes through the trouble of implementing \verb|G+R+P| by reading papers about \verb|G+R| and \verb|G+P| and disentangling orthogonality-related issues), there remains a problem at any interface between  components written using a different combination of features. An interface that  externally exposes a specialized construct particular to one language (e.g. a function that requires a quantity having a particular unit of measure) cannot necessarily be safely and naturally consumed from another language (e.g. a parallel programming language). Tool support is also lost when calling into different languages. %We call this the \emph{interoperability problem}. % programs written by clients of a certain collection of features cannot always interface with programs written by clients of other features  in a safe, performant and natural manner.

% % One strategy taken by proponents of a {language-oriented approach} \cite{journals/stp/Ward94} to partially address the interoperability problem is to  target an established intermediate language and use its constructs as a common language for communication between components written in different languages. Scala \cite{200464/IC} and F\# \cite{pickering2007foundations} are examples of prominent general-purpose languages that have taken this approach, and most DSL frameworks also rely on this strategy. As indicated in Figure \ref{approaches}a, this only enables interoperability in one direction. Calling into the common language becomes straightforward and safe, but calling in the other direction, or between the languages sharing the common target, does not, unless these languages are only trivially different from the common language. 

% % %This approach only works well when new languages consist of constructs that can also be expressed safely and almost as naturally in the common language.
% % %But many of the most innovative constructs found in modern languages (often, those that justify their creation) are difficult to define in terms of existing constructs in ways that guarantee all necessary invariants are statically maintained and that do not require large amounts boilerplate code and run-time overhead. 
% % As a simple example with significant contemporary implications, F\#'s type system does not admit \verb|null| as a value for any type both defined and used within F\# code, but maintaining this sensible internal invariant still requires dynamic  checks because the stricter typing rules of F\# do not apply when F\# data structures are constructed by other languages on the Common Language Infrastructure (CLI) like C\# or SML.NET. This is not an issue exclusive to intermediate languages that make regrettable choices regarding \verb|null|, however. The F\# type system also includes support for checking that units of measure are used correctly \cite{syme2012expert, kennedy1994dimension}, but this more specialized static invariant is left entirely unchecked at language boundaries. Indeed, guidelines for F\# suggest that exposing functions that operate over values having units of measure, datatypes or tuples is not recommended when a component ``might be used'' from another language \cite{syme2012expert} because it is awkward to construct and consume these from other languages without the convenient primitive operations (e.g. pattern matching) and syntax that F\# includes. SML.NET prohibits exposing such types at component boundaries altogether. Moreover, it also cannot naturally consume F\# data structures, despite having a rather similar syntax and semantics in most ways (both languages directly descend from ML). 
% %In Scala, traits that have default method implementations are difficult to implement from Java or other JVM languages and the workaround can break if the trait is modified \cite{scalatraitinterop}. 
% %In some cases, desirable features must be omitted entirely due to concerns about interoperability. F\#, for example, aimed to retain source compatibility with Ocaml code, but due to the need for bidirectional interoperability with CLI languages, it does not support features like polymorphic variants, modules or functors \cite{ocaml-manual} because they have no apparent analogs in the type system of the CLI.
% %\end{itemize}

% %\subsection{Language-Integrated Approaches}\label{language-integrated-approaches}


% %Such libraries have been called \emph{active libraries}  \cite{activelibraries}. %Features implemented within active libraries can be imported individually, unlike features implemented by external means, giving us a potential means to avoid the problems of orthogonality and interoperability just described.

% % We must proceed with caution, however: critical issues having to do with {safety} must be overcome before language-integrated extension mechanisms can be introduced into a system. If too much control over  these core features of the system is given  to developers, the system's metatheory may become weak. 
% % %For example, an extension could weaken important metatheoretic guarantees previously provided by the system. 
% % For example, parsing ambiguities might arise if the syntax can be modified arbitrarily. Type safety  may not hold if the static and dynamic semantics of the language can be modified or extended arbitrarily from within libraries. Furthermore, even if extensions can be shown not to cause such problems in isolation, there may still be conflicts between extensions that could weaken their semantics, leading to subtle problems that only appear when two extensions are used together. %As a simple example, if two active libraries introduce the same syntactic form but back it with differing (but individually valid) semantics, this ambiguity  would only manifest itself when both libraries were imported within the same scope. %Resolving these kinds of ambiguities requires significantly more expertise with parser technology than using the syntax itself does. 
% % These issues have plagued previous attempts to design language-integrated extensibility mechanisms.\todo{I can include a more detailed related work section if requested.}% We will briefly review some of these attempts below, then return to our approach.% To prevent them, our mechanisms will organize  extension logic around types to guarantee that extensions are both safe in isolation and also safely composable in any combination. 


%  %This represents a minimalist approach to system design -- the conventional distinction between built-in and user-defined constructs is blurred and most features of the system are orthogonally implemented as {libraries}, rather than by the maintainers of the system.

% %The mechanisms we describe will do so primarily by delimiting the scope of an extension to expressions of a single user-defined type or family of types. 

% %This can be thought of as a more pernicious form of the conflict that arises when two globally-accessible constructs are given the same name. n languages without universal namespacing mechanisms (e.g. C, JavaScript, \LaTeX, ML and many others). 

% %The extension mechanism\todo{elaborate on safety requirements + tension between expressiveness and safety, merge with next paragraph}. must be expressive enough to allow users to associate rich run-time, compile-time and edit-time behaviors with user constructs directly, while being sufficiently restrictive to maintain the global safety properties of the language and system as a whole, and to ensure that constructs cannot interfere with one another. 

% !TEX root = omar-thesis-proposal.tex
\section{Related Work}
We review existing approaches that are available to researchers and domain experts who wish to develop novel constructs below.

\subsection{Language-Oriented Approaches}
\subsubsection{Language Frameworks}
A number of tools have been developed to assist with this task of developing new languages and their associated tools, including compiler generators, language workbenches and domain-specific language frameworks (see \cite{fowler2010domain} for a review). In some cases, these tools allow language features to be defined modularly and composed differentially to produce a variety of different languages. To our knowledge, none of these mechanisms guarantee that any combination of features can be safely composed, nor do they guarantee safety properties about the resulting languages. User-defined code is still ultimately compiled against a particular language, and because language composition cannot be automatic or guaranteed safe, interoperability with code written in other languages is thus limited by the difficulties described above despite the modular construction.

\subsubsection{Extensible Compilers and Tools}
A related methodology is to implement language features as compiler and tool extensions directly. A number of extensible compilers have been developed to support this approach (see \cite{clements2008comparison}). As with language frameworks, this approach can lead to \emph{de facto} implementation of a new language, since user libraries must be compiled against a particular combination of extensions, and these extensions cannot be guaranteed to compose in general. Moreover, a language may have multiple competing compilers and other tools. By relying on implementation-specific features of a single tool to define core semantics and behaviors, the clean conceptual separation between languages and tools is broken, leading to compatibility issues and hard-to-anticipate behaviors for user code. We argue that this approach should be considered harmful.

\subsection{Library-Oriented Approaches}
\subsubsection{Embedded DSLs}
Embedded domain-specific languages are languages that creatively repurpose existing language constructs to create interfaces that resemble those of a distinct language. In languages with rich type systems, such as Haskell, this approach can be quite successful (e.g. \cite{svensson2011obsidian}). Ultimately, however, this approach is limited by the host system, and as discussed in Section 2, thus limits experts who want to express particularly novel constructs, relative to the host language. 

\subsubsection{Term Rewriting Systems}
Many languages and tools allow developers to rewrite expressions according to custom rules. These can broadly be classified as {\it term rewriting systems}. Macro systems, such as those characteristic of the LISP family of languages \cite{mccarthy1978history}, are the most prominent example. Some compile-time metaprogramming systems also allow users to manipulate syntax trees (e.g. MetaML \cite{Sheard:1999:UMS}), and external rewrite systems also exist for many languages. These systems are expressive if used correctly, but verifying correctness and non-interference properties is difficult for the same reason. Manipulating source trees directly is a complex task, even in languages with simple grammars like LISP. Finally, term rewriting systems focus on rewriting terms to support alternative language semantics but do not intrinsically support extensions that cover the full programming system.

\subsubsection{Active Libraries}
Active libraries, as proposed by Czarnecki et al. \cite{activelibraries}, ``are not passive collections of routines or objects, as are traditional libraries, but take an active role in generating code''. They go on to suggest a number of areas in which the library could interact with the programming system, including optimizing code, checking source code for correctness, reporting domain-specific errors and warnings, and ``rendering domain-specific textual and non-textual program representations and for interacting with such representations''. 

This paper was largely a proposal. The concrete implementations of this concept have largely been term rewriting systems described above. One prominent example within the active libraries literature is Blitz++, which uses the C++ template expansion system as a metalanguage to support optimizations of array operations. An example of tool support comes from the Tau package for tuning and analyzing parallel programs. Tau allows libraries to instrument themselves declaratively to hide internal details and complex internal representations from users. A more extensive system with support for active libraries is Xroma. Xroma allows users to provide annotations that intercept compilation of a component at various stages of compilation to support rewriting, custom error handling and custom error checking. The approach taken by Xroma, by still requiring direct syntax manipulation and allowing arbitrary interception, is flexible but not compositional, and generally more complex than may be necessary. There also does not appear to be a clear, well-founded theoretical foundation for this approach, nor for active libraries in general.

%\subsection{Type-Level Computation} %Haskell, Ur and $\Omega$mega
%System XX with simple case analysis provides the basis of type-level computation in Haskell (where type-level functions are called type families \cite{Chakravarty:2005:ATC}). Ur uses type-level records and names to support typesafe metaprogramming, with applications to web programming \cite{conf/pldi/Chlipala10}. $\Omega$mega adds algebraic data types at the type-level, using these to increase the expressive power of algebraic data types at the expression level \cite{conf/cefp/SheardL07}. Dependently-typed languages blur the traditional phase separation between types and expressions, so type-level computation is often implicitly used (though not always in its most general form, e.g. Deputy \cite{conf/icfp/ChenX05}, ATS \cite{conf/esop/ConditHAGN07}.)

%\subsubsection{Run-Time Indirection}
%{\it Operator overloading} \cite{vanWijngaarden:Mailloux:Peck:Koster:Sintzoff:Lindsey:Meertens:Fisker:acta:1975} and {\it metaobject dispatch} \cite{Kiczales91} are run-time protocols that translate operator invocations into function calls. The function is typically selected according to the type or value of one or more operands. These protocols share the notion of {\it inversion of control} with type-level specification. However, type-level specification is a {\it compile-time} protocol focused on enabling specialized verification and implementation strategies, rather than simply enabling run-time indirection.

%\subsection{Specification Languages}
%Several {\it specification languages} (or {\it logical frameworks}) based on these theoretical formulations exist, including the OBJ family of languages (e.g. CafeOBJ \cite{Diaconescu-Futatsugi01}). They provide support for verifying a program against a language specification, and can automatically execute these programs as well in some cases. The  language itself specifies which verification and execution strategies are used.
%
%Type-determined compilation takes a more concrete approach to the problem, focusing on combining {\it implementations} of different\- logics, rather than simply their specifications. In other words, it focuses on combining {\it type checkers} and {\it implementation strategies} rather than more abstract representations of a language's type system and dynamic semantics. In Section 4, we outlined a preliminary approach based on proof assistant available for the type-level language to unify these approaches, and we hope to continue this line of research in future work.
%

% !TEX root = omar-thesis-proposal.tex
\newcommand{\nat}{\textbf{nat}}
\newcommand{\natz}{\textbf{z}}
\newcommand{\nats}[1]{\textbf{s}(#1)}
\newcommand{\natrec}[3]{\textbf{natrec}(#1, #2, #3)}
\renewcommand{\ttdefault}{txtt}

\section{Background: Active Libraries}\label{alibs}\todo{move into separate tex file}
The term \emph{active libraries} was first used by Veldhuizen et al. \cite{atthesis, atpaper} to describe ``libraries that take an active role in compilation, rather than being passive collections of subroutines''. The authors went on to suggest a number of concrete reasons libraries might benefit from closer interactions with the programming system, including program optimization, checking programs for correctness against specialized criteria, reporting domain-specific errors and warnings, and ``rendering domain-specific textual and non-textual program representations and for interacting with such representations'' (anticipating interactions between libraries and tools other than just the compiler. Our definition, given in Section \ref{motivation}, differs  from theirs for this reason.) 

% This stuff isn't actually hooking into the compiler...
The first concrete realizations of active libraries, prompting the introduction of the term, were scientific libraries that performed domain-specific program optimization at compile-time by exploiting language mechanisms, like C++ templates, that require compile-time computation. A prominent example in the literature is Blitz++, a numerics library that uses C++ template metaprogramming to optimize compound operations on vectors and matrices by eliminating intermediate allocations \cite{blitzpp}. Although this and a number of other interesting optimizations\- are possible by this family of techniques, their expressiveness is limited because template expansion supports only substitution of compile-time constants into pre-written code, and templates are notoriously difficult to read, write and debug. %They are, however, composable, because templates are associated with a single function or class, and  the only metatheoretic issue is that template expansion can be non-terminating.

More powerful compile-time \emph{term rewriting mechanisms} integrated into some languages can also be used for optimization, as well as for inserting specialized error checking and reporting\- logic and extending the language with powerful domain-specific abstractions. These mechanisms are highly expressive because they allow users to programmatically manipulate syntax trees directly, but they suffer from problems of composability and safety. Macros, such as those in MetaML \cite{metaml}, Template Haskell \cite{template_haskell} and others\todo{reference other macro systems}, can take full control over all the code that they enclose, so there is no way to guarantee that nested macros will compose as intended. Outer macros can neglect to invoke or  manipulate the output of inner macros in ways that can cause conflicts or weaken important guarantees. Once data escapes a macro's scope, there is no way to rely on the guarantees and features that were available within its scope -- the output of a macro is a value in the underlying language, so the underlying language is all that governs it (a problem related to the interoperability problem of Section 1). It can also be difficult to reason about the semantics of code when a number of enclosing macros may be manipulating it.

Some rewriting systems go beyond the block scoping of macros. Xroma (pronounced ``Chroma''), for example, is designed around active libraries and allows users to insert custom rewriting and error checking passes into the compiler from within libraries \cite{xroma}. Similarly, the derivative of Haskell implemented by the Glasgow Haskell Compiler (GHC) allows libraries to define custom compile-time term rewriting logic if an appropriate flag is passed in \cite{haskell-rewrite}. In both cases, the user-defined logic can dispatch on arbitrary patterns of code throughout the component or program the extension is active in, so these mechanisms are highly expressive and avoid some of the difficulties of block-scoped macros. But libraries containing such global rewriting logic cannot be safely composed because two different libraries may attempt to rewrite the same piece of code differently. It is also difficult to guarantee that such logic is correct and difficult for users to reason about code when simply importing a library can change the semantics of the program in a non-local manner.
% http://www.haskell.org/ghc/docs/latest/html/users_guide/rewrite-rules.html

Another example of an active library approach to extensibility is SugarJ \cite{sugarj}. SugarJ allows libraries to extend the base syntax of Java in a nearly-arbitrary manner, and these extensions are imported transitively throughout a program. SugarJ libraries are thus also not safely composable. For example, a library that defines a literal syntax for HTML would conflict with another that defines a literal syntax for XML because they define differing semantics for some of the same syntactic forms. And again, it can be difficult for users to predict what an unfamiliar piece of syntax desugars into, leading to readability issues.

%Another example comes from the Tau package for tuning and analyzing parallel programs, which allows libraries to instrument themselves declaratively so that internal details can be hidden from users of the tool \cite{tau}. 

%The researchers behind these tools often claim that such conflicts are rare. There have, to our knowledge, been no studies of this claim. Nevertheless, the core fact remains: conflicts can arise, which makes it difficult to rely on such libraries 

%There also does not appear to be a clear, well-founded theoretical foundation for this approach, nor for active libraries in general.

\section{Active Types}
Developers specify type-specific compile-time and edit-time features by equipping type families, when they are defined, with functions written in an appropriate metalanguage. These are selectively invoked by tools, such as the parser, type checker, translator and editor, when they work with expressions of that type, and only in those situations.

\subsection{Example: Regular Expressions}
To make our approach more concrete, let us begin with a simple example that demonstrates the use cases examined in this thesis. If a programming system included full compile-time and edit-time support for regular expressions, it might support features like the following:

\begin{enumerate}
\item \textbf{built-in syntax for regex literals} so that malformed regular expression patterns result in intelligible compile-time parsing errors. One study found that malformed regular expressions are a common problem in Java \cite{regex-type-system}.
\item \textbf{type checking logic} that ensures that key constraints related to regular expressions are not violated, such as that out-of-bounds group indices are not used \cite{regex-type-system} and that only values with correct types are spliced into regular expressions. When a type error is found, the error message should be intelligible.
\item \textbf{translation logic} that partially or fully compiles regular expressions into the efficient internal representation that will be used by the regular expression matching engine at run-time. In most languages, this compilation step occurs at run-time, even if the pattern is fully known at compile-time, thereby introducing latency into programs. If the developer is not careful, regular expressions used repeatedly in a program might be needlessly re-compiled on each use. By performing this step ahead-of-time, these dangers can be avoided.
\item \textbf{editor-integrated tooling} for interactively testing patterns against example strings, quickly referring to documentation, searching databases of common patterns and other domain-specific edit-time facilities.
\end{enumerate}


%\subsection{Example: First-Class Natural Numbers}
%To make our approach more concrete and emphasize its foundational character, let us begin with a very simple example, first-class natural numbers. If a programming system included first-class support for natural numbers, it might support behaviors like the following:
%
%\begin{enumerate}
%\item type checking rules that constrain uses of the three operators associated with the $\nat$ type, as defined in G\"{o}del's System \textbf{T} \cite{tapl}: $\natz$, $\nats{e}$ and $\natrec{e_1}{e_2}{e_3}$\todo{phrasing of his}
%\item type checker error messages that provide domain-specific assistance, like detecting when the trailing $\textbf{z}$ in a constant has been omitted: \todo{phrasing / merge this into 1?}
%\begin{verbatim}
%    s(s(s))
%        ^ (Line 1, Character 5)
%    Error: Operator s requires 1 argument of type nat. Provided 0 arguments.
%    Did you mean s(z)?
%\end{verbatim}
%\item translation into an efficient integer representation at run-time
%\item syntax support \todo{elaborate on syntax support for nat literals}
%\item editor support, such as the ability to instantiate nat constants more easily \todo{elaborate on this / reference Graphite section}
%%generate statistical predictions useful for code completion (e.g. biasing smaller constants) and other editor interactions, customized syntactic/visual representations and so on.
%\end{enumerate}
%
In a conventional \emph{monolithic} programming system, support for each of these features would need to be built into the language and tools. No such system today has support for all of the features above. Instead, libraries  provide support for regular expressions by leveraging general-purpose abstractions. Unfortunately, it is impossible to fully encode the syntax and the specialized static and dynamic semantics described above in terms of standard general-purpose notations and abstractions. These libraries have thus needed to compromise, typically by representing regular expressions using strings and deferring parsing, typechecking and translation to run-time. This introduces performance overhead and can lead to unanticipated run-time errors (as shown in \cite{regex-type-system}) and security vulnerabilities (due to injection attacks when splicing is used, for example). Similarly, edit-time tools that make working with regular expressions easier, as described above, are rarely integrated into editors, and never in a way that facilitates their discovery and use directly when manipulating regular expressions. In most cases, such tools must discovered independently and accessed externally (for example, via a browser), making their use both less common and more awkward than necessary (we provide evidence for this \cite{active-code-completion}, see Section \ref{acc}).

Existing active library approaches could be used to implement some of these features, but the composition issues described in Section 2 make such solutions suboptimal. For example, if SugarJ was used by two different regular expression engines to provide literal syntax for regular expression patterns, there could easily be conflicts at link-time because both will introduce many of the same standard notations but back them with differing implementations. 

We can observe, however, that each of these features relates specifically to how terms representing regular expression patterns are processed by tools, and that such terms are always classified by a particular user-defined type\footnote{More generally, several different types within an indexed type family. For example, \texttt{Pattern(}$n$\texttt{)}  represents a pattern containing $n$ matching groups, so \texttt{Pattern} is  a type family indexed by a natural number, $n$ \cite{regex-type-system}. We will return to this distinction in Section \ref{att}.}. We will refer to it by a simple name, \verb|Pattern|, but it would need to have a fully-qualified name to ensure that conflicts due to name clashes cannot occur. It is only when creating, editing or compiling expressions of this type that the logic enumerated above would ever need to be invoked. Indeed, this is a common pattern -- types are widely seen as a natural organizational unit around which the semantics of programming languages and logics are defined (e.g. in both TAPL \cite{tapl} and PFPL \cite{pfpl}, most chapters simply describe the semantics and metatheory of a few new types without reference to other types). This suggests a principled alternative to the mechanisms described earlier that preserves most of their expressiveness but eliminates the possibility of conflict and makes it easier to reason locally about a piece of code: associating extension logic to a single type or type family as it is defined and scoping it only to expressions, or basic operations on expressions, in that type or type family. This is the common motif behind all of the {actively-typed} mechanisms proposed in this dissertation.

\subsection{Proposed Contributions}
In Section \ref{att}, we will describe how to extend the core static and dynamic semantics of a language in an actively-typed and safe manner. We will distill the essence of our approach, which we call \textbf{active typechecking and translation (AT\&T)}, by specifying an actively-typed lambda calculus called @$\lambda$, proving several key safety theorems, and examining the connections between active types and type-level computation, type abstraction and typed compilation techniques. We will then go on to demonstrate the expressiveness of this mechanism by designing and implementing a full-scale language, Ace, and implementing a number of interesting type families from existing full-scale languages as active libraries within Ace. A primary focus of these is on high-performance  programming abstractions, but we will also show some uses of Ace for other domains.

In both @$\lambda$ and Ace, semantic extensions operate over a fixed syntax. In Section \ref{aparsing}, we will show how domain-specific syntax can also be introduced in an actively-typed and safely composable manner. Our technique is called \textbf{actively-typed parsing} and it will be implemented within the Wyvern language. Our novel contributions include the design of a second parsing phase that runs in tandem with typechecking, a method for using whitespace to delimit domain-specific syntax, a generalization of standard literal forms so that they can be used for more than one type, an underlying mechanism for associating compile-time data and functionality with structural types, and a recursive use of the active parsing technique to introduce a domain-specific syntax for defining new actively-typed grammars. We will describe each of these contributions as implemented in Wyvern, give minimal formalisms showing how the core syntax and the type system of Wyvern support actively-typed parsing,  and show a number of examples expressible by this technique.

Finally, in Section \ref{acc}, we will show how editor-integrated domain-specific tooling for working with expressions of a single type can be introduced from within active libraries, by a technique known as \textbf{active code completion}. Developers associate
domain-specific user interfaces, called \emph{palettes}, with types. Users discover and invoke palettes from the code completion menu at edit-time, populated according to an actively-typed mechanism similar to that of actively-typed parsing. When they are done, the palette generates a term of that type based on the information received from the user. Using several empirical
methods, we survey\- the expressive power of this approach, describe the design and safety constraints governing
the system architecture as well as particular palette user 
interfaces, and develop one such system for the Java language\footnote{In Graphite, palettes are associated with Java classes, which are technically different from types. However, active code completion could also be implemented just as well in non-object-oriented languages.}, based on these constraints, called Graphite. Using Graphite,
we implement a palette for working with regular expressions to conduct a pilot study to provide evidence for the usefulness of this approach, and of contextually-relevant editor-integrated tooling generally.

Together, ...\todo{write something about a unified mechanism that can be used across concerns}

%
%This suggests that a natural place where these features can be defined is in the library containing the declaration of \verb|Pattern| itself, rather than in the language and tool implementations. An \emph{actively-typed} definition of \verb|Pattern| would thus be equipped with	 functions that described how the parser (item 1), type checker (item 2), translator (item 3) and editor (item 4) should operate when working with expressions of type \verb|Pattern|. We can abstractly denote this declaration, as it would exist within a user-defined library, as follows:
%\begin{equation*}
%{\sf type}~Pattern[f_{\text{editor}}]\{
%\textbf{z}[f_{\text{resolve-z}}, f_{\text{compile-z}}], 
%\textbf{s}[f_{\text{resolve-s}}, f_{\text{compile-s}}], 
%\textbf{natrec}[f_{\text{resolve-rec}}, f_{\text{compile-rec}}]\}
%\end{equation*}
%
%When type checking an expression like $\nats{\nats{\natz}}$, the type checker delegates to the user-provided type-level function $f_{\text{resolve-s}}$. This function would be tasked with assigning a type to expression as a whole, given information including the \emph{types} (but not necessarily the full syntax trees) of all its subexpressions, or if a type cannot be assigned, producing a specific error message. Similarly, the compiler calls the $f_{\text{compile-s}}$ function to determine a representation in the target language for the expression, checking to ensure that it is well-formed and type-correct with respect to the target type system. Finally, elements of the editor may call into the $f_{\text{editor}}$ function (or one of several such functions, more generally) to control behaviors like code completion and code prediction when an expression of type $\nat$ is being entered. 
%
%Note that these functions are \emph{not} to be conflated with methods or run-time functions -- they are functions written in a type-level language that are called at compile-time and edit-time to define the basic behaviors associated with the type that they are associated with.
%
%\subsection{Characteristics of an Actively-Typed Programming System}
%An actively-typed programming system can be characterized by its choice of type-level language, source grammar, target language and dispatch protocols.
%
%\paragraph{Type-Level Language} The type-level language is the language within which the type definitions and the functions that define their behaviors are defined. This language must be constrained so that different definitions do not interfere with one another and so that desirable safety properties for the system as a whole are maintained, as we discuss below.
%
%\paragraph{Source Language} The source language is the language with which run-time behavior is defined. In our example above, terms like $\nats{\nats{\natz}}$ are part of the source language. In the purest case, the source language is simply a grammar; its semantics are given entirely by active type specifications.
%
%\paragraph{Dispatch Protocol} For each syntactic form in the source language, there is a dispatch protocol that determines which type is delegated responsibility over it, and which specific function(s) are called for each behavior the system supports. This fixed protocol makes it possible for users to predict the meaning of a construct using information local to the term, a key differentiator of this approach compared to term-rewriting systems where there can be action at a distance.
%
%\paragraph{Target Language} The target language is the language that the front-end compilation phase of the system targets. The limitations and constraints imposed by the target language are final, because all constructs ultimately translate into terms in the target language. In other words, active type specifications can only add additional invariants to the language; they cannot violate invariants imposed by the target language.

%\subsection{Research Challenges}
%The example of natural numbers given above is relatively simple, and the solution we outline remains abstract. A key challenge is then to demonstrate that this approach is able to express the behaviors of more sophisticated language constructs that span diverse problem domains, and be implemented in the context of a realistic collection of tools. The resulting system should be usable by developers who lack the expertise needed to define new language constructs themselves.
%
%Simultaneously, we must also demonstrate that this model is well-motivated theoretically, place it within the broader context of the theory of typed programming languages, and demonstrate that it is possible for desirable system safety properties to be maintained. In particular, we are interested in properties like:
%
%\begin{itemize}
%\item Correctness of active type specifications, so that users of a library need not be forced to debug errors arising within the specifications themselves.
%\item Correctness of translations, so that the results of translation are guaranteed to be well-typed and consistent with respect to the target language.
%\item Termination of active type specifications, so that evaluation of the type-level functions cannot cause the compiler or editor to hang.
%\item Composability of active type specifications, so that the behaviors defined by one type cannot interfere with those defined by another, no matter the order in which they are imported. This property is essential if we wish to place these specifications within normal libraries.
%\end{itemize}



% !TEX root = omar-thesis-proposal.tex
\section{Actively-Typed Parsing}\label{aparsing}
Wyvern...
% !TEX root = omar-thesis-proposal.tex
\newcommand{\atlam}{@$\lambda$}

% Generic
\newcommand{\bindin}[2]{#1;~#2}
\newcommand{\pipe}{~\text{\large $\vert$}~}
\newcommand{\splat}[3]{#1_{#2};\ldots;#1_{#3}}
\newcommand{\splatC}[3]{#1_{#2}~~~~\cdots~~~~#1_{#3}}
\newcommand{\splatTwo}[4]{#1_{#3}#2_{#3},~\ldots~, #1_{#4}#2_{#4}}
\newcommand{\substn}[2]{[#1]#2}
\newcommand{\subst}[3]{\substn{#1/#2}{#3}}
\newcommand{\entails}[2]{#1 \vdash #2}

% Programs
\newcommand{\progsort}{\rho}
\newcommand{\pfam}[2]{\bindin{#1}{#2}}
\newcommand{\pdef}[4]{\bindin{{\sf def}~\tvar{#1}:#2=#3}{#4}}

% Families
\newcommand{\fvar}[1]{\textsc{#1}}

\newcommand{\family}[6]{{\sf family}~\fvar{#1}[#2]\sim \tvar{#5}.#6~\{#3\}}
\newcommand{\familyDf}{\family{Fam}{\kappaidx}{\opsort}{\opsigsort}{i}{\taurep}}

% Operators
\newcommand{\opsort}{\theta}
\newcommand{\opsigsort}{\Theta}
\newcommand{\opvar}[1]{\textbf{\textit{#1}}}

% Expressions
\newcommand{\evar}[1]{#1}
\newcommand{\elam}[3]{\lambda #1{:}#2.#3}
\newcommand{\eapp}[2]{{#1~#2}}
\newcommand{\eopapp}[2]{\eop{Arrow}{ap}{\tunit}{#1; #2}}
\newcommand{\eop}[4]{{\fvar{#1}.\opvar{#2}\langle#3\rangle(#4)}}
\newcommand{\elet}[4]{{\sf let}~#1 : #2 = #3~{\sf in}~#4}

% Type-Level Terms
\newcommand{\tvar}[1]{{\textbf{#1}}}

\newcommand{\tlam}[3]{\lambda \tvar{#1}{:}{#2}.#3}
\newcommand{\tapp}[2]{#1~#2}
\newcommand{\tifeq}[5]{{\sf if}~#1\equiv_{#3}#2{\sf ~then~}#4~{\sf else}~#5}
\newcommand{\tstr}[1]{\textit{``#1''}}
\newcommand{\tunit}{()}
\newcommand{\tpair}[2]{(#1, #2)}
\newcommand{\tfst}[1]{{\sf fst}(#1)}
\newcommand{\tsnd}[1]{{\sf snd}(#1)}
\newcommand{\tnil}[1]{[]_#1}
\newcommand{\tcons}[2]{#1 :: #2}
\newcommand{\tfold}[6]{{\sf fold}(#1; #2; \tvar{#3},\tvar{#4},\tvar{#5}.#6)}

\newcommand{\tfamSpec}[5]{{\sf family}~[#2]~::~#3~\{#4 : #5\}}
\newcommand{\tfamSpecStd}{\tfamSpec{fam}{\kappaidx}{\taurep}{\theta}{\Theta}}

\newcommand{\ttype}[2]{\fvar{#1}\langle#2\rangle}
\newcommand{\ttypestd}{\ttype{Fam}{\tau}}
\newcommand{\tfamcase}[5]{{\sf case}~#1~{\sf of}~\fvar{#2}\langle\tvar{#3}\rangle\Rightarrow#4~{\sf ow}~#5}
\newcommand{\trepof}[1]{{\sf rep}(#1)}

\newcommand{\tden}[2]{\llbracket #1~{\sf as}~#2 \rrbracket}
\newcommand{\ttypeof}[1]{{\sf typeof}~#1}
\newcommand{\tvalof}[2]{{\sf trans}(\tden{#1}{#2})}
\newcommand{\terr}{{\sf err}}
\newcommand{\tdencase}[5]{{\sf case}~#1~{\sf of}~\tden{\tvar{#2}}{\tvar{#3}}\Rightarrow#4~{\sf ow}~#5}

\newcommand{\titerm}[1]{\triangledown(#1)}
\newcommand{\titype}[1]{\blacktriangledown(#1)}

\newcommand{\tconst}[1]{{\sf const}(#1)}
\newcommand{\tOp}[1]{{\sf op}(#1)}

\newcommand{\tprog}[1]{{\sf program}(#1)}

\newcommand{\topsempty}{\cdot}
\newcommand{\tops}[5]{\opvar{#1}[#2](\tvar{#3},\tvar{#4}.#5)}
\newcommand{\topp}[2]{#1; #2}
\newcommand{\Tops}[2]{\tvar{#1} : #2}
\newcommand{\Topp}[2]{#1; #2}
\newcommand{\kOpEmpty}{\cdot}
\newcommand{\kOpS}[2]{\opvar{#1}[#2]}
\newcommand{\kOp}[3]{#1; \kOpS{#2}{#3}}

% Contexts
\newcommand{\fCtx}{\Sigma}
\newcommand{\itvarCtx}{\Omega}
\newcommand{\iCtx}{\Theta}
\newcommand{\eCtx}{\Gamma}
\newcommand{\etvarCtx}{\Omega}
\newcommand{\errCtx}{\mathcal{E}}
\newcommand{\famEvalCtx}{\Xi}

% Judgments
\newcommand{\emptyctx}{\emptyset}
\newcommand{\tvarCtx}{\Delta}
\newcommand{\tvarCtxX}[2]{\Delta, {\tvar{#1}} : {#2}}
\newcommand{\fvarCtx}{\Sigma}
\newcommand{\fvarCtxX}{\Sigma, \fvarOfType{Fam}{\kappaidx}{\Theta}}
\newcommand{\eivarCtx}{\Omega}
\newcommand{\eivarCtxX}[1]{\Omega, \evar{#1}}

\newcommand{\kEntails}[3]{#1 \vdash_{#2} #3}

\newcommand{\progProg}[1]{#1~{\tt prog}}
\newcommand{\progOK}[3]{\kEntails{#1}{#2}{\progProg{#3}}}
\newcommand{\progOKX}[1]{\progOK{\tvarCtx}{\fvarCtx}{#1}}

\newcommand{\fvarOfType}[3]{\fvar{#1}[#2,#3]}
\newcommand{\fvarOfTypeDf}{\fvarOfType{Fam}{\kappaidx}{\Theta}}

\newcommand{\opOfType}[2]{#1 : #2}
\newcommand{\opType}[4]{\kEntails{#1}{#2}{\opOfType{#3}{#4}}}

\newcommand{\tOfKind}[2]{#1 : #2}
\newcommand{\tKind}[4]{\kEntails{#1}{#2}{\tOfKind{#3}{#4}}}
\newcommand{\tKindX}[2]{\tKind{\tvarCtx}{\fvarCtx}{#1}{#2}}

\newcommand{\isExpr}[1]{#1~\texttt{expr}}
\newcommand{\exprOK}[4]{#1~#2 \vdash_{#3} \isExpr{#4}}
\newcommand{\exprOKX}[1]{\exprOK{\tvarCtx}{\eivarCtx}{\fvarCtx}{#1}}

\newcommand{\isIterm}[4]{#1~#2 \vdash_{#3} #4~\texttt{iterm}}
\newcommand{\isItermX}[1]{\isIterm{\tvarCtx}{\eivarCtx}{\fvarCtx}{#1}}

\newcommand{\isItype}[3]{#1 \vdash_{#2} #3~\texttt{itype}}
\newcommand{\isItypeX}[1]{\isItype{\tvarCtx}{\fvarCtx}{#1}}

\newcommand{\tEvalX}[2]{#1 \Downarrow #2}
\newcommand{\tiEvalX}[2]{#1 \curlyveedownarrow #2}

\newcommand{\fvalCtx}{\Phi}
\newcommand{\fvalCtxX}[1]{\Phi, #1}
\newcommand{\fval}[4]{\fvar{#1}[#2, \tvar{#3}.#4]}
\newcommand{\fvalDf}{\fval{Fam}{\theta}{i}{\tau}}

\newcommand{\pcompiles}[3]{\vdash_{#1} #2 \Longrightarrow #3}
\newcommand{\pcompilesX}[1]{\pcompiles{\fvalCtx}{#1}{\gamma}}

\newcommand{\ptcc}[2]{#1 \longrightarrow #2}

\newcommand{\etCtx}{\Gamma}
\newcommand{\etCtxX}[2]{\etCtx, \evar{#1} : #2}

\newcommand{\gtCtx}{\Psi}
\newcommand{\gtCtxX}[2]{\gtCtx, \evar{#1} : #2}

\newcommand{\ecompiles}[5]{#1 \vdash_{#2} #3 : #4 \Longrightarrow #5}
\newcommand{\ecompilesX}[3]{\ecompiles{\etCtx}{\fvalCtx}{#1}{#2}{#3}}

\newcommand{\delfromtau}[4]{\vdash^{\fvar{#1}}_{#2} #3 \sim #4}
\newcommand{\tauisdel}[5]{\vdash^{\fvar{#1}}_{#2} \ttype{#3}{#4} \sim #5}

\newcommand{\checkRC}[5]{#1 \vdash^{\fvar{#2}}_{#3} #4 \sim #5}
\newcommand{\checkRCX}[2]{\checkRC{\gtCtx}{Fam}{\fvalCtx}{#1}{#2}}

\newcommand{\erase}[3]{\vdash_{#1} #2 \leadsto #3}
\newcommand{\eraseX}[2]{\erase{\fvalCtx}{#1}{#2}}

\newcommand{\eCtxTogCtx}[4]{\vdash^{\fvar{#1}}_{#2} #3 \sim #4}
\newcommand{\eCtxTogCtxX}[2]{\eCtxTogCtx{Fam}{\fvalCtx}{#1}{#2}}

\newcommand{\ddbar}[4]{\vdash^{\fvar{#1}}_{#2} #3 \sim #4}
\newcommand{\ddbarX}[2]{\ddbar{Fam}{\fvalCtx}{#1}{#2}}

\newcommand{\iType}[3]{#1 \vdash #2 : #3}

\newcommand{\gtCtxH}{\hat{\gtCtx}}
\newcommand{\gtCtxHX}[2]{\gtCtxH, \evar{#1} : #2}

%
\newcommand{\fSpec}[3]{\tof{\fvar{#1}}{\kFam{#2}{#3}}}
\newcommand{\fSpecStd}{\fSpec{fam}{\kappaidx}{\Theta}}

% \tau
\newcommand{\taut}[1]{{\tau_{\text{#1}}}}
\newcommand{\tautype}{\taut{type}}
\newcommand{\tautrans}{\taut{trans}}
\newcommand{\tauproof}{\taut{proof}}
\newcommand{\tauidx}{\taut{idx}}
\newcommand{\taui}{\taut{i}}
\newcommand{\tauidxn}[1]{\taut{idx,#1}}
\newcommand{\taurep}{\taut{rep}}
\newcommand{\taurepn}[1]{\taut{rep,#1}}
\newcommand{\tauden}{\taut{den}}
\newcommand{\tauIT}{\taut{IT}}
\newcommand{\tauarrow}{\taut{arrow}}
\newcommand{\tauprod}{\taut{prod}}
\newcommand{\tauint}{\taut{int}}
\newcommand{\taubool}{\taut{bool}}
\newcommand{\tauprog}{\taut{prog}}
\newcommand{\tauiterm}{\taut{iterm}}
\newcommand{\tauval}{\taut{val}}
\newcommand{\tauop}{\taut{op}}

% \gamma
\newcommand{\ghat}{\hat{\gamma}}

% \sigma
\newcommand{\delt}[1]{\sigma_{\text{#1}}}
\newcommand{\delrep}{\delt{rep}}
\newcommand{\dhat}{\hat{\sigma}}
\newcommand{\dbar}{\bar{\sigma}}

% \kappa
\newcommand{\kappat}[1]{\kappa_{\text{#1}}}
\newcommand{\kappaidx}{\kappat{idx}}
\newcommand{\kappai}{\kappat{i}}

% Types

% IL terms
\newcommand{\ivar}[1]{\textrm{#1}}
\newcommand{\ilam}[3]{\lambda #1{:}#2.#3}
\newcommand{\ifix}[3]{{\sf fix~}#1{:}#2~{\sf is}~#3}
\newcommand{\iapp}[2]{#1~#2}
\newcommand{\ipair}[2]{(#1, #2)}
\newcommand{\ifst}[1]{{\sf fst}(#1)}
\newcommand{\isnd}[1]{{\sf snd}(#1)}
\newcommand{\iintlit}{\bar{
\textrm{z}}}
\newcommand{\iop}[2]{#1 \oplus #2}
\newcommand{\iIfEq}[5]{{\sf if}~#1\equiv_{#3}#2{\sf ~then~}#4~{\sf else}~#5}
\newcommand{\mvalof}[1]{{\sf valof}(#1)}
\newcommand{\iup}[1]{\vartriangle\hspace{-2.5pt}(#1)}

% Internal Types
\newcommand{\darrow}[2]{#1\rightarrow#2}
\newcommand{\dint}{\mathbb{Z}}
\newcommand{\dpair}[2]{#1\times#2}
\newcommand{\dup}[1]{\blacktriangle(#1)}
\newcommand{\drepof}[1]{{\sf repof}(#1)}

% Kinds
\newcommand{\kvar}[1]{\textrm{#1}}
\newcommand{\karrow}[2]{#1\rightarrow{#2}}
\newcommand{\kforall}[2]{\forall \kvar{#1}.#2}
\newcommand{\kstr}{\textsf{Str}}
\newcommand{\kunit}{\textsf{1}}
\newcommand{\kpair}[2]{#1 \times #2}
\newcommand{\klist}[1]{\textsf{list}[#1]}
\newcommand{\kTypeBlur}{\star}
\newcommand{\kDen}{\textsf{Den}}
\newcommand{\kIType}{\textsf{ITy}}
\newcommand{\kITerm}{\textsf{ITm}}

% Judgements
\newcommand{\tof}[2]{#1 : #2}
\newcommand{\mtof}[2]{#1 :: #2}
\newcommand{\tentails}[2]{#1 \vdash #2}
\newcommand{\tentailst}[3]{\tentails{#1}{\tof{#2}{#3}}}
\newcommand{\tStdCtx}{\fCtx~\tvarCtx}
\newcommand{\tCtxXF}[1]{\fCtx, #1~\tvarCtx}
\newcommand{\tCtxXT}[1]{\fCtx~\tvarCtx, #1}
%\newcommand{\tCtxXL}[1]{\fCtx~\tvarCtx~\lvarCtx, #1}
\newcommand{\tentailsX}[1]{\tentails{\tStdCtx}{#1}}
\newcommand{\tentailsXt}[2]{\tentailsX{\tof{#1}{#2}}}
\newcommand{\kentails}[2]{#1 \vdash #2}
\newcommand{\kentailsX}[1]{\kentails{\fCtx}{#1}}
\newcommand{\iMkCtx}[3]{#1~#2~#3}
\newcommand{\iStdCtx}{\iMkCtx{\fCtx}{\tvarCtx}{\itvarCtx}}
\newcommand{\ientails}[2]{#1 \vdash #2}
\newcommand{\ientailsX}[1]{\entails{\iStdCtx}{#1}}
\newcommand{\casemap}[2]{#1 : #2}
\newcommand{\mentails}[3]{#1, #2 \vdash #3}
\newcommand{\mentailsX}[1]{\mentails{\tvarCtx}{\itvarCtx}{#1}}
\newcommand{\eentails}[4]{#1~#2~#3 \vdash #4}
\newcommand{\eentailsX}[1]{\eentails{\fCtx}{\tvarCtx}{\etvarCtx}{#1}}
\newcommand{\mtentails}[2]{#1 \vdash #2}
\newcommand{\mtentailsX}[1]{\mtentails{\iCtx}{#1}}
\newcommand{\mtentailsXt}[2]{\mtentails{\iCtx}{\mtof{#1}{#2}}}
\newcommand{\kSimple}[1]{#1~{\sf simple}}
\newcommand{\Tentails}[3]{#1 \vdash_{#2} #3}
\newcommand{\TentailsX}[1]{\Tentails{\fCtx}{\fvar{fam}}{#1}}
\newcommand{\kEq}[1]{#1~\texttt{eq}}

% Verification and Translation
\newcommand{\translates}[4]{\entails{#1}{#2 \longrightarrow \tden{#3}{#4}}}

% Compilation Semantics
\newcommand{\compiless}[3]{#1 \Longrightarrow \tden{#2}{#3}}
\newcommand{\compiles}[3]{#1 \Longrightarrow \tden{#2}{#3}}
%\newcommand{\translates}[6]{\entails{#1}{\translatesTo{#2}{#3}{#4}{#5}{#6}}}
\newcommand{\translatesTo}[5]{#1 \longrightarrow \tden{#2}{\ttype{#3}{#4}{#5}{6}{7}}}
\newcommand{\translatesX}[5]{\translates{\eCtx}{#1}{#2}{#3}{#4}{#5}}

% !TEX root = omar-thesis-proposal.tex
\newcommand{\lam}[3]{\lambda #1{:}#2.#3}
\newcommand{\ap}[2]{#1~#2}
\newcommand{\z}{\textsf{z}}
\newcommand{\s}[1]{\textsf{s}(#1)}
\newcommand{\natrec}[5]{\textsf{natrec}(#1; #2; #3,#4.#5)}
\newcommand{\pair}[2]{(#1, #2)}
\newcommand{\fst}[1]{\textsf{fst}(#1)}
\newcommand{\snd}[1]{\textsf{snd}(#1)}

\newcommand{\tArrow}[2]{#1 \rightarrow #2}
\newcommand{\nat}{\texttt{nat}}
\renewcommand{\prod}[2]{#1 \times #2}

%\newcommand{\eCtx}{\Gamma}
\newcommand{\eCtxX}[3]{#1, #2 : #3}
\newcommand{\jet}[3]{#1 \vdash #2 : #3}
\newcommand{\jetX}[2]{\jet{\eCtx}{#1}{#2}}

\lstset{language=ML,
basicstyle=\ttfamily\footnotesize,
morekeywords={newcon,extends,tycon,opcon,err,schema,fix,is},
}

\section{Extending a Type System From Within}\label{att}
In this and the following two sections, we will turn our focus to language-integrated, type-oriented mechanisms for implementing extensions to the static semantics of a simply-typed programming language, to allow providers to express finer distinctions than allowed by the general-purpose constructs (recursive labeled sums and products) that we included in Wyvern. We will return to using a fixed  syntax. The methods in the previous section could also be applied to the languages we will now design, but we do not integrate them explicitly because their intellectual  contributions are fundamentally orthogonal. %We will also consider aspects of syntax other than literal forms, in various ways. % We will, however,  introduce a flexible dispatch mechanism for common elimination forms.
\subsection{Foundations}

\newcommand{\xty}[2]{\mathtt{#1}[#2]}
\newcommand{\xtyX}[1]{\xty{#1}{()}}
\newcommand{\xop}[3]{\mathtt{#1}[#2](#3)}
\newcommand{\xopX}[2]{\xop{#1}{()}{#2}}
\begin{figure}[t]
\small
\vspace{-8pt}
\begin{mathpar}
%\inferrule[var]{ }{
%	\jet{\eCtxX{\eCtx}{x}{\tau}}{x}{\tau}
%}
%
\inferrule[arrow-I]{
	\jet{\eCtxX{\eCtx}{x}{\tau}}{e}{\tau'}	
}{
	\jetX{\mathtt{lam}[\tau]({x}.e)}{\mathtt{arrow}[(\tau, \tau')]}
}

\inferrule[arrow-E]{
	\jetX{e_1}{\mathtt{arrow}[(\tau, \tau')]}\\
	\jetX{e_2}{\tau}
}{
	\jetX{\mathtt{ap}[()](e_1; e_2)}{\tau'}
}
\end{mathpar}
\vspace{-8pt}
\caption{Static semantics of the $\rightarrow$ fragment.}
\label{stlc}
\vspace{-8pt}
\end{figure}
\begin{figure}[t]
\small
\begin{mathpar}
\inferrule[nat-i1]{ }{\jetX{\mathtt{z}[()]()}{\mathtt{nat}[()]}}

\inferrule[nat-i2]{
	\jetX{e}{\mathtt{nat}[()]}
}{
	\jetX{\xopX{s}{e}}{\xtyX{nat}}
}

\inferrule[nat-e]{
	\jetX{e_1}{\xtyX{nat}}\\
	\jetX{e_2}{\tau}\\
	\jet{\eCtxX{\eCtxX{\eCtx}{x}{\xtyX{nat}}}{y}{\tau}}{e_3}{\tau}
}{	
	\jetX{\xopX{natrec}{e_1; e_2; x, y.e_3}}{\tau}
}
\end{mathpar}
\vspace{-8pt}
\caption{Static semantics of the $\mathtt{nat}$ fragment.}
\label{natfrag}
\vspace{-8pt}
\end{figure}
Simply-typed programming languages are often described in fragments, each consisting of a small number  of indexed type constructors (often one) and associated term-level operator constructors. The simply typed lambda calculus (STLC), for example, consists of a single fragment containing a single type constructor: $\mathtt{arrow}$, indexed by a pair of types, and two operator constructors: $\mathtt{lam}$, indexed by a type, and \verb|ap|, which we may think of as being indexed trivially. Their syntax and static semantics are shown in Fig. \ref{stlc}, written abstractly in a way that emphasizes this way of thinking about the type and term structure. A fragment is often identified by the type constructor it is organized around, so the STLC can more generically called $\mathcal{L}\{\rightarrow\}$, a notational convention used by Harper \cite{pfpl} and others. 

G\"odel's $\mathbf{T}$, or  $\mathcal{L}\{\rightarrow \mathtt{nat}\}$, adds the \verb|nat| fragment, defining natural numbers and a recursor that allows one to ``fold'' over a natural number. The static semantics of this fragment, also written abstractly with explicit type and operator indices, is shown in Fig. \ref{natfrag}. The language $\mathcal{L}\{\rightarrow \mathtt{nat}\}$ is  more powerful than $\mathcal{L}\{\rightarrow\}$ because the STLC can be embedded  into  $\mathbf{T}$ but the reverse is not true. Buoyed by this increase in power, we might go on by adding fragments that define sums, products and various forms of inductive types (e.g. lists), or perhaps a general mechanism for defining inductive types. Each fragment also increases the expressiveness of our language by the same argument.

If we consider the $\forall$ fragment, however, defining universal quantification over types (i.e. parametric polymorphism), we must take a moment to refine our understanding of embeddings. In $\mathcal{L}\{\rightarrow\,\forall\}$, studied by Girard as System $\mathbf{F}$ \cite{girard1971extension} and Reynolds as the polymorphic lambda calculus \cite{Reynolds94anintroduction}, it is known that sums, products, and inductive and co-inductive types can all be weakly defined by a technique analagous to Church encodings. This means that we can \emph{translate} well-typed terms of a language like $\mathcal{L}\{\rightarrow\forall~\mathtt{nat} + \times\}$ to well-typed terms in $\mathcal{L}\{\rightarrow\forall\}$ in a manner that preserves their dynamic semantics. But Reynolds, in a remark that recalls the ``Turing tarpit'' of Perlis \cite{Perl82a}, reminds us that discarding the source language and programming directly with the \emph{embedding} corresponding to this encoding may be unwise \cite{Reynolds94anintroduction}: 
\begin{quote}
To say that any reasonable function can be expressed by some program is not to say that it can be expressed by the most reasonable program. It is clear that the language requires a novel programming style. Moreover, it is likely that certain important functions cannot be expressed by their most efficient algorithms.
\end{quote}

%Adding new fragments directly to a language can make statically reasoning about programs more precise, support a more natural programming style and endow the language with a more favorable cost semantics. The latter two points might be relevant even if a strong embedding (i.e. one where there is a semantics-preserving isomorphism between terms and types of the fragment and the language) can be found. 
%
%Consistent with this view, typed programming languages like ML expose, for example, both $n$-ary product types and record types, despite the fact that any product type is definable using a record type (or in the other direction, using a module to provide the field selection operator). The compiler, on the other hand, \emph{is} mainly concerned with implementing the dynamic semantics correctly, and indeed, compilers often use the fact that weak definability results are quite readily derivable to their advantage, translating the constructs of an external language (EL) to those of a much simpler internal language (IL), e.g. with only binary products, during or directly after typechecking\todo{cite Harper-Stone}. %Correctly implementing the dynamic semantics of the language becomes the primary concern during this phase, so a simple IL that decreases the number of cases needing consideration when implementing and verifying the compiler is a wise choice.

%Having established why a minimal language like $\mathcal{L}\{\rightarrow\forall\}$, while suitable as an internal language, needs to be extended with additional fragments before it is suitable for use as a human-facing external language, 


An \emph{isomorphic embedding} of a typed language fragment, i.e. one that preserves static reasoning principles, in a sense that we will make more precise as we go on, can be far more difficult to establish than a weak embedding. For example, the aforementioned embedding of $\mathcal{L}\{{\rightarrow}~{\forall}~\mathtt{nat}~{+}~{\times}\}$ into $\mathcal{L}\{\rightarrow\forall\}$ does not preserve type disequality and some other equational reasoning principles (and we will see less subtle issues with more complex fragments soon). Establishing an isomorphic embedding that is also natural, as measured to a first approximation by the amount of boilerplate code that must be manually generated (and how likely one is to make a mistake when writing it), and that has a reasonable cost semantics is harder still. But these are the criteria one must satisfy to claim that a new language is unnecessary, because an embedding is sufficient.

Modern general-purpose languages do, of course, provide constructs, like datatypes, records, abstract types and objects, that admit satisfying embeddings of many language fragments, occupying what their designers and users see as ``sweet spots'' in the design space. However, ``general-purpose'' and ''all-purpose'' remain quite distinct, and situations continue to arise in both research and practice where desirable fragments can still only be weakly defined in terms of general-purpose constructs, or an isomorphic embedding, while possible, is widely seen as too verbose, brittle or inefficient:

\begin{enumerate}
%\compresslist
\item General-purpose constructs continue to evolve. There are many  variants of product types: records, records with functional record update, labeled tuples (records that specify a field ordering, unlike records) and record-like data structures with field delegation (of various sorts). Wyvern contains records with methods. Similarly, sum types also admit many variants (e.g. various forms of open sums). These are all either awkward or impossible to isomorphically embed into general-purpose languages that do not build in support. Even something as seemingly simple as a type-safe \verb|sprintf| operator requires special support from languages like Ocaml.
\item Perhaps more interestingly, specialized type systems that enforce stronger invariants than general-purpose constructs are capable of enforcing are often developed by researchers. One need take only a brief  excursion through the literature to discover language extensions that support data parallel programming \cite{Blelloch93,chakravarty2007data}, concurrency \cite{reppy1993concurrent}, distributed programming \cite{Murphy:2007:TDP:1793574.1793585}, dataflow programming \cite{mandel2005reactiveml}, authenticated data structures \cite{Miller:2014:ADS:2535838.2535851}, database queries \cite{Ohori:2011:MSM:2034773.2034815},  units of measure \cite{conf/cefp/Kennedy09} and many others.% All of these are implemented as dialects of existing languages, presumably because a strong encoding was not feasible.
\item Safe and natural foreign function interfaces (FFIs) require enforcing the type system of the foreign language within the calling language. %Using  a FFI that does not do this can lead to safety issues, even when both languages are separately known to be safe. 
%Safe FFIs generally require direct extensions to the language. 
For example, MLj extended Standard ML with constructs for safely and naturally interfacing with Java \cite{benton1999interlanguage}. For any other language, including  others on the JVM like Scala and the many languages that might be accessible via a native FFI, there is no way to guarantee that language-specific invariants are statically maintained, and the interface is far from natural.
\end{enumerate}

These sorts of innovations are, as these references suggest, generally disseminated as \emph{dialects} of an existing general-purpose language, constructed either as a fork, using tools like compiler generators, DSL frame\-works or  language workbenches, or directly within a particular compiler for the language, sometimes activated by a flag or pragma. This, as we have argued, is quite unsatisfying: a programmer can choose either a dialect supporting, for example, an innovative approach to data parallelism or one that builds in support for statically reasoning about units of measure, but there may not be an available dialect supporting both. Forming such a dialect is alarmingly non-trivial, even in the rare situation where a common framework has been used or the dialects are implemented within the same compiler, as these mechanisms do not guarantee that different combinations of individually sound dialects remain sound when combined. Metatheoretic and compiler correctness results can only be derived for the dialect \emph{resulting} from a language   composition operation, so in a sufficiently diverse software ecosystem, dialect providers have little choice but to leave this task to clients (providing, perhaps, some informal guidelines or partial automation). Avoiding dialect composition altogether in any large project is also difficult because interactions between components written in different dialects can lead to precisely the problems of item 3 (the \emph{client compatibility} problem previously discussed).% We will consider this further in our discussion of related work in Sec. \ref{related-work}.

These are not the sorts of problems usually faced by library providers. Well-designed languages preclude the  possibility of ``link-time'' conflicts between libraries and ensure that the semantics of one library  cannot be weakened by another by strictly enforcing abstraction barriers. For example, a module declaring an abstract type in ML can rely on any representation invariants that it internally maintains no matter which other modules are in use, so clients can assume that they will operate robustly in combination without needing to attend to burdensome ``link-time'' proof obligations. 

Inspired by this, we will begin in Sec. \ref{atlam} by developing a core calculus called @$\lambda$ where type and operator constructors are user-defined entities. The static semantics of a fragment can be implemented in a functional style by writing type-level functions. The dynamic semantics are implemented simultaneously by translation to a fixed typed internal language. We will show that by importing notions from the typed compilation literature and by treating fragment boundaries in a manner similar to how module boundaries are treated in conventional languages, we can derive type safety and conservativity theorems. As a result, if a fragment can be weakly defined in terms of the internal language, and its statics are of a general form that permit its implementation by this mechanism, it can be isomorphically embedded into the external language by implementing the ``missing'' rules. We call such an embedding an \emph{active embedding}. Once established, we show that a general class of lemmas necessary to prove that the embedding satisfies the isomorphism equations that we will discuss cannot be weakened by the introduction of any other fragment. While this work is focused on theoretical issues, and the core calculus has a decidedly uniform and thus awkward notation reminiscent of that in the two figures above, we will also sketch a flexible type-directed dispatch mechanism for standard syntactic forms that sits atop this core calculus that relies fundamentally on the idea of associating extension logic with types and recovers a more natural syntax. We then give examples of simple but interesting extensions that are built into, or are only weakly definable, in modern functional languages. 

This provides a bridge to Sec. \ref{ace}, where we approach the problem from the other direction, designing and implementing a full-scale actively typed language called Ace implemented itself as a library within an existing widely-used language, Python. Ace is aimed squarely at our goal of \emph{expressiveness}, giving today's programmers orthogonal access to a richer variety of type systems  from within a single   language. Python begins at the opposite extreme, initially providing effectively only a single type, \texttt{dyn}, so it poses quite an interesting challenge. We show how we can expose substantially more expressive examples of statically-typed general-purpose abstractions, as well as specialized abstractions (including the entirety of the OpenCL programming language for GPU programming), using Ace. Ace  supports user-defined base and target languages, rather than picking a fixed base language ($\mathcal{L}\{\rightarrow\}$) and target language ($\mathcal{L}\{\rightharpoonup \kint~\kunit \times{+}\}$) as in @$\lambda$. Ace also supports an extensible form of ``just-in-time'' staging: dynamic class tags extracted from Python values can propagate into the Ace compiler as static types at the point during execution where an Ace function is called from a Python script. We show that this is particularly well-suited for scientific workflows that make extensive use of FFIs. Though we include checks based on those in @$\lambda$ that make accidentally violating safety quite a bit more difficult than in related work, it is still possible. We discuss how this gap between @$\lambda$ and Ace could be filled by a bootstrapping technique, though we only sketch the details of this. The full details would require working with a formal dynamic semantics for Python and other intellectually uninteresting but unreasonably labor-intensive tasks, placing it  beyond the scope of this thesis.% characterized by periods of exploratory programming punctuated by computations that demand more performance, stronger correctness guarantees or specialized abstractions. 


%But, as mentioned in Section \ref{language-integrated-approaches}, some significant challenges must be addressed before such a mechanism can be relied upon. The desire for expressiveness must be balanced against  concerns about maintaining various safety properties in the presence of arbitrary combinations of user-defined  extensions to the language's core semantics. The mechanism must ensure that desirable \emph{metatheoretic properties} (e.g. type safety, decidability) of the language are maintained by extensions. Because multiple independently developed extensions might be used within one program, the mechanism must further guarantee that extensions are \emph{conservative}. These are the issues we seek to address in the next two sections.

\section{@$\lambda$}\label{atlam}
%Programming languages are typically organized around a monolithic collection of indexed type and operator constructors. Let us begin by considering the simply-typed lambda calculus (STLC). It provides a single type constructor, indexed by a pair of types, that we will call $\fvar{Arrow}$ for uniformity. It also provides two operator constructors: $\lambda$, indexed by a type, and $\texttt{ap}$, indexed trivially. We can specify the static semantics of the STLC as follows, using a uniform abstract syntax where type and operator indices are written in braces to emphasize this way of thinking about its structure:
%
%Although a researcher may casually speak of ``conservatively extending the STLC`` with a primitive natural number type and its corresponding operators (as in G\"odel's T; Figure \ref{nat-statics}) or primitive product types and their  corresponding operators (Figure \ref{pair-statics}), it is clearly not possible to directly introduce these from within the STLC.
%\begin{figure}[h]
%\small
%\vspace{-8pt}
%\begin{mathpar}
%\inferrule[nat-I1]{ }{
%	\jetX{\texttt{z}[()]()}{\fvar{Nat}[()]}
%}
%
%\inferrule[nat-I2]{
%	\jetX{e}{\fvar{Nat}[()]}
%}{
%	\jetX{\texttt{s}[()](e)}{\fvar{Nat}[()]}
%}
%
%\inferrule[nat-E]{
%	\jetX{e_1}{\fvar{Nat}[()]}\\
%	\jetX{e_2}{\tau}\\
%	\jet{\eCtxX{\eCtxX{\eCtx}{x}{\fvar{Nat}[()]}}{y}{\tau}}{e_3}{\tau}
%}{
%	\jetX{\texttt{natrec}[()](e_1; e_2; x.y.e_3)}{\tau}
%}
%\end{mathpar}
%\vspace{-8pt}
%\caption{Static semantics of natural numbers.}\label{nat-statics}
%\vspace{-15pt}
%\end{figure}
%\begin{figure}[h]
%\small
%\begin{mathpar}
%\inferrule[Prod-I]{
%	\jetX{e_1}{\tau_1}\\
%	\jetX{e_2}{\tau_2}
%}{
%	\jetX{\texttt{pair}[()](e_1; e_2)}{\fvar{Prod}[(\tau_1, \tau_2)]}
%}
%
%\inferrule[Prod-E1]{
%	\jetX{e}{\fvar{Prod}[(\tau_1, \tau_2)]}
%}{
%	\jetX{\texttt{prl}[()](e)}{\tau_1}
%}
%
%\inferrule[Prod-E2]{
%	\jetX{e}{\fvar{Prod}[(\tau_1, \tau_2)]}
%}{
%	\jetX{\texttt{prr}[()](e)}{\tau_1}
%}
%\end{mathpar}
%\vspace{-8pt}
%\caption{Static semantics of products}\label{pair-statics}
%\end{figure}
%
%One recourse researchers have in such situations is to attempt to encode new constructs in terms of existing constructs. Collections of such encodings are sometimes called \emph{embedded domain-specific languages (EDSLs)} \cite{fowler2010domain}. Unfortunately, finding an encoding that fully captures both the static and dynamic semantics of a desirable construct can be difficult\footnote{For example, Church encodings in System F were among the more challenging topics for students in our undergraduate programming languages course, 15-312.} and is not always possible. Even when it possible, it may be impractical to work with and optimize. For our example of adding natural numbers or products to the STLC, an encoding is impossible. In closely-related languages, like the polymorphic lambda calculus (a.k.a. System F), such constructs are only weakly definable via Church encodings \cite{reynoldsIntroToPolymorphism} and far from practical. Reynolds, echoing Perlis   \cite{perlisPearl}, puts it succinctly \cite{reynoldsIntroToPolymorphism}:
%
%\begin{quote}
%To say that any reasonable function can be expressed by some program is not to say that it can be expressed by the most reasonable program. It is clear that the language requires a novel programming style.
%Moreover, it is likely that certain important functions cannot be expressed by their most
%efficient algorithms.
%\end{quote}
%

%creating a new language. If this is not practical, the best one can attempt to do is encode the new types in terms of existing types (by a Church encoding, for example). This is generally unsatisfactory -- 

%Languages implemented using these common patterns are central planning by a language designer or design committee. 

%Researchers or domain experts who cannot work around such limitations must develop new standalone languages. In our simple scenario, we may simply copy our implementation of G\"odel's T or even edit it directly (a pernicious technique for implementing a new language where the prior one is overwritten). In a more complex scenario, we may instead employ a tool like a compiler generator or DSL framework \cite{fowler2010domain} that can generate a standalone implementation from declarative specifications of language constructs. Some of these tools allow you to package and reuse these specifications (with the important caveat that not all combinations of constructs are valid and free of conflicts, an important modularity issue that we will return to several times in this paper).
%
%The increasing sophistication and ease-of-use of these tools have led many to suggest a {\it language-oriented approach} \cite{journals/stp/Ward94} to software development where different components of an application are written in different languages. Unfortunately, this leads to problems at language boundaries: a library's external interface must only use constructs that can reasonably be expressed in \emph{all possible calling languages}. This can restrict domain-specific languages by, for example, precluding constructs that rely on statically-checked invariants stronger than those their underlying representation in a common target language normally supports. At best, constructs like these can be exposed by generating a wrapper where run-time checks have been inserted to guarantee necessary invariants. This compromises both verifiability and performance and requires the development of an interoperability layer for every DSL. Moreover, library clients must work with verbose and unnatural ``glue code'' when interfacing across languages, defeating the primary purpose of high-level programming languages: hiding the low-level details from the end-users of abstractions. We diagram this fundamental \emph{compatibility problem} in Figure \ref{approaches}(a).
%\begin{figure*}
%\begin{center}
%\includegraphics[scale=0.5]{approaches.pdf}
%\end{center}
%\vspace{-20px}
%\caption{\small (a) With a language-oriented approach, novel constructs are packaged into separate languages. Users can only safely and naturally call into languages consisting of common constructs (often only the common target language, such as C or Java bytecode). (b) With a language-internal extensibility approach, there is one system providing a common internal language, where additional primitive constructs that strictly strengthen its static guarantees or perform specialized code generation are specified and distributed within libraries. \label{approaches}}
%\end{figure*}

%As a result, domain-specific languages and new general-purpose abstractions alike have experienced relatively slow adoption in practice.
%
%Porting large codebases to new languages is difficult, and the dominant programming languages innovate slowly, so programming language.
%
%More specifically, such languages are neither \emph{internally extensible} because the language itself exposes only natural numbers and functions to its users, nor are they \emph{externally extensible} because no new behaviors can be added to the language's  implementation in a separate module from the one containing the initial implementation.

%This is the essence of a monolithic language implementation: it is impossible for anyone to modularly extend languages defined in this way. 

%\subsection{Theory}\label{atlam}
%%\begin{figure}
%%\begin{mathpar}
%%\inferrule{a}{b}
%%\end{mathpar}
%%\end{figure}
%In this section, we will describe a minimal calculus that captures our language-internal extensibility mechanism,  called \emph{active typechecking and translation (AT\&T)}. AT\&T allows developers to declare new primitive type families, associate operators with them, implement their static semantics in a functional style, and realize their dynamic semantics by simultaneously implementing a translation into a typed internal language. Note that this latter mechanism is closely related to how Standard ML was (re-)specified\todo{cite/read a bit about this/ask Bob/flesh this out}, but that we are fundamentally interested in extending language \emph{implementations}, not their  declarative specifications; proving the adequacy of such implementations against mechanized specifications, or extracting them directly from such specifications, will be investigated in future work.
%
%The AT\&T mechanism utilizes type-level computation of higher kind and integrates typed compilation techniques into the language to allow us to give strong metatheoretic guarantees,  and uses a mechanism notionally related to abstract types (such as those found in the ML module system) to guarantee that extensions cannot interfere with one another,  while remaining straightforward and expressive. In this section, we will develop a core calculus, called \atlam, which uses a minimal, uniform grammar for primitive operators. Then in Section \ref{ace}, we will show how to realize this minimal mechanism within a widely-used language with a more expressive grammar.
%
%AT\&T is general with respect to many choices about the type-level language, the typed internal language and syntax. Choices along these dimensions can affect both expressiveness and ease-of-use. We will begin in Sec. 2 by introducing a minimal system called $@\lambda$ (the ``actively-typed lambda calculus'') that distills the essence of the mechanism in a simply-typed, simply-kinded setting. This will allow us to fully and concisely formalize the language and compiler and give several key safety theorems. We will then continue in Sec. 3 by discussing variants of this mechanism based on other basic paradigms, considering dependently-typed functional languages and object-oriented languages, discussing trade-offs between expressivity and safety when doing so. We have developed a simple prototype called Ace and have used it to develop a number of full-scale language extensions as libraries. We will briefly discuss this language and these extensions in Sec. 4.

%We note at the outset that AT\&T focuses on extending the static semantics of languages with fixed, though flexible, syntax. Language-internal syntax extension mechanisms have been developed in the past (e.g. SugarJ \cite{sugarj}) but they have also suffered from safety problems because grammar composition is not always safe when done in an  unconstrained manner. Tyconstrained approaches that provide stronger safety guarantees have recently been outlined (e.g. Wyvern \cite{globaldsl13}) but we will leave integration of syntax extensions with semantic extensions as future work.


%\subsubsection{@$\lambda$: Conservatively Extending a Type System From Within}
\begin{figure}[t]
\small
$$\begin{array}{rccl}	
\textbf{programs} & \rho & ::= & \pfam{\familyDf}{\progsort} 
%\pipe \pdef{t}{\kappa}{\tau}{\progsort} 
\pipe e\\
		&	\theta	&	::= &	\tops{op}{\kappaidx}{i}{a}{\taut{def}} \pipe 
												\topp{\theta}{\theta}\\
\\
\textbf{external terms} 				&	e	&	::=	&	\evar{x} \pipe 
%														\efix{x}{\tau}{e} \pipe 
														\elam{\evar{x}}{\tau}{e} \pipe 
														\eop{Tycon}{op}{
															\tauidx
														}{
  												    		\splat{e}{1}{n}
														} \\
									& 		&		& 	\\


\textbf{internal terms} 				& 	\iota	&	::=	&	\evar{x} \pipe 
												\ifix{\evar{x}}{\sigma}{\iota} \pipe
												\ilam{\evar{x}}{\sigma}{\iota} \pipe 
												\iapp{\iota_{1}}{\iota_{2}} \pipe
												\iintlit \pipe \iop{\iota_{1}}{\iota_{2}} \pipe \iIfEq{\iota_{1}}{\iota_{2}}{\dint}{\iota_{3}}{\iota_{4}} \\
										& & \pipe & 
												\iunit \pipe
												\ipair{\iota_{1}}{\iota_{2}} \pipe 
												\ifst{\iota} \pipe
												\isnd{\iota} \pipe
												 \iinl{\sigma_2}{\iota_1} \pipe \iinr{\sigma_1}{\iota_2} \pipe \icase{e}{x}{e_1}{x}{e_2} \\
												
%\text{deabstracted}& \iota & ::= & \mathcal{G}[\iota, \sigma]\\
\textbf{internal types}			&	\sigma	&	::=	&    \darrow{\sigma_1}{\sigma_2} \pipe \dint \pipe \dunit \pipe \dpair{\sigma_1}{\sigma_2} \pipe \dsum{\sigma_1}{\sigma_2}  
												
\\
\\
							
\hspace{-5pt}\textbf{type-level terms} 	& \tau 	& ::= 	& 	\tvar{t} \pipe 
														\tlam{t}{\kappa}{\tau} \pipe 
														\tapp{\tau_1}{\tau_2} \pipe
														\tnil{\kappa} \pipe \tcons{\tau_1}{\tau_2} \pipe 
									                     \tfold{\tau_1}{\tau_2}{h}{t}{r}{\tau_3}
														\\
												
	 			& 		& \pipe	& 	 \iintlit \pipe \iop{\tau_1}{\tau_2} \pipe \tlabel{label} \pipe \tunit \pipe 
														\tpair{\tau_{1}}{\tau_{2}} \pipe 
														\tfst{\tau} \pipe 
														\tsnd{\tau} 
														\\	
									&       & \pipe & \tinl{\kappa_2}{\tau_1} \pipe \tinr{\kappa_1}{\tau_2} \pipe \tsumcase{\tau}{x}{\tau_1}{x}{\tau_2} \\
\text{equality}  & & \pipe & 					\tifeq{\tau_{1}}{\tau_{2}}{\kappa}{\tau_{3}}{\tau_{4}} 
														\\													\text{types} 						& 		& \pipe	& 	\ttypestd \pipe \tfamcase{\tau}{Tycon}{x}{\tau_1}{\tau_2}\\
						%				& & \pipe & \tfamcase{\tau}{Fam}{x}{\tau_1}{\tau_2}\\
																								
\text{derivates} 				& 		 & 	\pipe	&	\tden{\tauiterm}{\tautype}  \pipe \tderbind{x}{t}{\tau}{\tau_1}\\% \tdencase{\tau}{x}{t}{\tau_1}{\tau_2}\\
 %& & \pipe & 
%														\tdencase{\tau}{y}{x}{\tau_1}{\tau_2}
%														 \\

\text{reified IL}		&		&	\pipe	&	\titerm{\bar \iota} \pipe \titype{\bar \sigma} \\
% &  & \pipe & \tvalof{\tau_1}{\tau_2} \pipe \iup{\tau} \\
% 												\trepof{\tau} \pipe \dup{\tau}\\
	& \bar{\iota} & ::= & x \pipe \ifix{x}{\bar \sigma}{\bar \iota} 
	%\pipe \ilam{x}{\bar \sigma}{\bar \iota} \pipe \iapp{\bar \iota_1}{\bar \iota_2} 
	\pipe \cdots \pipe \iup{\tau} \pipe \tvalof{\tauiterm}{\tautype} \\						
 & \bar{\sigma} & ::= & \darrow{\bar \sigma_1}{\bar \sigma_2} \pipe \cdots \pipe \dup{\tau} \pipe \trepof{\tau} \\
											\\
\textbf{kinds} 					& \kappa	&	::=	&	\karrow{\kappa_1}{\kappa_2} \pipe \klist{\kappa} \pipe \kint \pipe
											    \klabel \pipe
												\kunit \pipe 
												\kpair{\kappa_{1}}{\kappa_{2}} \pipe 
												\ksum{\kappa_1}{\kappa_2} \pipe
												\kTypeBlur \pipe \kDen \pipe 
												\kITerm \pipe \kIType
												\\
\\												
%\textbf{ops signature}			& \Theta	&	::=	&	\kOpEmpty \pipe \kOp{\Theta}{op}{\kappai}\\
%											 							&		&		&	\\
\end{array}$$
\vspace{-10pt}
\caption{\small Syntax of Core \atlam. Here, $x$ ranges over external and internal language variables, $\tvar{t}$ ranges over type-level variables, $\fvar{Tycon}$ ranges over type constructor names, $\opvar{op}$ ranges over operator constructor names, $\iintlit$ ranges over integer literals, $\tlabel{label}$ ranges over label literals (see text) and $\oplus$ ranges over the standard total binary operations over integers (e.g. addition). The form $\tvalof{\tauiterm}{\tautype}$ is  used internally by the semantics (it need not be supported by a parser).%The productions related to the internal language are written using generators $\mathcal{G}$ and $\mathcal{S}$ to avoid duplicating the syntax of common terms.
\vspace{-10pt}
\label{grammar}}
\end{figure}
%\begin{figure}[t]
%\small
%\begin{flalign}
%\label{natfam}&\family{Nat}{\kunit}{\\
%\label{z}&\quad\tops{z}{\kunit}{i}{a}{\tlam{idx}{\kunit}{\tlam{args}{\klist{\kDen}}{
%	\tapp{\tapp{\tvar{if\_empty}}{\tvar{args}}}{
%		\tden{\titerm{0}}{\ttype{Nat}{\tunit}}
%	}}}};\\
%\label{s}&\quad\tops{s}{\kunit}{i}{a}{\tlam{idx}{\kunit}{\tlam{args}{\klist{\kDen}}{\\
%& \quad\quad
%	\tapp{
%		\tapp{\tvar{pop\_final}}{\tvar{args}}}{\tlam{x}{\kITerm}{\tlam{t}{\kTypeBlur}{
%			\\
%			&\quad\quad\tapp{\tapp{\tapp{\tvar{check\_type}}{\tvar{t}}}{\ttype{Nat}{\tunit}}}{
%				\tden{\titerm{\iup{\tvar{x}}+1}}{\ttype{Nat}{\tunit}}
%			}
%		}}
%		}
%	}}};\\
%\label{rec}&\quad\tops{rec}{\kunit}{i}{args}{\tlam{idx}{\kunit}{\tlam{args}{\klist{\kDen}}{\\
%& \quad\quad
%	\tapp{\tapp{\tvar{pop}}{\tvar{args}}}{\tlam{x1}{\kITerm}{\tlam{t1}{\kTypeBlur}{\tlam{args}{\klist{\kDen}}{\\
%	&\quad\quad\tapp{\tapp{\tvar{pop}}{\tvar{args}}}{\tlam{x2}{\kITerm}{\tlam{t2}{\kTypeBlur}{\tlam{args}{\klist{\kDen}}{\\
%	&\quad\quad \tapp{\tapp{\tvar{pop\_final}}{\tvar{args}}}{\tlam{x3}{\kITerm}{\tlam{t3}{\kTypeBlur}{\\
%	&\quad\quad \tapp{\tapp{\tapp{\tvar{check\_type}}{\tvar{t1}}}{\ttype{Nat}{\tunit}}}{(\\
%	&\quad\quad \tapp{\tapp{\tapp{\tvar{check\_type}}{\tvar{t3}}}{\ttype{Arrow}{(\ttype{Nat}{\tunit},\ttype{Arrow}{(\tvar{t2},\tvar{t2})})}}}{\\
%		\label{fix}&\quad\quad \tden{\titerm{\iapp{(\ifix{f}{\darrow{\dint}{\trepof{\tvar{t2}}}}{\ilam{x}{\dint}{\\
%		\label{lastop}&\quad\quad\quad \iIfEq{x}{0}{\dint}{\iup{\tvar{x2}}}{\iapp{\iapp{\iup{\tvar{x3}}}{(x-1)}}{(\iapp{f}{(x-1)})}}}})}{\iup{\tvar{x1}}}}}{\tvar{t2}}
%	)}}
%	}}}}}}}}}}}}}}
%\\
%&}{XXX}{i}{\tlam{idx}{\kunit}{\titype{\dint}}};\\
%\label{nattype}&\etdef{nat}{\kTypeBlur}{\ttype{Nat}{\tunit}}{\\
%& \elet{plus}{\ttype{Arrow}{(\tvar{nat}, \ttype{Arrow}{(\tvar{nat}, \tvar{nat})})}}{
%	\elam{x}{\tvar{nat}}{
%		\elam{y}{\tvar{nat}}{
%			\\&\quad \eop{Nat}{rec}{\tunit}{x; y; \elam{p}{\tvar{nat}}{\elam{r}{\tvar{nat}}{
%				\eop{Nat}{s}{\tunit}{r}
%			}}}
%		}
%	}
%}{\\
%& \elet{two}{\tvar{nat}}{\eop{Nat}{s}{\tunit}{\eop{Nat}{s}{\tunit}{\eop{Nat}{z}{\tunit}{}}}}{\\
%& \eop{Arrow}{ap}{\tunit}{plus; \eop{Arrow}{ap}{\tunit}{two; two}}
%}}}
%%{\\&\elet{two}{\tvar{nat}}{\eop{Nat}{s}{\tunit}{\eop{Nat}{s}{\tunit}{\eop{Nat}{z}{\tunit}{ }}}}{\eapp{\eapp{plus}{two}}{two}}}
%%}
%%\elam{x}{\tvar{nat}}{\elam{y}{\tvar{nat}}{\\
%	%&\quad \eop{Nat}{rec}{\tunit}{x; y; \elam{p}{\tvar{nat}}{\elam{r}{\tvar{nat}}{
%	%\eop{Nat}{s}{\tunit}{r}
%	%}}}}
%\end{flalign}
%\vspace{-15px}
%\caption{\small G\"odel's $\mathbf{T}$ in \atlam, used to calculate 2+2. The simple helper functions $\tvar{if\_empty}$, $\tvar{pop}$, $\tvar{pop\_final}$, $\tvar{check\_type}$ all have return kind $\ksum{\kDen}{\kunit}$ (see text). Their definitions can be found in the appendix. We use \textsf{let} to bind both external and type-level variables for clarity of presentation; these can be eliminated by manually performing the indicated substitutions (or added to the semantics).}
%\label{nat}
%\end{figure}
%
%\begin{figure}[t]
%\begin{lstlisting}
%tycon Nat of 1 with  
%    schema $\lambda$idx:1.$\blacktriangledown$($\mathbb{Z}$) 
%    opcon Z of 1 ($\lambda$idx:1.$\lambda$args:list[Elab].is_empty args $\llbracket$$\triangledown$(0) as Nat[()]$\rrbracket$)
%    opcon S of 1 ($\lambda$idx:1.$\lambda$args:list[Elab].pop_final args $\lambda$x:ITm.$\lambda$ty:$\star$. 
%      checktype ty Nat[()] $\llbracket$$\triangledown$($\vartriangle$(x)+1) as Nat[()]$\rrbracket$)
%    opcon Rec of 1 ($\lambda$idx:1.$\lambda$args:list[Elab].
%      pop args $\lambda$x1:ITm.$\lambda$ty1:$\star$.$\lambda$args':list[Elab].
%      pop args' $\lambda$x2:ITm.$\lambda$ty2:$\star$.$\lambda$args'':list[Elab].
%      pop_final args'' $\lambda$x3:ITm.$\lambda$ty3:$\star$.
%      check_type t1 Nat[()] (
%      check_type t3 Arrow[(Nat[()], Arrow[(t2, t2)])] 
%      	$\llbracket$$\triangledown$((fix f:$\mathbb{Z} \rightarrow$ rep(t2) is $\lambda$x:$\mathbb{Z}$.
%	        if x = 0 then $\vartriangle$(x2) else $\vartriangle$(x3) (x - 1) (f (x - 1))) $\vartriangle$(x1)) as t2$\rrbracket$))
%end
%\end{lstlisting}
%\caption{An implementation of primitive natural numbers as internal integers in @$\lambda$.}
%\label{nat-atlam}
%\end{figure}
In this section, we will develop an ``actively typed'' version of the simply-typed lambda calculus with simply-kinded type-level computation called @$\lambda$. More specifically, the level of types, $\tau$, will itself form a simply-typed lambda calculus. \emph{Kinds} classify type-level terms in the same way that types conventionally classify expressions. Types become just one kind  of type-level value (which we will write $\kTypeBlur$, though it is also variously written $\star$, \verb|T| and \verb|Type| in various settings). Rather than there being a fixed set of type constructors, we allow the programmer to declare new type  constructors, and give the static and dynamic semantics of their associated operators, by writing type-level functions. In the semantics for this calculus, our kind system combined with techniques borrowed from the typed compilation literature and a form of type abstraction allow us to prove strong type safety, decidability and conservativity theorems.

The syntax of @$\lambda$ is given in Fig. \ref{grammar}. An example of a program containing an extension that implements the static and dynamic semantics of G\"odel's \textbf{T} is given in Fig. \ref{nat}. We do not, of course, deny that natural numbers can be strongly encoded with a similar usage and asymptotic performance profile (but with potentially relevant additional function call overhead) by existing means (e.g. by an abstract type). We will provide more sophisticated examples where this is not the case below. %It consists of:
%a type constructor declaration, $\fvar{Nat}$, indexed trivially, together with three operator constructors, also all indexed trivially, that implement the standard introductory forms for natural numbers as well as the recursor operator (as in G\"odel's T \cite{pfpl}). Following the type constructor declaration, we apply $\fvar{Nat}$ with the trivial index, $\tunit$, to form the type $\tvar{nat}$. Finally, we write an external term that uses the operators associated with $\fvar{nat}$ and the built-in constructor $\fvar{Parr}$, governing partial functions, to define an addition function and compute the addition of the natural numbers  two and two. We will introduce a more convenient concrete syntax in later portions of this thesis; for now we will restrict ourselves to the abstract syntax so that this example can directly aid in understanding the semantics.

\subsection{Programs}

A program, $\rho$, consists of a series of static declarations followed by an external term, $e$. The syntax for external terms contains three core forms: variables, functions, and a form for all other operator invocations, which we will discuss below. Compiling a program consists of first \emph{kind checking} its static declarations and type-level terms then \emph{type checking} the external term and, simultaneously, \emph{translating} it to a term, $\iota$, in a \emph{typed internal language}. In our calculus, the internal language contains partial functions (via the generic fixpoint operator of Plotkin's PCF \cite{PCF}), simple product and sum types and a base type of integers (to make our example interesting and give a nod to practicality on contemporary machines). In practice, the internal language could be any intermediate language for which type safety and decidability of typechecking are known, as @$\lambda$ requires these to establish the corresponding theorems about the language as a whole.

The \emph{active typing judgement} relates an external term to a \emph{type} and an internal term, called its \emph{translation}, under \emph{typing context} $\Gamma$ and \emph{constructor context} $\Phi$: 
$$\ecompilesX{e}{\tau}{\iota}$$ 

This judgement follows standard conventions for the typing judgement of a simply-typed lambda calculus in most respects. The typing context $\Gamma$ maps variables to types and obeys standard structural properties. The key novelties here are that the available type and operator constructors are not fixed by the language, but rather are tracked by the constructor context, $\fvalCtx$, and that a translation is simultaneously derived. The dynamic semantics of external terms are defined by their translation to the internal language. For this judgement, the two contexts and the external term can be thought of as input, while the type and translation are output (we do not consider a bidirectional approach in this work, though this may be a promising avenue for future research).

\subsection{Types}
User-defined type constructors are declared at the top-level of a program (or package, in a practical implementation) using \textsf{tycon}. Each type constructor in the program must have a unique name, written e.g. \fvar{Nat}. We do not intrinsically consider the issue of namespacing, under the assumption that globally unique names can be generated by some extrinsic mechanism (e.g. a URI-based scheme). A type constructor must also declare an \emph{index kind}, $\kappaidx$. Types themselves are type-level values of kind $\kTypeBlur$ and are introduced by applying a previously declared type constructor to a type-level term of its index kind, $\ttype{Tycon}{\tauidx}$. The kind $\kTypeBlur$ also has an elimination form: a type can be case analyzed against a type constructor in scope to extract its index. Like open datatypes, there is no longer a notion of exhaustiveness. To maintain totality, we require the default case.

To permit the implementation of interesting type systems, the type-level language includes several other kinds of data alongside types. We lift several  standard functional constructs to the type level: unit ($\kunit$), binary sums ($\ksum{\kappa_1}{\kappa_2}$), binary products ($\kpair{\kappa_1}{\kappa_2}$), lists ($\klist{\kappa}$) and integers ($\dint$). We also include labels ($\klabel$), written in a slanted font, e.g. $\tlabel{myLabel}$, which are atomic string-like values that can only be compared, here only for equality, and play a distinguished role in the expanded syntax, as we will later discuss. Our first example, $\fvar{Nat}$, is indexed trivially, i.e. by unit kind, $\kunit$, because there is only one natural number type, $\ttype{Nat}{\tunit}$, but we will show examples of type constructors that are indexed in more interesting ways in later portions of this work. For example, $\fvar{Tuple}$ is indexed by a list of types and $\fvar{LabeledTuple}$ is indexed by a list pairing labels with types.  

Two type-level terms of kind $\kTypeBlur$ are equivalent if they apply the same constructor, identified by name, to equivalent indices. Going further, we ensure that deciding type equivalence requires only checking for syntactic equality after evaluation to normal form by imposing the restriction that type constructors can only be indexed by kinds for which equivalence can be decided in this way. All combinations of kinds introduced thus far  have this property and this is sufficient for many interesting examples. We leave  introducing a richer notion of  equivalence of types that  extension providers can extend as future work, as the metatheoretic guarantee that typing respects type equivalence would be quite a bit more complex in such a setting.% (a na\"ive approach to this would impose non-trivial extrinsic proof obligations onto extension developers that, unlike in others in this thesis, could threaten type safety).

Type constructors are not first-class; they do not themselves have arrow kind as in some kind systems (e.g.  \cite{Watkins08}\todo{cite}; Ch. 22 of \emph{PFPL} describes a related system \cite{pfpl}). The type-level language does, however, include total functions of conventional arrow kind, $\karrow{\kappa_1}{\kappa_2}$. Type constructor application can be wrapped in a type-level function to emulate first-class type constructors (and indeed, such a wrapper could be generated automatically, though we avoid this for simplicity in our semantics). Equivalence at arrow kind does not coincide with $\beta$-equivalence, so type-level functions cannot appear in type indices. Our treatment of equivalence in the type-level language is thus quite similar to the treatment of term-level equality using ``equality types'' in a language like Standard ML. Indeed, one might imagine that a practical implementation of our calculus would start with a pure, total subset of the term language of ML to construct the type-level language (consistent with its early development as a ``metalanguage'' and, as we will see, the role of our type-level language as a DSL for implementing type systems and translators, which functional languages are widely considered well-suited for as-is). We leave the development of such an implementation (and of bootstrapping type-level and internal languages that are themselves user-defined or extensible) as areas for future work.
\subsection{Operators}
User-defined operator constructors are declared using \textsf{opcon}. For reasons that we will discuss, our calculus associates every operator constructor with a type constructor. The \emph{fully-qualified name} of the operator constructor, e.g. $\fvar{Nat}.\opvar{z}$, must be unique. Operator constructors are indexed by type-level values, like type constructors, and so also declare an index kind $\kappaidx$. In our first example, all the operator constructors are indexed trivially, but later examples will use more interesting indices. For example, the projection operators for tuples and labeled tuples use numbers and labels as indices, respectively.
Note that neither operator constructors nor operators are first-class type-level values (we plan to investigate  lifting the latter restriction) and we do not impose any restrictions on their index kinds.

In the external language, an operator is selected and invoked by applying an operator constructor to a type-level index and $n \geq 0$ \emph{arguments}, $\eop{Tycon}{op}{\tauidx}{\splat{e}{1}{n}}$. For example, on line 18 of Fig. \ref{nat}, we see the operator constructors $\fvar{Nat}.\opvar{z}$ and $\fvar{Nat}.\opvar{s}$ being invoked to compute two.\footnote{Although our focus here is entirely on semantics, a brief note on syntax: in the expanded syntax, the trivial indices and empty argument lists can be omitted, so we could write \texttt{Nat.s(Nat.s(Nat.z))}. With the ability to ``open'' a type's operators into the context, we could shorten this still to \texttt{s(s(z))}. Alternatively, with the ability to define a TSL in a manner similar to that in Sec. \ref{aparsing}, we might instead just write \texttt{2}.}
To derive the active typing judgement for this form, the semantics invokes  
 the operator constructor's  \emph{definition}: a type-level function that must examine the provided operator index and the recursively determined \emph{derivates} of the arguments to determine a derivate for the operation as a whole,  or indicate an error. A derivate is a type-level representation of the result of deriving the active typing judgement\footnote{Technically, the abstract active typing judgement, which is similar in form and we will introduce shortly.}: an internal term, called the {translation},  paired with a type. Derivates have kind $\kDen$ and introductory form $\tden{\tautrans}{\tautype}$. The elimination form $\tderbind{x}{t}{\tau}{\tau'}$ extracts the translation and type from the derivate $\tau$, binding it to $\tvar{x}$ and $\tvar{t}$ respectively in $\tau'$. 
%Indeed, one might be tempted to simply define derivates as products. However, we treat translations with much care throughout the calculus, so this is not quite the case (we will return to this shortly). 
Because constructing a derivate is not always possible (e.g. when there is a type error in the client's code, or an invalid index was provided), the return kind of the function is an ``option kind'', $\ksum{\kDen}{\kunit}$, where the trivial case indicates a statically-detected error in the client's program. In practice, it would instead require providers to report information about the precise location of the error and an appropriate error message.
 
A translation, as we have said, is an internal term. To construct an internal term using a type-level function, as we must do to construct a derivate, we must expose a type-level representation of the internal language. A \emph{quoted internal term} is a type-level value of kind $\kITerm$ with introductory form $\titerm{\bar \iota}$ and a \emph{quoted internal type} is a type-level value of kind $\kIType$ with introductory form $\titype{\bar \sigma}$. Neither kind has an elimination form. Instead, the syntax for the quoted  internal language includes complementary \emph{unquote forms} $\iup{\tau}$ and $\dup{\tau}$ that permit the interpolation of another quoted term or type, respectively, into the one being formed. Interpolation is capture-avoiding, as our semantics will clarify. We have now described all the kinds in our language.

The definition of $\fvar{Nat}.\opvar{z}$ is quite simple: it returns the derivate $\tden{\titerm{0}}{\ttype{Nat}{\tunit}}$ if no arguments were provided, and indicates an error (an \emph{arity error}, though we do not distinguish this in our calculus) otherwise. This is done by calling a simple helper function, $\tvar{is\_empty} : \karrow{\klist{\kDen}}{\karrow{\kDen}{(\ksum{\kDen}{\kunit})}}$, that selects the appropriate case of the sum given the derivate that should be produced if the list is indeed empty. The definition of $\fvar{Nat}.\opvar{s}$ is only slightly more complex, because it requires inspecting a single argument. The helper function $\tvar{pop\_final} : \karrow{\karrow{\klist{\kDen}}{(\karrow{\kTypeBlur}{\karrow{\kITerm}{(\ksum{\kDen}{\kunit})})}}}{(\ksum{\kDen}{\kunit})}$ ``pops'' a derivate from the head of the list and, if no other arguments remain, passes its translation and type to the ``continuation'', returning the error case otherwise. The continuation checks if the argument type is equal to $\ttype{Nat}{\tunit}$ using $\tvar{check\_type} : \karrow{\kTypeBlur}{\karrow{\kTypeBlur}{\karrow{\kDen}{(\ksum{\kDen}{\kunit})}}}$, which operates similarly. If these arity and type checks succeed, the resulting derivate is composed by adding one to the translation of the argument, passed into the continuation as $\tvar{x}$ in our example, and pairing it with the natural number type: $\tden{\titerm{\iup{\tvar{x}} + 1}}{\ttype{Nat}{\tunit}}$. We will return to the definition of $\fvar{Nat}.\opvar{rec}$ after in the next subsection.

By writing the typechecking and translation logic in this way, as a total function, we are taking a rather ``implementation-focused view'' rather than attempting to extract such a function from a declarative specification. That is,  we leave to the provider (or to a program generator or kind-specific language that transforms such a specification into an operator definition) the problem of finding a deterministic algorithm that adequately implements their intended semantics, allowing us to prove decidability of typechecking for the language as a whole by essentially just citing the termination theorem for the type-level language. We also avoid difficulties with error reporting in practice. Rob Simmons' undergraduate thesis has a good discussion of these issues \cite{rjs-princeton}.

Because the input to an operator definition is the recursively determined derivate of the argument in the same context as the operator appears in, our mechanism does not presently permit the definition of operator constructors that bind variables themselves or require a different form of typing judgement (e.g. additional forms of contexts). This is also why the $\lambda$ operator constructor needs to be built in. It is the only operator in our language that can be used to bind variables. The type constructor $\fvar{Arrow}$ can be defined from within the language (and is included in the ``prelude'' constructor context), but only defines the operator constructor, $\opvar{ap}$. These two operators translate to their corresponding forms in the internal language directly, as we will discuss below. We leave adding the ability to declare and manipulate new contexts as future work.

\subsection{Representational Consistency Implies Type Safety}
In our example, natural numbers are represented internally as integers. Were this not the case -- if, for example, we added an operator constructor $\fvar{Nat}.\opvar{z2}$ that produced the derivate $\tden{\titerm{\iinl{\dint}{\iunit}}}{\ttype{Nat}{\tunit}}$, then there would be two different internal types, $\dint$ and $\dsum{\dunit}{\dint}$,  associated with a single external type, $\ttype{Nat}{\tunit}$. This makes it impossible to operate compositionally on the translation of an external term of type $\ttype{Nat}{\tunit}$, so our implementation of $\fvar{Nat}.\opvar{s}$ would produce ill-typed translations in some cases but not others. Similarly, we wouldn't be able to write functions over all natural numbers because there would not be a well-typed translation to give to such a function.

To reason compositionally about the semantics of well-typed external terms when they are given meaning by translation to a typed internal language, we must have the following property: for every  type, $\tau$, there must exist an internal type, $\sigma$, called its \emph{representation type}, such that the translation of every external term of type $\tau$ has internal type $\sigma$. This principle of \emph{representational consistency} arises essentially as a strengthening of the inductive hypothesis necessary to prove that all well-typed external terms translate to well-typed internal terms, precisely because operators like $\lambda$ and $\fvar{Nat}.\opvar{s}$ are defined compositionally. It is closely related to the concept of \emph{type-preserving compilation} developed by Morrisett et al. for the TIL compiler for Standard ML \cite{TIL}\todo{cite}. %That is, instead of being used as a necessary condition for compiler correctness, we are using it as a sufficient condition for type safety. 
%We emphasize the distinction between translation (which in our calculus endows terms with a dynamic semantics) and compilation (which must preserve the dynamic semantics of terms). %In the presence of extensions, checking for representational consistency becomes subtle if we wish to guarantee conservativity, as we will discuss in the next subsection.

It is easy to show by induction that, under a ``closed-world assumption'' where the only available operators are $\fvar{Nat}.\opvar{z}$ and $\fvar{Nat}.\opvar{s}$, the representation type of $\ttype{Nat}{\tunit}$ is $\dint$. If we can maintain this under an ``open-world assumption'' (requiring that, for example, applications of operator constructors like $\fvar{Nat}.\opvar{z2}$, above, are not well-typed), and we target a type safe internal language, then we will achieve type safety: well-typed external terms cannot go wrong, because they always translate to well-typed internal terms, which cannot go wrong. For the semantics to ensure that representational consistency is maintained by all operator definitions, we require that each type constructor must declare a \emph{representation schema} with the keyword \textsf{schema}. This must be a type-level function of kind $\karrow{\kappaidx}{\kIType}$, where $\kappaidx$ is the index kind of the type constructor. When the compiler needs to determine the representation type of the type $\ttype{Tycon}{\tauidx}$ it simply applies the representation schema of $\fvar{Tycon}$ to ${\tauidx}$.

As described above, the kind $\kIType$ has introductory form $\titype{\bar \sigma}$ and no elimination form. There are two forms in $\bar \sigma$ that do not correspond to forms in $\sigma$ that allow quoted internal types to be formed compositionally:
\begin{enumerate}
\item $\dup{\tau}$, already described, unquotes (or ``interpolates'') the quoted internal type $\tau$
\item $\trepof{\tau}$ refers to the representation type of type $\tau$ (we will see in the next subsection why this needs to be in the syntax for $\bar \sigma$ and not directly in the type-level language)
\end{enumerate}

These additional forms are not needed by the representation schema of $\fvar{Nat}$  because it is trivially indexed. In Fig. \ref{tuple}, we show an example of the type constructor $\fvar{Tuple}$, implementing the semantics of $n$-tuples by translation to nested binary products. Here, the representation schema requires referring to the representations of the tuple's constituent types, given in the type index\todo{add this example from appendix of ESOP}.

Operator constructor definitions might also need to refer to the representation of a type. We see this in the definition of the recursor on natural numbers, $\fvar{Nat}.\opvar{rec}$. After checking the arity and extracting the types and translations of its three arguments, it produces a derivate that implements the necessarily recursive dynamic semantics using a fixpoint computation in the internal language. This fixpoint computation produces a result of some arbitrary type, $\tvar{t2}$. We cannot know what the representation type of $\tvar{t2}$ is, we refer to it abstractly using $\trepof{\tvar{t2}}$. The translation is well-typed no matter what the representation type is. In other words, a proof of representational consistency of the derivate produced by the definition of $\fvar{Nat}.\opvar{rec}$ is parametric (in the metamathematics) over the representation type of $\tvar{t2}$. This leads us into the concept of conservativity.

\subsection{Parametricity Implies Conservativity}
In isolation, our definition of natural numbers can be shown to be a strong encoding of the semantics of G\"odel's $\mathbf{T}$. That is, there is a bijection between the terms and types of the two languages and the typing judgements preserve this mapping. Moreover, we can show by a bisimulation argument that the translation produced by @$\lambda$ implements the dynamic semantics of G\"odel's $\mathbf{T}$. We will provide more details on this later. A necessary lemma in the proof, however, is that the value of every translation of an external term of type $\ttype{Nat}{\tunit}$ is a non-negative integer. This is needed to show that the fixpoint computation in the definition of $\fvar{Nat}.\opvar{rec}$ is terminating, as the recursor always does in $\mathbf{T}$. A very similar lemma is needed to show that the translation of every external term of type $\ttype{Nat}{\tunit}$ is equivalent to the translation that would be produced by some combination of applications of $\fvar{Nat}.\opvar{z}$ and $\fvar{Nat}.\opvar{s}$ (the analog to a canonical forms lemma in our setting). Fortunately, the definitions we have provided thus far admit such lemmas.

However, these theorems are all quite precarious because they rely on exhaustive induction over the available operators. As soon as we load another library into the program, these theorems may no longer be \emph{conserved}. For example, the following operator declaration in some other type that we have loaded would topple our house of cards:





%If g : nat then [[g]] : Nat[()] -> i and i : Z and there exists an i' s.t. i |->* i' and i' val and geq i' -1 |->* 1.
%
%[[z]] := Nat.z[()]()
%[[s(e)]] := Nat.s[()]([[e]])
%If |- e' : t then [[rec(e; e'; x.y.e'')]] := Nat.rec[()]([[e]]; [[e']]; \x:Nat[()].\y:[[t]].e'')
%[[nat]] := Nat[()]
%[[t -> t']] := Arrow[([[t]], [[t']])]
%[[x]] := x
%[[\x:t.g]] := \x:[[t]].[[g]]
%[[g g']] := Arrow.ap[()]([[g]]; [[g']])
%
%If |- g : nat then g \evals g' and |- [[g]] : Nat[()] -> i
%  s.t. |- i : Z and if if i \evals i' then 
%    gt i' -1 |->* 1 and 
%    |- [[g']] : Nat[()] -> i'' s.t. 
%    if i'' \evals i''' then i' = i'''.
%    
%If |- e : Nat[()] -> i then |- [[e]] : nat and [[e]] |-> g' and |- [[g']] : Nat[()] -> i' and if i \evals i'' and i' evals i''' then i'' = i'''.
%
%The proof relies on the following lemma:
%If |- e : Nat[()] -> i then if i evals i', gt i' 0 evals 1.
%
%Without it, the case for the recursion operator, which requires fixpoint induction, would not terminate. 
%  
%g |-> g' s.t. g' val then |- [[g']] : Nat[()] -> i''
%  |- i'' : Z
%  i'' |->* i'
%
%[[Nat.z[()]()]] := z
%[[Nat.s[()](e)]] := s([[e]])
%[[Nat.rec[(()](e; e'; e'')]] := rec([[e]]; [[e']]; [[e'']])
%[[\x:tau.e]] := \x:[[tau]].[[e]]
%...
%
%If |- e : Nat[()] -> i then |- [[e]] : nat and [[e]] |-> g' and i -> i' and [[g']] : Nat[()] -> i'' s.t. i'' ->* i'.
%

%
%It can also be shown to maintain the invariant that natural numbers elaborate into \emph{non-negative} internal integers. Note that because we are beginning with a simply-typed, simply-kinded formulation, these kinds of statements must be proven metatheoretically (as with other sorts of complex invariants in simply-typed languages). With a naive formulation of this mechanism, these theorems would be quite a bit more precarious, however. If another provider, perhaps a malicious one, declares a new operator constructor that introduces natural numbers, these theorems may not be \emph{conserved}. For example, the following operator declaration is quite problematic

\todo{In progress below}:

\begin{lstlisting}
opcon badnat of 1 ($\lambda$idx:1.$\lambda$args:list[Elab].$\llbracket\triangledown$(-1) as Nat[()]$\rrbracket$)
\end{lstlisting}

With this declaration, there is now a new and unexpected introductory form for the natural number type, and even worse, it violates our previous implementation invariants. A naive check for type preservation would not catch this problem: \li{-1} does have type $\mathbb{Z}$, it is simply not an integer that could have been emitted by the original definition of \li{Nat}.

To prevent this problem, we \emph{hold the representation of a type abstract} outside of the operators explicitly associated with it. If \li{badnat} cannot know that the representation of \li{Nat[()]} is $\mathbb{Z}$, then the above operator implementation cannot be correct. Because we cannot prove relational correctness properties from within a simply-typed/kinded calculus, our semantics enforces this by using a form of value abstraction, similar to that described by Zdancewic et. al \cite{zdancewic}. In brief: the representation of a type, written $\trepof{\tautype}$, does not reduce further to the actual representation type (e.g. to $\mathbb{Z}$) when not in an operator definition associated with $\tautype$. The only internal form that can be checked against  $\trepof{\tautype}$ is the special form $\tvalof{\tauiterm}{\tautype}$, an abstracted internal term corresponding to the internal term reflected by $\tauiterm$. Rather than exposing it explicitly, it is wrapped so that it can't be checked against any type other than $\trepof{\tautype}$. A full description of this mechanism (and the others, described above) requires us to introduce the semantics in detail. A draft of the semantics of @$\lambda$ is available as a paper draft\footnote{\url{https://github.com/cyrus-/papers/tree/master/esop14}}.

%
%\subsection{Operators}
%As indicated, operators other than $\lambda$ and \textsf{fix} in @$\lambda$ are associated with type constructors (for reasons we will discuss), so an external term of the form $\eop{Tycon}{op}{\tauidx}{\splat{e}{1}{n}}$ applies the operator $\opvar{op}$ associated with type constructor $\fvar{Tycon}$ to operator index $\tauidx$ and zero or more arguments, written in our syntax and semantics using the shorthand ${\splat{e}{1}{n}}$ for concision. We can see on line X the application of the $\tvar{z}$ and $\tvar{s}$ constructors to form the natural number \texttt{s(s(z))}, for example:
%$$\eop{Nat}{s}{()}{\eop{Nat}{s}{()}{\eop{Nat}{z}{()}{ }}}$$
%
%Operator constructors, like type constructors, are indexed by type-level values. In our example, all the operators are indexed trivially, but we will see examples later where non-trivial operator indices are useful. For the sake of uniformity, we include trivial indices, but in a concrete syntax for such a language, these could be omitted. We will discuss why operator constructors are associated directly with type constructors, rather than existing as standalone entities, when we discuss the scoping mechanism for the representation schema. This coupling also forms the foundation for the higher-level elaboration mechanism that allows us to use more conventional and concise (though still fixed) syntactic idioms, which we intend to explore in this thesis, but have not yet formalized. For now, we simply note that intuitively, most operators are either introduction or elimination forms for a particular type, so we are simply encoding a common idiom. Free-floating operator constructors could be introduced to this calculus with little difficulty.
%
%The declarations of the three operator constructors related to natural numbers are given on lines 3-13 of Fig. \ref{nat-atlam}. Operator constructors are declared by specifying the kind of type-level value they are indexed by and giving an \emph{operator definition}, a type-level function of kind $\kappa \rightarrow \klist{\kDen} \rightarrow \kDen$. The kind $\kDen$ classifies \emph{elaborations}. There are two introductory forms for this kind: \emph{valid elaborations}, $\tden{\tauiterm}{\tautype}$, consist of a reflected internal language term paired with a type assignment, while \emph{error elaborations}, $\terr$, represent a type error (in a full implementation, the error elaboration would contain an error message, as discussed above, but for simplicity in our core calculus, we simply treat all type errors equivalently). Internal terms, $\iota$, like internal types, are reflected as type-level terms of kind $\kITerm$ via the introductory form $\titerm{\iota}$. The form $\iup{\tau}$ interpolates a type-level value of kind $\kITerm$ into an internal term.  Note that there are no elimination forms for the kinds $\kIType$ and $\kITerm$. Composition occurs entirely via interpolation -- syntax trees are never examined directly by extensions in @$\lambda$. 
%
%Operator definitions are called by the compiler to generate an elaboration for operator invocations. In our example, the definition of $\fvar{Nat}.\opvar{z}$ simply checks that there were no arguments (using a simple helper function for working with lists of elaborations, \li{is_empty}, not shown) and if so, provides an elaboration where the internal language integer \li{0} is associated with type $\ttype{Nat}{\tunit}$. If the list is not empty, \li{is_empty} simply returns $\terr$. The other two operator constructors are more verbose, but follow a similar pattern -- elaborations are popped off and broken down into their constituent internal terms and types by helper functions, then a functional program that implements the intended semantics is written to produce an elaboration as a whole. Note that the \li{Rec} operator does not itself bind new variables; it must use the built-in \li{Arrow} type constructor. There is no support in @$\lambda$ for introducing new forms of binding; $\lambda$ is the only available binding operator. Note that application is invoked like other operators: $\eop{Arrow}{ap}{\tunit}{e_1; e_2}$.
%
%
%\subsection{Kind Checking and Normalization}
%
%\subsection{Type Checking}

%\subsubsection{Abstract Internal Typechecking}
%
%\subsubsection{Deabstraction}
%
%\subsection{Safety}
%Type safety of TL. Type safety of IL. Type preserving compilation. Type safety of EL. Strong normalization of TL. Decidability of type checking. Conservativity.
%
%
% %Kind checking  can be expressed judgementally as follows:\todo{compilation judgement}
%
%
%
%
%
%\paragraph{Type Constructors} Type constructors are indexed by A type constructor is declared by providing a name, like $\fvar{Tuple}$, an index kind, like %$\list$
%
%Note that this calculus be compared to the ``core language'' of expressions and types in a language in Standard ML; a module language similar to ML's could be layered on top of this core in a completely standard way. Note also that we do not include polymorphism and recursive types in this calculus (nor can they be defined, because we do not support type binders). This is both to simplify the calculus and because it is known that polymorphism is conservative over simple types \cite{breazu1990polymorphism} (by similar arguments, recursive types are conservative over a simply typed language with fixpoints, which we do include).
%
%The semantics of the language are in the style of the Harper-Stone elaboration semantics for Standard ML: all terms in an external language, $e$, are simultaneously assigned a type and an elaboration to a term, $\iota$, in a fixed typed internal language. The mechanism checks that any generated elaborations are \emph{type preserving} -- that all external terms of a particular type, $\tau$, elaborate to internal terms with a consistent internal type, $\sigma$ -- following the approach taken in the TIL compiler for Standard ML by Morrisett et al. \cite{TIL}. This allows the semantics to be reasoned about compositionally and, when combined with type safety of the internal language, gives us type safety of the extensible external language. 
%
%This design ensures that global properties that the internal language maintains are guaranteed to hold no matter which extensions are introduced. In other words, extension providers cannot implement type systems weaker than the type system of the internal language. Providers are, however, able to implement type systems that maintain invariants stronger than those the internal type system can maintain. The internal language of @$\lambda$ is PCF with simple products and one base type, integers. Because of the availability of the fixpoint operator, the internal language is universal (i.e. Turing-complete).
%
%It is reasonable, if we wish to actually program with such a language, to permit only the addition of deterministic, decidable rules to the semantics. Extension logic is thus implemented in a functional style  (essentially, lifting fragments of the typechecker into the type-level language), rather than extracted from inductive specifications, like those at the beginning of Sec. \ref{att}\footnote{Though we will not explore this further in this thesis, a system for extracting such a function from a declarative specification could potentially be implemented using the techniques in Sec. \ref{aparsing} -- a \emph{kind-specific language}!}. This has the added benefit of giving providers finer-grained control over error reporting. When extracting an implementation from a typical inductive specification, which only specifies ``success conditions'', it is difficult to provide different error messages at different failure points within a single rule.
%%In this section, we will sketch a calculus where new type families and operator families can be introduced by users. The type synthesis and translation logic that implements the desired static and dynamic semantics, respectively, for these is implemented in the type-level language. 
%
%To enable this, the language must allow for the introduction of new type constructors, like \verb|Tuple|, operator constructors, like \verb|NewTuple| and \verb|Prj|, and their associated static and dynamic semantics. Type and operator constructors can no longer be indexed by compiler-internal values in such a design. 
%%We are also be considering only simply-typed languages -- those with a clear phase separation between compile-time and run-time -- so we will not consider dependent types, though extensibility mechanisms for dependently typed languages are an interesting avenue of future work.\footnote{Note that types indexed by expressions are distinct from type names that have metadata expressions associated with them, as in Wyvern. Metadata does not impact type equality, for example.} 
%We will instead use type-level values. This requires supporting more kinds of values in the type-level language, conventionally written $\tau$, than just types, and thus the type-level language must have a non-trivial semantics. The ``types'' of type-level terms are called \emph{kinds} for clarity. Types simply become a particular kind of type-level value (we will call it $\star$) and are introduced uniformly by naming an available type constructor and providing an index of the kind that constructor specifies. For example, the type constructor \verb|Tuple| is indexed by type-level values of kind \verb|list[|$\star$\verb|]| so a 3-tuple type might be written \verb|Tuple[cons(nat, cons(string, cons(nat, nil)))]|. Note that we do not identify type constructors and type-level functions, as in some kind systems (\verb|Tuple| is \emph{not} itself of kind \verb|list[|$\star$\verb|] |$\rightarrow$\verb| |$\star$). One can wrap constructor application with a type-level function to achieve the same effect (and avoid an awkward canonical forms lemma and beta-reduction rules).
%
%Operator constructors are similarly indexed by type-level values. For example, the projection operator written \verb|#2| in a language like Standard ML might be written \verb|prj[2]| in our work. Here, \verb|prj| is an operator constructor indexed by a type-level natural number. 
%To define an operator, we must also provide it with a static and dynamic semantics. We will, as in our work on Wyvern and consistent with our exposition above, take an approach similar to Harper and Stone in their definition of Standard ML where the dynamic semantics of external language terms are given by translation into a  typed internal language \cite{Harper00atype-theoretic}. Operator definitions must work with types, type indices and operator indices, all of which are type-level values, so we will argue that it is sensible to give these elaborations as type-level functions (appropriately constrained, as we will show). 
%
%-----
%
%To give a first example, the code in Fig. \ref{nat-atlam} (written using concrete syntax) shows how to introduce a new type constructor, $\fvar{Nat}$, indexed trivially (i.e. by a type-level value of kind 1, also known as unit). Types themselves are type-level values of kind $\kTypeBlur$ and are introduced by naming a type constructor and providing an index of the appropriate kind: $\ttype{Nat}{\tunit}$. To ensure that type equality is decidable, type constructors can only be indexed by values of a kind for which equality is ``trivially'' decidable (i.e. where semantic equality coincides with syntactic equality). This means that types themselves, as well as type-level data structures containing values of kinds with this property, can be used, while type-level functions cannot. We leave a richer consideration of type equality to future work for now.
%
%To ensure that compilation is type-preserving, as described above, each type must have an internal type, called its \emph{representation}, associated with it. The representation of a type might depend on its index, so each type constructor must declare a type-level function, called its \emph{representation schema}, that returns an internal type given a type's index. An internal type, $\sigma$, is reflected as a type-level term of kind $\kIType$ via the introductory form $\blacktriangledown(\sigma)$. Because $\fvar{Nat}$ is indexed trivially, it simply returns the reflected form of the internal type $\mathbb{Z}$. %The sort $\sigma$ also contains forms $\trepof{\tau}$ and $\dup{\tau}$. The former allows one to refer, abstractly, to the representation of the type $\tau$, while the latter enables the interpolatation of a type-level value of kind $\kIType$.
%
%%Two additional forms, $\tvalof{\tauiterm}{\tautype}$ and $\iup{\tau}$, will be discussed below. These are erased by the end of the elaboration process.
%
%--
%

\subsection{Expressiveness}
The use cases permitted by this mechanism are not simply subsumed by a module system with support for abstract types and functors. Indeed, such a module system is an orthogonal concern in that it must sit atop a core type system. More particularly, the distinction is the fundamental one between functions and operators. Functions cannot examine type indices at compile-time to determine a return type and implementation (or generate static error messages), as operators are able to do. It is encouraging that there are clear parallels between module systems with abstract types and type constructors with abstract schemas, and any full-scale language design based on this mechanism should certainly provide both mechanisms (with the former used in most cases). We plan to investigate this relationship more thoroughly, showing examples where abstract types are more clearly unsuitable, in the remainder of this work. 


\subsubsection{Remaining Tasks and Timeline}
We are actively working on this mechanism and plan to submit a paper to ICFP 2014 on Mar. 1 based on a submission to ESOP 2014 that was not accepted. The following key tasks, other than clarifying the writing, remain to be completed:

\begin{enumerate}
\item Reviewers of the previous submission to ESOP found a problem with our treatment of internal type variables that needs to be corrected by threading a context through the type-level evaluation semantics.
\item We must more rigorously state and prove the key lemmas and theorems related to type preservation, type safety, decidability and conservativity. 
\item We must develop examples that are more interesting than natural numbers. In particular, some interesting form of labeled product type that SML doesn't have (e.g. a record with prototypic dispatch) as well as sum types would be interesting.
\item We must consider whether properties related to termination can be conserved by this mechanism, because a fixpoint can be used to introduce a non-terminating expression of any type. This may be possible by being careful about how deabstraction interacts with fixpoints.
\item We must more clearly connect this work to the Harper-Stone semantics and other related work, and clarify  the relationship to abstract types.
\item We must show how to add additional syntactic forms to the external language that defer to the generic operator invocation form in a type-directed manner. This connects the theory to the work on Ace, described below. 
\end{enumerate}

\section{Ace}\label{ace}
 Using a type-level language that provides few strictly enforced semantic guarantees makes giving strong metatheoretic guarantees about Ace more difficult. Ace does include some checks similar to those in @$\lambda$ that catch accidental problems but they can be intentionally subverted by untrusted extension providers using the dynamic metaprogramming features of Python. This is sufficient for programmers who use a type system to help catch errors. We conjecture, but do not plan to show in detail, that it would be possible to use Ace to bootstrap a statically-typed subset of Python that, if used to compile extension logic, would be sufficient to provide theoretical guarantees comparable to @$\lambda$ (building, perhaps, on recent work that resulted in a mechanized semantics for Python \cite{pysem}).

\subsection{From Extensible Compilers to Extensible Languages}\label{evolution}
The monolithic character of most programming languages is reflected in, and perhaps influenced by, the most common constructs used for implementing programming languages.
Let us consider an implementation of the STLC. 
A compiler written using a functional language will invariably represent the primitive type  and operator constructors using {closed} recursive sums. 
For example, a simple implementation in Standard ML might be based around these datatypes (where variables, binders and operator application to arguments are handled separately, not shown)\footnote{This is essentially the approach that we took in the infrastructure for the coding assignments in 15-312.}:
\begin{lstlisting}
  datatype Ty = Arrow of Ty * Ty
  datatype Op = Lam of Ty | Ap 
\end{lstlisting}

%The compiler front-end, which consists of a typechecker and translator to a suitable intermediate language,    proceeds by exhaustive case analysis over the constructors of \verb|Op|.

In a class-based object-oriented implementation of the STLC, we might instead encode type and operator constructors as subclasses of abstract classes \verb|Ty| and \verb|Op|. If typechecking and translation proceed by the common \emph{visitor pattern}, dispatching against a fixed set of {known} subclasses of \verb|Op|, then we encounter a similar issue: everything is defined at once. 

This issue was first discussed by Reynolds \cite{Reynolds75}. A number of mechanisms have since been proposed that allow new cases to be added to data types and the functions that operate over them in a modular manner. 
In functional languages, we might turn to \emph{open datatypes and open functions} \cite{conf/ppdp/LohH06}. For example, we might add a new type constructor, \verb|Tuple|, indexed by a list of types, and two new operator constructors: an introductory form, \verb|NewTuple|, indexed trivially, and an elimination form, \verb|Prj|, indexed by a natural number (hypothetical syntax):
\begin{lstlisting}
  newcon Tuple of Ty list extends Ty
  newcon NewTuple extends Op
  newcon Prj of nat extends Op
\end{lstlisting}

The logic for functionality like typechecking and translation, if organized around open functions, can then be implemented for only these new cases. For example, the \verb|optype| function that synthesizes a type for an operator invocation given a  a list of argument types might be extended to support the new constructor \verb|Prj| like so:
\begin{lstlisting}
  optype (Prj(n), [t]) = case t of 
      Tuple(ts) => (case List.nth n ts of 
          Some t' => t'
        | _ => raise IndexError("<projection index out of bounds>"))
    | _ => raise TypeError("<tuple expected>")
  optype (ctx, Prj(n), _) = raise ArityError("<expected 1 argument>")
\end{lstlisting}

Using a class-based mechanism, we might dispense with the visitor pattern and instead invert control to abstract methods of \verb|Op|. 

\begin{lstlisting}[language=Java]
class Prj extends Op {
  Prj(Nat n) { this.n = n; }
  Nat n;
  Ty optype(List<Ty> args) {
    if (args.length != 1) throw new ArityError("<expected 1 argument>")
    Ty t = args[0];
    if (t instanceof Tuple) {
      Tuple t = (Tuple) t;
      try {
        return t.idx[this.n]
      } catch OutOfBoundsException(e) {
        throw new IndexError("<projection index out of bounds>");
      }
    } else throw new TypeError("<tuple expected>");
  }
}
\end{lstlisting} 

If we allowed programmers to load definitions like these into our compiler by, e.g., a pragma declaration, we will still have only succeeded in creating an extensible compiler. We have not created an extensible programming language because other compilers for the same language will not necessarily support the same extensions. These mechanisms are not truly \emph{language-integrated}, so if our newly-introduced constructs are exposed at a library's  interface boundary, clients of that library who are using different compilers face the same problems with client compatibility that those using different languages face (as described in Sec. \ref{external-approaches}). Worse still, these mechanisms allow the  metatheoretic properties of the language to be weakened by extensions and orthogonality is not guaranteed (in several senses of the word, which we will return to). %That is, {extending a language by extending a single compiler for it is morally equivalent to creating a new language}. 
Nevertheless, these two strategies give us the conceptual seeds for the approaches that we will take in Secs. \ref{atlam} and \ref{ace}. 
In both cases, we introduce into the language a mechanism similar to, but more constrained and specialized  than, the two above. Extensions become library imports, rather than compiler-specific pragmas or flags.\footnote{For unfortunate languages where a canonical implementation functions as the specification (e.g. Haskell and Scala), the mechanisms we introduce can be left in the compiler. Our metatheoretic advances apply also to this design, but we do not consider it further in this thesis.}


% We will show how, by integrating typed compilation techniques and a form of data hiding with similarities to abstract types into the semantics of operators, we can achieve type safety and conservativity of the language as a whole, despite this level of extensibility. Decidability is achieved by ensuring that the type-level language is total. Type equality is made decidable in a simple manner: by preventing type-level functions from appearing in type indices.

%The design described above suggests we may now need to add another layer to our language, above the type-level language (conventionally, $\tau$) and expression language (conventionally, $e$), where extensions are implemented. In fact, we will show that \textbf{a natural place for type system extensions is within the type-level language}. The intuition  is that extensions to a statically typed language's semantics will need to manipulate types as values at compile-time. Many languages already allow users to write type-level functions for various reasons, effectively supporting this notion of types as values at compile-time. The type-level language is often constrained by its own type system, where the types of type-level terms are called \emph{kinds} for clarity,  that prevents type-level functions from causing problems during compilation. The kind system is often more constraining than the type system is (e.g. it might permit only terminating functions, to ensure that typechecking is decidable). This is precisely the structure that a distinct extension layer would have, and so we will show that is quite natural to unify the two.

\lstset{
  language=Python,
  showstringspaces=false,
  formfeed=\newpage,
  tabsize=4,
  commentstyle=\itshape,
  basicstyle=\ttfamily\scriptsize,
  morekeywords={lambda, self, assert, as},
  numbers=left,
  numberstyle=\scriptsize\color{light-gray}\textsf,
  xleftmargin=2em,
  stringstyle=\color{mauve}
}
\lstdefinestyle{Bash}{
    language={}, 
    numbers=left,
    numberstyle=\scriptsize\color{light-gray}\textsf,
    moredelim=**[is][\color{blue}\bf\ttfamily]{`}{`}
}
\lstdefinestyle{OpenCL}{
	language=C++,
	morekeywords={kernel, __kernel, global, __global, size_t, get_global_id, sin, printf, int2}
}

%\usepackage{float}
%\floatstyle{ruled}
%\newfloat{codelisting}{tp}{lop}
%\floatname{codelisting}{Listing}
%\setlength{\floatsep}{10pt}
%\setlength{\textfloatsep}{10pt}



The key choices that a language must make when supporting the mechanism described in the previous section are:
\begin{enumerate}
\item What is the semantics of the type-level language?
\item What is the syntax of the external language, and how should the syntactic forms dispatch to  operator definitions?
\item What is the semantics of the internal language?
\end{enumerate}

In @$\lambda$, the type-level language was a form of the simply-typed lambda calculus with a few common data types, and the internal language was a variant of PCF. The external language currently only includes direct dispatch by naming an operator explicitly. This is useful for the purposes of clarifying the foundations of our work.

In this section, we wish to explore a different point in this design space, with an eye toward practicality and expressiveness. In particular, we want to show how to implement {the extension mechanism itself} as a library within an existing, widely used language, solving a \emph{bootstrapping problem} that has prevented  other work on extensibility from being effective as a means to bring more research into practice. This language, called Ace, makes the following choices:

\begin{enumerate}
\item Python is used as the type-level language (and more generally, as a compile-time metalanguage).
\item Python's syntax is used for the external language. We will describe the dispatch protocol below.
\item The internal language can be user-defined, rather than being pre-defined as in @$\lambda$.
\end{enumerate}

The choice of Python as the host language presents several challenges because we are attempting to embed an extensible static type system within a uni-typed (a.k.a. dynamically typed) language, without modifying its syntax in any way. We show that by leveraging Python's support for function quotations and by using its class-based object system to encode type constructors, we are able to accomplish this goal. Then, with a flexible statically-typed language under the control of libraries, we implement a variety of statically-typed abstractions, including common functional abstractions (e.g. inductive datatypes with pattern matching), low-level parallel abstractions (all of the OpenCL programming language), object systems and domain-specific abstractions (e.g. regular expression types, as described in the introduction). Ace can be used both as a standalone language and as a staged compilation environment from within Python.

\subsubsection{Language Design and Usage}\label{usage}
Listing \ref{map} shows an example of an Ace file. As promised, the top level of an Ace file is written directly in Python, requiring no modifications to the language (versions 2.6+ or 3.3+) nor features specific to CPython (so Ace supports alternative implementations like Jython, IronPython and PyPy). This choice pays  dividends on line 1: Ace's package system is Python's package system, so Python's build tools (e.g. \verb|pip|) and package repostories (e.g. \verb|PyPI|) are directly available for distributing Ace libraries. 

The top-level statements in an Ace file, like the \verb|print| statement on line 10, are executed at compile-time. That is, Python serves as the \emph{compile-time metalanguage} of Ace. %For readers familiar with C/C++, Python can be thought of as serving a role similar to (but more general than) its preprocessor and template system (as we will see).
Functions containing run-time behavior, like \verb|map|, are annotated as Ace functions and are then governed by a semantics that differs substantially from Python's (in ways that we will describe below). But Ace functions share Python's syntax. As a consequence, users of Ace benefit from an ecosystem of well-developed tools that work with Python syntax, including parsers, code highlighters, editor modes, style checkers and documentation generators. 

\subsubsection{OpenCL as an Active Library}
The code in this section uses \verb|clx|, an example of an library that implements the semantics of the OpenCL programming language, and extends it with some additional useful types, using Ace. Ace itself has no built-in support for OpenCL.

To briefly review, OpenCL provides a data-parallel SPMD programming model where developers define functions, called {\em kernels}, for execution across thousands of threads on \emph{compute devices} like GPUs or multi-core CPUs \cite{opencl11}. Each thread has access to a unique index, called its \emph{global ID}. Kernel code is written in the OpenCL kernel language, a somewhat simplified variant of C99 extended with some new primitive types and operators, which we will describe as needed in our examples below.

\subsubsection{Generic Functions}\label{genfn}
\begin{codelisting}
\lstinputlisting{../ace-pldi14/listing3.py}
\caption{[\texttt{listing\ref{map}.py}] A generic imperative data-parallel higher-order map function targeting OpenCL.}
\label{map}
\end{codelisting}
%\begin{codelisting}
%\lstinputlisting[commentstyle=\color{mauve}]{../ace-pldi14/listing7.py}
%\caption{[\texttt{listing\ref{metaprogramming}.py}] Metaprogramming with Ace, showing how to construct generic functions from abstract syntax trees.}
%\label{metaprogramming}
%\end{codelisting}
%\begin{codelisting}
%\begin{lstlisting}[style=Bash]
%$ `acec listing3.py`
%Hello, compile-time world!
%[ace] TypeError in listing1.py (line 6, col 28): 
%      'GenericFnType(negate)' does not support [].
%[acec] listing3.cl successfully generated.
%\end{lstlisting}
%\caption{Compiling \texttt{listing\ref{compscript}.py} using the \texttt{acec} compiler.}
%\label{mapc}
%\end{codelisting}
%\begin{codelisting}
%\lstinputlisting[style=OpenCL]{listing5.cl}
%\caption{[\texttt{listing\ref{compscript}.cl}] The OpenCL file generated by Listing \ref{mapc}.}
%\label{mapout}
%\end{codelisting}

Lines 3-4 introduce \verb|map|, an Ace function of three arguments that is governed by the \emph{active base} referred to by \verb|clx.base| and targets the \emph{active target} referred to by \verb|clx.opencl|. The active target determines which language the function will compile to (here, the OpenCL kernel language) and mediates code generation. The active target plays an analagous role to the internal language of @$\lambda$.

The body of this function, on lines 5-8, does not have Python's semantics. Instead, it will be governed by the active base together with any \emph{active types} used within it. No  types have yet been assigned, however. Because our type system is extensible, the code inside could be meaningful for many different assignments of types to the arguments (a form of \emph{ad hoc polymorphism}). We call functions awaiting types, like \verb|map|,  \emph{generic functions}. Once types have been assigned, they are called \emph{concrete functions}.

Generic functions are represented at compile-time as instances of \verb|ace.GenericFn| and consist of an abstract syntax tree, an {active base}, an {active target} and a read-only copy of the Python environment that they were defined within. The purpose of the \emph{decorator} on line 3 is to replace the Python function on lines 4-8 with an Ace generic function having the same syntax tree and environment and the provided active base and active target. 
A decorator in Python is simply syntactic sugar that applies another function directly to the function  being decorated \cite{python}. In other words, line 3 could be replaced by the following  statement on line 9: \verb|map = ace.fn(clx.base, clx.opencl)(map)|.
Ace extracts the abstract syntax tree for \verb|map| using the Python standard  library packages  \verb|inspect| (to retrieve its source code) and \verb|ast| (to parse it into a syntax tree). The ability to extract a function's syntax tree and inspect its closure directly are the two key ingredients for implementing a mechanism like this as a library within another language.

\subsubsection{Concrete Functions and Explicit Compilation}
\begin{codelisting}
\begin{lstlisting}
import listing1, ace, examples.clx as clx

@ace.fn(clx.base, clx.opencl)
def negate(x):
  return -x

T1 = clx.Ptr(clx.global_, clx.float)
T2 = clx.Ptr(clx.global_, clx.Cplx(clx.int))
TF = negate.ace_type

map_neg_f32 = listing1.map[[T1, T1, TF]]
map_neg_ci32 = listing1.map[[T2, T2, TF]]
\end{lstlisting}
%\lstinputlisting{../ace-pldi14/listing4.py}
\caption{[\texttt{listing2.py}] The generic \texttt{map} function compiled to map the \texttt{negate} function over two  types of input.}
\label{compscript}
\end{codelisting}

To compile a generic function to a particular \emph{concrete function}, a type must be provided for each argument, and typechecking and elaboration must then succeed. Listing \ref{compscript} shows how to explicitly provide type assignments to \verb|map| using the subscript operator (implemented using Python's operator overloading mechanism). We do so two times in Listing \ref{compscript}, on lines 11 and 12. Here, \verb|T1|, \verb|T2|, \verb|TF|, \verb|clx.float|, \verb|clx.int| and \verb|negate.ace_type| are types and \verb|clx.Ptr| and \verb|clx.Cplx| are type constructors. We will discuss these in the next section.
 
Concrete functions like \verb|map_neg_f32| and \verb|map_neg_ci32| are instances of \verb|ace.ConcreteFn|. They consist of a \emph{typed} abstract syntax tree, an elaboration into the target language and a reference to the originating generic function. %The typing and translation process is mediated by the logic in the active base, types and target that have been provided, as we will describe in more detail below. 

To produce an output file from an Ace ``compilation script'' like \verb|listing|\texttt{\ref{compscript}}\verb|.py|, the command \verb|acec| can be invoked from the shell, as shown in Listing \ref{mapc}. The result of compilation is the OpenCL file shown in Listing \ref{mapout}. The \verb|acec| compiler (a simple Python script) operates in two stages:
\begin{enumerate}
\item Executes the provided Python file (\verb|listing3.py|).
\item Extracts the elaborations from concrete functions and other top-level constructs that define elaborations (e.g. types requiring declarations) in the final Python environment.  This may produce one or more files, depending on which active targets were used (here, just \verb|listing3.cl|, but a web framework built upon Ace might produce separate HTML, CSS and JavaScript files).
\end{enumerate}

%In this case, stage 1 results in the output on lines \ref{mapc}.2-\ref{mapc}.4. The type error printed on lines \ref{mapc}.3-\ref{mapc}.4 will be explained in the next section. The compiler then enters stage 2 and concludes with the message on line \ref{mapc}.5 to indicate that one file was generated. This file is shown in Listing \ref{mapout} and can be used by any programs that consume OpenCL code (e.g. a C program that invokes the generated kernels via the OpenCL host API). 
We will show in the thesis, but omit in this proposal, that for targets with Python bindings, such as OpenCL, CUDA, C, Java or Python itself, generic functions can be executed directly, without any of the explicit compilation steps in Listings \ref{compscript} and \ref{mapc}. This represents a form of staged compilation.  In this setting, the dynamic type of a Python value determines, at the point of invocation, a static type assignment for the argument of an Ace generic function. 

\subsubsection{Types}
\begin{codelisting}
\begin{lstlisting}[style=Bash]
> `acec listing2.py`
Hello, compile-time world!
[acec] listing2.cl successfully generated.
\end{lstlisting}
%$ `acec listing3.py`
%Hello, compile-time world!
%[ace] TypeError in listing1.py (line 6, col 28): 
%      'GenericFnType(negate)' does not support [].
%[acec] listing3.cl successfully generated.
\caption{Compiling \texttt{listing\ref{compscript}.py} using the \texttt{acec} compiler.}
\label{mapc}
\end{codelisting}
\begin{codelisting}
\lstinputlisting[style=OpenCL]{../ace-pldi14/listing5.cl}
\caption{[\texttt{listing\ref{compscript}.cl}] The OpenCL file generated by Listing \ref{mapc}.}
\label{mapout}
\end{codelisting}
Lines 7-9 of Listing \ref{compscript} construct the types used to generate concrete functions from the generic function \verb|map| on lines 11 and 12. In Ace, types are themselves values that can be manipulated at compile-time. Python is thus Ace's type-level language. %This stands in contrast to other contemporary languages, where user-defined types (e.g. datatypes, classes, structs) are written declaratively at compile-time but cannot be constructed, inspected or passed around programmatically. 
More specifically, types are instances of a Python class that implements the \verb|ace.ActiveType| interface. Implementing this interface is analagous to defining a new type constructor in @$\lambda$. 

As Python values, types can be assigned to variables when convenient (removing the need for  facilities like \verb|typedef| in C or \verb|type| in Haskell). Types, like all compile-time objects derived from Ace base classes, do not have visible state and operate in a referentially transparent manner (by constructor memoization, which we do not detail here).% These types are all implemented in the \verb|clx| library imported on line 1, none are built into Ace itself.

The type named \verb|T1| on line 7 directly implements the logic of the underlying OpenCL type \verb|global float*|: a pointer to a 32-bit floating point number stored in the compute device's global memory (one of four address spaces defined by OpenCL \cite{opencl11}). It is constructed by applying \verb|clx.Ptr|, which is an Ace type constructor corresponding to pointer types, to a value representing the  address space, \verb|clx.global_|, and the type being pointed to. That type, \verb|clx.float|, is in turn the \verb|clx| type corresponding to \verb|float| in OpenCL (which, unlike in C99, is always 32 bits). 
The \verb|clx| library contains a full implementation of the OpenCL type system (including  complexities, like promotions, inherited from C99).
Ace is \emph{unopinionated} about issues like memory safety and the wisdom of such promotions. We will discuss how to implement, as libraries, abstractions that are higher-level than raw pointers, or simpler numeric types, but Ace does not prevent users from choosing a low level of abstraction or ``interesting'' semantics if the need arises (e.g. for compatibility with existing libraries). We also note that we are being more verbose than necessary for the sake of pedagogy. The \verb|clx| library includes more concise shorthand for OpenCL's types: \verb|T1| is equal to \verb|clx.gp(clx.f32)|. %Similarly, the decorators in Listings \ref{map} and \ref{metaprogramming} could have been written \verb|clx.cl_fn|.\todo{move this back there probably}

The type \verb|T2| on line 8 is a pointer to a \emph{complex integer} in global memory. It does not correspond directly to a type in OpenCL, because OpenCL does not include primitive support for complex numbers. Instead, the type constructor \verb|clx.Cplx| defines the necessary logic for typechecking operations on complex numbers and elaborating them to OpenCL. This constructor is parameterized by the numeric type that should be used for the real and imaginary parts, here \verb|clx.int|, which corresponds to 32-bit OpenCL integers. Arithmetic operations with other complex numbers, as well as with plain numeric types (treated as if their imaginary part was zero), are supported. When targeting OpenCL, Ace expressions assigned type \verb|clx.Cplx(clx.int)| are compiled to OpenCL expressions of type \verb|int2|, a  \emph{vector type} of two 32-bit integers (a type that itself is not inherited from C99). This can be observed in several places on lines \ref{mapout}.14-\ref{mapout}.21. This choice is merely an implementation detail that can be kept private to \verb|clx|. An Ace value of type \verb|clx.int2| (that is, an actual OpenCL vector) \emph{cannot} be used when a \verb|clx.Cplx(clx.int)| is expected (and attempting to do so will result in a static type error).

The type \verb|TF| on line 9 is extracted from the generic function \verb|negate| constructed in Listing \ref{compscript}. Generic functions, according to Sec. \ref{genfn}, have not yet had a type assigned to them, so it may seem perplexing that we are nevertheless extracting a type from \verb|negate|. Although a conventional arrow type cannot be assigned to \verb|negate|, we can give it a \emph{singleton type}: a type that simply means ``this expression is the \emph{particular} generic function \verb|negate|''. This type could also have been explicitly written as \verb|ace.GenericFnType(listing2.negate)|. During typechecking and translation of \verb|map_neg_f32| and \verb|map_neg_ci32|, the call to \verb|f| on line 6 of Listing \ref{map} uses the type of the argument to generate a concrete function from the generic function that inhabits the singleton type of \verb|f| (\verb|negate| in both cases shown). This is why there are two versions of \verb|negate| in the output in Listing \ref{mapout}. In other words, types \emph{propagate} into generic functions -- we didn't need to compile \verb|negate| explicitly. In effect, this scheme enables higher-order functions even when targeting languages, like OpenCL, that have no support for higher-order functions (OpenCL, unlike C99, does not support function pointers). Interestingly, because they have a singleton type, they are higher-order but not first-class functions. That is, the type system would prevent you from creating a heterogeneous list of generic functions. Concrete functions, on the other hand, can be given both a singleton type and a true function type. For example, \verb|listing2.negate[[clx.int]]| could be given type \verb|ace.Arrow(clx.int, clx.int)|. The base determines how to convert the Ace arrow type to an arrow type in the target language (e.g. a function pointer for C99, or an integer that indexes into a jump table constructed from knowledge of available functions of the appropriate type in OpenCL).

\subsubsection{Extensibility}
The core of Ace consists of about 1500 lines of Python code implementing its primary concepts: generic functions, concrete functions, active types, active bases and active targets.  The latter three comprise Ace's extension mechanism. Extensions provide semantics to, and govern the compilation of, Ace functions, rather than logic in Ace's core. %Indeed, the name ``Ace'' might itself be an acronym: it is an ``active compilation environment''.

Active types are the primary means for extending Ace with new abstractions. An active type, as mentioned previously, is an instance of a class implementing the \verb|ace.ActiveType| interface. Listing \ref{cplx} shows an example of such a class: the \verb|clx.Cplx| class used in Listing \ref{compscript}, which implements the logic of complex numbers. The constructor takes as a parameter any numeric type in \verb|clx| (line \ref{cplx}.2). 

\subsubsection{Dispatch Protocol}
\begin{codelisting}
\lstinputlisting{../ace-pldi14/listing9.py}
\caption{\texttt{[}in \texttt{examples/clx.py]} The active type family \texttt{Ptr} implements the semantics of OpenCL pointer types.}
\label{cplx}
\end{codelisting}
In a compiler for a monolithic language, there would be a \emph{syntax-directed} protocol governing typechecking and translation. In a compiler written in a functional language, for example, one would declare datatypes that captured all forms of types and expressions, and the typechecker would perform exhaustive case analysis over the expression forms. That is, all the semantics are implemented in one place. The visitor pattern typically used in object-oriented languages implements essentially the same protocol. This does not work for an extensible language because new cases and logic need to be added \emph{modularly}, in a safely composable manner.

Instead of taking a syntax-directed approach, the Ace compiler's typechecking and translation phases take a \emph{type-directed approach}. When encountering a compound term (e.g. \verb|e[e1]|), the compiler defers control over typechecking and translation to the active type of a designated subexpression (e.g. \verb|e|) determined by Ace's fixed \emph{dispatch protocol}. Below are examples of the choices made in Ace.
\begin{itemize}
\item Responsibility over {\bf attribute access} (\texttt{e.attr}), {\bf subscripting} (\texttt{e[e1]}),  \textbf{calls} (\verb|e(e1, ..., en)|) and {\bf unary operations} (e.g. \verb|-e|) is handed to the type recursively assigned to \texttt{e}.
\item Responsibility over {\bf binary operations} (e.g. \verb|e1 + e2|) is first handed to the type assigned to the left operand. If it indicates a type error, the type assigned to the right operand is handed responsibility, via a different method call. {Note that this operates like the corresponding rule in Python's \emph{dynamic} operator overloading mechanism.}
\item Responsibility over \textbf{constructor calls} (\verb|[t](e1, ..., en)|), where \verb|t| is a \emph{compile-time Python expression} evaluating to an active type, is handed to that type, by using Python's eval functionality to evaluate t. If \verb|t| evaluates to a type constructor, like \verb|clx.Cplx|, the type is first generated via a class method, as discussed below.
%\item Responsibility over {\bf simple assignment statements}, is handed to the type of the variable on the left (which, as we will see below, is determined by the active base). If this type does not provide special assignment semantics, the base must handle it.%Destructuring assignment is also supported by a somewhat more complex protocol.\todo{clarify}
\end{itemize}

\subsubsection{Typechecking}
When typechecking a compound expression or statement, the Ace compiler temporarily hands control to the object selected by the dispatch protocol by calling the method \verb|type_|$X$, where $X$ is  the name of the syntactic form, taken from the Python grammar \cite{python} (appended with a suffix in some cases). 

For example, if \verb|c| is a complex number, then the operations \verb|c.ni| and \verb|c.i| extract its non-imaginary and imaginary components, respectively. These expressions are of the form \verb|Attribute|, so the typechecker calls \verb|type_Attribute| (line 7).
This method receives the compilation context, \verb|context|, and the abstract syntax tree of the expression, \verb|node|, and must return a type assignment for it, or raise an \verb|ace.TypeError| if there is an error. In this case, a type assignment is possible if the attribute name is either \verb|"ni"| or \verb|"i"|, and an error is raised otherwise (lines 8-10). We note that error messages are an important and sometimes overlooked facet of {ease-of-use} \cite{marceau2011measuring}. A common frustration with using general-purpose abstraction mechanisms to encode an abstraction is that they can produce  verbose and cryptic error messages that reflect the implementation details instead of the semantics. Ace supports custom error messages.%Active types permit direct encoding of semantics and customization of error messages.

Complex numbers also support binary arithmetic operations partnered with both other complex numbers and with non-complex numbers, treating them as if their imaginary component is zero. The typechecking rules for this logic is implemented on lines 17-29. Arithmetic operations are usually symmetric, so the dispatch protocol checks the types of both subexpressions for support. To ensure that the semantics remain deterministic in the case that both types support the binary operation, Ace asks the left first (via \verb|type_BinOp_left|), asking the right (via \verb|type_BinOp_right|) only if the left indicates an error. In either position, our implementation begins by recursively assigning a type to the other operand in the current context via the \verb|context.type| method (line 24). If supported, it applies the C99 rules for arithmetic operations to determine the resulting type (via \verb|c99_binop_t|, not shown). 

Finally, a complex number can be constructed inside an Ace function using Ace's special constructor form: \verb|[clx.Cplx](3,4)| represents $3+4i$, for example. The term within the braces is evaluated at \emph{compile-time}. Because \verb|clx.Cplx| evaluates not to an active type, but to a class, this form is assigned a type by handing control to the class object via the \emph{class method} \verb|type_New|. It operates as expected, extracting the types of the two arguments to construct an appropriate complex number type (lines 50-57), raising a type error if the arguments cannot be promoted to a common type according to the rules of C99 or if two arguments were not provided.% (an exercise for the reader: modify this method to also allow a single argument for when the imaginary part is 0). 

%Note that in each example, subexpressions are not \emph{automatically} assigned types. Types must be requested from the context. This is useful because it allows types to reinterpret subexpressions of terms they control, to supporting more complex constructs (e.g. pattern matching on functional datatypes, see Sec. \ref{datatypes}).

\subsubsection{Translation}
Once typechecking a method is complete, the compiler enters the translation phase, where terms in the target language are generated from Ace terms. Terms in the target language are generated by calling methods of the \emph{active target} governing the function being compiled. The translation phase operates similarly to typechecking, using the dispatch protocol to invoke methods named \verb|trans_|$X$. These methods have access to the context and node just as during typechecking, as well as the active target (named \verb|target| here).

As seen in Listing \ref{mapout}, we are implementing complex numbers internally using OpenCL vector types, like \verb|int2|. Let us look first at \verb|trans_New| on lines 54-60, where new complex numbers are translated to vector literals by invoking \verb|target.VecLit|. This will ultimately  generate the necessary OpenCL code, as a string, to complete compilation (these strings are not directly manipulated by extensions, however, to avoid problems with, e.g. precedence). For it to be possible to reason compositionally about the correctness of compilation, all complex numbers must translate to terms in the target language that have a consistent target type. The \verb|trans_type| method of the \verb|ace.ActiveType| associates a type in the target language, here a vector type like \verb|int2|, with the active type. Ace supports a mode where this \emph{representational consistency} is dynamically checked during compilation (requiring that the active target know how to assign types to terms in the target language, which can be done  for our OpenCL target as of this writing).

The translation methods for attributes (line 12) and binary operations  (line 31) proceed in a straightforward manner. The context provides a method, \verb|trans|, for recursively determining the translation of subexpressions as needed. Of note is that the translation methods can assume that typechecking succeeded. For example, the implementation of \verb|trans_Attribute| assumes that if \verb|node.attr| is not \verb|'ni'| then it must have been \verb|'i'| on line 14, consistent with the implementation of \verb|type_Attribute| above it. Typechecking and translation are separated into two methods to emphasize that typechecking is not target-dependent, and to allow for more advanced uses, like type refinements and hypothetical typing judgements, that we do not describe here.

\subsubsection{Active Bases}\label{abases}
Each generic function is associated with an active base, which is an object implementing the \verb|ace.ActiveBase| interface. The active base specifies the \emph{base semantics} of that function. It controls the semantics of statements and expressions that do not have a clear ``primary'' subexpression for the dispatch protocol to defer to. A base is handed control over typechecking of statements and expressions in the same way as active types: via \verb|type_|$X$ and \verb|trans_|$X$ methods.  Each function can have a different base.

Literals are the most prominent form given their semantics by an active base. Our example active base, \verb|clx.base|, assigns integer literals the type \verb|clx.int32| while floating point literals have type \verb|clx.double|, consistent with the semantics of OpenCL and C99. The \verb|clx.base| object is an instance of \verb|clx.Base| that is pre-constructed for convenience. Alternative bases can be generated via different constructor arguments. For example, \verb|clx.Base(flit_t=clx.float)| is a base semantics where floating point literals have type \verb|clx.float|. This is useful because some OpenCL devices do not support double-precision numbers, or impose a significant performance penalty for their use. Indeed, to even use the \verb|double| type in OpenCL, a flag must be toggled by a \verb|#pragma| (The OpenCL target we have implemented inserts this automatically when needed, along with some other flags, and annotations like \verb|kernel|).  Similarly, for some applications, avoiding accidental overflows is more important than performance.  Infinite-precision integers can be implemented as an active type (not shown) and the base can be asked to use it for numeric literals. In all cases, this occurs by inserting a constructor call form, as described above.% the constructor call form, described above, around integers to support this.

The base also initializes the context and governs the semantics of variables and assignment. This can also permit it to allow for the use of ``built-in'' operators that were not explicitly imported. The semantics of \verb|get_global_id| and \verb|printf| are provided by \verb|clx.base|. In fact, the base provides extended versions of them (the former permits multi-dimensional global IDs, the latter supports typechecking format strings).A base which does not provide them as built-ins (requiring that they be imported explicitly) can be generated by passing the \verb|primitives=False| flag.

\subsubsection{Expressiveness}\label{examples}
Thus far, we have focused mainly on the OpenCL target and shown examples of fairly low-level active types: those that implement OpenCL's primitives (e.g. \verb|clx.Ptr|) and extend them in simple but convenient ways (e.g. \verb|clx.Cplx|).

But Ace has proven useful for more than low-level tasks like programming a GPU with OpenCL. We now describe some interesting extensions that implement the semantics of primitives drawn from a range of different language paradigms, to justify our claim that these mechanisms are highly expressive. 

\subsubsection{Growing a Statically-Typed Python Inside Ace}
Ace comes with a target, base and type implementing Python itself: \verb|ace.python.python|, \verb|ace.python.base| and \verb|ace.python.dyn|. These can be supplemented by additional active types and used as the foundation for writing statically-typed Python functions. These functions can either be compiled ahead-of-time to an untyped Python file for execution, or be immediately executed with just-in-time compilation, just like the OpenCL examples (not shown in this proposal).% Many of the examples in the next section support this target in addition to the OpenCL target we have focused on thus far.

\subsubsection{Recursive Labeled Sums}
\begin{codelisting}
\lstinputlisting{../ace-pldi14/datatypes_t.py}
\caption{\texttt{[datatypes\_t.py]} An example using statically-typed functional datatypes.}
\label{datatypest}
\end{codelisting}
\begin{codelisting}
\lstinputlisting{../ace-pldi14/datatypes.py}
\caption{\texttt{[datatypes.py]} The dynamically-typed Python code generated by running \texttt{acec datatypes\_t.py}.}
\label{datatypes}
\end{codelisting}
Listing \ref{datatypest} shows an example of the use of statically-typed polymorphic recursive labeled sum types together with the Python implementation just described. It shows two syntactic conveniences that were not mentioned previously: 1) if no target is provided to \verb|ace.fn|, the base can provide a default target (here, \verb|ace.python.python|); 2) a concrete function can be generated immediately by providing a type assignment immediately in braces (in Python 3, a more convenient syntax is available). Listing \ref{datatypest} generates Listing \ref{datatypes}.

Lines \ref{datatypest}.3-\ref{datatypest}.11 define a function that generates a recursive algebraic datatype representing a tree given a name and another Ace type for the data at the leaves. This type implemented by the active type family \verb|fp.Datatype|. A name for the datatype and the case names and types are provided programmatically on lines \ref{datatypest}.3-\ref{datatypest}.8. To support recursive datatypes, the case names are enclosed within a lambda term that will be passed a reference to the datatype itself. These lines also show two more active type families: units (the type containing just one value), and immutable pairs. Line \ref{datatypest}.13 calls this function with the Ace type for dynamic Python values, \verb|py.dyn|, to generate a type, aliased \verb|DT|. This type is implemented using class inheritance when the target is Python, as seen on lines \ref{datatypes}.1-\ref{datatypes}.13. For C-like targets, a \verb|union| type can be used (not shown).

The generic function \verb|depth_gt_2| demonstrates two features. First, the base has been setup to treat the final expression in a top-level branch as the return value of the function, consistent with typical functional languages. Second, the \verb|case| ``method'' on line \ref{datatypest}.17 creatively reuses the syntax for dictionary literals to express a nested pattern matching construct. The patterns to the left of the colons are not treated as expressions (that is, a type and translation is never recursively assigned to them). Instead, the active type implements the standard algorithm for ensuring that the cases are exhaustive and not redundant \cite{pfpl}. If the final case were omitted, for example, the algorithm would statically indicate a warning (or optionally an error, not shown), just as in ML.


%\subsubsection{Active Targets and Cross-Compilation}\label{atargets}
%An active target, which we describe being used above, is an instance of \verb|ace.ActiveTarget|. It has two primary roles in Ace: 1) providing an API for code generation to a target language, providing term and type constructors for use during translation (and, optionally, a type system implementation to support representational consistency checks); 2) implementing the interoperability layer to support calling of  generic or concrete function directly from Python, as described in Sec. \ref{implicit}.
%
%An active type or base can support multiple active targets by simply examining \verb|target| during translation (e.g. by performing capability checks using Python's \verb|hasattr| function). Alternatively, the active target can simulate the interface of a different target but generate code differently. For example, although our implementation of OpenCL's type system as a collection of active types and an active base relies on a target that supports OpenCL terms and types, our \verb|CUDA| and \verb|C99| targets provide partial support for the same term and type forms, in order to support cross-compilation. We will show the latter being used in Sec. \ref{examples}. 
%

%\subsection{Compositional Reasoning}\label{safety}
%Every statement and expression in an Ace function is governed by exactly one active type, determined by the dispatch protocol, or by the single active base associated with the Ace function. The representational consistency check ensures that compilation is successful without requiring that types that abstract over other types (e.g. container types) know about their internal implementation details. Together, this permits compositional reasoning about the semantics of Ace functions even in the presence of many extensions. This stands on contrast to many prior approaches to extensibility, where extensions could insert themselves into the semantics in conflicting ways, making the order of imports matter (see Sec. \ref{related}). 

%Type assignment to generic functions is similar in some ways to template specialization in C++. In effect, both a template header and type parameters at call sites are being generated automatically by Ace.% This simplifies a sophisticated feature of C++ and enables its use with other targets like OpenCL. %Other uses for C++ templates (and the preprocessor) are subsumed by the metaprogramming features discussed above\todo{cite template metaprogramming paper}.

%\subsection{Within-Function Type Resolution}
%\begin{codelisting}
%\lstinputlisting{listing6.py}
%\caption{\texttt{[listing6.py]} A function demonstrating whole-function type inference when multiple values with differing types are assigned to a single identifier, \texttt{y}.}
%\label{inference}
%\end{codelisting}
%On line 5 in the generic \verb|map| function in Listing \ref{map}, the variable \verb|gid| is initialized with the result of calling the OpenCL primitive \verb|get_global_id|.  The type for \verb|gid| is never given explicitly. This is a simple case of Ace's {\em within-function type resolution} strategy (we hesitate to call it \emph{type inference} because it does not, strictly speaking, take a constraint-solving approach). In this case, the type of \verb|gid| will resolve to \verb|size_t| because that is the return type of \verb|get_global_id| (as defined in the OpenCL specification, which the \verb|ace.OpenCL| module follows). The result can be observed on Lines 11 and 24 in Listing \ref{mapout}. 
%
%Inference is not restricted within single assignments, as in the \verb|map| example, however. Multiple assignments to the same identifier with values of differing types, or multiple return statements, can be combined if the types in each case are compatible with one another (e.g. by a subtyping relation or an implicit coercion). In Listing \ref{inference}, the \verb|threshold_scale| function assigns different values to \verb|y| in each branch of the conditional. In the first branch, the value \verb|0| is an \verb|int| literal. However, in the second branch of the loop, the type depends on the types of both arguments, \verb|x| and \verb|scale|. We show two choices for these types on Lines 11 and 12. Type inference correctly combines these two types according to OpenCL's C99-derived rules governing numeric types (defined by the user in the \verb|OpenCL| module, as we will describe in Section \ref{att}). We can verify this programmatically on Lines 12 and 13. Note that this example would also work correctly if the assignments to \verb|y| were replaced with \verb|return| statements (in other words, the return value of a function is treated as an assignable for the purpose of type inference).


\subsubsection{Remaining Tasks and Timeline}
A paper on this work was recently rejected from PLDI 2014. We plan to resubmit this work to OOPSLA 2014, due on Mar. 25th, after completing the following tasks (largely after the Mar. 1 ICFP deadline):
\begin{itemize}
\item The reviewers asked for more details about the composition properties of extensions, the relationship to abstract data types, to work on staging and on how variables and contexts are treated. We will give more space to these issues in the new draft.
\item We will provide more details, given the additional space available in an OOPSLA paper, of the examples other than OpenCL. We will consider using an example other than OpenCL to motivate the main body of the paper, so that the general distaste most PL researchers have for direct memory manipulation can be avoided. 
\item We will attempt to connect better to the theoretical work we have done on @$\lambda$, to emphasize that this work is about integrating that mechanism into an existing language (with the compromises this must necessarily entail).
\item The reviewers made several more concrete suggestions at different points in the paper that we will integrate.
\end{itemize}

% !TEX root = omar-thesis-proposal.tex
\section{Active Code Completion}\label{acc}

%\section{Preliminary Work}
We have started implementing a form of active typechecking and translation, as outlined above, by developing an actively-typed programming language embedded within Python called Ace. In addition, we have demonstrated and empirically evaluated active code completion support for Java, implemented using class annotations and a  plugin for the Eclipse editor called Graphite.

\subsection{Ace: Active Typechecking and Translation}
Ace is an actively-typed programming language with a fixed syntax, shared with the Python language, but user-extensible semantics specified using an implementation of the active typing mechanism described above. The type-level language of Ace is Python, and types are type-level instances of Python classes implementing a provided interface, \verb|ace.Type|. The compiler selectively invokes the methods of these instances when type checking and translating expressions, a mechanism we call {\em active type-checking and translation (AT\&T)}. We have demonstrated the basic practicality of this mechanism by implementing primitives from several widely-known low-level languages, including much of C99 and the entirety of OpenCL, as active libraries. 
\begin{codelisting}
\lstinputlisting{listing4.py}
\caption{The generic \texttt{map} function compiled to map the \texttt{add5} function over two  types of input.}
\label{mapadd5dbl}
\end{codelisting}

\begin{codelisting}
\lstinputlisting{listing9.py}
\caption{A portion of the implementation of OpenCL pointer types implementing subscripting logic using the Ace extension mechanism.}
\label{pointers}
\end{codelisting}

An example showing usage of Ace with the OpenCL active library is shown in Listing \ref{mapadd5dbl}. Here, the top-level of the file is type-level code. On line 13, two instances of an active type, corresponding to the OpenCL types \verb|__global double*| and \verb|__global int*|, are created (to emphasize, at compile-time). The \emph{generic functions} defined on lines 4-11 are then programmatically specialized with these types assigned to the input arguments on lines 12-13, triggering active typechecking and translation mechanism when operations on these arguments are encountered (e.g. line 7).

Listing \ref{pointers} shows the portion of the implementation of the \verb|GlobalPtrType| type family (which \verb|A| and \verb|B| in Listing 1 are members of) that is responsible for the subscripting operation on line 7. The \verb|resolve_Subscript| method is responsible for checking that the subscript is an integer type, returning the appropriate type for the expression as a whole -- the target type of the pointer -- if so or raising an appropriate error if not. The \verb|translate_Subscript| function is then responsible for translating the subscript expression into the target language. Here, our target language is OpenCL (that is, we have simply lifted a target language construct into Ace directly), so the translation is straightforward. In general, however, this method may contain sophisticated code generation logic. These two methods correspond to the $f_{\text{resolve-X}}$ and $f_{\text{compile-X}}$ functions described above.
 
In addition to AT\&T, Ace supports more general forms of metaprogramming, and functions can be launched directly from Python with standard numeric data structures as arguments. Using Ace, we designed a scientific simulation framework that allows users to modularly specify, compile and orchestrate the execution of parameterized families of scientific simulations on clusters of GPUs. This framework has been used to successfully conduct large-scale, high-performance neuroscience simulations, providing initial evidence that Ace is useful not only as a foundational tool for researchers, but also for the practice of high-performance computing today.

\subsection{Graphite: Active Code Completion}\label{graphite}
Graphite demonstrates an edit-time aspect of active types, corresponding to the $f_{\text{editor}}$ function above, in the context of an existing language, Java, and an existing editor, Eclipse. A paper describing and evaluating this work which was recently published at ICSE 2012 \cite{omar2012active}. 
In that paper, we introduced {\em active code completion}, a mechanism that allows library developers to associate interactive and highly-specialized code generation interfaces, called {\em palettes}, directly with types using Java's annotation system and HTML5 for implementation. A simple example of such a palette and its operation is shown in Figure \ref{color}. 

Using several empirical methods, we examined the contexts in which such a system could be useful, described the design constraints governing the system architecture as well as particular code completion interfaces, and detailed the design of our implementation, Graphite. Using Graphite, we implemented a more sophisticated palette for writing regular expressions and conducted a small pilot study that showed that this kind of domain-specific interface is useful for professional developers working in that domain.

\begin{figure*}\label{color}
\begin{center}
\includegraphics[width=40pc]{color_palette.png}\end{center}
\caption{(a) An example code completion palette associated with the \texttt{Color} class. (b) The source code generated by this palette.}
\end{figure*}

%\section{Work To Be Done}
\subsection{Completion and Evaluation of Ace}
The work done so far on the Ace language establishes the practicality of active typechecking and translation as a foundational language mechanism and begins to apply it to one realistic problem domain, high-performance scientific computing on GPUs and other coprocessors, by internalizing OpenCL. However, to more fully establish the practicality of this approach, it must be shown that Ace can support more sophisticated, higher-level language constructs as well, ideally from a variety of application domains. It must also be further shown that users who are not language design experts, and will not be utilizing the extension mechanism directly, can nevertheless use the language in practice. 

Thus, a portion of the resources allocated to this project will be dedicated to completing the implementation of Ace, documenting it and releasing it both to the research community and to practitioners in areas where a useful set of constructs have been developed, beginning with the members of the high-performance computing community. This will also support synergy with ongoing collaborations in our group with other problem domains, including front-end and back-end web development, security, functional programming and object-oriented programming.

With broader usage by the research community and in practice, we will be able to continue to conduct case studies and empirical evaluations examining the usability and usefulness of Ace and the active typing mechanisms it introduces. Indeed, because of the flexibility of active typing and our implementation strategy, we plan on instrumenting the Ace compiler to send data and direct feedback from willing users back to us for analysis. This will allow us to analyze questions like:
\begin{itemize}
\item How often are different constructs and mechanisms used in practice?
\item What kinds of errors appear most frequently in practice, and how long does it take users to resolve these errors?
\item How do users feel about the mechanisms, constructs and error messages that they encounter, as determined by direct feedback solicited at the time of occurrence.
\item How commonly do users utilize the active typing mechanism directly?
\end{itemize}

To our knowledge, no such detailed study of usage patterns of a programming language in the wild has been conducted previously, and we anticipate producing insights that are relevant to programming language design broadly.
\subsection{Active Type Theory}
The work on Ace is aimed at establishing a practical implementation of active typechecking and translation. However, because it uses a dynamically-typed language, Python, for the type-level language, it is not suitable for rigorous theoretical analysis, nor does it achieve safety in the strictest sense, due to high levels of dynamism exploitable in Python's object model. Moreover, it is a full language and so it does not necessarily distill the essential concepts that we wish to introduce to the research community in their simplest possible form or connect them to existing concepts in the theory of typed programming languages. For these reasons, we plan to develop a minimal type theoretic formulation, which we call $\lambda_{\text{A}}$.

\subsubsection{Type-Level Computation}

We have chosen to use the term \emph{type-level language}, rather than \emph{metalanguage} or, as it is called in the original work on active libraries, \emph{metacode} intentionally. This is because the concept of a type-level language is already well-established and includes a key feature necessary for active types -- the reification of types as compile-time entities that can be manipulated programmatically. We believe that this connection between type-level computation and active types will provide new insights into each mechanism individually, and provide a clean theoretical basis for active typechecking and translation as an extension to type-level computation.

\subsubsection{Active Types and Wadler's Expression Problem}
Another key insight is that in monolithic programming languages, the fixed set of base types can be written as an algebraic datatype. Indeed, in functional implementations of typecheckers, this is precisely what is done, for example for G\"{o}del's \textbf{T}:

\begin{verbatim}
    datatype Type = Arrow of Type * Type 
                  | Nat
\end{verbatim}

If we wish to allow users to introduce new base types and type families, it can then be seen that we face a variant of the expression problem, so named by Wadler \cite{wadler1998expression}. That is, we wish to add new constructors of kind \verb|Type|, but this conventional formulation of datatypes does not allow us to do so. There are competing approaches that aim to solve this problem. The most common is to use object-oriented inheritance, where subtypes of \verb|Type| represent new type families. This is the approach taken in Ace. The functional programming approach is to instead use \emph{open data types} \cite{conf/ppdp/LohH06}. We will thus aim to further explore this approach, drawing a clear connection between active types, the expression problem and type-level open data types.

\subsubsection{Rigorous Safety Proofs}
A benefit of a simple core theory for active typing is that it will allow us to formally state and prove key safety theorems, formalizing the somewhat informal notions described in Section 3.3. Based on early sketches of a theory that we have discussed, we have implemented some run-time checks into Ace that ensure some of these properties for compiled programs based on a notion borrowed from the area of verified compilation called \emph{type-directed compilation} \cite{tarditi+:til-OLD}. We would like to be able to ensure these properties for all possible programs, and a type theory will allow us to demonstrate the machinery needed to do this in a rigorous manner.

\subsubsection{Implementation With a Dependently-Typed Language}
Finally, we would like to implement a functional version of active type-checking and translation, both as a counterpoint to Ace and as a useful tool for theoretical researchers. We plan on doing so as an extension to the Coq theorem prover, which contains a dependently-typed functional programming language at its core, and is implemented in Ocaml. This will allow us to understand the challenges and opportunities of using a dependently-typed functional language at the core of an extensible programming system.
 

\section{Research Plan}

Our plan for carrying out the work described above involves building significantly on existing work on Ace, as well as iterating on existing highly-simplified prototypes we have developed of all of the components described above.

In the first year, we will fully develop the theory of active types and completing development of the Ace language. We plan on releasing Ace to the broader community, as described in Section 5.1, within the coming year. In addition, we will implement initial versions of a core active code prediction system for Java.

In the second year, we will begin conducting capstone evaluations of Ace, further applying the theory to functional languages, and begin working on applications of active code prediction to code completion and rehabilitative programming systems. 

In the third year, we will complete capstone evaluations of Ace, conduct capstone evaluations of active code prediction and the applications that we have developed, and begin prototyping an actively-typed structured editor, laying foundations for the long-term evolution of this research program beyond the scope of this grant.

Evaluations will consist of user studies, case studies, classroom studies and field studies based on instrumented versions of the tools that we develop. We plan on justifying not just the theoretical utility of active types, but their usefulness and usability in practice as well.
% !TEX root = omar-thesis.tex
\chapter{Conclusion \& Future Work}


\newpage

%% Bibliography magic
% dvips -n 15 -o description.ps description
% dvips -p=16 -o bibliography.ps description
% ps2pdf description.ps
% ps2pdf bibliography.ps
%
% OR, for pdflatex:
%
% pdftk description.pdf cat 1-15 output description1.pdf
% pdftk description.pdf cat 16-end output bibliography.pdf

\setcounter{page}{1}
\renewcommand{\thepage}{References - \arabic{page}}

\bibliographystyle{abbrv}
\bibliography{../research}
%,typestate-verification,objects,ownership,formal,dynamicweb,cloud
%\bibliographystyle{plainnat}
%\bibliography{references,dynamicweb,cloud}

\end{document}
