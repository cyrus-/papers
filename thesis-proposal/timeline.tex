% !TEX root = omar-thesis-proposal.tex
\section{Conclusion}\label{conclusion}
In arguing for language extensibility, we are accepting the philosophy of logical pluralism, explained by Carnap as follows \cite{carnap:1950b}:
\begin{quote}
Let us grant to those who work in any special fields of investigation the freedom to use any form of expression which seems useful to them. The work in the field will sooner or later lead to the elimination of those forms which have no useful function. Let us be cautious in making assertions and critical in examining them, but tolerant in permitting linguistic forms.
\end{quote}
But we take issue with this unbounded \emph{principle of tolerance}, which he explained more directly as follows \cite{CarnapR:logsl}:
\begin{quote}
In logic, there are no morals. Everyone is at liberty to build his own logic, i.e. his own form of language, as he wishes. All that is required of him is that, if he wishes to discuss it, he must state his methods clearly, and give syntactic rules instead of philosophical arguments.
\end{quote}

Carnap formulated his principle of tolerance because he wished to freely explore the consequences of reasoning with different connectives and  rules and emphasize that conducting philosophical reasoning by methodical logical analysis did not require one to accept any particular axioms. But he was also a prominent member of the Vienna circle, which sought a ``unified science'' the purpose of which was "to link and harmonise the achievements of individual investigators in their various fields of science" by the construction of a "constitutive system" \cite{hahn1929scientific}. These two programs were in conflict: were the Vienna circle's ``constitutive system'' to become na\"ively tolerant of the inference rules (what Carnap called L-rules) developed by individual investigators, it would find that this would then deny them autonomy of reasoning. According to social scientist Raz, the primary justification of tolerance is to maximize 
individual autonomy in a multi-party context \cite{raz1988autonomy}. Yossie Nehushtan argues that a liberal  society ``should not tolerate anything that denies the justifications of tolerance'' \cite{nehushtan2007limits}. Thus, we propose a more cooperative \emph{principle of reciprocal tolerance}:
\begin{quote}
In metalogic, there \emph{are} morals. Everyone is at liberty to build their own logical fragment, i.e. their own language extensions, as long as these can be implemented within a framework that ensures, by formal means instead of philosophical argument, that autonomously derived reasoning principles are conserved when fragments are combined, in \emph{any} combination.
\end{quote}

By following this principle, we believe that a research program that supports Carnap's logical pluralism within a ``constitutive system'' of the sort envisioned by the Vienna circle is both possible and practical. The work in this thesis gives first steps towards this goal.


%Aside from the archaic use of gendered pronouns, 
% If the reader will permit a brief digression into social science, the adoption of a tolerant viewpoint is difficult to sustain in a society that does not require \emph{reciprocity}.   And indeed, this was precisely Carnap's program. But

% http://www.trinitinture.com/documents/nehushtan.pdf
% https://en.wikipedia.org/wiki/Vienna_Circle
% http://consequently.org/papers/carnap.pdf
% https://www.academia.edu/4041290/How_tolerant_can_you_be_Carnap_on_the_normativity_of_logic

\section{Timeline}
To summarize, we plan on completing the work in this thesis per the following schedule:
\begin{itemize}
\item The work on type-specific languages has been accepted to ECOOP 2014. The deadline for final revisions is May 12, so we will complete the remaining work, including writing a technical report, by then. This should be sufficient for the chapter on Wyvern in the dissertation.
\item The work on @$\lambda$ and active embeddings is conceptually complete. We plan to complete the technical details and exposition and submit it to POPL in July. This should be sufficient for the chapter on @$\lambda$ in the dissertation.
\item The work described the core of Ace was submitted to OOPSLA 2014 on Mar. 25. Notification is May 26. A paper with Nathan Fulton, a former undergraduate student who I co-supervised, on regular expression types in Ace is in preparation for submission to PLAS 2014 on April 20. We plan to submit a paper to either SLE 2014 or GPCE 2014 on FFIs using Ace (in particular, a FFI for OpenCL, which was already developed); both conferences have due dates at the end of May. Together, these will form the chapters on Ace in the dissertation.
\item The work on active code completion has been completed and published at ICSE 2012 \cite{Omar:2012:ACC:2337223.2337324}. The remaining work is expository and will be completed in the course of writing the Graphite chapter of the dissertation.
\end{itemize}
