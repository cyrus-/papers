\section{Research Plan}

Our plan for carrying out the work described above involves building significantly on existing work on Ace, as well as iterating on existing highly-simplified prototypes we have developed of all of the components described above.

In the first year, we will fully develop the theory of active types and completing development of the Ace language. We plan on releasing Ace to the broader community, as described in Section 5.1, within the coming year. In addition, we will implement initial versions of a core active code prediction system for Java.

In the second year, we will begin conducting capstone evaluations of Ace, further applying the theory to functional languages, and begin working on applications of active code prediction to code completion and rehabilitative programming systems. 

In the third year, we will complete capstone evaluations of Ace, conduct capstone evaluations of active code prediction and the applications that we have developed, and begin prototyping an actively-typed structured editor, laying foundations for the long-term evolution of this research program beyond the scope of this grant.

Evaluations will consist of user studies, case studies, classroom studies and field studies based on instrumented versions of the tools that we develop. We plan on justifying not just the theoretical utility of active types, but their usefulness and usability in practice as well.