% !TEX root = omar-thesis-proposal.tex
\section{Conclusion}\label{conclusion}
In arguing for language extensibility, we are accepting Carnap's logical pluralism\todo{citations}. But we take issue with his \emph{principle of tolerance}, which he explained as follows:
\begin{quote}
In logic, there are no morals. Everyone is at liberty to build his own logic, i.e. his own form of language, as he wishes. All that is required of him is that, if he wishes to discuss it, he must state his methods clearly, and give syntactic rules instead of philosophical arguments.
\end{quote}
%Aside from the archaic use of gendered pronouns, 
Carnap formulated his principle of tolerance because he wished to freely explore the consequences of reasoning with different connectives and  rules and emphasize that conducting philosophical reasoning by methodical logical analysis did not require one to accept any particular axioms. But he was also a prominent member of the Vienna circle, which sought a ``unified science'' the purpose of which was "to link and harmonise the achievements of individual investigators in their various fields of science" by the construction of a "constitutive system" \cite{ViennaCircle}. These two programs were in conflict. If the reader will permit a brief digression into social science, the adoption of a tolerant viewpoint is difficult to sustain in a society that does not require \emph{reciprocity}. Yossie Nehushtan goes as far as to argue that a liberal  society ``should not tolerate anything that denies the justifications of tolerance'' \cite{Nehushtan}. According to Raz, the primary justification of tolerance is to maximize 
individual autonomy in a multi-party context \cite{Raz}. And indeed, this was precisely Carnap's program. But were the Vienna circle's ``constitutive system'' to become na\"ively tolerant of new inference rules (what Carnap called L-rules), it would find that this would then deny its investigators autonomy of reasoning. Thus, we propose a more cooperative \emph{principle of reciprocal tolerance}:
\begin{quote}
In metalogic, there \emph{are} morals. Everyone is at liberty to build their own logical fragment, i.e. their own language extensions, as long as these can be clearly shown, by formal means instead of philosophical argument, to conserve autonomously derived reasoning principles when used in \emph{any} combination.
\end{quote}

By following this principle, we believe that a research program that supports Carnap's logical pluralism within a ``constitutive system'' of the sort envisioned by Vienna circle is both possible and practical. The work in this thesis gives first steps towards this goal.

% http://www.trinitinture.com/documents/nehushtan.pdf
% https://en.wikipedia.org/wiki/Vienna_Circle
% http://consequently.org/papers/carnap.pdf
% https://www.academia.edu/4041290/How_tolerant_can_you_be_Carnap_on_the_normativity_of_logic

\section{Timeline}
To summarize, we plan on completing the work in this thesis per the following schedule:
\begin{itemize}
\item The work on type-specific languages has been accepted to ECOOP 2014. The deadline for final revisions is May 12, so we will complete the remaining work by then.
\item The work on Ace will be submitted to OOPSLA 2014 on Mar. 25.
\item The work on @$\lambda$ and active embeddings is substantially complete. We plan to complete some technical details and exposition after Mar. 25 and submit it to POPL in July (though we hope to be done with the thesis work before then).
\item The work on active code completion has been completed and published at ICSE 2012 \cite{Omar:2012:ACC:2337223.2337324}. The remaining work is expository and will be completed in the course of writing the dissertation.
\end{itemize}
