\documentclass[10pt,preprint]{sigplanconf}
\usepackage{bussproofs}    % Gentzen-style deduction trees *with aligned sequents!*
\usepackage{mathpartir}	% For type-settting type checking rule figures
\usepackage{syntax}		% For type-setting formal grammars.


\usepackage{hyperref}		% For links in citations

\usepackage{float}			% Make figures float if [H] option is passed.

\iffalse
\usepackage{listings}		% For typesetting code listings
\usepackage{callout}
\usepackage{titlesec}
\usepackage[T1]{fontenc}
\usepackage{upquote}
\lstset{upquote=true}
\fi

\usepackage{textcomp}		% For \textquotesingle as used in introduction
\usepackage{color}			% for box colors, like in TAPL.

\usepackage{amsmath}		% Begin Carthage default packages
\usepackage{makeidx}
\usepackage{amsthm}
\usepackage{amsfonts}
\usepackage{amssymb}
\usepackage{graphics}
\usepackage{enumerate}
\usepackage{multicol}
\usepackage{epsfig}
\usepackage{csquotes}
\usepackage{enumitem}

\newtheorem{thm}{Theorem}                                                       
\newtheorem{cor}[thm]{Corollary}                                                
\newtheorem{lem}[thm]{Lemma}                                                    
\newtheorem{prop}[thm]{Proposition}                                             
\newtheorem{ax}[thm]{Axiom}                                                     
\theoremstyle{definition}                                                       
\newtheorem{defn}[thm]{Definition}                                              
\newtheorem{exam}[thm]{Example}                                                 
\newtheorem{rem}[thm]{Remark} 

%
% For type setting inference rules with labels.
%
\newcommand{\inferlbl}[3]
			{\inferrule{#3}{#2}{\textsf{\footnotesize{\sc #1}}}}
\newcommand{\inferline}[3]
			{\inferrule{#3}{#2} & {\textsf{\footnotesize{\sc #1}}} \\ \\}

\newcommand{\Lagr}{\mathcal{L}}
\newcommand{\strin}{{\tt string\_in}}
\newcommand{\lang}[1]{\Lagr\{#1\}}
\newcommand{\stru}[2]{ {\tt string\_union}(#1,#2)}

\newcommand{\str}{ {\tt string} }
\newcommand{\istr}{ {\tt istring} }

\newcommand{\dconvert}[2]{ {\tt dconvert}(#1,#2) }
\newcommand{\filter}[2]{ {\tt filter}(#1,#2) }
\newcommand{\ifilter}[2]{ {\tt ifilter}(#1,#2) }

\newcommand{\reduces}{ \Rightarrow }
\newcommand{\ireduces}{ \leadsto }

\newcommand{\val}{ \ {\tt val} }
\newcommand{\ival}{ \ {\tt ival} }

\newcommand{\istrf}[1]{`#1\textrm'} %??
\newcommand{\strf}[1]{``#1"}

%
% Constrain the size of full-page diagrams and rule lists
%
%\newcommand{\pagewidth}{5in}
%\newcommand{\rulelistwidth}{3in}

%
% Names of type systems presented in paper
%
\newcommand{\lcs}{\lambda_{CS}}

\setlength{\grammarindent}{3em}

\begin{document}

\section{Introduction}
In web applications and other settings, incorrect input sanitation often causes
security vulnerabilities. For this reason, frameworks often provide methods
for sanitizing user input. When these methods are insufficient or unavailable,
developers often create  custom input sanitation algorithms. In both cases,
input sanitation techniques are ultimately implemented using the language's
regex library.

Formally, input sanitation is the problem of ensuring that an arbitrary string
is converted into a safe form before potentially unsafe use. For example, consider
SQL injection attacks. To prevent such attacks, we might ensure that any string
used as input to a query does not contain unescaped SQL command sequences. 

This appendix presents a type system, $\lcs$, for ensuring that input 
sanitation algorithms are implemented correctly with respect to use site 
specifications.

Unlike frameworks provided by general purpose languages such as Haskell and
Ruby, our type system provides a \emph{static} guarantee that input is always 
properly sanitized before use. We achieve this by defining a typing relation
which captures idiomatic sanitation algorithms. Type safety relies upon 
several closure and decidability results about regular languages.

XDuce typechecks XML. Like our work, XDuce relies upon some properties of
regular expressions to establish soundness and completeness results. Unlike
XDuce, our work is motivated by input sanitation and therefore considers arbitrary
strings (as opposed to tree-structured XML documents). Furthermore, our static
treatment of the ubiquitous regex filter (replace) function for regular expression is novel.

Finally, some research languages achieve similar safety guarantees via a specialized type
system. The existence of such domain-specific type systems is suggestive, and 
we believe the simplicity of $\lcs$ demonstrates the effectiveness of
extensibility. Instead of introducing entirely new languages for each domain,
language extension developers may invest incrementally in obtaining static
guarnatees which address common mistakes.

An outline of this appendix follows:

\begin{itemize}
  \item In \S 2, we define the type system's static and dynamic semantics.
  \item Section 3 recalls some classical results about regular expressions and presents a type safety proof for $\lcs$ based upon these properties.
  \item Finally, \S 4 discusses our implemention of $\lcs$ as a type system extension  within the Ace programming language.
\end{itemize}

\section{Definition of $\lcs$}

The $\lcs$ language is characterized by a type of strings indexed by regular
expressions, together with operations on such strings which correspond to common
input sanitation patterns.

This section presents the grammar and semantics of $\lcs$.
The semantics are defined in terms of an internal language with at least strings and a regex filter function.
These constraints are captured by the internal term valuations ($\ival$).
The internal language does not necessarily need a regex filter function because
any dynamic conversion is easily definable using a combination of filters and safe
casting.

%
% Begin Grammar 
%
\begin{figure}
\begin{grammar}
<r> ::= $\epsilon$ | $.$ | $a$ | $r \cdot r$ | $r + r$ | $r*$ \hfill Regex ($a \in \Sigma$).

<t> ::= 			\hfill terms: 					\\
%$x$					\hfill variable 				\alt
%${\tt lambda} x:t$		\hfill abstraction 			\alt
%$t\ t$				\hfill application				\alt
$\filter{\strin[r]}{t}$ 		\hfill filter substrings			\alt
$[\strin[r]](t)$ \hfill safe conversion \alt
$\dconvert{\strin[r]}{t}$ \hfill unsafe conversion

<iv> ::= 			\hfill internal values:					\\
$\istrf{s}$				\hfill internal string \alt
${\tt ifilter}(r, \istr(s))$ \hfill internal filter

<v> ::= \hfill values: \\
$\strf{s}$ \hfill string

<T> ::=				\hfill	types:					\\
$\str$  \hfill Strings \alt
$\strin[r]$				\hfill Regular language strings		
%\alt
%$T \rightarrow T$	\hfill	Functions

<$\Gamma$> ::=	$\emptyset$	 | $\Gamma,x:T$ \hfill typing context
\end{grammar}
\caption{Syntax of $\lcs$}
\label{fig:lcsSyntax}
\end{figure}

The $\lcs$ language gives static semantics for common regular expression library
functions. In this treatment, we include concatenation and filtering.
The ${\tt filter}$ function removes all instances of a regular expression in a string,
while concatenation ($+$) concatenates two strings.

\subsection{Typing}

The $\strin[r]$ type is parameterized by regular expressions; if $e:\strin[r]$,
then $e \in r$. Mapping from an arbitrary $\str$ to a $\strin[r]$ requires
defining an algorithm -- in terms of filter -- for converting a $\strin[.*]$
into a $\strin[r]$. The static semantics of the language defines the types of
operations on regular expressions in terms of well-understood properties about
regular lanugages; we recall these properties in section 3.

\subsection{Dynamics}

There are two evaluation judgements: $e:T \reduces e'$ and $e:T \ireduces i$.
The $\reduces$ relation is between $\lcs$ expressions, while the $\ireduces$
relation is a mapping from $\lcs$ expressions into internal language expressions
$i$ such that $i \ival$.

Safety of the evaluation relation depends upon an injective mapping from $\lcs$ types info
internal language types. This relation, $h$, is defined below.


\onecolumn
\begin{figure}
\begin{center}
\begin{tabular}{c r}
%\inferline{T-VAR}{\Gamma \vdash x : T}{x:T \in \Gamma}
%
%\inferline{T-ABS}{\Gamma \vdash ({\tt \textbf{lambda}\ } x: t_2) : T_1 \rightarrow T_2}
%			{\Gamma, x:T_1 \vdash t_2:T_2}
%
%\inferline{T-APP}
%			{\Gamma \vdash t_1t_2 : T_{12}}
%			{\Gamma \vdash t_1 : T_{11}\rightarrow T_{12} \\ \Gamma \vdash t_2 : T_{11}}	
%
\inferline{T-Equiv-Top}
{\str \equiv \strin[.*]}
{ }

\inferline{T-Equiv-string\_in}
{\strin[r] \equiv \strin[r']}
{\lang{r} = \lang{r'}}

\inferline{T-Equiv-Include}
{\Gamma \vdash e:\strin[r]}
{e \in \lang{r}}

\inferline{T-Minimum}
{ \Gamma \vdash e : \strin[e] }
{\Gamma \vdash e : \str}

%\inferline{T-Union}
%  {\Gamma \vdash \stru[r'](e) : \strin[r \cup r']}
%  { \Gamma \vdash e : \strin[r] }

\inferline{T-Concat}
{\Gamma \vdash e_1 + e_2 : \strin[r_1 \cdot r_2]}
{ \Gamma \vdash e_1 : \strin[r_1] \\ \Gamma \vdash e_2 : \strin[r_2] }

\inferline{T-Convert}
{\Gamma \vdash [\strin[r']](e):\strin[r']}
{\Gamma \vdash e : \strin[r] \\ \lang{r'} \subseteq \lang{r}}

\inferline{T-Dconvert (optional)}
{\Gamma \vdash \dconvert{\strin[r']}{e}:\strin[r']}
{\Gamma \vdash e : \strin[r]}

\inferline{T-Filter}
{\Gamma \vdash \filter{\strin[r']}{t} : \strin[(r \backslash r') + \emptyset]}
{\Gamma \vdash t:\strin[r]}

\end{tabular}
\caption{Typing relation for $\lcs$}
\label{fig:lcsTyping}
\end{center}
\end{figure}

%
% Begin Operational Semantics
%
\begin{figure}
\begin{center}
\begin{tabular}{c r c r}
%\inferline{E-App1}{t_1t_2 \Rightarrow t_1't_2}{t_1 \Rightarrow t_2}
%\inferline{E-App2}{v_1t_2 \Rightarrow v_1t_2'}{t_2 \Rightarrow t_2'}
%\inferline{E-AppAbs}{({\tt lambda } x : t_{12})v_2 \Rightarrow [x / v_2]t_{12}}{T_{11}}

\inferline{E-strinval}
  {\strf{s}:\strin[r] \val}
  { }

%\inferline{E-strval}
%{\strf{s}:\str \val}
%  { }

\inferline{E-ival}
  {\istrf{e}:\istr \ival }
  { }

\inferline{E-ifilterval}
{ {\tt ifilter}(r,e) \ival }
{ e {\tt ival} }

\inferline{E-string}
{e : \str \ireduces \istrf{e}}
{ e \val }

\inferline{E-string\_in}
{e : \strin[r] \ireduces  \istrf{e}}
{ e \val }

\inferline{E-concatval}
{e_1 + e_2 : \strin[r_1+r_2] \ireduces \istrf{e_1} + \istrf{e_2}}
{e_1 \val \\ e_2 \val }

\inferline{E-concatL}
{e_1 + e_2 : \strin[r_1+r_2] \reduces {e_1}' + {e_2}}
{ e_1 : \strin[r_1] \reduces {e_1}'}

\inferline{E-concatR}
{e_1 + e_2 : \strin[r_1+r_2] \reduces {e_1} + {e_2}'}
{ e_2 : \strin[r_2] \reduces {e_2}'}

\inferline{E-filterval}
{\filter{\strin[r']}{e} : \strin[r\backslash r' + \emptyset] \reduces {\tt rl\_filter}(r,e)}
{ e \val }
 
\inferline{E-filter}
{\filter{\strin[r']}{e} : \strin[r\backslash r' + \emptyset] \reduces  \filter{\strin[r]}{ e' }}     
{ e : \strin[r] \reduces e' }

\inferline{E-convertval}
{ [\strin[r']](e) : \strin[r'] \ireduces \istrf{e} }
{ e \val }

\inferline{E-convert}
{ [\strin[r']](e) : \strin[r'] \reduces [\strin[r']](e') }
{ e : \strin[r] \reduces e' }

\inferline{E-dconvertval (optional) }
{\dconvert{\strin[r']}{e} : \strin[r'] \reduces {\dconvert{\strin[r']}{e'}}}
{e:\strin[r] \reduces e'}

\inferline{E-dconvert (optional) }
{\dconvert{\strin[r']}{e} : \strin[r'] \ireduces \ifilter{r'}{\istrf{e}}}
{e \val }

\end{tabular}
\caption{SOS rules for $\lcs$}
\label{fig:lcsEval}
\end{center}
\end{figure}

\twocolumn

\section{Type Safety}

The type safety proof relies upon some assumptions about the type system and
dynamics of the internal language, as well as some properties of regular
languages.

There must exist a translation from $\lcs$ types to the types of the internal
language. For the remainder of this paper, we call the type translation function
$h$.

\begin{defn}[Type Translation Function $h$]
  The type translation function $h : Type \rightarrow IType$ is defined as follows:
  \begin{itemize}
    \item $\forall r. h(\strin[r]) = \istr$
    \item $h(\str) = \istr$
  \end{itemize}
\end{defn}

Additionally, we assume that the internal language contains an implementation of strings,
together with operations for concatenation and filtering by regular expression.

\begin{defn}[Types of internal values]
  Let $\istrf{s}$ range over string literals and $r$ over regular expressions.
  Internal values are typed as follows:
\begin{itemize}
  \item If $e=\istrf{s}$ then $e:\istr$.
  \item If $e=\ifilter{r}{\istrf{s}}$ then $e:\istr$.
  \item If $e=\istrf{s_1}+\istrf{s_2}$ then $e:\istr$.
\end{itemize}
\end{defn}

For simplicity, we assume a fixed translation from
$\lcs$ regular expressions to regular expressions recognizable by the internal
language's regex library (in practice, a fixed translation is acceptable.)
To summarize, we assume an internal language containing 
a string type together with operations for string concatentation and filtering. We
expect closure over strings for both operations.
Finally, recall that ${\tt ifilter}$ is only needed for dynamic casts, which may be removed without
descreasing the expressivity or even usability of the language.
Finally, the semantics of the filter function are defined in terms of ${\tt rl\_filter}$,
which is a static version of ${\tt ifilter}$.

\subsection{Properties of Regular Languages}

The regular languages are the smallest set generated by regular expressions
defined in Figure 1.

\begin{thm}{Closure Properties.} \label{thm:closure}
The regular expressions are closed under complements and concatenation.
\end{thm}
\begin{proof}
See \cite{cinderella}.
\end{proof}

\begin{thm}{Coercion Theorem.} \label{thm:coerce}
Suppose that $R$ and $L$ are regular expressions, and that $s \in R$ is a finite string.  Let $s' := coerce(R,L,s)$ with all maximal substrings recognized by $L$ replaced with $\epsilon$.  Then $s'$ is recognized by $(R \backslash L) + \emptyset$ and the construction of $R \backslash L$ is decidable.
\end{thm}
\begin{proof}
Let $F,G$ be FAs corresponding to $R$ and $L$, and let $G'$ be $G$ with its final states inverted (so that $G'$ is the complement of $L$).  Define an FA $H$ as a DFA corresponding to the NFA found by combining $F$ and $G'$ such that $H$ accepts only if $R$ and $L'$ accept or if $s$ is empty (this construction may result in an exponential blowup in state size.)  Clearly, $H$ corresponds to $R \backslash L + \emptyset$.  Thus, the construction of $R \backslash L + \emptyset$ is decidable.

If $R \subset L$, $s' = \emptyset$.  If $L \subset R$, either $s' = \emptyset$, or $s' \in R$ and $s' \not \in L$. If $R$ and $L$ are not subsets of one another, then it may be the case that $L$ recognizes part of $R$.  Consider $L$ as the union of two languages, one which is a subset of $R$ and one which is disjoint.  The subset language is considered above and the disjoint language is inconsequential.
\end{proof}

\subsection{Type Safety Proof}

%\begin{lem}[Canonical Forms for $\lcs$]
%We identify only canonical forms necessary for the type safety proof.
%\begin{itemize}
%  \item If $e \val$ and $e:\str$ then $e = \strf{s}$.
%\end{itemize}
%\end{lem}

\begin{thm}[Preservation]
  Let $T$ be a type in $\lcs$ and $h(T)=\sigma$ the corresponding type in the internal language.
  For all terms $e$:
  \begin{itemize}
    \item If $e:T$ and $e:T \ireduces i$ then $i : \sigma$ such that $h(T) = \sigma$.
    \item If $e:T$ and $e:T \reduces e'$ then $e':T$.
  \end{itemize}
\end{thm}
\begin{proof}
The proof is a straightforward induction on the derivation of the combined evaluation relation.
\begin{itemize}[label=$ $,itemsep=1ex]
  \item \textbf{E-Ival, E-Ifilterval}. Both cases hold since the terms at hand are not $\lcs$ terms.
  \item \textbf{E-Stringval, E-strval}. Both cases hold since no reduction is possible. 
  \item \textbf{E-String}. By the definition of typing for internal terms, $\istrf{e} : \istr$. It suffices to show that $h(\str) = \istr$, which follows from the definition of $h$.
  \item \textbf{E-String\_in}. By the definition of typing for internal terms, $\istrf{e} : \istr$. It suffices to show that $h(\strin[r]) = \istr$, which follows from the definition of $h$ for arbitrary $r$.
  \item \textbf{E-concatval}. By the definition of typing for internal terms, $\istrf{e_1} + \istrf{e_2} : \istr$. So it suffices to show that $h(\strin[r_1+r_2]) = \istr$, which follows from the definition of $h$ for arbitrary $r_1$, $r_2$.
  \item \textbf{E-concatR, E-concatL}. Consider E-concatR. By induction, $e_1' : \strin{l_1}$. By inversion of T-Concat at the premise, $e_2 : \strin{r_2}$. Therefore, $e_1 + e_2 : \strin[r_1 + r_2]$. The left rule is symmetric.
  \item \textbf{E-Filterval}. We have that $\filter{\strin[r']}{e}:\strin[r\backslash r' + \emptyset]$. By inversion of T-Filter, $e:\strin[r]$.
    By T-Equiv-string\_in (which is bidirectional), $e \in \lang{r}$.
    By the Coercion Theorem, ${\tt rl\_filter}(r,e) \in \lang{r\backslash r' + \emptyset}$.
    By T-Equiv-string\_in,  $e \in \lang{r}$ and ${\tt rl\_filter}(r,e) \in \lang{r\backslash r' + \emptyset}$ implies ${\tt rl\_filter}(r,e):\strin[r\backslash r' + \emptyset]$.
  \item \textbf{E-Filter}. By inversion, $e : \strin[r']$ $\reduces e'$ so by the induction $e':\strin[r']$.
    Therefore, \\${\tt filter}(\strin[r], e'):$ $\strin[r \backslash r' + \emptyset]$ by T-Filter.
  \item \textbf{E-Convertval}. It suffices to show that $h(\strin[r]) = \istr$, which is true by definition.
  \item \textbf{E-Convert}. By inversion and induction, $e' : \strin[r]$. We know that $[\strin[r']](e) : \strin[r']$, so by inversion of T-Convert $\lang{r'} \subseteq \lang{r}$.
    It follows that $[\strin[r']](e') : \strin[r']$.
  \item \textbf{E-DConvert}. By inversion, $e : \strin[r'] \reduces e'$. By the induction hypothesis, $e' : \strin[r']$; therefore, by T-Dconvert, $\dconvert[\strin[r]](e):\strin[r]$.
  \item \textbf{E-DConvertval}. By the definition of typing for internal terms, $\ifilter(r,\istrf{e}) : \istr$. It suffices to show that $h(\strin[r_1+r_2]) = \istr$, which follows from the definition of $h$.
\end{itemize}
\end{proof}

\begin{thm}[Progress]
  If $e:T$ then $e:T \reduces^* e' \ireduces^* i$ where $i \ival$.
\end{thm}
\begin{proof}
By induction on the derivation of $e:T$.
For \textbf{T-Equiv-Include}, note that $e \in \lang{r}$, $e = \strf{s}$ for some $s$; therefore, $e \val$ by E-Strinval. 
We have that $e \val$ and $e : \strin[r]$, so $e \ireduces \istrf{e}$ by E-string\_in.
The remaining cases follow by induction on the hypotheses and application of corresponding evaluation rules.
\end{proof}

\section{Implementation in Ace}

The $\lcs$ language is implemented as a type system extension in Ace; this extension
is illustrated in the examples at the beginning of this paper.

Computing a regular expression representating the language
$R \backslash S$ is necessary in order to type-check expressions in which the filter function
occurs. This language is computed by translating $R$ and $S$ into finite automata,
complementing the final states of $S$, then constructing the cross-product of $R$
and $S'$. Type checking terminates only because this construction is decidable.

In addition to this construction, other more typical operations -- such
as equality checks for regular expressions and regular expression matching -- are
also necessary. For these reasons, the Ace extension implementing $\lcs$ includes a library
implementing each of these constructions.

\end{document}

