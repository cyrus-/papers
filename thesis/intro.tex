% !TEX root = omar-thesis.tex
\chapter{Introduction}\label{chap:intro}
\section{Motivation}\label{sec:intro-motivation}
Functional programming language designers often study small typed lambda calculi to develop a principled understanding of  fundamental metatheoretic issues, like type safety, and to examine the mathematical character of language primitives of interest in isolation. These studies inform the design of ``full-scale''\footnote{Throughout this work, phrases that should be read as having an intuitive meaning, rather than a strict mathematical meaning, will be first introduced with quotation marks.} functional languages, which combine several such primitives and also typically feature various generalizations and primitive ``embellishments''  motivated by a consideration of human factors. 
For example, major languages like Standard ML (SML) \cite{mthm97-for-dart,harper1997programming}, OCaml \cite{ocaml-manual} and Haskell \cite{jones2003haskell} all primitively build in record types, generalizing the nullary and binary product types that suffice in simpler calculi, because explicitly labeled components are cognitively useful to human programmers. Similarly, these languages build in derived syntactic forms (colloquially, ``syntactic sugar'') that decrease the syntactic cost of  working with common library constructs, like lists.

The hope amongst many language designers is that a limited number of primitives like these will suffice to produce a ``general-purpose'' programming language, i.e. one where programmers can direct their efforts almost exclusively toward the development of useful libraries. However, a stable language design that fully achieves this ideal has yet to emerge, as evidenced by the diverse array of ``dialects'' that continue to proliferate around all major contemporary languages. 
In fact, tools that aid in the construction of so-called  ``domain-specific'' language dialects (DSLs)\footnote{In some parts of the literature, such dialects are called ``external DSLs'', to distinguish them from  ``internal'' or ``embedded DSLs'', which are actually  library interfaces that only ``resemble'' distinct dialects \cite{fowler2010domain}.} seem only to be becoming increasingly prominent.

\subsection*{Why are there so many dialects?}
{This calls for an investigation}: why is it that programmers and researchers are still so often unable to satisfyingly express the constructs that they need in libraries, as modes of use of the ``general-purpose'' primitives already available in major languages today, and instead see a need for new language dialects?

Perhaps the most common motivation may be that the \emph{syntactic cost} of expressing a construct of interest using contemporary general-purpose primitives is not always ideal. In response, library providers construct \emph{syntactic dialects} -- dialects that introduce only new derived syntactic forms. 
%Put another way, syntactic dialects can be specified by a context-independent expansion to the existing language that they are based on. 
For example, Ur/Web is a syntactic dialect of Ur (a language that itself descends from ML \cite{conf/pldi/Chlipala10}) that builds in derived forms for SQL queries, HTML elements and other datatypes used in the domain of web programming \cite{conf/popl/Chlipala15}. %Syntactic cost is often assessed qualitatively \cite{green1996usability}, though quantitative metrics can be defined. 
These are certainly not the only types of data that could similarly benefit from the availability of specialized derived forms -- we will consider a number of other examples in Sec. \ref{sec:syntax-examples}. 
Tools like Camlp4 \cite{ocaml-manual}, Sugar* \cite{erdweg2011sugarj,erdweg2013framework} and Racket \cite{Flatt:2012:CLR:2063176.2063195}, which we will discuss in Sec. \ref{sec:syntax-existing-approaches}, have lowered the engineering costs of constructing syntactic dialects in such situations, contributing to their proliferation. 

More advanced dialects introduce new type structure, going beyond what is possible with only new derived forms. As a simple example, the static and dynamic semantics of records cannot be expressed by context-independent expansion to a language with only nullary and binary products. Various languages have explored ``record-like'' primitives that go further, supporting functional update operators, width and depth coercions (sometimes implicit)%\cite{Cardelli:1984:SMI:1096.1098}
, methods, prototypic dispatch and other such ``semantic embellishments'' that in turn cannot be expressed by context-independent expansion to a language with only standard record types (we will detail an  example in Sec. \ref{sec:metamodules-motivating-examples}). OCaml primitively builds in the type structure of polymorphic variants, open datatypes and  operations that use format strings like $\mathtt{sprintf}$ \cite{ocaml-manual}. ReactiveML builds in primitives for functional reactive programming \cite{mandel2005reactiveml}. ML5 builds in high-level primitives for distributed programming based on a modal lambda calculus \cite{Murphy:2007:TDP:1793574.1793585}. Manticore \cite{conf/popl/FluetRRSX07} and AliceML  \cite{AliceLookingGlass} build in parallel programming primitives with a more elaborate type structure than is found in simpler accounts of parallelism. 
MLj builds in the type structure of the Java object system (motivated by a desire to interface safely and naturally with Java libraries) \cite{Benton:1999:IWW:317636.317791}. Other dialects do the same for other foreign languages, e.g. Furr and Foster describe a dialect of OCaml that builds in the type structure of C \cite{Furr:2005:CTS:1065010.1065019}. Tools like proof assistants and logical frameworks are used to specify and reason metatheoretically about dialects like these, and tools like compiler generators and language frameworks \cite{erdweg2013state} lower their implementation cost, again contributing to their proliferation. 

\subsection*{Dialects Considered Harmful}
One might view the ongoing proliferation of dialects described above as harmless or even positive, because programmers can simply choose the right language for the job at hand \cite{journals/stp/Ward94}. However, this ``dialect-oriented'' approach is, in an important sense, anti-modular: a library written in one dialect cannot, in general, safely and idiomatically interface with a library written in another dialect. 
As the study of MLj demonstrated, addressing this interoperability problem requires somehow ``combining'' the  dialects into a single language. However, in the most general setting where the dialects in question might be specified by judgements of arbitrary form, this is not a well-defined notion. Even if we restrict our interest to dialects specified using formalisms that do operationalize some notion of dialect combination, there is generally no guarantee that the combined dialect will conserve important syntactic and semantic properties that can be established about the dialects in isolation. %In other words, any putative ``combined language'' must formally be considered a  distinct system for which one must derive essentially all metatheorems of interest anew, guided only informally by those derived for the dialects individually. %There is no well-defined mechanism for constructing such a ``combined language'' in general. 
For example, consider two syntactic dialects, one specifying derived syntax for finite mappings, the other specifying a similar syntax for \emph{ordered} finite mappings. Though each dialect can be specified by an unambiguous grammar in isolation, when these grammars are na\"ively  combined by, for example, Camlp4,  ambiguities  arise. 
%It is thus infeasible to simply allow different contributors to a software system to choose their own favorite dialect for each component they are responsible for. 
%It it clear that dialects are better rhetorical devices than practical engineering artifacts. 
Due to this paucity of modular reasoning principles, the ``language-oriented'' approach is problematic for software development ``in the large''. %Large software projects and software ecosystems must pick a single language that does provide powerful modular reasoning principles and, to benefit from them, stay inside it.

Dialect designers have instead had to take a less direct approach to have an impact on large-scale software development: they have had to convince the designers in control of comparatively popular languages, like OCaml and Scala, to include some suitable variant of the primitives they've espoused into backwards compatible language revisions. %These decisions are increasingly influenced by community processes, e.g. the Scala Improvement Process.  %This approach concentrates power as well as responsibility over maintaining metatheoretic guarantees in the hands of a small group of language designers, though increasingly influenced by various community processes (e.g. the Scala Improvement Process). 
%Dialects thus serve the role of rhetorical vehicles for new ideas, rather than direct artifacts. 
%Over time, accepting such extensions has caused these languages to balloon in size. 
This \emph{ad hoc} approach is not sustainable, for three main reasons. First, as suggested by the diversity of examples given above, there are simply too  many potentially useful such primitives, and many of these are only relevant in relatively narrow application domains (for derived syntax, our group has  gathered initial data speaking to this \cite{TSLs}). Second, primitives introduced earlier in a language's lifespan can end up monopolizing finite ``syntactic resources'', forcing subsequent primitives to use ever more esoteric forms. And third, primitives that prove to be flawed in some way cannot be removed or changed without breaking backwards compatibility. %Because there is no data about how useful a construct is in practice until it is available in a major language, decisions about which constructs to include are often informed only by intuition.
%Recalling the words of  Reynolds, which are clearly as relevant today as they were almost half a century ago \cite{Reynolds70}: %This approach is antithetical to the ideal of a truly \emph{general-purpose language} described at the beginning of this section.
%\newpage
%\begin{quote}\textit{The recent development of programming languages suggests that the simul\-taneous achievement of simplicity 
%and generality in language design is a serious unsolved 
%problem.}\begin{flushright}--- John Reynolds (1970)\end{flushright}
%\end{quote}

This suggests that language designers should be quite careful and strive to keep general-purpose languages small and free of \emph{ad hoc} primitives. Given that we also want to avoid dialect formation, for the reasons described above, this leaves  two possible paths forward. One, exemplified (arguably) by SML, is to simply eschew further syntactic conveniences and innovative type structure and settle on the existing primitives, which might be considered to sit at a ``sweet spot'' in the overall language design space. % (accepting that in some circumstances, this trades away expressive power or leads to  high syntactic cost). 
The other path forward is to search for a small number of highly general primitives that allow us degrade many of the constructs that are built primitively into languages today insead to modularly composable library constructs. 
Encouragingly, primitives of this sort do occasionally arise. For example, a recent revision of OCaml added support for  ``generalized algebraic data types'' (GADTs), based on research on guarded recursive datatype constructors \cite{XiCheChe03}. Using GADTs, OCaml was able to move some of the \emph{ad hoc} machinery for typechecking operations that use format strings, like \texttt{sprintf}, out of the language and into a library (note, however, that syntactic machinery remains  built in). 

%Similarly, it recently introduced ``open datatypes'', which subsume its previous more specialized exception type, and captures many use cases for .

%Viewed ``dually'', one might equivalently ask for a language that builds in a core that is as small as possible, but provides expressive power comparable to languages with much larger cores. This is our goal in the work being proposed. 

%\vspace{-10px}
\section{Contributions}\label{sec:contributions}
%Our broad aim in the work being proposed is to introduce primitive language mechanisms that give library providers the ability to  express new syntactic expansions as well as new types and operators in a safe and modularly composable manner. 
In this work, we introduce primitive language constructs that take further steps down the second path just described. In particular, we introduce the following primitives:

\begin{enumerate}
\item \textbf{Typed syntax macros} (TSMs), introduced in Part \ref{part:syntax}, reduce the need to primitively build in derived concrete syntactic forms specific to library constructs (e.g. list syntax as in SML or XML syntax as in Scala and Ur/Web), by giving library providers static control over the parsing and expansion of delimited segments of textual syntax (at a specified type or parameterized family of types). We begin by showing explicitly-invoked TSMs, then show how to associate a privileged TSM with a type declaration and then rely on a local type inference scheme to invoke these implicitly.
\item \textbf{Metamodules}, introduced in Part \ref{part:metamodules}, reduce the need to primitively build in the type structure of constructs like records (and variants thereof),  labeled sums and other interesting constructs that we will introduce later by giving library providers programmatic ``hooks'' directly into the semantics, which are specified as a \emph{type-directed translation semantics} (introduced in Chap. \ref{chap:language-overview}). %For example, a library provider can implement the type structure of records with a metamodule that:
%\begin{enumerate}
%\item introduces a type constructor, \lstinline{record}, parameterized by finite mappings from labels to types, and defines, programmatically, a translation to unary and binary product types (which are built in to the internal language); and 
%\item introduces operators used to work with records, minimally record introduction and elimination (but perhaps also various functional update operators), and directly implements the logic governing their typechecking and translation to the IL (which builds in only nullary and binary products). 
%\end{enumerate}
%We will see direct analogies between ML-style modules (which our mechanisms also support) and metamodules later.
\end{enumerate} 
Both TSMs and metamodules make extensive use of \emph{static code generation} (also called \emph{static} or \emph{compile-time metaprogramming}), meaning that the relevant rules in the static semantics of the language call for the evaluation of \emph{static functions} that generate static representations of expressions and types. %Library providers write these static functions using the Verse \emph{static language} (SL). 
This design also has conceptual roots in earlier work on \emph{active libraries}, which similarly envisioned using compile-time computation to give library providers more control over aspects of the language and compilation process \cite{active-libraries-thesis}. %Maintaining a separation between the static (or ``compile-time'') phase and the dynamic (or ``run-time'') phase is an important facet of Verse's design. % static code generation. %We will  also introduce a simple variant of each of these primitives that leverages Verse's support for local type inference to further reduce syntactic cost in certain common situations. 

As vehicles for this work, we plan to formally specify small typed lambda calculi that capture each of the novel primitives that we introduce ``minimally''. We will also describe (but not formally specify) a new ``full-scale'' functional language called Verse.\footnote{We distinguish Verse from Wyvern, which is the language referred to in prior publications about some of the work that we will describe, because Wyvern is a group effort evolving independently in some important ways.} The reason we will not follow Standard ML \cite{mthm97-for-dart} in giving a complete formal specification of Verse is both to emphasize that the primitives we introduce can be considered for inclusion in a variety of language designs, and to avoid distracting the reader with specifications for ``orthogonal'' primitives that are already well-understood in the literature. %We anticipate that future full-scale language specifications will be able to combine the ideas  in the proposed work without trouble. %The purpose of the work being proposed is to serve as a reference for those interested in the new constructs we introduce, not to serve as a language specification. 
We will give a brief overview covering how these languages are organized in Chap. \ref{chap:language-overview}.

The main challenge in the design of these primitives will come in ensuring that they are metatheoretically well-behaved. If we are not careful, many of the problems  that arise when combining language dialects, discussed earlier, could simply shift into the semantics of these primitives.\footnote{This is why languages  like Verse are often called ``extensible languages'', though this is somewhat of a misnomer. The defining characteristic of an extensible language is that it \emph{doesn't} need to be extended in situations where other languages would need to be extended. We will avoid this somewhat confusing terminology.} Our main technical contributions will be in rigorously showing how to address these problems in a principled manner. In particular, syntactic conflicts will be impossible by construction and the semantics will validate code statically generated by TSMs and metamodules to maintain a strong \emph{hygienic type discipline} and, most uniquely, powerful \emph{modular reasoning principles}. In other words, library providers will have the ability to reason about the constructs that they have defined in isolation, and clients will be able to use them safely in any program context and in any combination, without the possibility of conflict.\footnote{This is not quite true -- simple naming conflicts can arise. We will tacitly assume that they are being avoided extrinsically, e.g. by using a URI-based naming scheme as in the Java ecosystem.} We will make these notions completely precise as we continue.



\subsubsection*{Thesis Statement}
In summary, we propose a thesis defending the following statement:
\begin{quote}
A functional programming language can give library providers the ability to %meta\-pro\-gram\-matic\-ally 
express new syntactic expansions and new types and operators atop a small type-theoretic internal language while maintaining a hygienic type discipline and modular reasoning principles. %These  primitives are  expressive enough to subsume the need for a variety of primitives that are, or would need to be, built in to comparable contemporary languages.
\end{quote}

\section{Disclaimers}
Before we continue, it may be useful to explicitly acknowledge that completely eliminating the need for dialects would indeed be asking for too much: certain design decisions are fundamentally incompatible with others or require coordination across a language design. We aim only to decrease the need for dialects.

It may also be useful to explicitly acknowledge that library providers could leverage the primitives we introduce   to define constructs that are in ``poor taste''. We  expect that in practice, languages like Verse will come with a standard library defining a carefully curated collection of standard constructs, as well as guidelines for advanced users regarding when it would be sensible to use the mechanisms we introduce (following the example of languages that support features like operator overloading or type classes \cite{Hall:1996:TCH:227699.227700}, which also have the potential for such ``abuse''). For most programmers, using Verse should not be substantially different from using a language like ML or one of its dialects.%The vast majority of programmers should not use the primitives that we introduce directly.

Finally, Verse is by design not a dependently-typed language like Coq, Agda or Idris, because these languages do not maintain a phase separation between ``compile-time'' and ``run-time''. This phase separation is useful for programming tasks (where one would like to be able to discover errors before running a program, particularly programs that may have an effect) but less so for theorem proving tasks (where it is primarily the fact that a pure expression is well-typed that is of interest, by the propositions-as-types principle). Verse is designed to be used for programming tasks where SML, OCaml, Haskell or Scala would be used today, not for advanced theorem proving tasks. That said, we conjecture that the primitives we describe could be added to languages like Gallina (the ``external language'' of the Coq proof assistant  \cite{Coq:manual}) or to the program extraction mechanisms of proof assistants like Coq with  modifications, but do not plan to pursue this line of research in this dissertation.
