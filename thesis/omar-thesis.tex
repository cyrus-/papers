%for a more compact document, add the option openany to avoid
%starting all chapters on odd numbered pages
\documentclass[12pt]{cmuthesis}

% This is a template for a CMU thesis.  It is 18 pages without any content :-)
% The source for this is pulled from a variety of sources and people.
% Here's a partial list of people who may or may have not contributed:
%
%        bnoble   = Brian Noble
%        caruana  = Rich Caruana
%        colohan  = Chris Colohan
%        jab      = Justin Boyan
%        josullvn = Joseph O'Sullivan
%        jrs      = Jonathan Shewchuk
%        kosak    = Corey Kosak
%        mjz      = Matt Zekauskas (mattz@cs)
%        pdinda   = Peter Dinda
%        pfr      = Patrick Riley
%        dkoes = David Koes (me)

% My main contribution is putting everything into a single class files and small
% template since I prefer this to some complicated sprawling directory tree with
% makefiles.
\newcommand{\todolater}[1]{{\color{magenta} TODO (Later): #1}}
\newcommand{\todo}[1]{{\color{red} TODO: #1}}

% some useful packages
\usepackage{times}
\usepackage{fullpage}
\usepackage{graphicx}
\usepackage{amsmath}
\usepackage[numbers,sort]{natbib}
\usepackage[backref,pageanchor=true,plainpages=false, pdfpagelabels, bookmarks,bookmarksnumbered,
pdfborder={0 0 0},  %removes outlines around hyper links in online display
]{hyperref}
\usepackage{subfigure}

% Approximately 1" margins, more space on binding side
%\usepackage[letterpaper,twoside,vscale=.8,hscale=.75,nomarginpar]{geometry}
%for general printing (not binding)
\usepackage[letterpaper,twoside,vscale=.8,hscale=.75,nomarginpar,hmarginratio=1:1]{geometry}

% Provides a draft mark at the top of the document. 
\draftstamp{\today}{DRAFT}


% I hate hyphenation.
\lefthyphenmin=7

\begin{document} 
\frontmatter

%initialize page style, so contents come out right (see bot) -mjz
\pagestyle{empty}

\title{ %% {\it \huge Thesis Proposal}\\
{\bf Modularly Programmable Syntax\\and Type Structure}}
\author{Cyrus Omar}
\date{\today}
\Year{2015}
%\trnumber{\todolater{TR Number}}

\committee{Jonathan Aldrich, Chair\\\todo{confirm rest of committee}}

%\support{\todolater{Support}
%Funded by DOE CSGF, NSF GRFP, NSF, NSA, ...?
%}
\disclaimer{}
%\permission{\todolater{Creative Commons license.}}

% copyright notice generated automatically from Year and author.
% permission added if \permission{} given.

%\keywords{\todolater{keywords}}

\maketitle

%\begin{dedication}
%\todolater{Dedication}
%\end{dedication}

\pagestyle{plain} % for toc, was empty

%% Obviously, it's probably a good idea to break the various sections of your thesis
%% into different files and input them into this file...

\begin{abstract}
\noindent
Functional programming languages like ML descend conceptually from minimal lambda calculi, but to be pragmatic, expose a concrete syntax and type structure to programmers of a  more elaborate design. 
Language designers have many viable choices along these dimensions, as evidenced by the diversity of dialects that continue to proliferate  around these languages. 
But such language dialects cannot be modularly combined, limiting the choices available to programmers. 
We describe and formally specify new language primitives designed to decrease the need for dialects by giving library providers the ability to safely and {modularly} control syntactic expansion, typechecking and translation to a minimal type-theoretic internal language.
\end{abstract}

%\begin{acknowledgments}
%\todolater{Acknowledgments}
%\end{acknowledgments}


\tableofcontents
\listoffigures
%\listoftables

\mainmatter

% \section*{Global TODO}
% Current:
% \begin{itemize}
% \item \todo{Rename OCaml to O'Caml (?)}
% \end{itemize}

% \noindent
% Before finalizing:
% \begin{itemize}
% \item \todolater{Add acknowledgements}
% \item \todolater{Add dedication}
% \item \todolater{Add keywords}
% \item \todolater{Add TR number}
% \item \todolater{Add CC license}
% \item \todolater{Add funding acknowledgement.}
% \end{itemize}
%% Double space document for easy review:
%\renewcommand{\baselinestretch}{1.66}\normalsize

% The other requirements Catherine has:
%
%  - avoid large margins.  She wants the thesis to use fewer pages, 
%    especially if it requires colour printing.
%
%  - The thesis should be formatted for double-sided printing.  This
%    means that all chapters, acknowledgements, table of contents, etc.
%    should start on odd numbered (right facing) pages.
%
%  - You need to use the department standard tech report title page.  I
%    have tried to ensure that the title page here conforms to this
%    standard.
%
%  - Use a nice serif font, such as Times Roman.  Sans serif looks bad.
%
% Other than that, just make it look good...



\section{Introduction}
\label{s:intro}

Domain-specific languages (DSLs) \cite{shellchapter, coldfusion,
  sinatra}
% have been widely-studied because they 
enable the expression of  
domain-specific ideas naturally and permit domain-specific
verification and compilation strategies.  For DSLs to reach their full
potential, however, it must be possible to easily use multiple
DSLs at once within a host general-purpose language (GPL), such that DSL code
and GPL glue code can inter-operate to form a complete application. More specifically, we propose the following essential design criteria:

\begin{itemize}

\item \emph {Composability}: It should be possible to use multiple DSLs and a GPL
%, in addition to a GPL,
within a single program unit.  %Here 
Within the text file-based paradigm familiar to most developers, this 
means including multiple DSLs within a single file.
 Moreover, it should be possible to embed code written in one DSL
  within another DSL without them requiring knowledge of each other or interfering with one another. So, more specifically, DSLs should be \emph{safely composable}.

\item \emph{Interoperability}: DSLs should be able to \emph{interoperate}. A necessary condition for interoperability is that it should be
  possible to pass and operate on values 
%such as functions or data structures
  defined in foreign DSLs. Additional requirements, such as the ability to do so naturally (rather than requiring a large amount of ``glue code'') and safely may also be relevant in many settings. 
%Moreover,
  Minimally, however, if one DSL defines a function or data structure satisfying an interface specified in
  a common interface description language (such as the type system of
  a host GPL), code written in another DSL should be able to use those
  values according to that interface, without requiring that the client DSL
  have knowledge of the details of the provider DSL or \emph{vice versa}.

\end{itemize}


%\item type system: one DSL depends on types defined by another DSL,
%  can use objects of that type in special ways [this goes beyond the
%    scope - save for a later paper]


In addition to composability and interoperability, we believe that to be most useful, a system supporting DSLs should have the following characteristics:

\begin{itemize}

\item \emph{Flexibility}: Have flexible enough DSL syntax to support a wide variety of text-based notations;

\item \emph{Identifiability}: Make it easy for programmers to identify which code is written in which DSL;

\item \emph{Consistency}: Encourage DSLs written in similar styles in order to enhance readability and learnability of each DSL;

\item \emph{Simplicity}: Keep the cost of defining a DSL low.
 
%\item Share conventions between DSLs and a host language, making each DSL easier for programmers to learn, helping programmers to identify which code is written in which DSL, and avoiding unintended conflicts between DSLs. \todo{Avoid conflicts (visual and real), enhance learnability}
 
%\item Reuse low-level mechanisms and design decisions from a host language, thereby reducing the cost of defining DSLs.  \todo{Easy to define DSLs}

\end{itemize}

We are developing a comprehensive approach that has the potential to meet these design requirements well, based on defining DSLs within a language specifically designed for extensibility from within.
The approach inherently supports multiple DSLs in the same file, and
supports interoperability because the DSLs are defined by translation
into a common internal language.  One key aspect of our proposed solution, and one of the focal points of this paper, is
that we restrict the syntax of DSLs in a few key ways that greatly simplify
the mechanism without significantly decreasing expressiveness:
\begin{itemize}

\item The host language and its DSLs share a tokenization and lexing 
  strategy, standardizing conventions for identifiers, operators,
  constants, and comments.  This avoids the cost of defining lexing
  within each DSL, avoids many kinds of low-level clashes between
  languages, and makes the composed language more readable. Note that this does not limit the ability of a DSL to define new keywords and other constructs.
 
\item Constructs in the host language and its DSLs are
  \emph{whitespace-delimited} at the top level, according to a particular strategy that we will describe.  Various forms of parentheses 
  can also serve as delimiters.  Thus, indentation levels or
  parenthesized expressions clearly delimit blocks that are governed by a particular 
  language.  This makes the boundaries of each DSL clear to
  the programmer and the compiler, enhancing usability and guaranteeing that 
  subtle conflicts cannot arise.
  
\end{itemize}

Within this syntactic framework, we then propose a novel syntax extension
mechanism in which the \emph{expected type of an expression}, rather than an explicit keyword, 
%, rather than a keyword, 
signals the switch to a new DSL that parses the tokenized and lexed code within a subsequent delimited block according to a type-associated grammar. The  more common keyword-directed strategy arises as a special case of this type-directed strategy. 
We will begin with some motivating examples (\S\ref{s:motivation}), describe our approach (\S\ref{s:approach}) and implementation (\S\ref{s:implementation}), and conclude with related work (\S\ref{s:related}).

% keep discussion of type-based parsing brief - active typing for parsing (only)
% avoids conflict with Cyrus' paper

%The rest of the paper is organized as follows.  The next section
%illustrates our approach by example, discussing the components of the
%solution in more detail. Section~\ref{s:approach} outlines our
%approach, shows a wider variety of examples and discusses variations
%of our approach.  Our in-progress implementation of the proposal is
%described in Section~\ref{s:implementation}.  Section~\ref{s:related}
%compares to related work, and Section~\ref{s:conclusion} concludes.

% with a discussion of future work.

% !TEX root = omar-thesis.tex
\chapter{Language Overview}\label{chap:language-overview}
\todo{integrate this point}
An important observation is that dialects that have substantially different type structure are still often implemented by translation to a simple and stable intermediate language. For example, the Glasgow Haskell Compiler's intermediate language has remained stable for many years, even as the number of Haskell dialects that the compiler supports has grown. Similarly, the primary Scala compiler uses the Java virtual machine (JVM) bytecode language as an intermediate language, which has also evolved relatively slowly.
\section{External Language}\label{sec:EL-overview}
\section{Internal Language}\label{sec:IL-overview}
\section{Static Language}\label{sec:SL-overview}
\section{Module Language}\label{sec:ML-overview}



\part{Modularly Programmable Syntax}
% !TEX root = omar-thesis.tex

\chapter{Motivation}
\section{Motivating Examples}
\subsection{Lists}
\subsection{HTML}
\subsection{Regular Expressions}
\subsection{Monadic Commands}
\subsection{Quasiquotation}
\section{Existing Approaches}
\subsection{Dynamic String Parsing}
\subsection{Direct Syntax Extension}

Related work I haven't mentioned yet:
\begin{itemize}
\item Fan: http://zhanghongbo.me/fan/start.html
\item Well-Typed Islands Parse Faster: \\\url{http://www.ccs.neu.edu/home/ejs/papers/tfp12-island.pdf}
\end{itemize}
\subsection{Term Rewriting}
% !TEX root = omar-thesis.tex
\chapter{Typed Syntax Macros}
\section{Examples}
\section{Minimal Formalization}
\section{Parameterized TSMs}

% !TEX root = omar-thesis.tex
\chapter{Type-Specific Languages}\label{chap:tsls}
\section{Examples}\label{sec:tsls-examples}
\section{Minimal Formalization}\label{sec:tsls-minimal-formalization}
\section{Parameterized TSLs}\label{sec:tsls-parameterized}


% !TEX root = omar-thesis.tex
\section{Conclusion \& Future Work}\label{chap:syntax-conclusion}



\part{Modularly Programmable Type Structure}
% !TEX root = omar-thesis.tex
\chapter{Motivation}
\section{Motivating Examples}
\section{Existing Approaches}


% !TEX root = omar-thesis.tex
\chapter{Metamodules}


% !TEX root = globaldsl13.tex
%\section{Conclusion} % and Future Work}
%\label{s:conclusion}
%
%In this paper, we described how extensible parsing in Wyvern makes for
%a solid platform to support whitespace-delimited, type-directed embedded DSLs. In the
%future, we aim to implement a wide variety of DSLs in Wyvern tweaking
%our approach and implementation thereof to provide a comprehensive example of
%supporting multiple interacting DSLs in a safe and easy-to-use manner.
%
% \todo{tie features to goals}

% \todo{implementation and validation plans}

\section*{Acknowledgements}
We thank the anonymous reviewers for helpful comments, and acknowledge the support of the Department of Defense and the Air Force Research Laboratory. CO is supported by the NSF Graduate Research Fellowship.

\documentclass{llncs}
\usepackage{amsmath}
\usepackage{llncsdoc}
\usepackage{amssymb} 
%\usepackage{amsthm}
\usepackage{ stmaryrd }
\usepackage{mathpartir}
%\usepackage{cite}
%\renewcommand{\citepunct}{,\,} % IEEEtran wants to use ],\,[ for this but that looks dumb...

\newcommand{\atlam}{@$\lambda$}

% Generic
\newcommand{\bindin}[2]{#1;~#2}
\newcommand{\pipe}{~\text{\large $\vert$}~}
\newcommand{\splat}[3]{#1_{#2};\ldots;#1_{#3}}
\newcommand{\splatC}[3]{#1_{#2}~~~~\cdots~~~~#1_{#3}}
\newcommand{\splatTwo}[4]{#1_{#3}#2_{#3},~\ldots~, #1_{#4}#2_{#4}}
\newcommand{\substn}[2]{[#1]#2}
\newcommand{\subst}[3]{\substn{#1/#2}{#3}}
\newcommand{\entails}[2]{#1 \vdash #2}

% Programs
\newcommand{\progsort}{\rho}
\newcommand{\pfam}[2]{\bindin{#1}{#2}}
\newcommand{\pdef}[4]{\bindin{{\sf def}~\tvar{#1}:#2=#3}{#4}}

% Families
\newcommand{\fvar}[1]{\textsc{#1}}

\newcommand{\family}[6]{{\sf family}~\fvar{#1}[#2]\sim \tvar{#5}.#6~\{#3\}}
\newcommand{\familyDf}{\family{Fam}{\kappaidx}{\opsort}{\opsigsort}{i}{\taurep}}

% Operators
\newcommand{\opsort}{\theta}
\newcommand{\opsigsort}{\Theta}
\newcommand{\opvar}[1]{\textbf{\textit{#1}}}

% Expressions
\newcommand{\evar}[1]{#1}
\newcommand{\elam}[3]{\lambda #1{:}#2.#3}
\newcommand{\eapp}[2]{{#1~#2}}
\newcommand{\eopapp}[2]{\eop{Arrow}{ap}{\tunit}{#1; #2}}
\newcommand{\eop}[4]{{\fvar{#1}.\opvar{#2}\langle#3\rangle(#4)}}
\newcommand{\elet}[4]{{\sf let}~#1 : #2 = #3~{\sf in}~#4}

% Type-Level Terms
\newcommand{\tvar}[1]{{\textbf{#1}}}

\newcommand{\tlam}[3]{\lambda \tvar{#1}{:}{#2}.#3}
\newcommand{\tapp}[2]{#1~#2}
\newcommand{\tifeq}[5]{{\sf if}~#1\equiv_{#3}#2{\sf ~then~}#4~{\sf else}~#5}
\newcommand{\tstr}[1]{\textit{``#1''}}
\newcommand{\tunit}{()}
\newcommand{\tpair}[2]{(#1, #2)}
\newcommand{\tfst}[1]{{\sf fst}(#1)}
\newcommand{\tsnd}[1]{{\sf snd}(#1)}
\newcommand{\tnil}[1]{[]_#1}
\newcommand{\tcons}[2]{#1 :: #2}
\newcommand{\tfold}[6]{{\sf fold}(#1; #2; \tvar{#3},\tvar{#4},\tvar{#5}.#6)}

\newcommand{\tfamSpec}[5]{{\sf family}~[#2]~::~#3~\{#4 : #5\}}
\newcommand{\tfamSpecStd}{\tfamSpec{fam}{\kappaidx}{\taurep}{\theta}{\Theta}}

\newcommand{\ttype}[2]{\fvar{#1}\langle#2\rangle}
\newcommand{\ttypestd}{\ttype{Fam}{\tau}}
\newcommand{\tfamcase}[5]{{\sf case}~#1~{\sf of}~\fvar{#2}\langle\tvar{#3}\rangle\Rightarrow#4~{\sf ow}~#5}
\newcommand{\trepof}[1]{{\sf rep}(#1)}

\newcommand{\tden}[2]{\llbracket #1~{\sf as}~#2 \rrbracket}
\newcommand{\ttypeof}[1]{{\sf typeof}~#1}
\newcommand{\tvalof}[2]{{\sf trans}(\tden{#1}{#2})}
\newcommand{\terr}{{\sf err}}
\newcommand{\tdencase}[5]{{\sf case}~#1~{\sf of}~\tden{\tvar{#2}}{\tvar{#3}}\Rightarrow#4~{\sf ow}~#5}

\newcommand{\titerm}[1]{\triangledown(#1)}
\newcommand{\titype}[1]{\blacktriangledown(#1)}

\newcommand{\tconst}[1]{{\sf const}(#1)}
\newcommand{\tOp}[1]{{\sf op}(#1)}

\newcommand{\tprog}[1]{{\sf program}(#1)}

\newcommand{\topsempty}{\cdot}
\newcommand{\tops}[5]{\opvar{#1}[#2](\tvar{#3},\tvar{#4}.#5)}
\newcommand{\topp}[2]{#1; #2}
\newcommand{\Tops}[2]{\tvar{#1} : #2}
\newcommand{\Topp}[2]{#1; #2}
\newcommand{\kOpEmpty}{\cdot}
\newcommand{\kOpS}[2]{\opvar{#1}[#2]}
\newcommand{\kOp}[3]{#1; \kOpS{#2}{#3}}

% Contexts
\newcommand{\fCtx}{\Sigma}
\newcommand{\itvarCtx}{\Omega}
\newcommand{\iCtx}{\Theta}
\newcommand{\eCtx}{\Gamma}
\newcommand{\etvarCtx}{\Omega}
\newcommand{\errCtx}{\mathcal{E}}
\newcommand{\famEvalCtx}{\Xi}

% Judgments
\newcommand{\emptyctx}{\emptyset}
\newcommand{\tvarCtx}{\Delta}
\newcommand{\tvarCtxX}[2]{\Delta, {\tvar{#1}} : {#2}}
\newcommand{\fvarCtx}{\Sigma}
\newcommand{\fvarCtxX}{\Sigma, \fvarOfType{Fam}{\kappaidx}{\Theta}}
\newcommand{\eivarCtx}{\Omega}
\newcommand{\eivarCtxX}[1]{\Omega, \evar{#1}}

\newcommand{\kEntails}[3]{#1 \vdash_{#2} #3}

\newcommand{\progProg}[1]{#1~{\tt prog}}
\newcommand{\progOK}[3]{\kEntails{#1}{#2}{\progProg{#3}}}
\newcommand{\progOKX}[1]{\progOK{\tvarCtx}{\fvarCtx}{#1}}

\newcommand{\fvarOfType}[3]{\fvar{#1}[#2,#3]}
\newcommand{\fvarOfTypeDf}{\fvarOfType{Fam}{\kappaidx}{\Theta}}

\newcommand{\opOfType}[2]{#1 : #2}
\newcommand{\opType}[4]{\kEntails{#1}{#2}{\opOfType{#3}{#4}}}

\newcommand{\tOfKind}[2]{#1 : #2}
\newcommand{\tKind}[4]{\kEntails{#1}{#2}{\tOfKind{#3}{#4}}}
\newcommand{\tKindX}[2]{\tKind{\tvarCtx}{\fvarCtx}{#1}{#2}}

\newcommand{\isExpr}[1]{#1~\texttt{expr}}
\newcommand{\exprOK}[4]{#1~#2 \vdash_{#3} \isExpr{#4}}
\newcommand{\exprOKX}[1]{\exprOK{\tvarCtx}{\eivarCtx}{\fvarCtx}{#1}}

\newcommand{\isIterm}[4]{#1~#2 \vdash_{#3} #4~\texttt{iterm}}
\newcommand{\isItermX}[1]{\isIterm{\tvarCtx}{\eivarCtx}{\fvarCtx}{#1}}

\newcommand{\isItype}[3]{#1 \vdash_{#2} #3~\texttt{itype}}
\newcommand{\isItypeX}[1]{\isItype{\tvarCtx}{\fvarCtx}{#1}}

\newcommand{\tEvalX}[2]{#1 \Downarrow #2}
\newcommand{\tiEvalX}[2]{#1 \curlyveedownarrow #2}

\newcommand{\fvalCtx}{\Phi}
\newcommand{\fvalCtxX}[1]{\Phi, #1}
\newcommand{\fval}[4]{\fvar{#1}[#2, \tvar{#3}.#4]}
\newcommand{\fvalDf}{\fval{Fam}{\theta}{i}{\tau}}

\newcommand{\pcompiles}[3]{\vdash_{#1} #2 \Longrightarrow #3}
\newcommand{\pcompilesX}[1]{\pcompiles{\fvalCtx}{#1}{\gamma}}

\newcommand{\ptcc}[2]{#1 \longrightarrow #2}

\newcommand{\etCtx}{\Gamma}
\newcommand{\etCtxX}[2]{\etCtx, \evar{#1} : #2}

\newcommand{\gtCtx}{\Psi}
\newcommand{\gtCtxX}[2]{\gtCtx, \evar{#1} : #2}

\newcommand{\ecompiles}[5]{#1 \vdash_{#2} #3 : #4 \Longrightarrow #5}
\newcommand{\ecompilesX}[3]{\ecompiles{\etCtx}{\fvalCtx}{#1}{#2}{#3}}

\newcommand{\delfromtau}[4]{\vdash^{\fvar{#1}}_{#2} #3 \sim #4}
\newcommand{\tauisdel}[5]{\vdash^{\fvar{#1}}_{#2} \ttype{#3}{#4} \sim #5}

\newcommand{\checkRC}[5]{#1 \vdash^{\fvar{#2}}_{#3} #4 \sim #5}
\newcommand{\checkRCX}[2]{\checkRC{\gtCtx}{Fam}{\fvalCtx}{#1}{#2}}

\newcommand{\erase}[3]{\vdash_{#1} #2 \leadsto #3}
\newcommand{\eraseX}[2]{\erase{\fvalCtx}{#1}{#2}}

\newcommand{\eCtxTogCtx}[4]{\vdash^{\fvar{#1}}_{#2} #3 \sim #4}
\newcommand{\eCtxTogCtxX}[2]{\eCtxTogCtx{Fam}{\fvalCtx}{#1}{#2}}

\newcommand{\ddbar}[4]{\vdash^{\fvar{#1}}_{#2} #3 \sim #4}
\newcommand{\ddbarX}[2]{\ddbar{Fam}{\fvalCtx}{#1}{#2}}

\newcommand{\iType}[3]{#1 \vdash #2 : #3}

\newcommand{\gtCtxH}{\hat{\gtCtx}}
\newcommand{\gtCtxHX}[2]{\gtCtxH, \evar{#1} : #2}

%
\newcommand{\fSpec}[3]{\tof{\fvar{#1}}{\kFam{#2}{#3}}}
\newcommand{\fSpecStd}{\fSpec{fam}{\kappaidx}{\Theta}}

% \tau
\newcommand{\taut}[1]{{\tau_{\text{#1}}}}
\newcommand{\tautype}{\taut{type}}
\newcommand{\tautrans}{\taut{trans}}
\newcommand{\tauproof}{\taut{proof}}
\newcommand{\tauidx}{\taut{idx}}
\newcommand{\taui}{\taut{i}}
\newcommand{\tauidxn}[1]{\taut{idx,#1}}
\newcommand{\taurep}{\taut{rep}}
\newcommand{\taurepn}[1]{\taut{rep,#1}}
\newcommand{\tauden}{\taut{den}}
\newcommand{\tauIT}{\taut{IT}}
\newcommand{\tauarrow}{\taut{arrow}}
\newcommand{\tauprod}{\taut{prod}}
\newcommand{\tauint}{\taut{int}}
\newcommand{\taubool}{\taut{bool}}
\newcommand{\tauprog}{\taut{prog}}
\newcommand{\tauiterm}{\taut{iterm}}
\newcommand{\tauval}{\taut{val}}
\newcommand{\tauop}{\taut{op}}

% \gamma
\newcommand{\ghat}{\hat{\gamma}}

% \sigma
\newcommand{\delt}[1]{\sigma_{\text{#1}}}
\newcommand{\delrep}{\delt{rep}}
\newcommand{\dhat}{\hat{\sigma}}
\newcommand{\dbar}{\bar{\sigma}}

% \kappa
\newcommand{\kappat}[1]{\kappa_{\text{#1}}}
\newcommand{\kappaidx}{\kappat{idx}}
\newcommand{\kappai}{\kappat{i}}

% Types

% IL terms
\newcommand{\ivar}[1]{\textrm{#1}}
\newcommand{\ilam}[3]{\lambda #1{:}#2.#3}
\newcommand{\ifix}[3]{{\sf fix~}#1{:}#2~{\sf is}~#3}
\newcommand{\iapp}[2]{#1~#2}
\newcommand{\ipair}[2]{(#1, #2)}
\newcommand{\ifst}[1]{{\sf fst}(#1)}
\newcommand{\isnd}[1]{{\sf snd}(#1)}
\newcommand{\iintlit}{\bar{
\textrm{z}}}
\newcommand{\iop}[2]{#1 \oplus #2}
\newcommand{\iIfEq}[5]{{\sf if}~#1\equiv_{#3}#2{\sf ~then~}#4~{\sf else}~#5}
\newcommand{\mvalof}[1]{{\sf valof}(#1)}
\newcommand{\iup}[1]{\vartriangle\hspace{-2.5pt}(#1)}

% Internal Types
\newcommand{\darrow}[2]{#1\rightarrow#2}
\newcommand{\dint}{\mathbb{Z}}
\newcommand{\dpair}[2]{#1\times#2}
\newcommand{\dup}[1]{\blacktriangle(#1)}
\newcommand{\drepof}[1]{{\sf repof}(#1)}

% Kinds
\newcommand{\kvar}[1]{\textrm{#1}}
\newcommand{\karrow}[2]{#1\rightarrow{#2}}
\newcommand{\kforall}[2]{\forall \kvar{#1}.#2}
\newcommand{\kstr}{\textsf{Str}}
\newcommand{\kunit}{\textsf{1}}
\newcommand{\kpair}[2]{#1 \times #2}
\newcommand{\klist}[1]{\textsf{list}[#1]}
\newcommand{\kTypeBlur}{\star}
\newcommand{\kDen}{\textsf{Den}}
\newcommand{\kIType}{\textsf{ITy}}
\newcommand{\kITerm}{\textsf{ITm}}

% Judgements
\newcommand{\tof}[2]{#1 : #2}
\newcommand{\mtof}[2]{#1 :: #2}
\newcommand{\tentails}[2]{#1 \vdash #2}
\newcommand{\tentailst}[3]{\tentails{#1}{\tof{#2}{#3}}}
\newcommand{\tStdCtx}{\fCtx~\tvarCtx}
\newcommand{\tCtxXF}[1]{\fCtx, #1~\tvarCtx}
\newcommand{\tCtxXT}[1]{\fCtx~\tvarCtx, #1}
%\newcommand{\tCtxXL}[1]{\fCtx~\tvarCtx~\lvarCtx, #1}
\newcommand{\tentailsX}[1]{\tentails{\tStdCtx}{#1}}
\newcommand{\tentailsXt}[2]{\tentailsX{\tof{#1}{#2}}}
\newcommand{\kentails}[2]{#1 \vdash #2}
\newcommand{\kentailsX}[1]{\kentails{\fCtx}{#1}}
\newcommand{\iMkCtx}[3]{#1~#2~#3}
\newcommand{\iStdCtx}{\iMkCtx{\fCtx}{\tvarCtx}{\itvarCtx}}
\newcommand{\ientails}[2]{#1 \vdash #2}
\newcommand{\ientailsX}[1]{\entails{\iStdCtx}{#1}}
\newcommand{\casemap}[2]{#1 : #2}
\newcommand{\mentails}[3]{#1, #2 \vdash #3}
\newcommand{\mentailsX}[1]{\mentails{\tvarCtx}{\itvarCtx}{#1}}
\newcommand{\eentails}[4]{#1~#2~#3 \vdash #4}
\newcommand{\eentailsX}[1]{\eentails{\fCtx}{\tvarCtx}{\etvarCtx}{#1}}
\newcommand{\mtentails}[2]{#1 \vdash #2}
\newcommand{\mtentailsX}[1]{\mtentails{\iCtx}{#1}}
\newcommand{\mtentailsXt}[2]{\mtentails{\iCtx}{\mtof{#1}{#2}}}
\newcommand{\kSimple}[1]{#1~{\sf simple}}
\newcommand{\Tentails}[3]{#1 \vdash_{#2} #3}
\newcommand{\TentailsX}[1]{\Tentails{\fCtx}{\fvar{fam}}{#1}}
\newcommand{\kEq}[1]{#1~\texttt{eq}}

% Verification and Translation
\newcommand{\translates}[4]{\entails{#1}{#2 \longrightarrow \tden{#3}{#4}}}

% Compilation Semantics
\newcommand{\compiless}[3]{#1 \Longrightarrow \tden{#2}{#3}}
\newcommand{\compiles}[3]{#1 \Longrightarrow \tden{#2}{#3}}
%\newcommand{\translates}[6]{\entails{#1}{\translatesTo{#2}{#3}{#4}{#5}{#6}}}
\newcommand{\translatesTo}[5]{#1 \longrightarrow \tden{#2}{\ttype{#3}{#4}{#5}{6}{7}}}
\newcommand{\translatesX}[5]{\translates{\eCtx}{#1}{#2}{#3}{#4}{#5}}


\renewcommand{\ttdefault}{txtt}
\usepackage{alltt}
\usepackage{listings}
\lstset{language=ML,
showstringspaces=false,
basicstyle=\ttfamily\footnotesize,
morekeywords={newcase,extends}}

\usepackage{float}
\floatstyle{ruled}
\newfloat{codelisting}{tp}{lop}
\floatname{codelisting}{Listing}

\usepackage{url}
% url.sty was written by Donald Arseneau. It provides better support for
% handling and breaking URLs. url.sty is already installed on most LaTeX
% systems. The latest version can be obtained at:
% http://www.ctan.org/tex-archive/macros/latex/contrib/misc/
% Read the url.sty source comments for usage information. Basically,
% \url{my_url_here}.

\usepackage{placeins}

%\lefthyphenmin=4
\hyphenation{op-tical net-works semi-conduc-tor}

\usepackage{todonotes}

\begin{document}
\mainmatter  % start of an individual contribution

% first the title is needed
\title{Active Typechecking and Translation: A Safe Language-Internal Extension Mechanism}
\subtitle{Appendix}

% a short form should be given in case it is too long for the running head
\titlerunning{Lecture Notes in Computer Science: Authors' Instructions}

% the name(s) of the author(s) follow(s) next
%
% NB: Chinese authors should write their first names(s) in front of
% their surnames. This ensures that the names appear correctly in
% the running heads and the author index.
%
\author{Cyrus Omar%
\and Jonathan Aldrich}
%
\authorrunning{Lecture Notes in Computer Science: Authors' Instructions}
% (feature abused for this document to repeat the title also on left hand pages)

% the affiliations are given next; don't give your e-mail address
% unless you accept that it will be published
\institute{Carnegie Mellon University, Pittsburgh, PA 15213, USA\\
\texttt{\{comar,aldrich\}@cs.cmu.edu}
}

%
% NB: a more complex sample for affiliations and the mapping to the
% corresponding authors can be found in the file "llncs.dem"
% (search for the string "\mainmatter" where a contribution starts).
% "llncs.dem" accompanies the document class "llncs.cls".
%

\toctitle{Lecture Notes in Computer Science}
\tocauthor{Authors' Instructions}
\maketitle

\appendix
\section{\atlam}
\subsection{Helper Functions}\label{helper}
The following helper functions are useful for working with lists of denotations:
\[\begin{array}{rcl}
\tvar{const} & := & (\tlam{a}{\klist{\kDen}}{\tlam{d}{\kDen}{\tfold{\tvar{a}}{\tvar{d}}{\_}{\_}{\_}{\terr}}})\\
\tvar{pop} & := & (\tlam{a}{\klist{\kDen}}{\tlam{f}{\karrow{\kITerm}{\karrow{\kTypeBlur}{\karrow{\klist{\kDen}}{\kDen}}}}{\\& & \quad\tfold{\tvar{a}}{\terr}{d}{b}{\_}{\tdencase{\tvar{d}}{x}{t}{
	\tapp{\tapp{\tapp{\tvar{f}}{\tvar{x}}}{\tvar{t}}}{\tvar{b}}}{\terr}}}})\\
\tvar{pop\_final} & := & (\tlam{a}{\klist{\kDen}}{\tlam{f}{\karrow{\kITerm}{\karrow{\kTypeBlur}{\kDen}}}{\\& & \quad\tfold{\tvar{a}}{\terr}{d}{b}{\_}{\\& & \quad \tfold{\tvar{b}}{\tdencase{\tvar{d}}{x}{t}{\tapp{\tapp{\tvar{f}}{\tvar{x}}}{\tvar{t}}}{\terr}}{\_}{\_}{\_}{\terr}}}})
\end{array}\]
\noindent
The following helper function is useful for checking the equivalence of two types before producing a denotation. If the two types are not equivalent, an error is returned.
\[\begin{array}{rcl}
\tvar{check\_type} & := & (\tlam{t1}{\kTypeBlur}{\tlam{t2}{\kTypeBlur}{\tlam{d}{\kDen}{\tifeq{\tvar{t1}}{\tvar{t2}}{\kTypeBlur}{\tvar{d}}{\terr}}}})
\end{array}\]

\subsection{Definition of Arrow types}
The definition of the $\fvar{Arrow}$ type family is built into \atlam~as follows:
\[
\begin{array}{rcl}
\Sigma_0 & := & \fvarOfType{Arrow}{\kpair{\kTypeBlur}{\kTypeBlur}}{\kOpS{ap}{\kunit}}\\
\Phi_0 & := & \fval{Arrow}{\theta_0}{i}{\titype{\darrow{\trepof{\tfst{\tvar{i}}}}{\trepof{\tsnd{\tvar{i}}}}}}\\
\theta_0 & := & \tops{ap}{\kunit}{i}{a}{
	\tapp{\tapp{\tvar{pop}}{\tvar{a}}}{\tlam{x1}{\kDen}{\tlam{t1}{\kTypeBlur}{\tlam{b}{\klist{\kDen}}{
		\\ & & \quad \tapp{\tapp{\tvar{pop\_final}}{\tvar{b}}}{\tlam{x2}{\kDen}{\tlam{t2}{\kTypeBlur}{
		\\ & & \quad \tfamcase{\tvar{t1}}{Arrow}{j}{
		\\ & & \quad \quad \tapp{\tapp{\tapp{\tvar{check\_type}}{\tfst{\tvar{j}}}}{\tvar{t2}}}{\tden{\titerm{\iapp{\iup{\tvar{x1}}}{\iup{\tvar{x2}}}}}{\tsnd{\tvar{j}}}}
		\\ & & \quad\hspace{-.33em}}{\terr}
		}}}
	}}}}
	%\tapp{\tapp{\tvar{check\_type}}{\tvar{a}}}{c}
}\\
\Xi_0 & := & \fvar{Arrow}
\end{array}
\]

\subsection{Deabstraction}\label{deabstraction}
%\begin{figure}[t]
\small
$\fbox{\inferrule{}{\eraseX{\sigma}{\sigma'}}}$
\begin{mathpar}
\inferrule[deabs-int]{ }{
	\eraseX{\dint}{\dint}
}

\inferrule[deabs-arrow]{
	\eraseX{\sigma_1}{\sigma_1'}\\
	\eraseX{\sigma_2}{\sigma_2'}
}{
	\eraseX{\darrow{\sigma_1}{\sigma_2}}{\darrow{\sigma_1'}{\sigma_2'}}
}

\inferrule[deabs-prod]{
	\eraseX{\sigma_1}{\sigma_1'}\\
	\eraseX{\sigma_2}{\sigma_2'}
}{
	\eraseX{\dpair{\sigma_1}{\sigma_2}}{\dpair{\sigma_1'}{\sigma_2'}}
}

\inferrule[deabs-rep]{
	\fvalDf \in \fvalCtx\\
	\tEvalX{[\tauidx/\tvar{i}]\tau}{\titype{\sigma}}\\
	\ddbar{\Xi_0}{\fvalCtx}{\sigma}{\sigma'}\\
	\eraseX{\sigma'}{\sigma''}
}{
	\eraseX{\trepof{\ttype{Fam}{\tauidx}}}{\sigma''}
}
\end{mathpar}
$\fbox{\inferrule{}{\eraseX{\gamma}{\gamma'}}}$
\begin{mathpar}
\inferrule[deabs-var]{ }{
	\eraseX{\evar{x}}{\evar{x}}
}

\inferrule[deabs-lam]{
	\ddbar{\Xi_0}{\fvalCtx}{\sigma}{\sigma'}\\
	\eraseX{\sigma'}{\sigma''}\\\\
	\eraseX{\gamma}{\gamma'}
}{
	\eraseX{\ilam{x}{\sigma}{\gamma}}{\ilam{x}{\sigma''}{\gamma'}}
}

%\inferrule[deabs ap]{
%	\eraseX{\gamma_1}{\gamma_1}\\
%	\eraseX{\gamma_2}{\gamma_2}
%}{
%	\eraseX{\iapp{\gamma_1}{\gamma_2}}{\iapp{\gamma_1}{\gamma_2}}
%}
%
\inferrule[deabs-fix]{
	\ddbar{\Xi_0}{\fvalCtx}{\sigma}{\sigma'}\\
	\eraseX{\sigma'}{\sigma''}\\\\
	\eraseX{\gamma}{\gamma'}
}{
	\eraseX{\ifix{x}{\sigma}{\gamma}}{\ifix{x}{\sigma''}{\gamma'}}
}
\\
\text{\color{gray} (omitted forms have trivially recursive rules)}
\\
\inferrule[deabs-cancel]{
	\eraseX{\gamma}{\gamma'}
}{
	\eraseX{\iup{\titerm{\gamma}}}{\gamma'}
}

\inferrule[deabs-trans]{
	\eraseX{\gamma}{\gamma'}
}{
	\eraseX{\tvalof{\titerm{\gamma}}{\ttype{Fam}{\tauidx}}}{\gamma'}
}
\end{mathpar}
%\caption{\small Deabstraction rules}
%\label{deabs}
%\end{figure}

\section{Examples}
\subsection{G\"odel's T}\label{T}
The implementation of G\"odel's T in \atlam~encodes the following static and dynamic semantics.
\subsubsection{Statics}
\newcommand{\lam}[3]{\lambda #1{:}#2.#3}
\newcommand{\ap}[2]{#1~#2}
\newcommand{\z}{\textsf{z}}
\newcommand{\s}[1]{\textsf{s}(#1)}
\newcommand{\natrec}[3]{\textsf{rec}(#1; #2; #3)}
\newcommand{\pair}[2]{(#1, #2)}
\newcommand{\fst}[1]{\textsf{fst}(#1)}
\newcommand{\snd}[1]{\textsf{snd}(#1)}

\newcommand{\tArrow}[2]{#1 \rightarrow #2}
\newcommand{\nat}{\texttt{nat}}
\renewcommand{\prod}[2]{#1 \times #2}

%\newcommand{\eCtx}{\Gamma}
\newcommand{\eCtxX}[3]{#1, #2 : #3}
\newcommand{\jet}[3]{#1 \vdash #2 : #3}
\newcommand{\jetX}[2]{\jet{\eCtx}{#1}{#2}}
%\begin{figure}
%\small
\begin{mathpar}
\small
\inferrule[var]{ }{
	\jet{\eCtxX{\eCtx}{x}{\tau}}{x}{\tau}
}

\inferrule[arrow-intro]{
	\jet{\eCtxX{\eCtx}{x}{\tau}}{e}{\tau'}	
}{
	\jetX{\lam{x}{\tau}{e}}{\tArrow{\tau}{\tau'}}
}

\inferrule[arrow-elim]{
	\jetX{e_1}{\tArrow{\tau}{\tau'}}\\
	\jetX{e_2}{\tau}
}{
	\jetX{\ap{e_1}{e_2}}{\tau'}
}

\inferrule[nat-intro-z]{ }{
	\jetX{\z}{\nat}
}
~~~~~~~~
\inferrule[nat-intro-s]{
	\jetX{e}{\nat}
}{
	\jetX{\s{e}}{\nat}
}
~~~~~~~~
\inferrule[nat-elim]{
	\jetX{e_1}{\nat}\\
	\jetX{e_2}{\tau}\\
	\jetX{e_3}{\tArrow{\nat}{\tArrow{\tau}{\tau}}}
}{
	\jetX{\natrec{e_1}{e_2}{e_3}}{\tau}
}
\end{mathpar}
%\caption{Static semantics of G\"odel's T}
%\end{figure}
\subsubsection{Dynamics}
\newcommand{\steps}[2]{#1 \mapsto #2}
\newcommand{\val}[1]{#1~\texttt{val}}
\begin{mathpar}
\inferrule[lam-val]{ }{
	\val{\lam{x}{\tau}{e}}
}

\inferrule[ap-step-l]{
	\steps{e_1}{e_1'}
}{
	\steps{\ap{e_1}{e_2}}{\ap{e_1'}{e_2}}
}

\inferrule[ap-step-r]{
	\steps{e_2}{e_2'}
}{
	\steps{\ap{\lam{x}{\tau}{e}}{e_2}}{\ap{\lam{x}{\tau}{e}}{e_2'}}
}

\inferrule[ap-subst]{
	\val{e_2}
}{
	\steps{\ap{\lam{x}{\tau}{e}}{e_2}}{[e_2/x]e}
}
\\
\inferrule[z-val]{ }{
	\val{\z}
}

\inferrule[s-step]{
	\steps{e}{e'}
}{
	\steps{\s{e}}{\s{e'}}
}

\inferrule[s-val]{
	\val{e}
}{
	\val{\s{e}}
}

\inferrule[rec-step]{
	\steps{e_1}{e_1'}
}{
	\steps{\natrec{e_1}{e_2}{e_3}}{\natrec{e_1'}{e_2}{e_3}}
}

\inferrule[rec-z]{ }{
	\steps{\natrec{\z}{e_2}{e_3}}{e_2}
}

\inferrule[rec-s]{ }{
	\steps{\natrec{\s{e}}{e_2}{e_3}}{\ap{\ap{e_3}{e}}{{\natrec{e}{e_2}{e_3}}}}
}
\end{mathpar}
\subsection{$n$-tuples}
The static and dynamic semantics of $n$-tuples (that is, $n$-ary products) shown below can be encoded in \atlam. Notice that this judgemental description of $n$-tuples relies on elipses frequently. This  correspond to folds over the list of types that an $n$-tuple is indexed by, and underscores the need for a more concrete functional language for extensions, rather than a simple declarative scheme which is only completely precise and decidable when complex side conditions, elipses and search schemes are not necessary.
\newcommand{\pr}[2]{\textsf{pr}[#1](#2)}
\subsubsection{Statics}
\begin{mathpar}
\inferrule[$n$-tuple-intro]{
	\jetX{e_1}{\tau_1}\\
	\cdots\\
	\jetX{e_n}{\tau_n}
}{
	\jetX{(e_1, \ldots, e_n)}{(\tau_1, \ldots, \tau_n)}
}~(n \geq 0)

\inferrule[$n$-tuple-elim]{
	\jetX{e}{(\tau_1, \ldots, \tau_n)}
}{
	\jetX{\pr{i}{e}}{\tau_i}
}~(1 \leq i \leq n)
\end{mathpar}
\subsubsection{Dynamics}
\begin{mathpar}
\inferrule[$n$-tuple-step]{
	\val{e_1}\\
	\cdots\\
	\val{e_{i-1}}\\
	\steps{e_i}{e_i'}
}{
	\steps{(e_1,  \ldots, e_i, \ldots, e_n)}{(e_1, \ldots, e_i', \ldots, e_n)}
}

\inferrule[$n$-tuple-val]{
	\val{e_1}\\
	\cdots\\
	\val{e_n}
}{
	\val{(e_1, \ldots, e_n)}
}

\inferrule[proj-$i$-step]{
	\steps{e}{e'}
}{
	\steps{\pr{i}{e}}{\pr{i}{e'}}
}~(1 \leq i \leq n)

\inferrule[proj-$i$-of-$n$]{
	\val{(e_1, \ldots, e_n)}
}{
	\steps{\pr{i}{(e_1, \ldots, e_n)}}{e_i}
}~(1 \leq i \leq n)
\end{mathpar}
\subsubsection{Implementation in \atlam}
$n$-tuples are implemented differently depending on $n$. For $n=0$ (that is, the unit value), the integer $0$ is used. For $n=1$, the tuple only contains 1 element so it is represented without any additional run-time adornment. For $n > 1$, nested pairs are used.
\begin{flalign}
 & \family{Ntuple}{\klist{\kTypeBlur}}{\\
 & \quad \tops{new}{\kunit}{\_}{a}{
 	\tfold{\tvar{a}}{\tden{\titerm{0}}{\ttype{Ntuple}{\tnil{\kTypeBlur}}}}{d}{b}{r}{\\
		& \quad\quad\quad \tdencase{\tvar{d}}{dx}{dt}{\tdencase{\tvar{r}}{rx}{rt}{\\
			& \quad\quad\quad \tfamcase{\tvar{rt}}{Ntuple}{i}{\tfold{\tvar{b}}{\tden{\tvar{dx}}{\ttype{Ntuple}{\tcons{\tvar{dt}}{\tvar{i}}}}}{\_}{\_}{\_}{\\
			& \quad\quad\quad\quad \tden{\titerm{(\iup{\tvar{dx}},\iup{\tvar{rx}})}}{\ttype{Ntuple}{\tcons{{\tvar{dt}}}{\tvar{i}}}}}
			}{\terr}
		}{\terr}}{\terr}
	}
 };\\
 & \quad \tops{pr}{\dint}{i}{a}{
 	\tapp{\tvar{pop\_final}}{\tapp{\tvar{a}}{\tlam{x}{\kDen}{\tlam{nt}{\kTypeBlur}{\tfamcase{\tvar{nt}}{Ntuple}{nl}{\\
 & \quad\quad \tfold{\tvar{nl}}{\terr}{t1}{j}{\_}{\\
 & \quad\quad \tfold{\tvar{j}}{\tifeq{\tvar{i}}{0}{\dint}{\tden{\tvar{x}}{\tvar{t1}}}{\terr}}{\_}{\_}{\_}{\\
 & \quad\quad\quad \tfst{\tapp{\tvar{foldl}}{\tapp{\tvar{nl}}{\tapp{(\tden{\tvar{x}}{\tvar{nt}},0)}{
 	\tlam{r}{\kpair{\kDen}{\dint}}{\tlam{t}{\kTypeBlur}{\tlam{ts}{\klist{\kTypeBlur}}{\\
	& \quad\quad\quad\quad \tifeq{\tvar{i}}{\tsnd{\tvar{r}}}{\dint}{\tvar{r}}{\tdencase{\tfst{\tvar{r}}}{rx}{\_}{\\
	& \quad\quad\quad\quad\quad \tifeq{\tvar{i}}{\tsnd{\tvar{r}}+1}{\dint}{\\
	& \quad\quad\quad\quad\quad\quad \tifeq{\tvar{ts}}{\tnil{\kTypeBlur}}{\klist{\kTypeBlur}}{(\tden{\tvar{rx}}{\tvar{t}},\tvar{i})\\
	& \quad\quad\quad\quad\quad\quad}{(\tden{\titerm{\ifst{\iup{\tvar{rx}}}}}{\tvar{t}},\tvar{i})}\\	
	& \quad\quad\quad\quad\quad}{(\tden{\titerm{\isnd{\iup{\tvar{rx}}}}}{\tvar{t}},\tsnd{\tvar{r}}+1)}}{\terr}}}}}
 }}}}}
 }}{\terr}}}}}
 }\\ & 
%& \quad \tops{pr}a}{b}{c}{d}
 }{d}{i}{
 	\\& \quad \quad \tfold{\tvar{i}}{\titype{\dint}}{s}{j}{r}{
 		\tfold{\tvar{j}}{\titype{\trepof{\tvar{s}}}}{\_}{\_}{\_}{
 		\titype{\dpair{\trepof{\tvar{s}}}{\dup{\tvar{r}}}}
 		}
 	}
 }
\end{flalign}
\noindent
This definition uses a left fold function, $\tvar{foldl}$, defined in terms of the right fold operator built into the type-level language in the usual way, as follows:
\[
\begin{array}{rcl}
\tvar{foldl} & := & \tlam{l}{\klist{\kTypeBlur}}{\tlam{b}{\kpair{\kDen}{\dint}}{\tlam{f}{\karrow{(\kpair{\kDen}{\dint})}{\karrow{\kTypeBlur}{\karrow{\klist{\kTypeBlur}}{(\kpair{\kDen}{\dint})}}}}{\\
& & \quad \tapp{\tfold{\tvar{l}}{\tlam{x}{\kpair{\kDen}{\dint}}{\tvar{x}}}{t}{ts}{r}{\tlam{x}{\kpair{\kDen}{\dint}}{\tapp{\tvar{r}}{(\tapp{\tvar{f}}{\tapp{\tvar{x}}{\tapp{\tvar{t}}{\tvar{ts}}}})}}}}}}{\tvar{b}}}
\end{array}
\]

\end{document}

\backmatter

%\renewcommand{\baselinestretch}{1.0}\normalsize
% By default \bibsection is \chapter*, but we really want this to show
% up in the table of contents and pdf bookmarks.
\renewcommand{\bibsection}{\chapter{\bibname}}
\newcommand{\bibpreamble}{TODO: Explain abbreviations.}
\bibliographystyle{plainnat}
\bibliography{../research} %your bib file

\end{document}
