\documentclass[english]{sigplanconf}
\usepackage{palatino}
\usepackage[T1]{fontenc}
\usepackage[latin9]{inputenc}
\usepackage{listings}
\usepackage{url}
\usepackage{babel}
\usepackage{graphicx}
%\usepackage{microtype}

%\lstset{
%  numbers=left,
%  basicstyle=\ttfamily,
%  tabsize=4,
%  frame=single,
%}  

\lstset{tabsize=3, basicstyle=\ttfamily\small, commentstyle=\itshape\rmfamily, numbers=left, numberstyle=\tiny, language=java,moredelim=[il][\sffamily]{?},mathescape=true,showspaces=false,showstringspaces=false,columns=fullflexible,xleftmargin=15pt,escapeinside={(@}{@)}, morekeywords=[1]{application,policy,external,component,connects,to,meth,let,in,val,where,return,group,by,within,connect,with,attr,html,head,title,style,body,div,keyword,location,schema,must,salt,select,text}}
\lstloadlanguages{Java,VBScript,XML,HTML,Python}



\newcommand{\todo}[1]{\textbf{[TODO: #1]}}
\newcommand{\keyw}[1]{\texttt{\textbf{#1}}}

\begin{document}

\title{Whitespace-Delimited, Type-Directed Parsing for Embedded DSLs}

\authorinfo{Cyrus Omar, Benjamin Chung$^{1}$, Darya Kurilova, Alex Potanin$^{2}$,
  and Jonathan Aldrich}
{Carnegie Mellon University}
{\{comar, darya, aldrich\}@cs.cmu.edu, bwchung@andrew.cmu.edu$^{1}$, and alex@ecs.vuw.ac.nz$^{2}$}
% Maybe just keep my cs account active after I leave for a few years?
% Otherwise: alex@ecs.vuw.ac.nz is my email...
\maketitle

\begin{abstract}

Domain-specific languages enhance ease of use, expressiveness and
verifiability, 
but defining and using different 
DSLs within a single application remains difficult.  
We present an approach to embedded DSLs where:
1) a whitespace-delimited lexer is shared across DSLs, and 
2) the parsing and type checking phases occur in tandem so that types can define domain-specific parsing strategies governing whitespace-delimited blocks.
We argue that this approach occupies a sweet spot with both 
high expressiveness and ease of use. We outline a design
and ongoing implementation of this strategy
and examine the relationship between the type-directed and keyword-directed strategies for parsing of embedded DSLs.

\end{abstract}

\begin{sloppypar}

\section{Introduction}
\label{s:intro}

Domain-specific languages (DSLs) \cite{shellchapter, coldfusion,
  sinatra}
% have been widely-studied because they 
enable the expression of  
domain-specific ideas naturally and permit domain-specific
verification and compilation strategies.  For DSLs to reach their full
potential, however, it must be possible to easily use multiple
DSLs at once within a host general-purpose language (GPL), such that DSL code
and GPL glue code can inter-operate to form a complete application. More specifically, we propose the following essential design criteria:

\begin{itemize}

\item \emph {Composability}: It should be possible to use multiple DSLs and a GPL
%, in addition to a GPL,
within a single program unit.  %Here 
Within the text file-based paradigm familiar to most developers, this 
means including multiple DSLs within a single file.
 Moreover, it should be possible to embed code written in one DSL
  within another DSL without them requiring knowledge of each other or interfering with one another. So, more specifically, DSLs should be \emph{safely composable}.

\item \emph{Interoperability}: DSLs should be able to \emph{interoperate}. A necessary condition for interoperability is that it should be
  possible to pass and operate on values 
%such as functions or data structures
  defined in foreign DSLs. Additional requirements, such as the ability to do so naturally (rather than requiring a large amount of ``glue code'') and safely may also be relevant in many settings. 
%Moreover,
  Minimally, however, if one DSL defines a function or data structure satisfying an interface specified in
  a common interface description language (such as the type system of
  a host GPL), code written in another DSL should be able to use those
  values according to that interface, without requiring that the client DSL
  have knowledge of the details of the provider DSL or \emph{vice versa}.

\end{itemize}


%\item type system: one DSL depends on types defined by another DSL,
%  can use objects of that type in special ways [this goes beyond the
%    scope - save for a later paper]


In addition to composability and interoperability, we believe that to be most useful, a system supporting DSLs should have the following characteristics:

\begin{itemize}

\item \emph{Flexibility}: Have flexible enough DSL syntax to support a wide variety of text-based notations;

\item \emph{Identifiability}: Make it easy for programmers to identify which code is written in which DSL;

\item \emph{Consistency}: Encourage DSLs written in similar styles in order to enhance readability and learnability of each DSL;

\item \emph{Simplicity}: Keep the cost of defining a DSL low.
 
%\item Share conventions between DSLs and a host language, making each DSL easier for programmers to learn, helping programmers to identify which code is written in which DSL, and avoiding unintended conflicts between DSLs. \todo{Avoid conflicts (visual and real), enhance learnability}
 
%\item Reuse low-level mechanisms and design decisions from a host language, thereby reducing the cost of defining DSLs.  \todo{Easy to define DSLs}

\end{itemize}

We are developing a comprehensive approach that has the potential to meet these design requirements well, based on defining DSLs within a language specifically designed for extensibility from within.
The approach inherently supports multiple DSLs in the same file, and
supports interoperability because the DSLs are defined by translation
into a common internal language.  One key aspect of our proposed solution, and one of the focal points of this paper, is
that we restrict the syntax of DSLs in a few key ways that greatly simplify
the mechanism without significantly decreasing expressiveness:
\begin{itemize}

\item The host language and its DSLs share a tokenization and lexing 
  strategy, standardizing conventions for identifiers, operators,
  constants, and comments.  This avoids the cost of defining lexing
  within each DSL, avoids many kinds of low-level clashes between
  languages, and makes the composed language more readable. Note that this does not limit the ability of a DSL to define new keywords and other constructs.
 
\item Constructs in the host language and its DSLs are
  \emph{whitespace-delimited} at the top level, according to a particular strategy that we will describe.  Various forms of parentheses 
  can also serve as delimiters.  Thus, indentation levels or
  parenthesized expressions clearly delimit blocks that are governed by a particular 
  language.  This makes the boundaries of each DSL clear to
  the programmer and the compiler, enhancing usability and guaranteeing that 
  subtle conflicts cannot arise.
  
\end{itemize}

Within this syntactic framework, we then propose a novel syntax extension
mechanism in which the \emph{expected type of an expression}, rather than an explicit keyword, 
%, rather than a keyword, 
signals the switch to a new DSL that parses the tokenized and lexed code within a subsequent delimited block according to a type-associated grammar. The  more common keyword-directed strategy arises as a special case of this type-directed strategy. 
We will begin with some motivating examples (\S\ref{s:motivation}), describe our approach (\S\ref{s:approach}) and implementation (\S\ref{s:implementation}), and conclude with related work (\S\ref{s:related}).

% keep discussion of type-based parsing brief - active typing for parsing (only)
% avoids conflict with Cyrus' paper

%The rest of the paper is organized as follows.  The next section
%illustrates our approach by example, discussing the components of the
%solution in more detail. Section~\ref{s:approach} outlines our
%approach, shows a wider variety of examples and discusses variations
%of our approach.  Our in-progress implementation of the proposal is
%described in Section~\ref{s:implementation}.  Section~\ref{s:related}
%compares to related work, and Section~\ref{s:conclusion} concludes.

% with a discussion of future work.

\section{Motivating Examples}
\label{s:motivation}

\begin{figure}
  \centering
  \begin{lstlisting}
val hotProductDashboard = new Application
    external component twitter : Feed
        location [www.twitter.com]
    external component client : Browser
        connects to servlet
    component servlet : DashServlet
        connects to productDB, twitter
        location [intranet.nameless.com]
    component productDB : Database
        location [db.nameless.com]
    policy NamelessSecurity
        must salt [servlet.login.password]
        connect * -> servlet with HTTPS
        connect servlet -> productDB with TLS
  \end{lstlisting}
  \caption{Wyvern DSL: Architecture Specification}
  \label{f:dsl-arch}
\end{figure}

We start with a few examples meant to illustrate the expressiveness of our approach and the breadth of DSLs we plan for it to support.  The examples are presented in the proposed syntax for Wyvern, a new language being developed by our group  initially targeted toward building secure web and mobile applications. We will further clarify how these examples are parsed in Section \ref{s:approach}.

The first example, shown in Figure~\ref{f:dsl-arch}, describes the overall architecture of a ``hot product dashboard'' application.  The constructor \verb|Application| takes an argument of type \verb|ComponentArch|. Rather than being provided explicitly, a DSL for specifying the component architecture of the application is used to provide this argument because it appears in a whitespace-delimited block. The example architecture declares several components, some of which are marked \keyw{external} to indicate that they are used by this application but are not part of it directly. Component types, connectivity and properties, such as location, are declared within further-indented blocks.

The \keyw{policy} keyword introduces a security policy, which constrains the communication protocols to be used as well as
enforces secure handling of passwords.  Again, indentation delimits
the scope of the security policy.  This
%example
illustrates that one DSL may be nested within another.

\begin{figure}
  \centering
  \begin{lstlisting}
val newProds = productDB.query 
    select twHandle 
    where introduced - today < 3 months
this.prodTwt = Feed(newProds)
return prodTwt.query
    select *
    group by followed
    where count > POPULAR  
  \end{lstlisting}
  \caption{Wyvern DSL: Queries}
  \label{f:dsl-query}
\end{figure}

Figure~\ref{f:dsl-query} shows how a DSL for querying can be used from within ordinary Wyvern code.  The example shows the body of a method for computing a feed that is derived from tweets about the company's new products.  In this example, the use of a querying DSL is triggered by the use of a function, query, expecting a \verb|DBQuery|.  This type defines a particular syntax for database queries, delimited by the indented block as the reader would expect. This is similar to what can be expressed with built-in query syntax such as Microsoft's LINQ \cite{mslinq}, but in this case, defined in a user-defined type. The feed query operator may define different features, as seen on Line 5.
\begin{figure}
  \centering
  \begin{lstlisting}
html:
    head:
        title: [text "Hot Products"]
        style:
            myStylesheet
    body:
        div id="search":
            SearchBox("products")
        div id="products":
            FeedBox(servlet.hotProds())
  \end{lstlisting}
  \caption{Wyvern DSL: Presentation}
  \label{f:dsl-presentation}
\end{figure}

Finally, Figure~\ref{f:dsl-presentation} shows a DSL for rendering the hot product application in a web browser.  The DSL is based on HTML but uses our indentation-based syntax and allows Wyvern code to be integrated -- function/method calls and variables of the appropriate type can be used at any point. Literal tags are introduced by a tag name followed by any tag attributes followed by a colon.

% !TEX root = omar-thesis-proposal.tex
\newcommand{\nat}{\textbf{nat}}
\newcommand{\natz}{\textbf{z}}
\newcommand{\nats}[1]{\textbf{s}(#1)}
\newcommand{\natrec}[3]{\textbf{natrec}(#1, #2, #3)}
\renewcommand{\ttdefault}{txtt}

\section{Background: Active Libraries}\label{alibs}\todo{move into separate tex file}
The term \emph{active libraries} was first used by Veldhuizen et al. \cite{atthesis, atpaper} to describe ``libraries that take an active role in compilation, rather than being passive collections of subroutines''. The authors went on to suggest a number of concrete reasons libraries might benefit from closer interactions with the programming system, including program optimization, checking programs for correctness against specialized criteria, reporting domain-specific errors and warnings, and ``rendering domain-specific textual and non-textual program representations and for interacting with such representations'' (anticipating interactions between libraries and tools other than just the compiler. Our definition, given in Section \ref{motivation}, differs  from theirs for this reason.) 

% This stuff isn't actually hooking into the compiler...
The first concrete realizations of active libraries, prompting the introduction of the term, were scientific libraries that performed domain-specific program optimization at compile-time by exploiting language mechanisms, like C++ templates, that require compile-time computation. A prominent example in the literature is Blitz++, a numerics library that uses C++ template metaprogramming to optimize compound operations on vectors and matrices by eliminating intermediate allocations \cite{blitzpp}. Although this and a number of other interesting optimizations\- are possible by this family of techniques, their expressiveness is limited because template expansion supports only substitution of compile-time constants into pre-written code, and templates are notoriously difficult to read, write and debug. %They are, however, composable, because templates are associated with a single function or class, and  the only metatheoretic issue is that template expansion can be non-terminating.

More powerful compile-time \emph{term rewriting mechanisms} integrated into some languages can also be used for optimization, as well as for inserting specialized error checking and reporting\- logic and extending the language with powerful domain-specific abstractions. These mechanisms are highly expressive because they allow users to programmatically manipulate syntax trees directly, but they suffer from problems of composability and safety. Macros, such as those in MetaML \cite{metaml}, Template Haskell \cite{template_haskell} and others\todo{reference other macro systems}, can take full control over all the code that they enclose, so there is no way to guarantee that nested macros will compose as intended. Outer macros can neglect to invoke or  manipulate the output of inner macros in ways that can cause conflicts or weaken important guarantees. Once data escapes a macro's scope, there is no way to rely on the guarantees and features that were available within its scope -- the output of a macro is a value in the underlying language, so the underlying language is all that governs it (a problem related to the interoperability problem of Section 1). It can also be difficult to reason about the semantics of code when a number of enclosing macros may be manipulating it.

Some rewriting systems go beyond the block scoping of macros. Xroma (pronounced ``Chroma''), for example, is designed around active libraries and allows users to insert custom rewriting and error checking passes into the compiler from within libraries \cite{xroma}. Similarly, the derivative of Haskell implemented by the Glasgow Haskell Compiler (GHC) allows libraries to define custom compile-time term rewriting logic if an appropriate flag is passed in \cite{haskell-rewrite}. In both cases, the user-defined logic can dispatch on arbitrary patterns of code throughout the component or program the extension is active in, so these mechanisms are highly expressive and avoid some of the difficulties of block-scoped macros. But libraries containing such global rewriting logic cannot be safely composed because two different libraries may attempt to rewrite the same piece of code differently. It is also difficult to guarantee that such logic is correct and difficult for users to reason about code when simply importing a library can change the semantics of the program in a non-local manner.
% http://www.haskell.org/ghc/docs/latest/html/users_guide/rewrite-rules.html

Another example of an active library approach to extensibility is SugarJ \cite{sugarj}. SugarJ allows libraries to extend the base syntax of Java in a nearly-arbitrary manner, and these extensions are imported transitively throughout a program. SugarJ libraries are thus also not safely composable. For example, a library that defines a literal syntax for HTML would conflict with another that defines a literal syntax for XML because they define differing semantics for some of the same syntactic forms. And again, it can be difficult for users to predict what an unfamiliar piece of syntax desugars into, leading to readability issues.

%Another example comes from the Tau package for tuning and analyzing parallel programs, which allows libraries to instrument themselves declaratively so that internal details can be hidden from users of the tool \cite{tau}. 

%The researchers behind these tools often claim that such conflicts are rare. There have, to our knowledge, been no studies of this claim. Nevertheless, the core fact remains: conflicts can arise, which makes it difficult to rely on such libraries 

%There also does not appear to be a clear, well-founded theoretical foundation for this approach, nor for active libraries in general.

\section{Active Types}
Developers specify type-specific compile-time and edit-time features by equipping type families, when they are defined, with functions written in an appropriate metalanguage. These are selectively invoked by tools, such as the parser, type checker, translator and editor, when they work with expressions of that type, and only in those situations.

\subsection{Example: Regular Expressions}
To make our approach more concrete, let us begin with a simple example that demonstrates the use cases examined in this thesis. If a programming system included full compile-time and edit-time support for regular expressions, it might support features like the following:

\begin{enumerate}
\item \textbf{built-in syntax for regex literals} so that malformed regular expression patterns result in intelligible compile-time parsing errors. One study found that malformed regular expressions are a common problem in Java \cite{regex-type-system}.
\item \textbf{type checking logic} that ensures that key constraints related to regular expressions are not violated, such as that out-of-bounds group indices are not used \cite{regex-type-system} and that only values with correct types are spliced into regular expressions. When a type error is found, the error message should be intelligible.
\item \textbf{translation logic} that partially or fully compiles regular expressions into the efficient internal representation that will be used by the regular expression matching engine at run-time. In most languages, this compilation step occurs at run-time, even if the pattern is fully known at compile-time, thereby introducing latency into programs. If the developer is not careful, regular expressions used repeatedly in a program might be needlessly re-compiled on each use. By performing this step ahead-of-time, these dangers can be avoided.
\item \textbf{editor-integrated tooling} for interactively testing patterns against example strings, quickly referring to documentation, searching databases of common patterns and other domain-specific edit-time facilities.
\end{enumerate}


%\subsection{Example: First-Class Natural Numbers}
%To make our approach more concrete and emphasize its foundational character, let us begin with a very simple example, first-class natural numbers. If a programming system included first-class support for natural numbers, it might support behaviors like the following:
%
%\begin{enumerate}
%\item type checking rules that constrain uses of the three operators associated with the $\nat$ type, as defined in G\"{o}del's System \textbf{T} \cite{tapl}: $\natz$, $\nats{e}$ and $\natrec{e_1}{e_2}{e_3}$\todo{phrasing of his}
%\item type checker error messages that provide domain-specific assistance, like detecting when the trailing $\textbf{z}$ in a constant has been omitted: \todo{phrasing / merge this into 1?}
%\begin{verbatim}
%    s(s(s))
%        ^ (Line 1, Character 5)
%    Error: Operator s requires 1 argument of type nat. Provided 0 arguments.
%    Did you mean s(z)?
%\end{verbatim}
%\item translation into an efficient integer representation at run-time
%\item syntax support \todo{elaborate on syntax support for nat literals}
%\item editor support, such as the ability to instantiate nat constants more easily \todo{elaborate on this / reference Graphite section}
%%generate statistical predictions useful for code completion (e.g. biasing smaller constants) and other editor interactions, customized syntactic/visual representations and so on.
%\end{enumerate}
%
In a conventional \emph{monolithic} programming system, support for each of these features would need to be built into the language and tools. No such system today has support for all of the features above. Instead, libraries  provide support for regular expressions by leveraging general-purpose abstractions. Unfortunately, it is impossible to fully encode the syntax and the specialized static and dynamic semantics described above in terms of standard general-purpose notations and abstractions. These libraries have thus needed to compromise, typically by representing regular expressions using strings and deferring parsing, typechecking and translation to run-time. This introduces performance overhead and can lead to unanticipated run-time errors (as shown in \cite{regex-type-system}) and security vulnerabilities (due to injection attacks when splicing is used, for example). Similarly, edit-time tools that make working with regular expressions easier, as described above, are rarely integrated into editors, and never in a way that facilitates their discovery and use directly when manipulating regular expressions. In most cases, such tools must discovered independently and accessed externally (for example, via a browser), making their use both less common and more awkward than necessary (we provide evidence for this \cite{active-code-completion}, see Section \ref{acc}).

Existing active library approaches could be used to implement some of these features, but the composition issues described in Section 2 make such solutions suboptimal. For example, if SugarJ was used by two different regular expression engines to provide literal syntax for regular expression patterns, there could easily be conflicts at link-time because both will introduce many of the same standard notations but back them with differing implementations. 

We can observe, however, that each of these features relates specifically to how terms representing regular expression patterns are processed by tools, and that such terms are always classified by a particular user-defined type\footnote{More generally, several different types within an indexed type family. For example, \texttt{Pattern(}$n$\texttt{)}  represents a pattern containing $n$ matching groups, so \texttt{Pattern} is  a type family indexed by a natural number, $n$ \cite{regex-type-system}. We will return to this distinction in Section \ref{att}.}. We will refer to it by a simple name, \verb|Pattern|, but it would need to have a fully-qualified name to ensure that conflicts due to name clashes cannot occur. It is only when creating, editing or compiling expressions of this type that the logic enumerated above would ever need to be invoked. Indeed, this is a common pattern -- types are widely seen as a natural organizational unit around which the semantics of programming languages and logics are defined (e.g. in both TAPL \cite{tapl} and PFPL \cite{pfpl}, most chapters simply describe the semantics and metatheory of a few new types without reference to other types). This suggests a principled alternative to the mechanisms described earlier that preserves most of their expressiveness but eliminates the possibility of conflict and makes it easier to reason locally about a piece of code: associating extension logic to a single type or type family as it is defined and scoping it only to expressions, or basic operations on expressions, in that type or type family. This is the common motif behind all of the {actively-typed} mechanisms proposed in this dissertation.

\subsection{Proposed Contributions}
In Section \ref{att}, we will describe how to extend the core static and dynamic semantics of a language in an actively-typed and safe manner. We will distill the essence of our approach, which we call \textbf{active typechecking and translation (AT\&T)}, by specifying an actively-typed lambda calculus called @$\lambda$, proving several key safety theorems, and examining the connections between active types and type-level computation, type abstraction and typed compilation techniques. We will then go on to demonstrate the expressiveness of this mechanism by designing and implementing a full-scale language, Ace, and implementing a number of interesting type families from existing full-scale languages as active libraries within Ace. A primary focus of these is on high-performance  programming abstractions, but we will also show some uses of Ace for other domains.

In both @$\lambda$ and Ace, semantic extensions operate over a fixed syntax. In Section \ref{aparsing}, we will show how domain-specific syntax can also be introduced in an actively-typed and safely composable manner. Our technique is called \textbf{actively-typed parsing} and it will be implemented within the Wyvern language. Our novel contributions include the design of a second parsing phase that runs in tandem with typechecking, a method for using whitespace to delimit domain-specific syntax, a generalization of standard literal forms so that they can be used for more than one type, an underlying mechanism for associating compile-time data and functionality with structural types, and a recursive use of the active parsing technique to introduce a domain-specific syntax for defining new actively-typed grammars. We will describe each of these contributions as implemented in Wyvern, give minimal formalisms showing how the core syntax and the type system of Wyvern support actively-typed parsing,  and show a number of examples expressible by this technique.

Finally, in Section \ref{acc}, we will show how editor-integrated domain-specific tooling for working with expressions of a single type can be introduced from within active libraries, by a technique known as \textbf{active code completion}. Developers associate
domain-specific user interfaces, called \emph{palettes}, with types. Users discover and invoke palettes from the code completion menu at edit-time, populated according to an actively-typed mechanism similar to that of actively-typed parsing. When they are done, the palette generates a term of that type based on the information received from the user. Using several empirical
methods, we survey\- the expressive power of this approach, describe the design and safety constraints governing
the system architecture as well as particular palette user 
interfaces, and develop one such system for the Java language\footnote{In Graphite, palettes are associated with Java classes, which are technically different from types. However, active code completion could also be implemented just as well in non-object-oriented languages.}, based on these constraints, called Graphite. Using Graphite,
we implement a palette for working with regular expressions to conduct a pilot study to provide evidence for the usefulness of this approach, and of contextually-relevant editor-integrated tooling generally.

Together, ...\todo{write something about a unified mechanism that can be used across concerns}

%
%This suggests that a natural place where these features can be defined is in the library containing the declaration of \verb|Pattern| itself, rather than in the language and tool implementations. An \emph{actively-typed} definition of \verb|Pattern| would thus be equipped with	 functions that described how the parser (item 1), type checker (item 2), translator (item 3) and editor (item 4) should operate when working with expressions of type \verb|Pattern|. We can abstractly denote this declaration, as it would exist within a user-defined library, as follows:
%\begin{equation*}
%{\sf type}~Pattern[f_{\text{editor}}]\{
%\textbf{z}[f_{\text{resolve-z}}, f_{\text{compile-z}}], 
%\textbf{s}[f_{\text{resolve-s}}, f_{\text{compile-s}}], 
%\textbf{natrec}[f_{\text{resolve-rec}}, f_{\text{compile-rec}}]\}
%\end{equation*}
%
%When type checking an expression like $\nats{\nats{\natz}}$, the type checker delegates to the user-provided type-level function $f_{\text{resolve-s}}$. This function would be tasked with assigning a type to expression as a whole, given information including the \emph{types} (but not necessarily the full syntax trees) of all its subexpressions, or if a type cannot be assigned, producing a specific error message. Similarly, the compiler calls the $f_{\text{compile-s}}$ function to determine a representation in the target language for the expression, checking to ensure that it is well-formed and type-correct with respect to the target type system. Finally, elements of the editor may call into the $f_{\text{editor}}$ function (or one of several such functions, more generally) to control behaviors like code completion and code prediction when an expression of type $\nat$ is being entered. 
%
%Note that these functions are \emph{not} to be conflated with methods or run-time functions -- they are functions written in a type-level language that are called at compile-time and edit-time to define the basic behaviors associated with the type that they are associated with.
%
%\subsection{Characteristics of an Actively-Typed Programming System}
%An actively-typed programming system can be characterized by its choice of type-level language, source grammar, target language and dispatch protocols.
%
%\paragraph{Type-Level Language} The type-level language is the language within which the type definitions and the functions that define their behaviors are defined. This language must be constrained so that different definitions do not interfere with one another and so that desirable safety properties for the system as a whole are maintained, as we discuss below.
%
%\paragraph{Source Language} The source language is the language with which run-time behavior is defined. In our example above, terms like $\nats{\nats{\natz}}$ are part of the source language. In the purest case, the source language is simply a grammar; its semantics are given entirely by active type specifications.
%
%\paragraph{Dispatch Protocol} For each syntactic form in the source language, there is a dispatch protocol that determines which type is delegated responsibility over it, and which specific function(s) are called for each behavior the system supports. This fixed protocol makes it possible for users to predict the meaning of a construct using information local to the term, a key differentiator of this approach compared to term-rewriting systems where there can be action at a distance.
%
%\paragraph{Target Language} The target language is the language that the front-end compilation phase of the system targets. The limitations and constraints imposed by the target language are final, because all constructs ultimately translate into terms in the target language. In other words, active type specifications can only add additional invariants to the language; they cannot violate invariants imposed by the target language.

%\subsection{Research Challenges}
%The example of natural numbers given above is relatively simple, and the solution we outline remains abstract. A key challenge is then to demonstrate that this approach is able to express the behaviors of more sophisticated language constructs that span diverse problem domains, and be implemented in the context of a realistic collection of tools. The resulting system should be usable by developers who lack the expertise needed to define new language constructs themselves.
%
%Simultaneously, we must also demonstrate that this model is well-motivated theoretically, place it within the broader context of the theory of typed programming languages, and demonstrate that it is possible for desirable system safety properties to be maintained. In particular, we are interested in properties like:
%
%\begin{itemize}
%\item Correctness of active type specifications, so that users of a library need not be forced to debug errors arising within the specifications themselves.
%\item Correctness of translations, so that the results of translation are guaranteed to be well-typed and consistent with respect to the target language.
%\item Termination of active type specifications, so that evaluation of the type-level functions cannot cause the compiler or editor to hang.
%\item Composability of active type specifications, so that the behaviors defined by one type cannot interfere with those defined by another, no matter the order in which they are imported. This property is essential if we wish to place these specifications within normal libraries.
%\end{itemize}



\input{implementation}
\section{Related Work}
\label{s:related}
The most common mechanism for creating programming language extensions is macros, as exemplified by hygienic macros in Scheme. Macros in Scheme and other Lisp-style languages are written in Lisp itself, and benefit from the regular syntax of the underlying language and the consistent use of parentheses as expression delimiters.  The syntactic nature of macros allows them to control parsing within a delimited scope, and this delimited control of parsing inspired our approach in Wyvern.  Wyvern's use of types to trigger custom parsers avoids the Lisp limitation of starting each DSL with a keyword or macro name.

Wyvern chooses whitespace delimitation rather than parenthesis delimitation because, anecdotally, many programmers find syntax in that style to be more readable.  There are several proposals for making Lisp and its dialects whitespace-delimited, including sweet-expressions \cite{sweet-expr} and SRFI-49 \cite{srfi-49} for Scheme, P4P proposal for Racket \cite{p4p-proposal}, Kernel's f-expressions \cite{fexprthesis} for Common Lisp \cite{f-expr}, and others. None of them seem to be successfully implemented and deployed within the Lisp communities, however.  This may be because Lisp programmers don't want to switch away from parentheses, yet having an alternate syntax is not by itself enough to draw non-Lisp programmers into the community.

%Therefore, our approach considers the readability of language extensions from the beginning: we attempt to devise the most comfortable and natural notations for the language extensions without bothersome extra keywords and try to keep it succinct.

Some language extensibility projects provide extension at levels of abstraction above parsing. For instance, OJ (previously, OpenJava)
\cite{Tatsubori00openjava:a} provides a macro system based on a meta-object
protocol, and Backstage Java~\cite{Palmer:2011:BJM:2048066.2048137} employs compile-time meta-programming.  Both of these systems provide macro-style rewriting of existing source code, but they provide at most limited extension of language parsing.

Other systems aim at providing more flexible forms of syntax extension compared to our whitespace-delimited approach, at the potential cost of complexity or conflicts.  Camlp4 \cite{camlp4} is a preprocessor for OCaml that offers the developer the ability to extend the concrete syntax of the language via the use of parsers and extensible grammars.  SugarJ \cite{Erdweg:2011:SLL:2048147.2048199} takes a library-centric approach which allows to syntactically extend the Java language by adding libraries. In Wyvern, the core language (that is, the \verb|Exp| sort) does not get extended directly, so conflicts cannot arise and reasoning about which DSL is applicable is much simpler.

Finally, researchers have developed a number of DSL frameworks and workbenches, such as the Meta Programming System (MPS) \cite{mps}, Spoofax \cite{KatsVisser2010}, the Intentional Domain Workbench (IDW) \cite{idw}, and Ens\={o} \cite{enso}, that provide support for extending a variety of programming languages by applying generic rules and concepts.  The Marco language \cite{lee:2012:marco} similarly provides macro definition at a level of abstraction that is largely independent of the target language.

Compared to these approaches, Wyvern focuses on making one programming language extensible with embedded DSLs, provides a common lexer and extensible parsing confined to whitespace-delimited regions, and investigates a novel type-based approach to triggering parser extensions.

%programming
%language extensions to a higher level of abstraction and creates a
%system that consumes the target language and rules for creating
%macros and produces a safe and expressive language. In contrast, our
%approach aims only at the Wyvern programming language and tries to
%enhance its extensibility.

%However, Camlp4 does not support types for the language extensions and thus is missing the additional layer of safety that our approach provides.

%While aiming for the same goal, Backstage Java  and SugarJ
% take very different approaches.   

% !TEX root = omar-thesis.tex
\chapter{Conclusion \& Future Work}


%TODO TODO TODO TODO TODO TODO TODO TODO add acknowledgments here!!!!!!  ARO

\end{sloppypar}

\bibliographystyle{plain}
\bibliography{biblio}
\end{document}