\documentclass[english]{sigplanconf}
\usepackage{palatino}
\usepackage[T1]{fontenc}
\usepackage[latin9]{inputenc}
\usepackage{listings}
\usepackage{url}
\usepackage{babel}
\usepackage{graphicx}
%\usepackage{microtype}

%\lstset{
%  numbers=left,
%  basicstyle=\ttfamily,
%  tabsize=4,
%  frame=single,
%}  

\lstset{tabsize=3, basicstyle=\ttfamily\small, commentstyle=\itshape\rmfamily, numbers=left, numberstyle=\tiny, language=java,moredelim=[il][\sffamily]{?},mathescape=true,showspaces=false,showstringspaces=false,columns=fullflexible,xleftmargin=15pt,escapeinside={(@}{@)}, morekeywords=[1]{application,policy,external,component,connects,to,meth,let,in,val,where,return,group,by,within,connect,with,attr,html,head,title,style,body,div,keyword,location,schema,must,salt,select,text}}
\lstloadlanguages{Java,VBScript,XML,HTML,Python}



\newcommand{\todo}[1]{\textbf{[TODO: #1]}}
\newcommand{\keyw}[1]{\texttt{\textbf{#1}}}

\begin{document}

\title{Whitespace-Delimited, Type-Directed Parsing for Embedded DSLs}

\authorinfo{Cyrus Omar, Benjamin Chung$^{1}$, Darya Kurilova, Alex Potanin$^{2}$,
  and Jonathan Aldrich}
{Carnegie Mellon University}
{\{comar, darya, aldrich\}@cs.cmu.edu, bwchung@andrew.cmu.edu$^{1}$, and alex@ecs.vuw.ac.nz$^{2}$}
% Maybe just keep my cs account active after I leave for a few years?
% Otherwise: alex@ecs.vuw.ac.nz is my email...
\maketitle

\begin{abstract}

Domain-specific languages enhance ease of use, expressiveness and
verifiability, 
but defining and using different 
DSLs within a single application remains difficult.  
We present an approach to embedded DSLs where:
1) a whitespace-delimited lexer is shared across DSLs, and 
2) the parsing and type checking phases occur in tandem so that types can define domain-specific parsing strategies governing whitespace-delimited blocks.
We argue that this approach occupies a sweet spot with both 
high expressiveness and ease of use. We outline a design
and ongoing implementation of this strategy
and examine the relationship between the type-directed and keyword-directed strategies for parsing of embedded DSLs.

\end{abstract}

\begin{sloppypar}

\section{Introduction}
\label{s:intro}

Domain-specific languages (DSLs) \cite{shellchapter, coldfusion,
  sinatra}
% have been widely-studied because they 
enable the expression of  
domain-specific ideas naturally and permit domain-specific
verification and compilation strategies.  For DSLs to reach their full
potential, however, it must be possible to easily use multiple
DSLs at once within a host general-purpose language (GPL), such that DSL code
and GPL glue code can inter-operate to form a complete application. More specifically, we propose the following essential design criteria:

\begin{itemize}

\item \emph {Composability}: It should be possible to use multiple DSLs and a GPL
%, in addition to a GPL,
within a single program unit.  %Here 
Within the text file-based paradigm familiar to most developers, this 
means including multiple DSLs within a single file.
 Moreover, it should be possible to embed code written in one DSL
  within another DSL without them requiring knowledge of each other or interfering with one another. So, more specifically, DSLs should be \emph{safely composable}.

\item \emph{Interoperability}: DSLs should be able to \emph{interoperate}. A necessary condition for interoperability is that it should be
  possible to pass and operate on values 
%such as functions or data structures
  defined in foreign DSLs. Additional requirements, such as the ability to do so naturally (rather than requiring a large amount of ``glue code'') and safely may also be relevant in many settings. 
%Moreover,
  Minimally, however, if one DSL defines a function or data structure satisfying an interface specified in
  a common interface description language (such as the type system of
  a host GPL), code written in another DSL should be able to use those
  values according to that interface, without requiring that the client DSL
  have knowledge of the details of the provider DSL or \emph{vice versa}.

\end{itemize}


%\item type system: one DSL depends on types defined by another DSL,
%  can use objects of that type in special ways [this goes beyond the
%    scope - save for a later paper]


In addition to composability and interoperability, we believe that to be most useful, a system supporting DSLs should have the following characteristics:

\begin{itemize}

\item \emph{Flexibility}: Have flexible enough DSL syntax to support a wide variety of text-based notations;

\item \emph{Identifiability}: Make it easy for programmers to identify which code is written in which DSL;

\item \emph{Consistency}: Encourage DSLs written in similar styles in order to enhance readability and learnability of each DSL;

\item \emph{Simplicity}: Keep the cost of defining a DSL low.
 
%\item Share conventions between DSLs and a host language, making each DSL easier for programmers to learn, helping programmers to identify which code is written in which DSL, and avoiding unintended conflicts between DSLs. \todo{Avoid conflicts (visual and real), enhance learnability}
 
%\item Reuse low-level mechanisms and design decisions from a host language, thereby reducing the cost of defining DSLs.  \todo{Easy to define DSLs}

\end{itemize}

We are developing a comprehensive approach that has the potential to meet these design requirements well, based on defining DSLs within a language specifically designed for extensibility from within.
The approach inherently supports multiple DSLs in the same file, and
supports interoperability because the DSLs are defined by translation
into a common internal language.  One key aspect of our proposed solution, and one of the focal points of this paper, is
that we restrict the syntax of DSLs in a few key ways that greatly simplify
the mechanism without significantly decreasing expressiveness:
\begin{itemize}

\item The host language and its DSLs share a tokenization and lexing 
  strategy, standardizing conventions for identifiers, operators,
  constants, and comments.  This avoids the cost of defining lexing
  within each DSL, avoids many kinds of low-level clashes between
  languages, and makes the composed language more readable. Note that this does not limit the ability of a DSL to define new keywords and other constructs.
 
\item Constructs in the host language and its DSLs are
  \emph{whitespace-delimited} at the top level, according to a particular strategy that we will describe.  Various forms of parentheses 
  can also serve as delimiters.  Thus, indentation levels or
  parenthesized expressions clearly delimit blocks that are governed by a particular 
  language.  This makes the boundaries of each DSL clear to
  the programmer and the compiler, enhancing usability and guaranteeing that 
  subtle conflicts cannot arise.
  
\end{itemize}

Within this syntactic framework, we then propose a novel syntax extension
mechanism in which the \emph{expected type of an expression}, rather than an explicit keyword, 
%, rather than a keyword, 
signals the switch to a new DSL that parses the tokenized and lexed code within a subsequent delimited block according to a type-associated grammar. The  more common keyword-directed strategy arises as a special case of this type-directed strategy. 
We will begin with some motivating examples (\S\ref{s:motivation}), describe our approach (\S\ref{s:approach}) and implementation (\S\ref{s:implementation}), and conclude with related work (\S\ref{s:related}).

% keep discussion of type-based parsing brief - active typing for parsing (only)
% avoids conflict with Cyrus' paper

%The rest of the paper is organized as follows.  The next section
%illustrates our approach by example, discussing the components of the
%solution in more detail. Section~\ref{s:approach} outlines our
%approach, shows a wider variety of examples and discusses variations
%of our approach.  Our in-progress implementation of the proposal is
%described in Section~\ref{s:implementation}.  Section~\ref{s:related}
%compares to related work, and Section~\ref{s:conclusion} concludes.

% with a discussion of future work.

% !TEX root = globaldsl13.tex
\section{Motivating Examples}
\label{s:motivation}

\begin{figure}
  \centering
  \begin{lstlisting}
val dashboardArchitecture : Architecture = ~
    external component twitter : Feed
        location www.twitter.com
    external component client : Browser
        connects to servlet
    component servlet : DashServlet
        connects to productDB, twitter
        location intranet.nameless.com
    component productDB : Database
        location db.nameless.com
    policy mainPolicy = ~
        must salt servlet.login.password
        connect * -> servlet with HTTPS
        connect servlet -> productDB with TLS
  \end{lstlisting}
  \caption{Wyvern DSL: Architecture Specification}
  \label{f:dsl-arch}
\end{figure}

We start with a few examples to illustrate the expressiveness of our approach and the breadth of DSLs we plan for it to support.  The examples are presented in the proposed syntax for Wyvern, a new language being developed by our group that is targeted toward building secure web and mobile applications. We will informally describe each of these examples here, and further explain how such code is parsed in Section \ref{s:approach}.

The first example, shown in Figure~\ref{f:dsl-arch}, describes the overall architecture of a ``hot product dashboard'' application.  The variable \lstinline{dashboardArchitecture} is explicitly ascribed type \lstinline{Architecture}. Rather than explicitly providing a value of this type, we instead use a DSL that makes specifying the component architecture of the application more concise and readable. This DSL code appears in the subsequent whitespace-delimited block and is introduced by a tilde (\lstinline{~}). The example architecture declares several components, some of which are declared \keyw{external} to indicate that they are used by this application but are not part of it directly. Component types are declared after a colon and attributes like connectivity location, are declared after the type (formatted in an indented block for readability). 
The \keyw{policy} keyword (line 11) introduces a security policy, which constrains the communication protocols that can be used and 
enforces the secure handling of passwords. A separate type, \lstinline{Policy}, is associated with such policies. Although we could instantiate this type explicitly using a Wyvern expression, we use a DSL for defining policies instead, again within a whitespace-delimited block introduced by a tilde.

\begin{figure}
  \centering
  \begin{lstlisting}
val newProds = productDB.query(~)
    select twHandle 
    where introduced - today < 3 months
val prodTwt = new Feed(newProds)
return prodTwt.query(~)
    select *
    group by followed
    where count > 1000
  \end{lstlisting}
  \caption{Wyvern DSL: Queries}
  \label{f:dsl-query}
\end{figure}

Figure~\ref{f:dsl-query} shows how a DSL for database queries can be used from within ordinary Wyvern code.  The example shows code for computing a feed that is derived from tweets about a company's new products.  In this example, the use of a querying DSL is triggered by the use of methods named \lstinline{query} expecting an argument of type \lstinline{DBQuery} (line 1) or \lstinline{FeedQuery} (line 5) respectively.  These types define related but distinct syntax for queries, determined by the expected type of expression where the tilde appears (tildes need not appear only at the ends of lines). Queries are again delimited by indentation. This mechanism is similar to what can be expressed in languages with built-in query syntax like LINQ \cite{mslinq}, but in this case, it is entirely user-defined, rather than built into the language.\begin{figure}
  \centering
  \begin{lstlisting}
serve(page, loc) where 
  val page = ~ 
    html:
      head:
        title: Hot Products
        style: {myStylesheet}
        body:
          div id="search":
            {SearchBox("products")}
          div id="products":
            {FeedBox(servlet.hotProds())}
  val loc = ~
    products.nameless.com
  \end{lstlisting}
  \caption{Wyvern DSLs: Presentation and URLs}
  \label{f:dsl-presentation}
\end{figure}

Finally, Figure~\ref{f:dsl-presentation} shows a DSL for presenting the hot product application to a web browser, served at a particular URL. Here, two DSLs are used within a single function call. To allow this without introducing ambiguity, the user can use a \keyw{where} clause, similar to that found in Haskell \cite{jones2003haskell}. The presentation DSL is based on HTML and associated with a type, \lstinline{HTMLElement}. It uses an indentation-sensitive syntax and allows integration of Wyvern code of the appropriate type using curly braces. The second DSL simply canonicalizes URL literals into Wyvern values of type \lstinline{URL}.

% !TEX root = omar-thesis-proposal.tex
\section{Active Types}\label{contributions}
The language-integrated extension mechanisms that we will introduce in this thesis are designed to be highly \textbf{expressive}, permitting library-based implementations of features comparable to the built-in features found in modern programming systems, but without the kinds of \textbf{safety} problems that have been an issue in previous mechanisms, as described above. We also aim to maintain the ability to understand and reason locally about code. This is accomplished by organizing extension logic around types and scoping it to or around expressions of the type it is associated with, rather than applying it globally or within an explicitly delimited scope as in previous mechanisms. 

To motivate this approach, let us return to our example of regular expressions. Observe that every feature described in Sec. \ref{regex} relates specifically to how terms  classified by a single user-defined type or family of types should behave. In fact, nearly all the features relate to the type representing regular expression patterns (let us call it \verb|Pattern|\footnote{We should note at the outset that to fully prevent conflicts between libraries, naming conflicts must also be avoided. Suitable namespacing mechanisms (e.g. URI-based schemes, as Java uses) are already widely used in practice  and will be assumed implicitly whenever needed.}). Feature 1 calls for specialized syntax for the introductory form for this type. Features 2a and 2b relate to its static and dynamic semantics. Feature 3 is about its edit-time behavior. The remaining feature, 2c, also relates to the semantics of a single family of types: \verb|StringIn[r]|, which classifies strings known to be in the language of the statically-known regular expression \verb|r|. It is exclusively when editing or compiling expressions of the associated type that the logic in Sec. \ref{regex}  needs to be considered. 

Indeed, this is not a property unique to our chosen example, but a commonly-seen pattern in programming language design. The semantics of a programming language or logic is often organized around its types (equivalently, its propositions). In two major textbooks about programming languages, TAPL \cite{tapl} and PFPL \cite{pfpl}, most chapters describe the semantics and metatheory of a few new types and their associated primitive operations without reference to other types. The composition of the types and associated operations from different chapters into complete languages is a language-external operation. For example, in PFPL, the notation $\mathcal{L}$\{$\rightarrow$ \verb|nat| \verb|dyn|\} represents a language composed of the arrow ($\rightarrow$), \verb|nat| and \verb|dyn| types and their associated operators. Each of these are defined in separate chapters, and it is generally left unstated that the semantics and metatheory developed separately will compose without trouble (justified by the fact that, upon careful examination, it is indeed the case that almost any combination of types defined separately in PFPL can be combined to form a language with little trouble). 

This ubiquitous type-oriented organization suggests a principled language-integrated alternative to the mechanisms described in Section \ref{alibs} that preserves much of their expressiveness but eliminates the possibility of conflict and makes it easier to reason locally about a piece of code: associating extension logic directly with a single type (or family of types) and scoping it to affect only expressions classified by that type (or by a type in that family). This guarantees that different extensions don't have overlapping scope, preventing a range of common conflicts. By constraining the extension logic itself by various means we will show that the system as a whole can also maintain many other important safety and non-interference properties that have not previously been achieved. We call types with such logic associated with them \emph{active types} and systems that support them \emph{actively-typed programming systems}. 

\subsection{Proposed Contributions}
This thesis will introduce several language-integrated extensibility mechanisms, each based on active types, that give providers control over a different aspect of the system from within libraries (that is, in a decentralized manner). In each case, we will show that the system remains fundamentally safe and that extensions cannot interfere with one another. We will also discuss various points in the design space related to extension correctness (as distinct from safety, which is guaranteed even if an incorrect extension is used). To justify the  expressiveness of each approach, we will give a number of examples of non-trivial features that are, or would need to be, built into other systems, but that can be expressed within libraries using our mechanisms. To help us gather a broad, unbiased collection of examples and demonstrate the scope and applicability of our approaches in practice, we conduct empirical studies.

We begin in Sec. \ref{aparsing} by considering \textbf{syntax}. The availability of specialized syntax can bring numerous cognitive benefits \cite{green1996usability}, and discourage the use of problematic techniques like using strings to represent structured data \cite{Bravenboer:2007:PIA:1289971.1289975}. But allowing library providers to add arbitrary new syntactic forms to a language's grammar can lead to interference issues, as described above. We observe that many syntax extensions are motivated by the desire to add alternative  introductory forms (a.k.a. \emph{literal forms}) for a particular type. For example, regular expression pattern literals as described in Sec. \ref{regex} are an introductory form for the \verb|Pattern| type.  In the mechanism we introduce, syntax extensions are associated directly with a type and are active only where an expression of that type is expected (shifting part of the burden of parsing into the typechecker). This avoids extension interference problems because the base grammar of the language is never extended directly. We call such an extension a \emph{type-specific language (TSL)} and introduce TSLs in the context of a new language, Wyvern. We begin by showing how interference issues between the base language and the TSL syntax can be avoided by either introducing minor syntactic constraints on the body of a literal or by using a novel layout-delimited literal form. We then develop a formal semantics, basing it on work on bidirectional type systems and elaboration semantics, and introduce a novel mechanism that statically prevents unsafe variable capture and shadowing by extensions (providing a form of \emph{hygiene}). Finally, we conduct a corpus analysis to examine this technique's expressiveness, finding that a substantial fraction of string literals in existing code could be replaced by TSL literals.

Wyvern has an extensible syntax but a fixed general-purpose static and dynamic semantics. The general-purpose abstraction mechanisms we have included in Wyvern are powerful, and implementation techniques for these are well-developed, but there remain situations where providers may wish to extend the \textbf{semantics} of a language directly, by introducing new primitive types and operations. Examples of type system extensions that require this level of control abound in the research literature. For example, to implement the features in Sec. \ref{regex}, new logic must be added to the type system to statically track information related to backreferences (feature 2b, see \cite{spishak2012type}) or to execute a decision procedure for language inclusion when determining whether a coercion is possible (feature 2c, see \cite{fulton-thesis}). We discuss more examples from the literature where general-purpose abstraction mechanisms proved inadequate and researchers had to turn to language-external approaches in Sec. \ref{att}. To support these more advanced use cases in a decentralized manner, we next develop language-integrated mechanisms for implementing semantic extensions, while leaving the syntax fixed. We begin in Sec. \ref{foundations} with a type theoretic treatment, specifying an ``actively typed'' lambda calculus called @$\lambda$. By beginning from first principles, we are able to cleanly state and prove the key safety and non-interference theorems and examine the connections between active types and several prior notions, including type-level computation, typed compilation and abstract types. We then go on in Sec. \ref{ace} to demonstrate the expressiveness of this mechanism by designing and implementing a full-scale actively typed language, Ace. Interestingly, Ace is itself bootstrapped as a library within an existing language, Python. We discuss how we accomplish this, relate Ace to the core calculus and implement a number of powerful primitives from existing languages as libraries, giving examples from a variety of paradigms, including low-level parallel programming, functional programming, object-oriented programming and specialized domains, like regular expression types.

Finally, in Sec. \ref{acc}, we will show how \textbf{editor services} can be implemented from within active libraries, by a technique we call \emph{active code completion}. To provide a new editor service, providers associate
specialized user interfaces, called \emph{palettes}, with types. Clients discover and invoke palettes from the code completion menu at edit-time, populated according to the expected type at the cursor (a protocol similar to the one we use for syntax extensions in Wyvern). When the interaction between the client and the palette is complete, the palette generates a term of the type it is associated with based on the information received from the user. Using several empirical
methods, we surveyed the expressive power of this approach and developed design criteria. Based on these initial studies, we then developed an active code completion system for Java called Graphite. Using Graphite,
we implemented a palette for working with regular expressions and conduct a small study that demonstrates the usefulness of this approach.

Taken together, these mechanisms demonstrate that actively-typed mechanisms can be introduced throughout a programming system to allow users to extend both its compile-time and edit-time semantics from within libraries, without  weakening the safety guarantees that the system provides. They also further reinforce the idea that types are a natural organizational unit for defining programming system features and show how a type-oriented approach can make it easier to guarantee that the features will be safely composable in any combination. By implementing our techniques within a variety of different host systems, we demonstrate that actively-typed mechanisms are relevant across traditional paradigms. In the future, we anticipate developing a programming system that will bring together several actively-typed mechanisms, organized around a minimal, well-specified and formally verified core, where nearly every feature is specified, implemented and verified in a decentralized manner and distributed as a library. 

%
%This suggests that a natural place where these features can be defined is in the library containing the declaration of \verb|Pattern| itself, rather than in the language and tool implementations. An \emph{actively-typed} definition of \verb|Pattern| would thus be equipped with	 functions that described how the parser (item 1), type checker (item 2), translator (item 3) and editor (item 4) should operate when working with expressions of type \verb|Pattern|. We can abstractly denote this declaration, as it would exist within a user-defined library, as follows:
%\begin{equation*}
%{\sf type}~Pattern[f_{\text{editor}}]\{
%\textbf{z}[f_{\text{resolve-z}}, f_{\text{compile-z}}], 
%\textbf{s}[f_{\text{resolve-s}}, f_{\text{compile-s}}], 
%\textbf{natrec}[f_{\text{resolve-rec}}, f_{\text{compile-rec}}]\}
%\end{equation*}
%
%When type checking an expression like $\nats{\nats{\natz}}$, the type checker delegates to the user-provided type-level function $f_{\text{resolve-s}}$. This function would be tasked with assigning a type to expression as a whole, given information including the \emph{types} (but not necessarily the full syntax trees) of all its subexpressions, or if a type cannot be assigned, producing a specific error message. Similarly, the compiler calls the $f_{\text{compile-s}}$ function to determine a representation in the target language for the expression, checking to ensure that it is well-formed and type-correct with respect to the target type system. Finally, elements of the editor may call into the $f_{\text{editor}}$ function (or one of several such functions, more generally) to control behaviors like code completion and code prediction when an expression of type $\nat$ is being entered. 
%
%Note that these functions are \emph{not} to be conflated with methods or run-time functions -- they are functions written in a type-level language that are called at compile-time and edit-time to define the basic behaviors associated with the type that they are associated with.
%
%\subsection{Characteristics of an Actively-Typed Programming System}
%An actively-typed programming system can be characterized by its choice of type-level language, source grammar, target language and dispatch protocols.
%
%\paragraph{Type-Level Language} The type-level language is the language within which the type definitions and the functions that define their behaviors are defined. This language must be constrained so that different definitions do not interfere with one another and so that desirable safety properties for the system as a whole are maintained, as we discuss below.
%
%\paragraph{Source Language} The source language is the language with which run-time behavior is defined. In our example above, terms like $\nats{\nats{\natz}}$ are part of the source language. In the purest case, the source language is simply a grammar; its semantics are given entirely by active type specifications.
%
%\paragraph{Dispatch Protocol} For each syntactic form in the source language, there is a dispatch protocol that determines which type is delegated responsibility over it, and which specific function(s) are called for each behavior the system supports. This fixed protocol makes it possible for users to predict the meaning of a construct using information local to the term, a key differentiator of this approach compared to term-rewriting systems where there can be action at a distance.
%
%\paragraph{Target Language} The target language is the language that the front-end compilation phase of the system targets. The limitations and constraints imposed by the target language are final, because all constructs ultimately translate into terms in the target language. In other words, active type specifications can only add additional invariants to the language; they cannot violate invariants imposed by the target language.

%\subsection{Research Challenges}
%The example of natural numbers given above is relatively simple, and the solution we outline remains abstract. A key challenge is then to demonstrate that this approach is able to express the behaviors of more sophisticated language constructs that span diverse problem domains, and be implemented in the context of a realistic collection of tools. The resulting system should be usable by developers who lack the expertise needed to define new language constructs themselves.
%
%Simultaneously, we must also demonstrate that this model is well-motivated theoretically, place it within the broader context of the theory of typed programming languages, and demonstrate that it is possible for desirable system safety properties to be maintained. In particular, we are interested in properties like:
%
%\begin{itemize}
%\item Correctness of active type specifications, so that users of a library need not be forced to debug errors arising within the specifications themselves.
%\item Correctness of translations, so that the results of translation are guaranteed to be well-typed and consistent with respect to the target language.
%\item Termination of active type specifications, so that evaluation of the type-level functions cannot cause the compiler or editor to hang.
%\item Composability of active type specifications, so that the behaviors defined by one type cannot interfere with those defined by another, no matter the order in which they are imported. This property is essential if we wish to place these specifications within normal libraries.
%\end{itemize}



% !TEX root = globaldsl13.tex

%\section{Implementation Considerations}
%\label{s:implementation}
%
%The Wyvern compiler
%was developed from the ground up to support the extensible parsing
%interface. The compiler is currently implemented in Java, with an eventual goal of self-hosting it within Wyvern itself.
%
%Wyvern uses a fixed whitespace-indented lexing approach similar to that used by languages like Python. The current indentation convention is based on the number of whitespace characters. In order for a block to be formed, each of its lines must share the same initial whitespace string. If the indentation level is decreased but does not return to the initial indentation level, a keyword block is created as shown in Figure \ref{f:dsl-multi}. This particular approach is not found in other whitespace-delimited languages to our knowledge.
%
%The Wyvern front-end effectively combines the parsing and type checking
%stages that are usually separated in more traditional compilers, such
%as \texttt{javac} or \texttt{gcc}. Before any block can be parsed,
%Wyvern requires a context that contains the types of the arguments of the surrounding function and any variables and types in scope. This context is passed into the delimited block for portions of the type-associated grammar that contain terms of sort \verb|Exp| or \verb|Type|, and thus may contain other function calls, variables, or types. 

%For example, the Wyvern default environment contains bindings for
%keywords like \keyw{class} or \keyw{meth} that are able to parse
%class and method declarations respectively. Parsers are bound to types and, as such, may be shadowed within application code. Only operators and their associated precedence are not exposed in the environment and, thus, have to be reused by the DSLs, though it is feasible to lift them up to the
%environment with an appropriate parser.
%
%% , an infix nature of operators
%% might make parsing definition more complex with arguably small gains
%% in expressiveness for potential DSLs.

%The core parsing in Wyvern is an extremely simple mechanism. The language uses a core parser that for every token encountered delegates to the appropriate parser in the environment. This extension parser processes the token and any relevant ones that follow. Once the extension parser is finished, the core parser continues through the remaining tokens, terminating when it either runs out of tokens or encounters a token with no associated extension parser.

%% In a similar manner to keywords, we also allow types and values to
%% attach appropriate bindings that can potentially define how to parse a
%% particular type and value as well as connecting them to an appropriate
%% type representation. For example, currently all primitive types (such
%% as integers, booleans, or strings) are defined using a binding as well
%% as basic values such as \texttt{true} or \texttt{false}.

%% Such flexible mechanism, especially when defined in Wyvern itself
%% makes any DSL extension as flexible and natural to define as writing
%% any normal Wyvern program itself.


%% \subsection{Examples}

%% We now present a small number of examples that demonstrate and explain
%% the operation of Wyvern.

%% \begin{figure}
%%   \centering
%%   \label{f:eg-let-eg}
%%   \caption{Wyvern Example: \texttt{let}}
%% \end{figure}
%% The \lstinline{let} parser knows that the first few lines following
%% the \lstinline{let} keyword will be variable definitions of the form
%% \lstinline{varname = value} and therefore can easily parse the lines
%% that following until a token \lstinline{in} on an indented line of its
%% own is encountered.  The body of the \lstinline{let} statement can now
%% be parsed using a continuation of the original Wyvern parser with an
%% addition of variables \lstinline{x} and \lstinline{z} to the current
%% environment. When the indented block finishes, the main Wyvern parser
%% can continue without \lstinline{let} variables in the environment. One
%% can observe that such parsing is very suitable to performing parsing
%% and type checking at the same time.

%% Our next example presents a simple \lstinline{if} statement in
%% Figure~\ref{f:eg-if}. Just like with the \lstinline{let} statement,
%% one can see how upon seeing the \lstinline{if} keyword, Wyvern parser
%% can invoke an appropriate \lstinline{if} parser that can easily be
%% replaced or modified. The default \lstinline{if} parsers parses the
%% conditional and is able to parse the body using the continuation of
%% the main parser with the right value of the condition taken into
%% account. One can observe that extending Wyvern to perform flow
%% sensitive checks such as non-nullness can be done easily even during
%% the parsing stage.

%% \begin{figure}
%%   \centering
%%   \begin{lstlisting}
%% if x = 5
%%     do stuff
%%   else
%%     do stuff
%%   \end{lstlisting}
%%   \label{f:eg-if}
%%   \caption{Wyvern Example: \texttt{if}}
%% \end{figure}

%% A more interesting variation of the \lstinline{if} statement shown in
%% Figure~\ref{f:eg-if-smalltalk} can be one that attaches a parser to
%% the boolean type rather than a specific keyword. In the spirit of
%% Smalltalk, one can provide a parser in Wyvern that is invoked when an
%% expression of a boolean type is encountered. Such parser can then
%% check if the next keyword is \lstinline{iftrue} that can be followed
%% by the true block parsed by the continuation of the normal
%% parser. Furthermore, an additional \lstinline{iffalse} keyword can be
%% allowed that contains the else part.

%% \begin{figure}
%%   \centering
%%   \begin{lstlisting}
%% condition iftrue
%%     [block]
%%   iffalse
%%     [block]
%%   \end{lstlisting}
%%   \label{f:eg-if-smalltalk}
%%   \caption{Wyvern Example: \texttt{if} in Smalltalk style}
%% \end{figure}

%% As a more complex example, consider an attempt to express a simple
%% Domain Specific Language that roughly corresponds to
%% HTML. Figure~\ref{f:eg-html} shows such extension that makes each
%% major html tag a keyword with its own parser attached to it. In fact,
%% each keyword can be treated as a first class value that can be stored
%% in a variable resulting in body being passed around or used as part of
%% a \lstinline{let} expression. The indentation makes clear when each
%% tag's parser needs to stop and in our opinion makes a body of a web
%% page much more readable, yet structured and even suitable for type
%% checking.

%% \begin{figure}
%%   \centering
%%   \begin{lstlisting}
%% let 
%%     theBody = body
%%         h1
%%             "This is a section header"
%%   in  
%% 	html
%% 	    head
%% 	        title: "The title of the page"
%% 	        script(src="http://.../")
%% 	        style
%% 	            body
%% 	                background-color: "white"
%% 	    theBody
%%   \end{lstlisting}
%%   \label{f:eg-html}
%%   \caption{Wyvern Example: HTML}
%% \end{figure}

%% \begin{figure}
%%   \centering
%%   \label{f:eg-arch}
%%   \begin{lstlisting}
%%   \end{lstlisting}
%%   \caption{Wyvern Example: Architecture}
%% \end{figure}

%% \begin{figure}
%%   \centering
%%   \label{f:eg-security}
%%   \begin{lstlisting}
%%   \end{lstlisting}
%%   \caption{Wyvern Example: Security Policy}
%% \end{figure}

%% \begin{figure*}
%%   \centering
%%   \label{f:eg-dsl-db}
%%   \begin{lstlisting}
%% assuming 
%%   theDB : OracleDB 
%% (that is, theDB is a DB that supports oracle-specific queries)

%% type-directed parsing:

%% theDB query
%%   (* stuff in the syntax of Oracle DBs *)

%% vs

%% keyword-directed parsing:

%% OracleDBquery theDB
%%   (* stuff in the syntax of Oracle DBs *)
%%   \end{lstlisting}
%%   \caption{Wyvern Example: DSL for DB Definition}
%% \end{figure*}



%For example, the Wyvern default environment contains bindings for
%keywords like \keyw{class} or \keyw{meth} that are able to parse
%class and method declarations respectively. Parsers are bound to types and, as such, may be shadowed within application code. Only operators and their associated precedence are not exposed in the environment and, thus, have to be reused by the DSLs, though it is feasible to lift them up to the
%environment with an appropriate parser.
%
%% , an infix nature of operators
%% might make parsing definition more complex with arguably small gains
%% in expressiveness for potential DSLs.

%The core parsing in Wyvern is an extremely simple mechanism. The language uses a core parser that for every token encountered delegates to the appropriate parser in the environment. This extension parser processes the token and any relevant ones that follow. Once the extension parser is finished, the core parser continues through the remaining tokens, terminating when it either runs out of tokens or encounters a token with no associated extension parser.

%% In a similar manner to keywords, we also allow types and values to
%% attach appropriate bindings that can potentially define how to parse a
%% particular type and value as well as connecting them to an appropriate
%% type representation. For example, currently all primitive types (such
%% as integers, booleans, or strings) are defined using a binding as well
%% as basic values such as \texttt{true} or \texttt{false}.

%% Such flexible mechanism, especially when defined in Wyvern itself
%% makes any DSL extension as flexible and natural to define as writing
%% any normal Wyvern program itself.


%% \subsection{Examples}

%% We now present a small number of examples that demonstrate and explain
%% the operation of Wyvern.

%% \begin{figure}
%%   \centering
%%   \label{f:eg-let-eg}
%%   \caption{Wyvern Example: \texttt{let}}
%% \end{figure}


\section{Discussion and Future Work}\label{s:discussion}
\paragraph{Keyword-Directed Invocation}
In most DSL frameworks, a switch to a DSL is indicated by a keyword or function call naming the DSL to be used. Wyvern eliminates this overhead in many cases by determining the DSL based on the expected type of an expression. This lightweight mechanism is particularly useful for small DSLs, like the one associated with \lstinline{URL}. Keyword-directed invocation of a DSL is simply a special case of this approach. In particular, a keyword macro can be defined as a function with a single argument of a type specific to that keyword. The type contains the implementation of the domain-specific syntax associated with that keyword. In the most general sense, it may simply allow the entire \keyw{EXP} grammar, manipulating it in later phases of compilation. 

As an example, consider control flow operators like \verb|if|. This can be defined as a polymorphic method of the \verb|bool| type with signature $(\texttt{unit} \rightarrow \alpha, \texttt{unit} \rightarrow \alpha) \rightarrow \alpha$. That is, it takes the two branches as functions and chooses which to invoke based on the value of the boolean, using perhaps a more primitive control flow operator, like case analysis, or even a Church encoding of booleans as functions. In Wyvern, the branches could be packaged together into a type, \verb|IfBranches|, with an associated grammar that accepts the two branches as unwrapped expressions. Thus, \verb|if| could be defined entirely in a library and used as follows: 

\begin{lstlisting}
<guard>.if(~)
  then
    <any EXP>
  else
    <any EXP>
\end{lstlisting}

For methods like \verb|if| where constructing the argument explicitly will almost never be done, it may be useful to mark the method in a way that allows Wyvern to assume it is being called with a DSL argument immediately following its use. This would eliminate the need for the \verb|(~)| portion, supporting even more conventional notation. We have not considered this possibility in detail.

\paragraph{Explicit Delimiters}
Throughout this paper, DSLs have been delimited by whitespace. This allows arbitrary syntax within DSLs, since no delimiters need to be reserved to indicate the end of the DSL and thus there is no need for escaping internal uses of these delimiters. In cases where DSL expressions are expected to be reasonably short, such as the \lstinline{URL} example, or where delimiters are more natural than whitespace, such as for array or dictionary literals, it may be desirable to support other forms of delimited ``DSL literals''. 

One possible strategy for this is to reserve a number of common delimiter forms, such as quotation marks and  forms of braces, as equivalent DSL literal forms. The traditional meaning of these delimiters, such as quotation marks for strings and square brackets for lists, would then simply be convention in Wyvern. That is, the following expressions, as well as several similar ones, would be precisely equivalent (the programmer could choose the most convenient form, given the enclosed term):
\begin{verbatim}
  f("http://github.com/wyvernlang")  
  f([http://github.com/wyvernlang])
\end{verbatim}

Alternatively, types could specify the set of permitted delimiters so that conventions can be enforced by the compiler, improving identifiability. We have not yet explored either of these possibilities in detail, nor explored options that allow \emph{arbitrary} type-specified delimiters (a naive strategy for which would require that the first phase of parsing also be type-directed, which we wish to avoid).

\paragraph{Interaction with Subtyping}
The mechanism described here does not consider the case where multiple subtypes of a base type define a grammar. This can be resolved in several ways. We could require that only the \emph{declared} type's grammar is used (if a subtype's grammar is desired, an explicit type annotation on the tilde can be used). Alternatively, we could attempt to parse against all relevant subtypes, only requiring explicit disambiguation when ambiguities arise. Wyvern does not currently support subtyping, so we leave this as future work.
%%\paragraph{Custom Contexts}
%% One can observe a number of interesting issues that can be seen from
%% our Wyvern examples. For example, our \lstinline{let} statement
%% requires a corresponding \lstinline{in} part to be indented
%% differently from the other sub-blocks. This means that the entire
%% expression cannot be parsed by simply processing the appropriately
%% indented line sequence and then returing the control back to the
%% Wyvern parser, but rather there needs to be a way to detect that the
%% parser is now back to the level of the original \lstinline{let}
%% statement and the environment can be reset to what it was before the
%% statement.
%
%Presently, grammars cannot define their own contexts that are passed through nested grammars, but this would likely be a useful addition in many cases. As an example, consider a \lstinline{let} expression defined by a functional programming based DSL in 
%Wyvern:
%
%\begin{lstlisting}
%let 
%    x = 5
%    z = 4
%  in
%    x + z
%\end{lstlisting}
%
%\noindent 
%This expression requires maintaining a context of bound variables. This context must be maintained such that any expression within the \verb|in| block that is written in this DSL has access to the appropriate context, even if it appears as a subexpression within a different DSL. Multiple DSLs may define different contexts, and these must be threaded throughout nested DSLs in a modular manner.
%
%\paragraph{Custom Lexers} 
%Our existing lexing strategy may be too restrictive, requiring all DSLs to be hierarchical in nature. One potential expansion would be to enable DSLs to define their own lexers, still perhaps delimited by indentation or parentheses. Such an extension would sacrifice some readability.
%%However, we have been unable to find convincing use cases for this facet of extensibility. 
%
%Wyvern's operators are built in and their precedence follows that of
%C++ operators. We do not allow a replacement parser for infix
%operators as we considered it to unnecessarily complicate the current
%prefixed parsing approach. In the future, we plan to further support
%redefining operators.
%
%Finally, following Python and some other whitespace delimited
%languages, we may wish to allow parenthesized expressions to avoid the need for
%following the indentation level. This is still subject to debate and,
%as we try to express more and more DSL kinds in Wyvern, we may decide
%to enforce indentation levels even inside the parentheses.

\section{Related Work}
\label{s:related}
The most common mechanism for creating programming language extensions is macros, as exemplified by hygienic macros in Scheme. Macros in Scheme and other Lisp-style languages are written in Lisp itself, and benefit from the regular syntax of the underlying language and the consistent use of parentheses as expression delimiters.  The syntactic nature of macros allows them to control parsing within a delimited scope, and this delimited control of parsing inspired our approach in Wyvern.  Wyvern's use of types to trigger custom parsers avoids the Lisp limitation of starting each DSL with a keyword or macro name.

Wyvern chooses whitespace delimitation rather than parenthesis delimitation because, anecdotally, many programmers find syntax in that style to be more readable.  There are several proposals for making Lisp and its dialects whitespace-delimited, including sweet-expressions \cite{sweet-expr} and SRFI-49 \cite{srfi-49} for Scheme, P4P proposal for Racket \cite{p4p-proposal}, Kernel's f-expressions \cite{fexprthesis} for Common Lisp \cite{f-expr}, and others. None of them seem to be successfully implemented and deployed within the Lisp communities, however.  This may be because Lisp programmers don't want to switch away from parentheses, yet having an alternate syntax is not by itself enough to draw non-Lisp programmers into the community.

%Therefore, our approach considers the readability of language extensions from the beginning: we attempt to devise the most comfortable and natural notations for the language extensions without bothersome extra keywords and try to keep it succinct.

Some language extensibility projects provide extension at levels of abstraction above parsing. For instance, OJ (previously, OpenJava)
\cite{Tatsubori00openjava:a} provides a macro system based on a meta-object
protocol, and Backstage Java~\cite{Palmer:2011:BJM:2048066.2048137} employs compile-time meta-programming.  Both of these systems provide macro-style rewriting of existing source code, but they provide at most limited extension of language parsing.

Other systems aim at providing more flexible forms of syntax extension compared to our whitespace-delimited approach, at the potential cost of complexity or conflicts.  Camlp4 \cite{camlp4} is a preprocessor for OCaml that offers the developer the ability to extend the concrete syntax of the language via the use of parsers and extensible grammars.  SugarJ \cite{Erdweg:2011:SLL:2048147.2048199} takes a library-centric approach which allows to syntactically extend the Java language by adding libraries. In Wyvern, the core language (that is, the \verb|Exp| sort) does not get extended directly, so conflicts cannot arise and reasoning about which DSL is applicable is much simpler.

Finally, researchers have developed a number of DSL frameworks and workbenches, such as the Meta Programming System (MPS) \cite{mps}, Spoofax \cite{KatsVisser2010}, the Intentional Domain Workbench (IDW) \cite{idw}, and Ens\={o} \cite{enso}, that provide support for extending a variety of programming languages by applying generic rules and concepts.  The Marco language \cite{lee:2012:marco} similarly provides macro definition at a level of abstraction that is largely independent of the target language.

Compared to these approaches, Wyvern focuses on making one programming language extensible with embedded DSLs, provides a common lexer and extensible parsing confined to whitespace-delimited regions, and investigates a novel type-based approach to triggering parser extensions.

%programming
%language extensions to a higher level of abstraction and creates a
%system that consumes the target language and rules for creating
%macros and produces a safe and expressive language. In contrast, our
%approach aims only at the Wyvern programming language and tries to
%enhance its extensibility.

%However, Camlp4 does not support types for the language extensions and thus is missing the additional layer of safety that our approach provides.

%While aiming for the same goal, Backstage Java  and SugarJ
% take very different approaches.   

% !TEX root = globaldsl13.tex
%\section{Conclusion} % and Future Work}
%\label{s:conclusion}
%
%In this paper, we described how extensible parsing in Wyvern makes for
%a solid platform to support whitespace-delimited, type-directed embedded DSLs. In the
%future, we aim to implement a wide variety of DSLs in Wyvern tweaking
%our approach and implementation thereof to provide a comprehensive example of
%supporting multiple interacting DSLs in a safe and easy-to-use manner.
%
% \todo{tie features to goals}

% \todo{implementation and validation plans}

\section*{Acknowledgements}
We thank the anonymous reviewers for helpful comments, and acknowledge the support of the Department of Defense and the Air Force Research Laboratory. CO is supported by the NSF Graduate Research Fellowship.

%TODO TODO TODO TODO TODO TODO TODO TODO add acknowledgments here!!!!!!  ARO

\end{sloppypar}

\bibliographystyle{plain}
\bibliography{biblio}
\end{document}