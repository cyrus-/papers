\section{Related Work}
\label{s:related}
The most common mechanism for creating programming language extensions is macros, as exemplified by hygienic macros in Scheme. Macros in Scheme and other Lisp-style languages are written in Lisp itself, and benefit from the regular syntax of the underlying language and the consistent use of parentheses as expression delimiters.  The syntactic nature of macros allows them to control parsing within a delimited scope, and this delimited control of parsing inspired our approach in Wyvern.  Wyvern's use of types to trigger custom parsers avoids the Lisp limitation of starting each DSL with a keyword or macro name.

Wyvern chooses whitespace delimitation rather than parenthesis delimitation because, anecdotally, many programmers find syntax in that style to be more readable.  There are several proposals for making Lisp and its dialects whitespace-delimited, including sweet-expressions \cite{sweet-expr} and SRFI-49 \cite{srfi-49} for Scheme, P4P proposal for Racket \cite{p4p-proposal}, Kernel's f-expressions \cite{fexprthesis} for Common Lisp \cite{f-expr}, and others. None of them seem to be successfully implemented and deployed within the Lisp communities, however.  This may be because Lisp programmers don't want to switch away from parentheses, yet having an alternate syntax is not by itself enough to draw non-Lisp programmers into the community.

%Therefore, our approach considers the readability of language extensions from the beginning: we attempt to devise the most comfortable and natural notations for the language extensions without bothersome extra keywords and try to keep it succinct.

Some language extensibility projects provide extension at levels of abstraction above parsing. For instance, OJ (previously, OpenJava)
\cite{Tatsubori00openjava:a} provides a macro system based on a meta-object
protocol, and Backstage Java~\cite{Palmer:2011:BJM:2048066.2048137} employs compile-time meta-programming.  Both of these systems provide macro-style rewriting of existing source code, but they provide at most limited extension of language parsing.

Other systems aim at providing more flexible forms of syntax extension compared to our whitespace-delimited approach, at the potential cost of complexity or conflicts.  Camlp4 \cite{camlp4} is a preprocessor for OCaml that offers the developer the ability to extend the concrete syntax of the language via the use of parsers and extensible grammars.  SugarJ \cite{Erdweg:2011:SLL:2048147.2048199} takes a library-centric approach which allows to syntactically extend the Java language by adding libraries. In Wyvern, the core language (that is, the \verb|Exp| sort) does not get extended directly, so conflicts cannot arise and reasoning about which DSL is applicable is much simpler.

Finally, researchers have developed a number of DSL frameworks and workbenches, such as the Meta Programming System (MPS) \cite{mps}, Spoofax \cite{KatsVisser2010}, the Intentional Domain Workbench (IDW) \cite{idw}, and Ens\={o} \cite{enso}, that provide support for extending a variety of programming languages by applying generic rules and concepts.  The Marco language \cite{lee:2012:marco} similarly provides macro definition at a level of abstraction that is largely independent of the target language.

Compared to these approaches, Wyvern focuses on making one programming language extensible with embedded DSLs, provides a common lexer and extensible parsing confined to whitespace-delimited regions, and investigates a novel type-based approach to triggering parser extensions.

%programming
%language extensions to a higher level of abstraction and creates a
%system that consumes the target language and rules for creating
%macros and produces a safe and expressive language. In contrast, our
%approach aims only at the Wyvern programming language and tries to
%enhance its extensibility.

%However, Camlp4 does not support types for the language extensions and thus is missing the additional layer of safety that our approach provides.

%While aiming for the same goal, Backstage Java  and SugarJ
% take very different approaches.   
