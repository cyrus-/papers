%\documentclass[12pt]{article}
\documentclass[10pt,preprint]{sigplanconf}
% The following \documentclass options may be useful:
%
% 10pt          To set in 10-point type instead of 9-point.
% 11pt          To set in 11-point type instead of 9-point.
% authoryear    To obtain author/year citation style instead of numeric.
\usepackage{amsmath}
\usepackage{amssymb} 
\usepackage{amsthm}
\usepackage{ stmaryrd }
\usepackage{mathpartir}
\usepackage[usenames,dvipsnames,svgnames,table]{xcolor}

\newcommand{\keyw}[1]{{\sf #1}}
\newcommand{\mvlbl}[1]{\textbf{\textsf{#1}}}
\newcommand{\technical}[1]{{\color[gray]{0.6} #1}}
\newcommand{\karrow}[2]{#1 \rightarrow #2}
\newcommand{\kalpha}{\boldsymbol{\alpha}}
\newcommand{\kforall}[2]{\forall(#1.#2)}
\newcommand{\kvar}{\textbf{\textit{k}}}
\newcommand{\kmu}[2]{\mu_\text{ind}(#1.#2)}
\newcommand{\kunit}{\keyw{1}}
\newcommand{\kprod}[2]{#1\times#2}
\newcommand{\ksum}[2]{#1+#2}
\newcommand{\kty}{\keyw{Ty}}
\newcommand{\kity}{\keyw{ITy}}
\newcommand{\kitm}{\keyw{ITm}}
\newcommand{\st}{\sigma}
\newcommand{\stx}[1]{\st_\text{#1}}
\newcommand{\sttyidx}{\stx{tyidx}}
\newcommand{\sttmidx}{\stx{tmidx}}
\newcommand{\svar}[1]{{\textbf{\textit{#1}}}}
\newcommand{\sx}{\svar{x}}
\newcommand{\sy}{\svar{y}}
\newcommand{\sz}{\svar{z}}
\newcommand{\slam}[3]{\lambda #2{:}#1.#3}
\newcommand{\sap}[2]{#1~#2}
\newcommand{\sLam}[2]{\Lambda(#1.#2)}
\newcommand{\skap}[2]{#2\texttt{[}#1\texttt{]}}
\newcommand{\sfold}[3]{\keyw{fold}[#1.#2](#3)}
\newcommand{\srec}[4]{\keyw{rec}[#1](#2; #3.#4)}
\newcommand{\striv}{()}
\newcommand{\spair}[2]{(#1, #2)}
\newcommand{\sletpair}[4]{\keyw{letpair}(#1; #2,#3.#4)}
\newcommand{\sinl}[2]{\keyw{inl}[#1](#2)}
\newcommand{\sinr}[2]{\keyw{inr}[#1](#2)}
\newcommand{\scase}[5]{\keyw{case}(#1; #2.#3; #4.#5)}
\newcommand{\tcvar}[1]{\textsc{#1}}
\newcommand{\tc}{\tcvar{tc}}
\newcommand{\sty}[2]{\keyw{ty}[#1](#2)}
\newcommand{\sparr}[1]{\keyw{ty}[\rightharpoonup](#1)}
\newcommand{\sotherty}[2]{\keyw{otherty}[#1; #2]}
\newcommand{\stycase}[5]{\keyw{tycase}[#1](#2; #3.#4; #5)}
\newcommand{\sparrcase}[4]{\keyw{tycase}[\rightharpoonup](#1; #2.#3; #4)}
\newcommand{\sqity}[1]{{\blacktriangleright}(#1)}
\newcommand{\qity}{\grave{\tau}}
\newcommand{\srep}[1]{\keyw{rep}(#1)}
\newcommand{\sity}[1]{\keyw{ity}[#1]}
\newcommand{\ity}{\tau}
\newcommand{\sqitm}[1]{{\rhd}(#1)}
\newcommand{\qitm}{\grave{\iota}}
\newcommand{\sitm}[1]{\keyw{itm}[#1]}
\newcommand{\itm}{\iota}
\newcommand{\sana}[2]{\keyw{ana}[#1](#2)}
\newcommand{\ssyn}[1]{\keyw{syn}[#1]}
\newcommand{\sraise}[1]{\keyw{raise}[#1]}
\newcommand{\quqty}[1]{{\blacktriangleleft}(#1)}
\newcommand{\quqtm}[1]{{\lhd}(#1)}
\newcommand{\kGamma}{\boldsymbol\Gamma}
\newcommand{\kDelta}{\boldsymbol\Delta}
\newcommand{\kTheta}{\boldsymbol\Theta}
\newcommand{\kXi}{\boldsymbol\Xi}
\newcommand{\tcsig}[2]{\keyw{tcsig}[#1]\,\{#2\}}
\newcommand{\opsigs}{\Omega}
\newcommand{\litsig}[1]{\keyw{lit}[#1]}
\newcommand{\opsig}[2]{#1[#2]}
\newcommand{\tycons}[1]{#1~\mathtt{tycons}}
\newcommand{\tcdeclsok}[1]{#1~\mathtt{decls}}
\newcommand{\tcdefsok}[1]{#1~\mathtt{defs}}
\newcommand{\kok}[3]{#1~#2 \vdash #3}
\newcommand{\keq}[2]{#1 \vdash #2~\mathtt{eq}}
\newcommand{\kpos}[2]{#1.#2~\mathtt{pos}}
\newcommand{\sofkn}[6]{#1~#2 \vdash_{#3}^{#4} #5 : #6}
\newcommand{\sofk}[5]{\sofkn{#1}{#2}{#3}{n}{#4}{#5}}
\newcommand{\sofkz}[5]{\sofkn{#1}{#2}{#3}{0}{#4}{#5}}
\newcommand{\sofkzX}[2]{\sofkz{\kDelta}{\kGamma}{\kXi}{#1}{#2}}
\newcommand{\sqofk}[4]{#1~#2 \vdash_{#3}^{n} #4}
\newcommand{\sofkX}[2]{\sofk{\kDelta}{\kGamma}{\kXi}{#1}{#2}}
\newcommand{\sqofkX}[1]{\sqofk{\kDelta}{\kGamma}{\kXi}{#1}}
\newcommand{\serr}[4]{#1 \parallel #2~#3~\mathtt{err}_{#4}}
\newcommand{\serrX}[1]{\serr{#1}{\memD}{\memG}{\aCtx}}
\newcommand{\snorm}[7]{#1 \parallel #2~#3 \Downarrow_{#4} #5 \parallel #6~#7}
\newcommand{\memD}{\mathcal{D}}
\newcommand{\memG}{\mathcal{G}}
\newcommand{\aCtx}{\mathcal{C}}
\newcommand{\snormX}[2]{\snorm{#1}{\memD}{\memG}{\aCtx}{#2}{\memD'}{\memG'}}
\newcommand{\snormT}[2]{#1 \Downarrow #2}
\newcommand{\kctxok}[2]{#1 \vdash #2}
\newcommand{\tccok}[2]{\vdash #1 \sim #2}
\newcommand{\tccokX}{\tccok{\Phi}{\kXi}}
\newcommand{\tcok}[4]{\vdash_{#1} #2 \sim #3[#4]}
\newcommand{\tcsigok}[2]{\vdash #1 \sim #2}
\newcommand{\ocok}[4]{\vdash_{#1} #3~\{#2\} : #4}
\newcommand{\ocsigok}[1]{\vdash #1}
\newcommand{\sfinalrep}[3]{\vdash_{#1} #2 \leadsto #3}
\newcommand{\sfinalrepX}[2]{\sfinalrep{\Phi}{#1}{#2}}
\newcommand{\gfinalrep}[3]{\vdash_{#1} #2 \leadsto #3}
\newcommand{\gfinalrepX}[2]{\gfinalrep{\Phi}{#1}{#2}}
\newcommand{\svalofk}[2]{\sofkz{\emptyset}{\emptyset}{\kXi}{#1}{#2}}
\newcommand{\stype}[2]{#1~\mathtt{type}_{#2}}
\newcommand{\stypeX}[1]{#1~\mathtt{type}_{\Phi}}
\newcommand{\sstep}[7]{#1 \parallel #2~#3 \mapsto_{#4} #5 \parallel #6~#7}
\newcommand{\sstepX}[2]{\sstep{#1}{\memD}{\memG}{\aCtx}{#2}{\memD'}{\memG'}}
\newcommand{\sval}[4]{#1 \parallel #2~#3~\mathtt{val}_{#4}}
\newcommand{\svalX}[1]{\sval{#1}{\memD}{\memG}{\aCtx}}
\newcommand{\qparr}[2]{#1 \rightharpoonup #2}
\newcommand{\qforall}[2]{\forall(#1.#2)}
\newcommand{\qunit}{\keyw{1}}
\newcommand{\qprod}[2]{#1 \times #2}
\newcommand{\qsum}[2]{#1 + #2}
\newcommand{\qmu}[2]{\mu(#1.#2)}
\newcommand{\qtuq}[1]{{\blacktriangleleft}(#1)}
\newcommand{\qfix}[3]{\keyw{qfix}[#1](#2.#3)}
\newcommand{\qlam}[3]{\lambda[#1](#2.#3)}
\newcommand{\qap}[2]{#1~#2}
\newcommand{\qLam}[2]{\Lambda(#1.#2)}
\newcommand{\qAp}[2]{#1\texttt{[}#2\texttt{]}}
\newcommand{\qtriv}{()}
\newcommand{\qpair}[2]{(#1, #2)}
\newcommand{\qletpair}[4]{\keyw{letpair}(#1; #2, #3.#4)}
\newcommand{\qinl}[2]{\keyw{inl}[#1](#2)}
\newcommand{\qinr}[2]{\keyw{inr}[#1](#2)}
\newcommand{\qcase}[5]{\keyw{case}(#1; #2.#3; #4.#5)}
\newcommand{\qfold}[3]{\keyw{fold}[#1.#2](#3)}
\newcommand{\qunfold}[1]{\keyw{unfold}(#1)}
\newcommand{\quq}[1]{{\lhd}(#1)}
\newcommand{\eslet}[3]{\keyw{letstatic}[#1](#2.#3)}
\newcommand{\easc}[2]{\keyw{asc}[#2](#1)}
\newcommand{\elet}[3]{\keyw{let}(#1; #2.#3)}
\newcommand{\efix}[2]{\keyw{fix}(#1.#2)}
\newcommand{\eanalam}[2]{\lambda(#1.#2)}
\newcommand{\elam}[3]{\lambda[#1](#2.#3)}
\newcommand{\eap}[2]{\keyw{ap}(#1; #2)}
\newcommand{\eLam}[2]{\Lambda(#1.#2)}
\newcommand{\eAp}[2]{#1\texttt{[}#2\texttt{]}}
\newcommand{\efold}[1]{\keyw{fold}(#1)}
\newcommand{\eunfold}[1]{\keyw{unfold}(#1)}
\newcommand{\elit}[2]{\keyw{lit}[#1](#2)}
\newcommand{\etarg}[4]{\keyw{targ}[#1; #2](#3; #4)}
\newcommand{\eother}[1]{\keyw{other}[#1]}
\newcommand{\es}{\overline{e}}
\newcommand{\eDelta}{\Delta}
\newcommand{\eTheta}{\Theta}
\newcommand{\tcdecl}[2]{#1 \hookrightarrow #2}
\newcommand{\tcdef}[3]{\keyw{tycon}~#1 ~\{#3\} : #2}
\newcommand{\opdefs}{\omega}
\newcommand{\opname}[1]{\textbf{#1}}
\newcommand{\analitop}[1]{\keyw{ana~lit}=#1}
\newcommand{\anaop}[2]{\keyw{ana}~#1=#2}
\newcommand{\synop}[2]{\keyw{syn}~#1=#2}
\newcommand{\tcdeclstd}{\tcdecl{\tc}{\tcdefmv}}
\newcommand{\tcdefmv}{\psi}
\newcommand{\ktyidx}{\kappa_\text{tyidx}}
\newcommand{\strep}{\st_\text{rep}}
\newcommand{\klitidx}{\kappa_\text{tmidx}}
\newcommand{\sanalit}{\st_\text{lit}}
\newcommand{\ktargidx}{\kappa_\text{tmidx}}
\newcommand{\ssyntarg}{\st_\text{targ}}
\newcommand{\kargs}{\klist{\karg}}
\newcommand{\karg}{\kappa_\text{arg}}
\newcommand{\kana}{\kappa_\text{ana}}
\newcommand{\ksyn}{\kappa_\text{syn}}
\newcommand{\stmidx}{\sigma_\text{tmidx}}
\newcommand{\esyn}[5]{#1 \vdash_{#2} #3 \Rightarrow #4 \leadsto #5}
\newcommand{\eana}[5]{#1 \vdash_{#2} #3 \Leftarrow #4 \leadsto #5}
\newcommand{\esynX}[3]{\esyn{\Upsilon}{\Phi}{#1}{#2}{#3}}
\newcommand{\eanaX}[3]{\eana{\Upsilon}{\Phi}{#1}{#2}{#3}}
\newcommand{\mkargs}[4]{#1 \looparrowright #2 \looparrowright #3; #4}
\newcommand{\mkargsX}{\mkargs{\es}{\stx{args}}{\memG_0}{n}}
\newcommand{\ctxok}[2]{\vdash_{#1} #2}
\newcommand{\iparr}[2]{#1 \rightharpoonup #2}
\newcommand{\iforall}[2]{\forall(#1.#2)}
\newcommand{\imu}[2]{\mu(#1.#2)}
\newcommand{\iunit}{\keyw{1}}
\newcommand{\iprod}[2]{#1 \times #2}
\newcommand{\isum}[2]{#1 + #2}
\newcommand{\ifix}[3]{\keyw{fix}[#1](#2.#3)}
\newcommand{\ilam}[3]{\lambda[#1](#2.#3)}
\newcommand{\iap}[2]{\keyw{ap}(#1; #2)}
\newcommand{\iLam}[2]{\Lambda(#1.#2)}
\newcommand{\iAp}[2]{#2\texttt{[}#1\texttt{]}}
\newcommand{\ifold}[3]{\keyw{fold}[#1.#2](#3)}
\newcommand{\iunfold}[1]{\keyw{unfold}(#1)}
\newcommand{\itriv}{\keyw{triv}}
\newcommand{\ipair}[2]{\keyw{pair}(#1; #2)}
\newcommand{\iletpair}[4]{\keyw{letpair}(#1; #2, #3.#4)}
\newcommand{\iinl}[2]{\keyw{inl}[#1](#2)}
\newcommand{\iinr}[2]{\keyw{inr}[#1](#2)}
\newcommand{\icase}[5]{\keyw{case}(#1; #2.#3; #4.#5)}
\newcommand{\iGamma}{\Gamma}
\newcommand{\iDelta}{\Delta}
\newcommand{\iTheta}{\Theta}
\newcommand{\swrong}{\keyw{wrong}}
\newcommand{\linenumber}[1]{{\tt{\color{gray} #1}}}
\newcommand{\concty}[2]{\fvar{#1}~#2}
\newcommand{\concrx}[1]{\texttt{\scriptsize /#1/}}
\newcommand{\concstr}[1]{\texttt{\scriptsize "#1"}}
\newcommand{\conctarg}[1]{.\opname{#1}}
\newcommand{\conclbl}[1]{\texttt{\scriptsize #1}}
\newcommand{\digit}{\textbackslash d}
\newcommand{\rdot}{\textbackslash .}
\newcommand{\conckap}[2]{#1\texttt{[}#2\texttt{]}}
\newcommand{\desugar}[1]{\lfloor #1 \rfloor}
\newcommand{\klbl}{\keyw{Lbl}}
\newcommand{\kstr}{\keyw{Str}}
\newcommand{\krx}{\keyw{Rx}}
\newcommand{\knat}{\keyw{Nat}}
\newcommand{\scolorc}{\color{JungleGreen}}
\newcommand{\scolor}[1]{{\scolorc #1}}
\newcommand{\qcolorc}{\color{Cerulean}} %\color[gray]{0.6}
\newcommand{\qcolor}[1]{{\qcolorc #1}}
\newcommand{\cnl}{\\\scolorc}
\newcommand{\klist}[1]{\keyw{List}[#1]}

\usepackage{times}
\renewcommand{\ttdefault}{txtt}
\usepackage{alltt}
\usepackage{listings}
\lstset{language=ML,
showstringspaces=false,
basicstyle=\ttfamily\footnotesize,
morekeywords={newcase,extends}}

\usepackage{url}
\usepackage{todonotes}
\lefthyphenmin=5
\sloppy

\newcommand{\moutput}{^{\color{gray}+}}
\newcommand{\rulename}[1]{(#1)}
\def \TirNameStyle #1{\small\rulename{#1}}
\renewcommand{\MathparLineskip}{\lineskiplimit=.6\baselineskip\lineskip=.6\baselineskip plus .2\baselineskip}

\newtheorem{theorem}{Theorem}
\newtheorem{lemma}{Lemma}
\newtheorem{corollary}{Corollary}
\newtheorem{definition}{Definition}
\newtheorem{tyconinvariant}{Tycon Invariant}
\newenvironment{proof-sketch}{\noindent{\emph{Proof Sketch.}}}{\qed}
\makeatletter

% Heather-added packages for the fancy table
\usepackage{longtable}
\usepackage{booktabs}
\usepackage{pdflscape}
\usepackage{colortbl}%
\newcommand{\myrowcolour}{\rowcolor[gray]{0.925}}
\usepackage{wasysym}

\renewcommand\topfraction{0.85}
\renewcommand\bottomfraction{0.85}
\renewcommand\textfraction{0.1}
\renewcommand\floatpagefraction{0.85}

\begin{document}

\conferenceinfo{-}{-} 
\copyrightyear{-} 
\copyrightdata{[to be supplied]} 

%\titlebanner{{\tt \textcolor{Red}{{\small Under Review -- distribute within CMU only.}}}}        % These are ignored unless
%\preprintfooter{Distribute within CMU only.}   % 'preprint' option specified.

\title{Modularly Composing Typed Language Fragments}
\subtitle{Supplemental Material}
\authorinfo{}{}{}
%\authorinfo{Cyrus Omar \and Jonathan Aldrich}
 %         {Carnegie Mellon University}
  %         {\{comar, aldrich\}@cs.cmu.edu}   

\maketitle

%\category{D.3.2}{Programming Languages}{Language Classifications}[Extensible Languages]
%\category{D.3.4}{Programming Languages}{Processors}[Compilers]
%\category{F.3.1}{Logics \& Meanings of Programs}{Specifying and Verifying and Reasoning about Programs}[Specification Techniques]
%\keywords
%extensible languages; module systems; type abstraction; typed compilation; type-level computation

\section{Internal Language}
\todo{copy from one of the 312 HWs}

\subsection{Substitutions}

\subsection{Abstraction Theorem}
\todo{hm...}

\section{Tycon Contexts}
\subsection{Tycon Context Well-Definedness}

\subsection{Equality Kinds}
Need an equational theory for SL to state equality kind property, but not important for other metatheory. 

\subsection{Full Examples}

\section{Static Language}
\subsection{Kind Formation}

\subsection{Kinding Context Formation}

\subsection{Kinding}

\subsection{Dynamic Semantics}

\subsection{Kind Safety}

\section{Types}
\subsection{Type Translations}

\subsection{Typing Context Translations}

\paragraph{Unicity}
The rules are structured so that if a term is well-typed, both its type and translation are unique.\todo{could move this whole thing to supplement if room needed}
\begin{theorem}[Unicity]
If $\vdash \Phi$ and $\vdash_\Phi \Upsilon \leadsto \Gamma$ and $\vdash_\Phi \st \leadsto \tau$ and $\vdash_\Phi \st' \leadsto \tau'$ and $\eanaX{e}{\st}{\iota}$ and $\eanaX{e}{\st'}{\iota'}$ then $\st = \st'$ and $\tau=\tau'$ and $\iota = \iota'$.
\end{theorem}

\section{External Language}
\subsection{Additional Desugarings}
\subsection{Typing}
\subsection{Proof of Regular String Soundness Tycon Invariant}


\newpage
% We recommend abbrvnat bibliography style.

\bibliographystyle{abbrv}

% The bibliography should be embedded for final submission.

\bibliography{../research}
%\softraggedright
%P. Q. Smith, and X. Y. Jones. ...reference text...

\newpage
\appendix
\section{Appendix}
\begin{mathpar}
\small
\inferrule[s-ty-step]{
    \sstep{\st}{\argEnv}{\st'}
}{
    \sstep{\sty{c}{\st}}{\argEnv}{\sty{c}{\st'}}
}

\inferrule[s-ty-err]{
    \serr{\st}{\argEnv}
}{
    \serr{\sty{c}{\st}}{\argEnv}
}

\inferrule[s-ty-v]{
    \sval{\st}{\argEnv}
}{
    \sval{\sty{c}{\st}}{\argEnv}
}

\inferrule[s-otherty-v]{ }{
    \sval{\sotherty{m}{\tau}}{\argEnv}
}
\end{mathpar}

\begin{mathpar}
\small
\inferrule[s-tycase-step]{
    \sstep{\st}{\argEnv}{\st'}
}{
    \sstep{\stycase{c}{\st}{\sx}{\st_1}{\st_2}}{\argEnv}{\stycase{c}{\st'}{x}{\st_1}{\st_2}}
}

\inferrule[s-tycase-err]{
    \serr{\st}{\argEnv}
}{
    \serr{\stycase{c}{\st}{\sx}{\st_1}{\st_2}}{\argEnv}
}
\end{mathpar}
\begin{mathpar}\small
\inferrule[s-tycase-match]{
    \sval{\sty{c}{\st}}{\argEnv}
}{
    \sstep{\stycase{c}{\sty{c}{\st}}{\sx}{\st_1}{\st_2}}{\argEnv}{[\st/x]\st_1}
}

\inferrule[s-tycase-fail]{
    \st \neq \sty{c}{\st'}
}{
    \sstep{\stycase{c}{\st}{\sx}{\st_1}{\st_2}}{\argEnv}{\st_2}
}
\end{mathpar}
\begin{mathpar}
\small
\inferrule[keq-k]{\kvar \in \kDelta}{\keq{\kDelta}{\kvar}}

\inferrule[keq-ind]{\keq{\kDelta, \kvar}{\kappa}}{
  \keq{\kDelta}{\kmu{\kvar}{\kappa}}}

\inferrule[keq-unit]{ }{\keq{\kDelta}{\kunit}}
\end{mathpar}\begin{mathpar}
\small
\inferrule[keq-prod]{
  \keq{\kDelta}{\kappa_1}~~~~
  \keq{\kDelta}{\kappa_2}
}{
  \keq{\kDelta}{\kprod{\kappa_1}{\kappa_2}}
}

\inferrule[keq-sum]{
  \keq{\kDelta}{\kappa_1}~~~~
  \keq{\kDelta}{\kappa_2}
}{
  \keq{\kDelta}{\ksum{\kappa_1}{\kappa_2}}
}

\inferrule[keq-ty]{ }{\keq{\kDelta}{\kty}}
\end{mathpar}
 %We will discuss strategies for fragments that benefit from more sophisticated equivalences in Section \ref{examples}.\todo{do this}

\begin{mathpar}

\inferrule[k-ity-alpha]{ }{\sofkX{\sqity{\alpha}}{\kity}}
\end{mathpar}

\begin{mathpar}
\small
\inferrule[s-ity-lam-step-1]{
    \sstep{\sqity{\qity_1}}{\argEnv}{\sqity{\qity_1'}}%\sstep{\sqity{\qity_1}}{\argEnv}{\sstep{\sqity{\qity_1'}}}
}{
    \sstep{\sqity{\qity_1\times\qity_2}}{\argEnv}{\sqity{\qity_1' \times \qity_2}}
}

\inferrule[s-ity-lam-step-2]{
    \sval{\sqity{\qity_1}}{\argEnv}\\
    \sstep{\sqity{\qity_2}}{\argEnv}{\sqity{\qity_2'}}%\sstep{\sqity{\qity_1}}{\argEnv}{\sstep{\sqity{\qity_1'}}}
}{
    \sstep{\sqity{\qity_1\times\qity_2}}{\argEnv}{\sqity{\qity_1 \times \qity_2'}}
}

\inferrule[s-ity-lam-err-1]{
    \serr{\sqity{\qity_1}}{\argEnv}
}{
    \serr{\sqity{\qity_1\times\qity_2}}{\argEnv}
}

\inferrule[s-ity-lam-err-2]{
    \serr{\sqity{\qity_2}}{\argEnv}
}{
    \serr{\sqity{\qity_1\times\qity_2}}{\argEnv}
}

\inferrule[s-ity-lam-v]{
    \sval{\sqity{\qity_1}}{\argEnv}\\
    \sval{\sqity{\qity_2}}{\argEnv}
}{
    \sval{\sqity{\qity_1\times\qity_2}}{\argEnv}
}

\inferrule[s-ity-alpha-v]{ }{
    \sval{\sqity{\alpha}}{\argEnv}
}
\end{mathpar}
\begin{mathpar}
\small
\inferrule[k-tycase-parr]{
    \sofkX{\st}{\kty}\\
    \sofk{\kDelta}{\kGamma, \sx :: \kty \times \kty}{\Phi}{\st_1}{\kappa}\\
    \sofkX{\st_2}{\kappa}
}{
    \sofkX{\stycase{\rightharpoonup}{\st}{\sx}{\st_1}{\st_2}}{\kappa}
}
\end{mathpar}
\begin{mathpar}
\small
\inferrule[s-ity-unquote-step]{
    \sstep{\st}{\argEnv}{\st'}
}{
    \sstep{\sqity{\qtuq{\st}}}{\argEnv}{\sqity{\qtuq{\st'}}}
}

\inferrule[s-ity-unquote-err]{
    \serr{\st}{\argEnv}
}{
    \serr{\sqity{\qtuq{\st}}}{\argEnv}
}
\end{mathpar}
\begin{mathpar}
\inferrule[s-ity-trans-step]{
    \sstep{\st}{\argEnv}{\st'}
}{
    \sstep{\sqity{\srep{\st}}}{\argEnv}{\sqity{\srep{\st'}}}
}

\inferrule[s-ity-trans-err]{
    \serr{\st}{\argEnv}
}{
    \serr{\sqity{\srep{\st}}}{\argEnv}
}
\end{mathpar}
This judgement is defined by the following straightforward rules:
\begin{mathpar}\small
\inferrule[tstore-emp]{ }{\emptyset \leadsto \emptyset : \emptyset}

\inferrule[tstore-ext]{\memD \leadsto \delta : \Delta}{(\memD, \st \leftrightarrow \tau/\alpha) \leadsto (\delta, \tau/\alpha) : (\Delta, \alpha)}
\end{mathpar}

\[\small
\begin{array}{lll}
\textbf{Description} ~~~~& \textbf{Concrete Form}~~~~ & \textbf{Desugared Form}\\
\text{sequences} & (e_1, \ldots, e_n) ~\text{or}~ \texttt{[}e_1, \ldots, e_n\texttt{]} & \eintro{()}{e_1; \ldots; e_n}\\
\text{labeled sequences} & \{\mathtt{lbl}_1=e_1, \ldots, \mathtt{lbl}_n=e_n\} & \eintro{\texttt{[}\mathtt{lbl}_1, \ldots, \mathtt{lbl}_n\texttt{]}}{e_1; \ldots; e_n}\\
\text{label application} & \mathtt{lbl}\langle e_1, \ldots, e_n \rangle & \eintro{\mathtt{lbl}}{e_1, \ldots, e_n}\\
\text{numerals} & n & \eintro{n}{\cdot}\\
\text{labeled numerals} & n\texttt{lbl} & \eintro{(n, \texttt{lbl})}{\cdot}\\
\text{strings} & \texttt{"s"} & \eintro{\texttt{"s"}}{\cdot}
\end{array}
\]

\begin{mathpar}\small
\inferrule[s-itm-var-v]{ }{
    \sval{\sqitm{x}}{\argEnv}
}

\inferrule[s-itm-lam-step-1]{
    \sstep{\sqity{\qity}}{\argEnv}{\sqity{\qity'}}
}{
    \sstep{\sqitm{\ilam{\qity}{x}{\qitm}}}{\argEnv}{\sqitm{\ilam{\qity'}{x}{\qitm}}}
}

\inferrule[s-itm-lam-step-2]{
    \sval{\sqity{\qity}}{\argEnv}\\
    \sstep{\sqitm{\qitm}}{\argEnv}{\sqitm{\qitm'}}
}{
    \sstep{\sqitm{\ilam{\qity}{x}{\qitm}}}{\argEnv}{\sqitm{\ilam{\qity}{x}{\qitm'}}}
}
\end{mathpar}
\begin{mathpar}\small
\inferrule[s-itm-lam-err-1]{
    \serr{\sqity{\qity}}{\argEnv}
}{
    \serr{\sqitm{\ilam{\qity}{x}{\qitm}}}{\argEnv}
}

\inferrule[s-itm-lam-err-2]{
    \serr{\sqitm{\qitm}}{\argEnv}
}{
    \serr{\sqitm{\ilam{\qity}{x}{\qitm}}}{\argEnv}
}

\inferrule[s-itm-lam-v]{
    \sval{\sqity{\qity}}{\argEnv}\\
    \sval{\sqitm{\qitm}}{\argEnv}
}{
    \sval{\sqitm{\ilam{\qity}{x}{\qitm}}}{\argEnv}
}
\end{mathpar}
\begin{mathpar}\small
\inferrule[k-itm-unquote]{
    \sofkX{\st}{\kitm}
}{
    \sofkX{\sqitm{\quq{\st}}}{\kitm}
}
\end{mathpar}

\begin{mathpar}\small
\inferrule[s-itm-unquote-step]{
    \sstep{\st}{\argEnv}{\st'}
}{
    \sstep{\sqitm{\quq{\st}}}{\argEnv}{\sqitm{\quq{\st'}}}
}

\inferrule[s-itm-unquote-err]{
    \serr{\st}{\argEnv}
}{
    \serr{\sqitm{\quq{\st}}}{\argEnv}
}

\inferrule[s-itm-unquote-elim]{
    \sval{\sqitm{\qitm}}{\argEnv}
}{
    \sstep{\sqitm{\quq{\sqitm{\qitm}}}}{\argEnv}{\sqitm{\qitm}}
}
\end{mathpar}
\begin{mathpar}
\inferrule[s-ana-step]{
    \sstep{\st}{\argEnv}{\st'}
}{
    \sstep{\sana{n}{\st}}{\argEnv}{\sana{n}{\st'}}
}

\inferrule[s-ana-err]{
    \serr{\st}{\argEnv}
}{
    \serr{\sana{n}{\st}}{\argEnv}
}

\inferrule[s-itm-anatrans-step]{
    \sstep{\st}{\argEnv}{\st'}
}{
    \sstep{\sqitm{\anatrans{n}{\st}}}{\argEnv}{\sqitm{\anatrans{n}{\st'}}}
}

\inferrule[s-itm-anatrans-err]{
    \serr{\st}{\argEnv}
}{
    \serr{\sqitm{\anatrans{n}{\st}}}{\argEnv}
}
\end{mathpar}
, as specified by the judgement $\memG \leadsto \gamma : \Gamma$ defined by the following rules:
\begin{mathpar}
\inferrule[ttrs-emp]{ }{\emptyset \leadsto \emptyset : \emptyset}

\inferrule[ttrs-ext]{
    \memG \leadsto \gamma : \Gamma
}{
    (\memG, n : \st \leadsto \iota/x : \tau) \leadsto (\gamma, \iota/x) : (\Gamma, x : \tau)
}
\end{mathpar}


\begin{mathpar}\small
\inferrule[k-itm-lam]{
    \sofkX{\sqity{\qity}}{\kity}\\
    \sofkX{\sqitm{\qitm}}{\kitm}
}{
    \sofkX{\sqitm{\ilam{\qity}{x}{\qitm}}}{\kitm}
}
\end{mathpar} 
\begin{mathpar}\small
\inferrule[k-raise]{\kDelta \vdash \kappa}{\sofkX{\sraise{\kappa}}{\kappa}}

\inferrule[s-raise]{ }{\serr{\sraise{\kappa}}{\argEnv}}
\end{mathpar}

\begin{mathpar}\small
\inferrule[s-syn-success]{
    \keyw{nth}[n](\es) = e\\
    \esynX{e}{\st}{\iota}
}{
    \sstep{\ssyn{n}}{\es;\Upsilon;\Phi}{(\st, \sqitm{\syntrans{n}})}
}

\inferrule[s-syn-fail]{
    \keyw{nth}[n](\es) = e\\
    [\Upsilon \vdash_\Phi e \nRightarrow]
}{
    \serr{\ssyn{n}}{\es;\Upsilon;\Phi}
}
\end{mathpar}

\begin{mathpar}\small
\inferrule[k-itm-syntrans]{
    n' < n
}{
    \sofkX{\sqitm{\syntrans{n'}}}{\kitm}
}
\end{mathpar}
, e.g. for lambdas:
\begin{mathpar}\small
%\inferrule[abs-var]{ }{\edeabs{\tc}{\argEnv}{x}{\memD}{\memG}{x}{\memD}{\memG}}
%
\inferrule[abs-lam]{
    \tdeabs{\Phi}{\tc}{\qity}{\memD}{\ity}{\memD'}\\
    \edeabs{\tc}{\es;\Upsilon;\Phi}{\qitm}{\memD'}{\memG}{\itm}{\memD''}{\memG'}
}{
    \edeabs{\tc}{\es;\Upsilon;\Phi}{\ilam{\qity}{x}{\qitm}}{\memD}{\memG}{\ilam{\tau}{x}{\itm}}{\memG'}{\memD''}
}
\end{mathpar}
\begin{mathpar}\small
\inferrule[abs-anatrans-stored]{
    n : \st \leadsto \iota/x : \tau \in \memG
}{
    \edeabs{\tc}{\argEnv}{\anatrans{n}{\st}}{\memG}{\memD}{x}{\memG}{\memD}
}

\inferrule[abs-syntrans-stored]{
    n : \st \leadsto \iota/x : \tau \in \memG
}{
    \edeabs{\tc}{\argEnv}{\syntrans{n}}{\memG}{\memD}{x}{\memG}{\memD}
}\end{mathpar}
\begin{mathpar}\small
\inferrule[k-itm-anatrans]{
    n' < n\\
    \sofkX{\st}{\kty}
}{
    \sofkX{\sqitm{\anatrans{n'}{\st}}}{\kitm}
}
\end{mathpar}
\begin{mathpar}\small
\inferrule[abs-syntrans-new]{
    n \notin \text{dom}(\memG)\\
    \keyw{nth}[n](\es) = e\\
    \esynX{e}{\st}{\iota}\\
    \tdeabs{\Phi}{\tc}{\srep{\st}}{\memD}{\ity}{\memD'}\\
    (x~\text{fresh})
}{
    \edeabs{\tc}{\es;\Upsilon;\Phi}{\syntrans{n}}{\memG}{\memD}{x}{\memG, n : \st \leadsto \iota/x : \tau}{\memD'}
}
\end{mathpar}

\begin{mathpar}\small
\inferrule[etctx-emp]{ }{\vdash_\Phi \emptyset \leadsto \emptyset}

\inferrule[etctx-ext]{\vdash_\Phi \Upsilon \leadsto \Gamma\\\istype{\st}{\Phi}\\\vdash_\Phi \st \leadsto \tau}{\vdash_\Phi \Upsilon, x \Rightarrow \st \leadsto \Gamma, x : \tau}
\end{mathpar}

\[\small
\begin{array}{lll}
\textbf{Description}& \textbf{Concrete Form}~~~~ & \textbf{Desugared Form}\\
\text{index projection} & e_\text{targ}\texttt{\#}n & \etarg{\opname{idx}}{n}{e_\text{targ}}{\cdot}\\
\text{label projection} & e_\text{targ}\texttt{\#}\conclbl{lbl} & \etarg{\opname{prj}}{\conclbl{lbl}}{e_\text{targ}}{\cdot}\\
\text{explicit invocation}~~~~ & e_\text{targ}{\cdot}\opname{op}[\sttmidx](\es) & \etarg{\opname{op}}{\sttmidx}{e_\text{targ}}{\es}\\
& e_\text{targ}{\cdot}\opname{op}(\es) & \etarg{\opname{op}}{()}{e_\text{targ}}{\es}\\
& e_\text{targ}{\cdot}\opname{op}(\conclbl{lbl}_1=e_1, \ldots, \conclbl{lbl}_n=e_n) & \keyw{targ}[\opname{op}; \texttt{[}\conclbl{lbl}_1, \ldots, \conclbl{lbl}_n\texttt{]}](e_\text{targ}; \\
& & \quad e_1; \ldots; e_n)\\
\text{labeled case analysis} & e_\text{targ}{\cdot}\opname{case}\,\{ & \keyw{targ}[{\opname{case}}; {\texttt{[}\st_1, \ldots, \st_n\texttt{]}}]({e_\text{targ}}; \\
 & ~|~~\st_1\langle x_1, \ldots, x_k \rangle \Rightarrow e_1 & \quad \eanalam{x_1}{\ldots \eanalam{x_k}{e_1}};\\
 & ~|~~\ldots & \quad \ldots;\\ 
 & ~|~~\st_n\langle x_1, \ldots, x_k \rangle \Rightarrow e_n\} & \quad \eanalam{x_1}{\ldots \eanalam{x_k}{e_n}})
\end{array}
\]

 For example, 
\begin{mathpar}
\small
\inferrule[abs-prod]{
    \tdeabs{\Phi}{\tc}{\qity_1}{\memD}{\ity_1}{\memD'}\\
    \tdeabs{\Phi}{\tc}{\qity_2}{\memD'}{\ity_2}{\memD''}
}{
    \tdeabs{\Phi}{\tc}{\qity_1 \times \qity_2}{\memD}{\ity_1 \times \ity_2}{\memD''}
}
\end{mathpar}

The argument interfaces that populate the list provided to opcon definitions is derived from the argument list by the judgement $\keyw{args}(\es)=_n \stx{args}$, defined as follows:
\begin{mathpar}\small
\inferrule[args-z]{ }{
    \keyw{args}(\cdot)=_0 \skap{\karg}{\svar{nil}}
}

\inferrule[args-s]{
    \keyw{args}(\es)=_{n} \st
}{
    \keyw{args}(\es;e)=_{n+1} \skap{\karg}{\svar{rcons}}~\st~(\slam{\kty}{\svar{ty}}{\sana{n}{\svar{ty}}},\slam{\kunit}{\_}{\ssyn{n}})
}
\end{mathpar}
We assume that the definitions of the standard helper functions $\small\svar{nil} :: \kforall{\kalpha}{\klist{\kalpha}}$ and $\small\svar{rcons} :: \kforall{\kalpha}{\klist{\kalpha} \rightarrow \kalpha \rightarrow \klist{\kalpha}}$, which adds an item to the end of a list, have been substituted into these rules. The result is that the $n$th element of the argument interface list simply wraps the static terms $\sana{n}{\st}$ and $\ssyn{n}$.\end{document}
