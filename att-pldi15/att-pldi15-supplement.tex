%\documentclass[12pt]{article}
\documentclass[12pt]{article}
\usepackage{fullpage,cmu-titlepage2}
\usepackage{times}
% The following \documentclass options may be useful:
%
% 10pt          To set in 10-point type instead of 9-point.
% 11pt          To set in 11-point type instead of 9-point.
% authoryear    To obtain author/year citation style instead of numeric.
\usepackage{amsmath}
\usepackage{amssymb} 
\usepackage{amsthm}
\usepackage{ stmaryrd }
\usepackage{mathpartir}
\usepackage{enumerate}
\usepackage[usenames,dvipsnames,svgnames,table]{xcolor}

\newcommand{\keyw}[1]{{\sf #1}}
\newcommand{\mvlbl}[1]{\textbf{\textsf{#1}}}
\newcommand{\technical}[1]{{\color[gray]{0.6} #1}}
\newcommand{\karrow}[2]{#1 \rightarrow #2}
\newcommand{\kalpha}{\boldsymbol{\alpha}}
\newcommand{\kforall}[2]{\forall(#1.#2)}
\newcommand{\kvar}{\textbf{\textit{k}}}
\newcommand{\kmu}[2]{\mu_\text{ind}(#1.#2)}
\newcommand{\kunit}{\keyw{1}}
\newcommand{\kprod}[2]{#1\times#2}
\newcommand{\ksum}[2]{#1+#2}
\newcommand{\kty}{\keyw{Ty}}
\newcommand{\kity}{\keyw{ITy}}
\newcommand{\kitm}{\keyw{ITm}}
\newcommand{\st}{\sigma}
\newcommand{\stx}[1]{\st_\text{#1}}
\newcommand{\sttyidx}{\stx{tyidx}}
\newcommand{\sttmidx}{\stx{tmidx}}
\newcommand{\svar}[1]{{\textbf{\textit{#1}}}}
\newcommand{\sx}{\svar{x}}
\newcommand{\sy}{\svar{y}}
\newcommand{\sz}{\svar{z}}
\newcommand{\slam}[3]{\lambda #2{:}#1.#3}
\newcommand{\sap}[2]{#1~#2}
\newcommand{\sLam}[2]{\Lambda(#1.#2)}
\newcommand{\skap}[2]{#2\texttt{[}#1\texttt{]}}
\newcommand{\sfold}[3]{\keyw{fold}[#1.#2](#3)}
\newcommand{\srec}[4]{\keyw{rec}[#1](#2; #3.#4)}
\newcommand{\striv}{()}
\newcommand{\spair}[2]{(#1, #2)}
\newcommand{\sletpair}[4]{\keyw{letpair}(#1; #2,#3.#4)}
\newcommand{\sinl}[2]{\keyw{inl}[#1](#2)}
\newcommand{\sinr}[2]{\keyw{inr}[#1](#2)}
\newcommand{\scase}[5]{\keyw{case}(#1; #2.#3; #4.#5)}
\newcommand{\tcvar}[1]{\textsc{#1}}
\newcommand{\tc}{\tcvar{tc}}
\newcommand{\sty}[2]{\keyw{ty}[#1](#2)}
\newcommand{\sparr}[1]{\keyw{ty}[\rightharpoonup](#1)}
\newcommand{\sotherty}[2]{\keyw{otherty}[#1; #2]}
\newcommand{\stycase}[5]{\keyw{tycase}[#1](#2; #3.#4; #5)}
\newcommand{\sparrcase}[4]{\keyw{tycase}[\rightharpoonup](#1; #2.#3; #4)}
\newcommand{\sqity}[1]{{\blacktriangleright}(#1)}
\newcommand{\qity}{\grave{\tau}}
\newcommand{\srep}[1]{\keyw{rep}(#1)}
\newcommand{\sity}[1]{\keyw{ity}[#1]}
\newcommand{\ity}{\tau}
\newcommand{\sqitm}[1]{{\rhd}(#1)}
\newcommand{\qitm}{\grave{\iota}}
\newcommand{\sitm}[1]{\keyw{itm}[#1]}
\newcommand{\itm}{\iota}
\newcommand{\sana}[2]{\keyw{ana}[#1](#2)}
\newcommand{\ssyn}[1]{\keyw{syn}[#1]}
\newcommand{\sraise}[1]{\keyw{raise}[#1]}
\newcommand{\quqty}[1]{{\blacktriangleleft}(#1)}
\newcommand{\quqtm}[1]{{\lhd}(#1)}
\newcommand{\kGamma}{\boldsymbol\Gamma}
\newcommand{\kDelta}{\boldsymbol\Delta}
\newcommand{\kTheta}{\boldsymbol\Theta}
\newcommand{\kXi}{\boldsymbol\Xi}
\newcommand{\tcsig}[2]{\keyw{tcsig}[#1]\,\{#2\}}
\newcommand{\opsigs}{\Omega}
\newcommand{\litsig}[1]{\keyw{lit}[#1]}
\newcommand{\opsig}[2]{#1[#2]}
\newcommand{\tycons}[1]{#1~\mathtt{tycons}}
\newcommand{\tcdeclsok}[1]{#1~\mathtt{decls}}
\newcommand{\tcdefsok}[1]{#1~\mathtt{defs}}
\newcommand{\kok}[3]{#1~#2 \vdash #3}
\newcommand{\keq}[2]{#1 \vdash #2~\mathtt{eq}}
\newcommand{\kpos}[2]{#1.#2~\mathtt{pos}}
\newcommand{\sofkn}[6]{#1~#2 \vdash_{#3}^{#4} #5 : #6}
\newcommand{\sofk}[5]{\sofkn{#1}{#2}{#3}{n}{#4}{#5}}
\newcommand{\sofkz}[5]{\sofkn{#1}{#2}{#3}{0}{#4}{#5}}
\newcommand{\sofkzX}[2]{\sofkz{\kDelta}{\kGamma}{\kXi}{#1}{#2}}
\newcommand{\sqofk}[4]{#1~#2 \vdash_{#3}^{n} #4}
\newcommand{\sofkX}[2]{\sofk{\kDelta}{\kGamma}{\kXi}{#1}{#2}}
\newcommand{\sqofkX}[1]{\sqofk{\kDelta}{\kGamma}{\kXi}{#1}}
\newcommand{\serr}[4]{#1 \parallel #2~#3~\mathtt{err}_{#4}}
\newcommand{\serrX}[1]{\serr{#1}{\memD}{\memG}{\aCtx}}
\newcommand{\snorm}[7]{#1 \parallel #2~#3 \Downarrow_{#4} #5 \parallel #6~#7}
\newcommand{\memD}{\mathcal{D}}
\newcommand{\memG}{\mathcal{G}}
\newcommand{\aCtx}{\mathcal{C}}
\newcommand{\snormX}[2]{\snorm{#1}{\memD}{\memG}{\aCtx}{#2}{\memD'}{\memG'}}
\newcommand{\snormT}[2]{#1 \Downarrow #2}
\newcommand{\kctxok}[2]{#1 \vdash #2}
\newcommand{\tccok}[2]{\vdash #1 \sim #2}
\newcommand{\tccokX}{\tccok{\Phi}{\kXi}}
\newcommand{\tcok}[4]{\vdash_{#1} #2 \sim #3[#4]}
\newcommand{\tcsigok}[2]{\vdash #1 \sim #2}
\newcommand{\ocok}[4]{\vdash_{#1} #3~\{#2\} : #4}
\newcommand{\ocsigok}[1]{\vdash #1}
\newcommand{\sfinalrep}[3]{\vdash_{#1} #2 \leadsto #3}
\newcommand{\sfinalrepX}[2]{\sfinalrep{\Phi}{#1}{#2}}
\newcommand{\gfinalrep}[3]{\vdash_{#1} #2 \leadsto #3}
\newcommand{\gfinalrepX}[2]{\gfinalrep{\Phi}{#1}{#2}}
\newcommand{\svalofk}[2]{\sofkz{\emptyset}{\emptyset}{\kXi}{#1}{#2}}
\newcommand{\stype}[2]{#1~\mathtt{type}_{#2}}
\newcommand{\stypeX}[1]{#1~\mathtt{type}_{\Phi}}
\newcommand{\qparr}[2]{#1 \rightharpoonup #2}
\newcommand{\qforall}[2]{\forall(#1.#2)}
\newcommand{\qunit}{\keyw{1}}
\newcommand{\qprod}[2]{#1 \times #2}
\newcommand{\qsum}[2]{#1 + #2}
\newcommand{\qmu}[2]{\mu(#1.#2)}
\newcommand{\qtuq}[1]{{\blacktriangleleft}(#1)}
\newcommand{\qfix}[3]{\keyw{qfix}[#1](#2.#3)}
\newcommand{\qlam}[3]{\lambda[#1](#2.#3)}
\newcommand{\qap}[2]{#1~#2}
\newcommand{\qLam}[2]{\Lambda(#1.#2)}
\newcommand{\qAp}[2]{#1\texttt{[}#2\texttt{]}}
\newcommand{\qtriv}{()}
\newcommand{\qpair}[2]{(#1, #2)}
\newcommand{\qletpair}[4]{\keyw{letpair}(#1; #2, #3.#4)}
\newcommand{\qinl}[2]{\keyw{inl}[#1](#2)}
\newcommand{\qinr}[2]{\keyw{inr}[#1](#2)}
\newcommand{\qcase}[5]{\keyw{case}(#1; #2.#3; #4.#5)}
\newcommand{\qfold}[3]{\keyw{fold}[#1.#2](#3)}
\newcommand{\qunfold}[1]{\keyw{unfold}(#1)}
\newcommand{\quq}[1]{{\lhd}(#1)}
\newcommand{\eslet}[3]{\keyw{letstatic}[#1](#2.#3)}
\newcommand{\easc}[2]{\keyw{asc}[#2](#1)}
\newcommand{\elet}[3]{\keyw{let}(#1; #2.#3)}
\newcommand{\efix}[2]{\keyw{fix}(#1.#2)}
\newcommand{\eanalam}[2]{\lambda(#1.#2)}
\newcommand{\elam}[3]{\lambda[#1](#2.#3)}
\newcommand{\eap}[2]{\keyw{ap}(#1; #2)}
\newcommand{\eLam}[2]{\Lambda(#1.#2)}
\newcommand{\eAp}[2]{#1\texttt{[}#2\texttt{]}}
\newcommand{\efold}[1]{\keyw{fold}(#1)}
\newcommand{\eunfold}[1]{\keyw{unfold}(#1)}
\newcommand{\elit}[2]{\keyw{lit}[#1](#2)}
\newcommand{\etarg}[4]{\keyw{targ}[#1; #2](#3; #4)}
\newcommand{\eother}[1]{\keyw{other}[#1]}
\newcommand{\es}{\overline{e}}
\newcommand{\eDelta}{\Delta}
\newcommand{\eTheta}{\Theta}
\newcommand{\tcdecl}[2]{#1 \hookrightarrow #2}
\newcommand{\tcdef}[3]{\keyw{tycon}~#1 ~\{#3\} : #2}
\newcommand{\opdefs}{\omega}
\newcommand{\opname}[1]{\textbf{#1}}
\newcommand{\analitop}[1]{\keyw{ana~lit}=#1}
\newcommand{\anaop}[2]{\keyw{ana}~#1=#2}
\newcommand{\synop}[2]{\keyw{syn}~#1=#2}
\newcommand{\tcdeclstd}{\tcdecl{\tc}{\tcdefmv}}
\newcommand{\tcdefmv}{\psi}
\newcommand{\ktyidx}{\kappa_\text{tyidx}}
\newcommand{\strep}{\st_\text{rep}}
\newcommand{\klitidx}{\kappa_\text{tmidx}}
\newcommand{\sanalit}{\st_\text{lit}}
\newcommand{\ktargidx}{\kappa_\text{tmidx}}
\newcommand{\ssyntarg}{\st_\text{targ}}
\newcommand{\kargs}{\klist{\karg}}
\newcommand{\karg}{\kappa_\text{arg}}
\newcommand{\kana}{\kappa_\text{ana}}
\newcommand{\ksyn}{\kappa_\text{syn}}
\newcommand{\stmidx}{\sigma_\text{tmidx}}
\newcommand{\esyn}[5]{#1 \vdash_{#2} #3 \Rightarrow #4 \leadsto #5}
\newcommand{\eana}[5]{#1 \vdash_{#2} #3 \Leftarrow #4 \leadsto #5}
\newcommand{\esynX}[3]{\esyn{\Upsilon}{\Phi}{#1}{#2}{#3}}
\newcommand{\eanaX}[3]{\eana{\Upsilon}{\Phi}{#1}{#2}{#3}}
\newcommand{\mkargs}[4]{#1 \looparrowright #2 \looparrowright #3; #4}
\newcommand{\mkargsX}{\mkargs{\es}{\stx{args}}{\memG_0}{n}}
\newcommand{\ctxok}[2]{\vdash_{#1} #2}
\newcommand{\iparr}[2]{#1 \rightharpoonup #2}
\newcommand{\iforall}[2]{\forall(#1.#2)}
\newcommand{\imu}[2]{\mu(#1.#2)}
\newcommand{\iunit}{\keyw{1}}
\newcommand{\iprod}[2]{#1 \times #2}
\newcommand{\isum}[2]{#1 + #2}
\newcommand{\ifix}[3]{\keyw{fix}[#1](#2.#3)}
\newcommand{\ilam}[3]{\lambda[#1](#2.#3)}
\newcommand{\iap}[2]{\keyw{ap}(#1; #2)}
\newcommand{\iLam}[2]{\Lambda(#1.#2)}
\newcommand{\iAp}[2]{#2\texttt{[}#1\texttt{]}}
\newcommand{\ifold}[3]{\keyw{fold}[#1.#2](#3)}
\newcommand{\iunfold}[1]{\keyw{unfold}(#1)}
\newcommand{\itriv}{\keyw{triv}}
\newcommand{\ipair}[2]{\keyw{pair}(#1; #2)}
\newcommand{\iletpair}[4]{\keyw{letpair}(#1; #2, #3.#4)}
\newcommand{\iinl}[2]{\keyw{inl}[#1](#2)}
\newcommand{\iinr}[2]{\keyw{inr}[#1](#2)}
\newcommand{\icase}[5]{\keyw{case}(#1; #2.#3; #4.#5)}
\newcommand{\iGamma}{\Gamma}
\newcommand{\iDelta}{\Delta}
\newcommand{\iTheta}{\Theta}
\newcommand{\sval}{\hat{\sigma}}
\newcommand{\swrong}{\keyw{wrong}}
\newcommand{\linenumber}[1]{{\tt{\color{gray} #1}}}
\newcommand{\concty}[2]{\fvar{#1}~#2}
\newcommand{\concrx}[1]{\texttt{\scriptsize /#1/}}
\newcommand{\concstr}[1]{\texttt{\scriptsize "#1"}}
\newcommand{\conctarg}[1]{.\opname{#1}}
\newcommand{\conclbl}[1]{\texttt{\scriptsize #1}}
\newcommand{\digit}{\textbackslash d}
\newcommand{\rdot}{\textbackslash .}
\newcommand{\conckap}[2]{#1\texttt{[}#2\texttt{]}}
\newcommand{\desugar}[1]{\lfloor #1 \rfloor}
\newcommand{\klbl}{\keyw{Lbl}}
\newcommand{\kstr}{\keyw{Str}}
\newcommand{\krx}{\keyw{Rx}}
\newcommand{\knat}{\keyw{Nat}}
\newcommand{\scolorc}{\color{JungleGreen}}
\newcommand{\scolor}[1]{{\scolorc #1}}
\newcommand{\qcolorc}{\color{Cerulean}} %\color[gray]{0.6}
\newcommand{\qcolor}[1]{{\qcolorc #1}}
\newcommand{\cnl}{\\\scolorc}
\newcommand{\klist}[1]{\keyw{List}[#1]}

\usepackage{times}
\renewcommand{\ttdefault}{txtt}
\usepackage{alltt}
\usepackage{listings}
\lstset{language=ML,
showstringspaces=false,
basicstyle=\ttfamily\footnotesize,
morekeywords={newcase,extends}}

\usepackage{url}
\usepackage{todonotes}
\lefthyphenmin=5
\sloppy

\newcommand{\moutput}{^{\color{gray}+}}
\newcommand{\rulename}[1]{(#1)}
\def \TirNameStyle #1{\small\rulename{#1}}
\renewcommand{\MathparLineskip}{\lineskiplimit=.6\baselineskip\lineskip=.6\baselineskip plus .2\baselineskip}

\newtheorem{theorem}{Theorem}
\newtheorem{lemma}{Lemma}
\newtheorem{corollary}{Corollary}
\newtheorem{definition}{Definition}
\newtheorem{tyconinvariant}{Tycon Invariant}
\newenvironment{proof-sketch}{\noindent{\emph{Proof Sketch.}}}{\qed}
\makeatletter

% Heather-added packages for the fancy table
\usepackage{longtable}
\usepackage{booktabs}
\usepackage{pdflscape}
\usepackage{colortbl}%
\newcommand{\myrowcolour}{\rowcolor[gray]{0.925}}
\usepackage{wasysym}

\renewcommand\topfraction{0.85}
\renewcommand\bottomfraction{0.85}
\renewcommand\textfraction{0.1}
\renewcommand\floatpagefraction{0.85}

\begin{document}

%\titlebanner{{\tt \textcolor{Red}{{\small Under Review -- distribute within CMU only.}}}}        % These are ignored unless
%\preprintfooter{Distribute within CMU only.}   % 'preprint' option specified.

\title{Modularly Composing Typed Language Fragments (Supplemental Material)}
%\subtitle{Supplemental Material}
%\authorinfo{}{}{}
%\authorinfo{Cyrus Omar \and Jonathan Aldrich}
 %         {Carnegie Mellon University}
  %         {\{comar, aldrich\}@cs.cmu.edu}   

\author{ }

\abstract{
    This document provides the full technical development described in the paper ``Modularly Composing Typed Language Fragments'' submitted to PLDI 2015. 
}
\keywords{Statistical Model Checking, Markov Decision Processes, Reinforcement Learning}

\trnumber{ }

\maketitle

%\category{D.3.2}{Programming Languages}{Language Classifications}[Extensible Languages]
%\category{D.3.4}{Programming Languages}{Processors}[Compilers]
%\category{F.3.1}{Logics \& Meanings of Programs}{Specifying and Verifying and Reasoning about Programs}[Specification Techniques]
%\keywords
%extensible languages; module systems; type abstraction; typed compilation; type-level computation

\section{Internal Language}
We assume the statics of the IL are specified in the standard way by judgements for  type formation {$\iDelta \vdash \tau$}, typing context formation { $\iDelta \vdash \iGamma$} and type assignment {$\iDelta~\iGamma \vdash \iota : \tau\moutput$}. 
%The typing and type formation contexts obey standard structural properties (i.e. weakening, exchange and contraction). 
The internal dynamics are specified as a structural operational semantics with a stepping judgement {\small $\iota \mapsto \iota\moutput$} and a value judgement {$\iota~\mathtt{val}$}. The multi-step judgement $\iota \mapsto^{*} \iota\moutput$ is the reflexive, transitive closure of the stepping judgement and the evaluation judgement $\iota \Downarrow \iota'$ is defined iff $\iota \mapsto^{*} \iota'$ and $\iota'~\mathtt{val}$. Both the static and dynamic semantics of the IL can be found in any standard textbook covering typed lambda calculi (we directly follow \cite{pfpl}), so we assume familiarity and give the lemmas in this section without proof.

We use {$\mathcal{L}\{{\rightharpoonup}\,{\forall}\,{\mu}\,{\iunit}\,{\times}\,{+}\}$}, the syntax for which is shown in Figure \ref{syntax-IL}, as representative of any intermediate language for a typed functional language. 

\begin{figure}[h!]
\small
\hspace{-5px}$\begin{array}{llcl}%\\
\textbf{internal types} & \ity & ::= & \multicolumn{1}{l}{\iparr{\ity}{\ity} ~|~ \alpha ~|~ \iforall{\alpha}{\ity} ~|~ t ~|~ \imu{t}{\ity} ~|~ \iunit ~|~ \iprod{\ity}{\ity} ~|~ \isum{\ity}{\ity}}\\
\textbf{internal terms} & \itm & ::= & \multicolumn{1}{l}{x ~|~ \ilam{\ity}{x}{\itm} ~|~ \iap{\itm}{\itm} ~|~ \ifix{\ity}{x}{\itm} ~|~ \iLam{\alpha}{\itm} ~|~ \iAp{\ity}{\itm} ~|~ \ifold{t}{\ity}{\itm} ~|~ \iunfold{\itm} }\\
&& ~|~ & \multicolumn{1}{l}{\itriv ~|~ \ipair{\itm}{\itm} ~|~ \iletpair{\itm}{x}{y}{\itm} ~|~ \iinl{\ity}{\itm} ~|~ \iinr{\ity}{\itm} ~|~ \icase{\itm}{x}{\itm}{x}{\itm}}\\
\multicolumn{4}{l}{\textbf{internal typing contexts}~\Gamma ::= \emptyset ~|~ \Gamma, x : \tau}\\
\multicolumn{4}{l}{\textbf{internal type formation contexts}~\Delta ::= \emptyset ~|~ \Delta, \alpha ~|~ \Delta, t}\\
\end{array}%\\
$
\caption{Syntax of {$\mathcal{L}\{{\rightharpoonup}\,{\forall}\,{\mu}\,{\iunit}\,{\times}\,{+}\}$}, our internal language (IL). Metavariable $x$ ranges over term variables and $\alpha$ and $t$ both range over type variables.}
\label{syntax-IL}
\end{figure}

In fact, our intention is not to prescribe a particular choice of IL, so we will here only review the key metatheoretic properties that the IL must possess. Each choice of IL is technically a distinct dialect of @$\lambda$, but for the broad class of ILs that enjoy these properties, the metatheory in the remainder of the supplement should follow without trouble.

\begin{lemma}[Internal Type Assignment]
If $\Delta \vdash \Gamma$ and $\Delta~\Gamma \vdash \iota : \tau$ then $\Delta \vdash \tau$.
\end{lemma}

Internal type safety follows the standard methodology of proving preservation and progress lemmas.

\begin{lemma}[Internal Progress]
If $\emptyset~\emptyset \vdash \iota : \tau$ then either $\iota~\texttt{val}$ or $\iota \mapsto \iota'$.
\end{lemma}
\begin{lemma}[Internal Preservation]
If $\emptyset~\emptyset \vdash \iota : \tau$ and $\iota \mapsto \iota'$ then $\emptyset~\emptyset \vdash \iota' : \tau$. 
\end{lemma}

We assume substitution and simultaneous $n$-ary substitutions, $\delta$ and $\gamma$, have the standard semantics, and that typing and type formation contexts, $\Delta$ and $\Gamma$, obey standard properties of weakening, exchange and contraction. We assume that the names of variables and type variables are unimportant, so that $\alpha$-equivalent terms, types and contexts can be identified implicitly throughout this work. In particular, we need the definitions of substitution validity shown in Figures \ref{fig:internal-type-substitution-validity} and \ref{fig:internal-term-substitution-validity} and the following lemmas.

\begin{figure}[h]
\small\fbox{$\Delta \vdash \delta : \Delta$}~~$\delta ::=  \emptyset ~|~ \delta, \tau/\alpha$
\begin{mathpar}
\inferrule[tysub-emp]{ }{\Delta \vdash \emptyset : \emptyset}

\inferrule[tysub-ext]{
    \Delta \vdash \delta : \Delta'\\
    \Delta \vdash \tau
}{
    \Delta \vdash \delta, \tau/\alpha : \Delta', \alpha
}
\end{mathpar}
\caption{Internal Type Substitution Validity}
\label{fig:internal-type-substitution-validity}
\end{figure}

\begin{lemma}[Internal Type Substitution on Types] If $\Delta \vdash \delta : \Delta'$ and $\Delta\Delta' \vdash \tau$ then $\Delta \vdash [\delta]\tau$.\end{lemma}

\begin{lemma}[Internal Type Substitutions on Typing Contexts] If $\Delta \vdash \delta : \Delta'$ and $\Delta\Delta' \vdash \Gamma$ then $\Delta \vdash [\delta]\Gamma$.
\end{lemma}

\begin{lemma}[Internal Type Substitutions on Terms] If $\Delta \vdash \delta : \Delta'$ and $\Delta\Delta' \vdash \tau$ and $\Delta\Delta' \vdash \Gamma$ and $\Delta\Delta'~\Gamma \vdash \iota : \tau$ then $\Delta~[\delta]\Gamma \vdash [\delta]\iota : [\delta]\tau$.
\end{lemma}

\begin{figure}[h]
\small\fbox{$\Delta~\Gamma \vdash \gamma : \Gamma$}~~$\gamma ::= \emptyset ~|~ \gamma, \iota/x$
\begin{mathpar}
\inferrule[tmsub-emp]{ }{\Delta~\Gamma \vdash \emptyset : \emptyset}

\inferrule[tmsub-ext]{
    \Delta~\Gamma \vdash \gamma : \Gamma'\\
    \Delta \vdash \tau\\
    \Delta~\Gamma \vdash \iota : \tau
}{
    \Delta~\Gamma \vdash \gamma, \iota/x : \Gamma', x : \tau
}
\end{mathpar}
\caption{Internal Term Substitution Validity}
\label{fig:internal-term-substitution-validity}
\end{figure}

\begin{lemma}[Internal Term Substitutions] If $\Delta \vdash \Gamma$ and $\Delta \vdash \Gamma'$ and $\Delta~\Gamma \vdash \gamma : \Gamma'$ and $\Delta~\Gamma\Gamma' \vdash \iota : \tau$ then $\Delta~\Gamma \vdash [\gamma]\iota : \tau$.
\end{lemma}


\section{Tycons}
\begin{figure}[h]
\small
$\begin{array}{lrcl}
\textbf{tycons} & c & ::= & \rightharpoonup ~|~ \tc ~|~ \keyw{other}[m]\\
\textbf{tycon contexts} & \Phi & ::= & \cdot ~|~ \Phi, \tcdef{\tc}{\psi}{\theta}\\
\textbf{tycon structures} & \theta & ::= & \tcstruct{\st}{\omega} \\
\textbf{opcon structures} & \omega & ::= & \tcstructn{\st} ~|~ \tcstructc{\omega}{op}{\st}\\
\textbf{tycon sigs} & \psi & ::= & \tcsig{\kappa}{\chi}\\
\textbf{opcon sigs}& \chi & ::= & \introsig{\kappa} ~|~ \opsigS{\chi}{op}{\kappa}\\
\end{array}$
\caption{Syntax of tycons. Metavariables $\tc$ and $\opname{op}$ range over user-defined tycon and opcon names, respectively, and $m$ ranges over natural numbers.}
\label{syntax-TC}
\end{figure}

\subsection{Tycon Contexts}
Tycon contexts are ordered mappings from tycon names $\tc$ to tycon definitions. The tycon context well-definedness judgement $\vdash \Phi$ is specified in Figure \ref{fig:tycon-ctxs}. 

\begin{figure}[h]
\fbox{$\vdash \Phi$}\vspace{-25px}
\begin{mathpar}
\small
\inferrule[tcc-emp]{ }{\vdash \cdot}

\inferrule[tcc-ext]{
    \vdash \Phi\\
    \tc \notin \text{dom}(\Phi)\\
    \keq{\emptyset}{\ktyidx}\\
    \sofkz{\emptyset}{\emptyset}{\Phi}{\stx{schema}}{\ktyidx \rightarrow \kity}\\\\
    \vdash_{\Phi,  \tcdef{\tc}{\tcsig{\ktyidx}{\chi}}{\tcstruct{\stx{schema}}{\omega}}} \omega \sim \tcsig{\ktyidx}{\chi}
}{
    \vdash \Phi, \tcdef{\tc}{\tcsig{\ktyidx}{\chi}}{\tcstruct{\stx{schema}}{\omega}}
}
\end{mathpar}
\caption{Tycon context well-definedness.}
\label{fig:tycon-ctxs}
\end{figure}



Opcon structures are checked against the tycon's signature.

\begin{figure}[h]
\small\fbox{$\vdash_\Phi \omega \sim \psi$}
\begin{mathpar}\small
\inferrule[ocstruct-intro]{
    \introsig{\klitidx} \in \chi\\
    \emptyset \vdash \klitidx\\\\
    \sofkz{\emptyset}{\emptyset}{\Phi}{\stx{def}}{\ktyidx \rightarrow \klitidx \rightarrow \kargs \rightarrow \kitm}
}{
    \vdash_\Phi \tcstructn{\stx{def}} \sim \tcsig{\ktyidx}{\chi}
}

\inferrule[ocstruct-targ]{
    \vdash_\Phi \omega \sim \tcsig{\ktyidx}{\chi}\\
    \opname{op}\notin\text{dom}(\chi)\\
    \emptyset \vdash \klitidx\\\\
    \sofkn{\emptyset}{\emptyset}{\Phi}{0}{\stx{def}}{\ktyidx \rightarrow \klitidx \rightarrow \kargs \rightarrow (\kty \times \kitm)}
}{
    \vdash_\Phi \tcstructc{\omega}{op}{\stx{def}} \sim \tcsig{\ktyidx}{\chi; \opsig{\opname{op}}{\klitidx}}
}
\end{mathpar}
\caption{Checking opcon structures against tycon signatures.}
\label{ocstruct}
\end{figure}

For the purposes of the SL, only the type signatures are relevant. We thus define  judgements $\vdash \psi$, $\vdash \chi$ and $\vdash \Phi~\mathtt{sigsok}$ in Figure \ref{fig:tycon-and-opcon-signature-well-formedness} for checking this only.

The following lemmas characterize these judgements. 
\begin{figure}[h]
\fbox{$\vdash \psi$}\vspace{-25px}
\begin{mathpar}
\small
\inferrule[tcsig-ok]{
    \keq{\emptyset}{\ktyidx}\\
    \vdash \chi
}{
    \vdash \tcsig{\ktyidx}{\chi}
}
\end{mathpar}
\fbox{$\vdash \chi$}\vspace{-25px}
\begin{mathpar}
\small
\inferrule[ocsig-intro-ok]{
    \emptyset \vdash \kappa
}{
    \vdash \introsig{\kappa}
}

\inferrule[ocsig-targ-ok]{
    \emptyset \vdash \kappa\\
    \opname{op} \notin \text{dom}(\chi)
}{
    \vdash \opsigS{\chi}{op}{\kappa}
}
\end{mathpar}
\fbox{$\vdash \Phi~\mathtt{sigsok}$}\vspace{-25px}
\begin{mathpar}
\small
\inferrule[tcc-sigsok-emp]{ }{
    \vdash \emptyset~\mathtt{sigsok}
}

\inferrule[tcc-sigsok-ext]{
    \vdash \Phi~\mathtt{sigsok}\\
    \vdash \psi
}{
    \vdash \Phi, \tcdef{\tc}{\psi}{\theta}~\mathtt{sigsok}
}
\end{mathpar}
\caption{Tycon and Opcon Signature Well-Formedness}
\label{fig:tycon-and-opcon-signature-well-formedness}
\end{figure}


\begin{lemma}[Inversion on Tycon Context Well-Definedness]
\label{lemma:inversion-on-tycon-context-well-definedness}
If (1) $$\vdash \Phi, \tcdef{\tc}{\tcsig{\ktyidx}{\chi}}{\tcstruct{\stx{schema}}{\omega}}$$ then (a) $\vdash \Phi$ and (b) $\tc \notin \text{dom}(\Phi)$ and (c) $\keq{\emptyset}{\ktyidx}$ and (d) $\sofkz{\emptyset}{\emptyset}{\Phi}{\stx{schema}}{\ktyidx \rightarrow \kity}$ and (e) $\vdash_{\Phi,  \tcdef{\tc}{\tcsig{\ktyidx}{\chi}}{\tcstruct{\stx{schema}}{\omega}}} \omega \sim \tcsig{\ktyidx}{\chi}$.
\end{lemma}
\begin{proof}
Rule induction on assumption (1). Rule (tcc-emp) does not apply. Rule (tcc-ext) applies. The conclusions (a-e) are the five premises of (tcc-ext) and thus follow immediately.
\end{proof}


\begin{lemma}[Tycon Context Composition]
If (i) $\vdash \Phi_1$ and (ii) $\vdash \Phi_2$ and (iii) $\text{dom}(\Phi_1) \cap \text{dom}(\Phi_2) = \emptyset$ then (a) $\vdash \Phi_1\Phi_2$.
\end{lemma}
\begin{proof}
By syntactic case analysis on $\Phi_1$ and $\Phi_2$.

If $\Phi_1=\emptyset$ and $\Phi_2=\emptyset$, then $\Phi_1\Phi_2=\emptyset$ and (a) follows by (tcc-emp).

If $\Phi_2=\emptyset$ then $\Phi_1\Phi_2=\Phi_1$ and (a) follows by (1).

If $\Phi_1=\emptyset$ then $\Phi_1\Phi_2=\Phi_2$ and (a) follows by (2).

If $\Phi_1=\Phi_1', \tcdef{\tc_1}{\psi_1}{\theta_1}$ and $\Phi_2=\Phi_2',\tcdef{\tc_2}{\psi_2}{\theta_2}$ then $\Phi_1\Phi_2=\Phi_1'\Phi_2',\tcdef{\tc_1}{\psi_1}{\theta_1},\tcdef{\tc_2}{\psi_2}{\theta_2}$. Let (1-5) be the result of applying Lemma \ref{lemma:inversion-on-tycon-context-well-definedness} on (i), and (6-10) be the result of applying Lemma \ref{lemma:inversion-on-tycon-context-well-definedness} on (ii). By (iii) and the usual properties of domains of finite mappings, we have that (11) $\text{dom}(\Phi_1') \cap \text{dom}(\Phi_2') = \emptyset$ and further by (2) we have that (12) $\tc\notin\keyw{dom}(\Phi_1'\Phi_2')$ and by (8) that (13) $\tc'\notin\text{dom}({\Phi_1'\Phi_2'})$.  By the IH on (1), (6) and (11) we have that (14) $\vdash \Phi_1'\Phi_2'$. By (tcc-ext) on (14), (12), (3), (4) and \todo{weakening of kinding and opcon well-definedness} on (5), we have that (15) $\vdash \Phi_1'\Phi_2', \tcdef{\tc_1}{\psi_1}{\theta_1}$. By (tcc-ext) on (15), (13), (8), (9) and SAME on (10), we have (a).
\end{proof}

\begin{lemma}[Intro Opcon Existence and Well-Definedness]
If (1) $\vdash_\Phi \omega \sim \tcsig{\ktyidx}{\chi}$ then (a) $\keyw{intro}[\klitidx] \in \chi$ and (b) $\emptyset \vdash \klitidx$ and (c) $\keyw{ana~intro}=\stx{def}\in\omega$ and (d) $\sofkz{\emptyset}{\emptyset}{\Phi}{\stx{def}}{\ktyidx \rightarrow \klitidx \rightarrow \kargs \rightarrow \kitm}$.
\end{lemma}
\begin{proof}
Rule induction on assumption (1). If rule (ocstruct-intro) applies then conclusion (c) follows syntactically and (a), (b) and (d) are the three premises and thus follow immediately. If rule (ocstruct-targ) applies, then we apply the IH to the first premise.
\end{proof}

\begin{lemma}[Targeted Opcon Well-Definedness and Unicity]
If (1) $\vdash_\Phi \omega \sim \tcsig{\ktyidx}{\chi}$ and (2) $\keyw{syn}~\opname{op}=\stx{def} \in \omega$ then (a) $\opname{op}[\klitidx] \in \chi$ and (b) $\emptyset \vdash \klitidx$ and (c) $
    \sofkn{\emptyset}{\emptyset}{\Phi}{0}{\stx{def}}{\ktyidx \rightarrow \klitidx \rightarrow \kargs \rightarrow (\kty \times \kitm)}$ and (d) if $\keyw{syn}~\opname{op}=\stx{def}' \in \omega$ then $\stx{def}'=\stx{def}$ and (e) if $\opname{op}[\klitidx']\in\chi$ then $\klitidx'=\klitidx$.
\end{lemma}
\begin{proof}
Rule induction on assumption (1). Rule (ocstruct-intro) does not apply by assumption (2). Rule (ocstruct-targ) applies. Conclusion (a) follows syntactically. Conclusions (b) and (c) are premises 3 and 4. Conclusion (d) follows because either $\stx{def}'=\stx{def}$ (i.e. we considering the current definition) or we can apply the IH to premise 1 and the assumption of (d). Conclusion (e) follows because premise 2 checks the unicity condition directly.
\end{proof}


\subsection{Full Examples}
\subsubsection{Regular Strings}

\subsubsection{Labeled Products}

\subsubsection{Records}

\section{Static Language}
\subsection{Kind Formation}

\begin{figure}[h]
\small\fbox{$\kDelta \vdash \kappa$}
\begin{mathpar}
\inferrule[kf-arrow]{
    \kDelta \vdash \kappa_1\\
    \kDelta \vdash \kappa_2
}{
    \kDelta \vdash \karrow{\kappa_1}{\kappa_2}
}

\inferrule[kf-alpha]{
    \kalpha \in \kDelta
}{
    \kDelta \vdash \kalpha
}

\inferrule[kf-forall]{
    \kDelta, \kalpha \vdash \kappa
}{
    \kDelta \vdash \kforall{\kalpha}{\kappa}
}

\inferrule[kf-k]{
    \kvar \in \kDelta
}{
    \kDelta \vdash \kvar
}

\inferrule[kf-ind]{
    ???
}{
    \kDelta \vdash \kmu{\kvar}{\kappa}
}

\inferrule[kf-unit]{ }{
    \kDelta \vdash \kunit
}

\inferrule[kf-prod]{
    \kDelta \vdash \kappa_1\\
    \kDelta \vdash \kappa_2
}{
    \kDelta \vdash \kprod{\kappa_1}{\kappa_2}
}

\inferrule[kf-sum]{\kDelta \vdash \kappa_1\\
    \kDelta \vdash \kappa_2
}{
    \kDelta \vdash \ksum{\kappa_1}{\kappa_2}
}

\inferrule[kf-ty]{ }{\kDelta \vdash \kty}

\inferrule[kf-ity]{ }{\kDelta \vdash \kity}

\inferrule[kf-itm]{ }{\kDelta \vdash \kitm}
\end{mathpar}
\caption{Kind Formation}
\label{fig:kind-formation}
\end{figure}
\begin{lemma}[Kind Variable Substitution - Kinds]~\\
\begin{enumerate}
\item If $\kDelta, \kalpha \vdash \kappa$ and $\kDelta \vdash \kappa'$ then $\kDelta \vdash [\kappa'/\kalpha]\kappa$.
\item If $\kDelta, \kvar \vdash \kappa$ and $\kDelta \vdash \kappa'$ then $\kDelta \vdash [\kappa'/\kvar]\kappa$.
\end{enumerate}
\end{lemma}
\begin{figure}[h]
???
\caption{Positive Kinds}
\label{fig:positive-kinds}
\end{figure}

\begin{figure}[h]
\small\fbox{$\kDelta \vdash \kGamma$}
\begin{mathpar}
\inferrule[kctx-emp]{ }{\kDelta \vdash \emptyset}

\inferrule[kctx-ext]{
    \kDelta \vdash \kGamma\\
    \kDelta \vdash \kappa
}{
    \kDelta \vdash \kGamma, \sx :: \kappa
}
\end{mathpar}
\caption{Kinding Context Formation}
\label{fig:kinding-context-formation}
\end{figure}

\subsection{Equality Kinds}
Need an equational theory for SL to state equality kind property, but not important for other metatheory. 

\begin{lemma}[Equality Kind Well-Formedness]
If $\keq{\emptyset}{\kappa}$ then $\emptyset \vdash \kappa$. 
\end{lemma}

\subsection{Kinding}

\subsection{Static Dynamics}

\begin{lemma}[Static Canonical Forms]
...
\end{lemma}

Discussion of decidability and weak normalization.

\subsection{Type Translations}

\subsection{Typing Context Translations}

\subsection{Arguments}
\begin{definition}[Argument Interface Kind]
The kind abbreviated $\karg$ is defined as\[\karg := (\kty \rightarrow \kitm) \times (\kunit \rightarrow \kty\times\kitm)\]
\end{definition}

\subsection{Kind Safety}
\paragraph{Kind Safety} Kind safety ensures that normalization of well-kinded static terms cannot go wrong. We can take a standard progress and preservation based approach. 
\begin{theorem}[Static Progress]\label{thm:static-progress}
If $\sofkn{\emptyset}{\emptyset}{\Phi}{n}{\st}{\kappa}$ and $\vdash \Phi$ and $|\es|=n$ and $\vdash_\Phi \Upsilon \leadsto \Gamma$ then $\sval{\st}{\es; \Upsilon; \Phi}$ or $\serr{\st}{\es; \Upsilon; \Phi}$ or $\sstep{\st}{\es; \Upsilon; \Phi}{\st'}$.
\end{theorem}

\begin{theorem}[Static Preservation]\label{thm:static-preservation}
If $\sofkn{\emptyset}{\emptyset}{\Phi}{n}{\st}{\kappa}$ and $\vdash \Phi$ and $|\es|=n$ and $\vdash_\Phi \Upsilon \leadsto \Gamma$ and $\sstep{\st}{\es; \Upsilon; \Phi}{\st'}$ then $\sofkn{\emptyset}{\emptyset}{\Phi}{n}{\st'}{\kappa}$.
\end{theorem}
The case in the proof of Theorem \ref{thm:static-preservation} for $\ssyn{n}$ requires that the following theorem be mutually defined. The mutual induction is well-founded because the total number of argument lists in the terms being considered decreases.
\begin{theorem}[Type Synthesis]
If $\vdash \Phi$ and $\vdash_\Phi \Upsilon \leadsto \Gamma$ and $\esynX{e}{\st}{\iota}$ then $\vdash_\Phi \st \leadsto \tau$ (and thus $\istype{\st}{\Phi}$). 
\end{theorem}

\section{External Language}
\subsection{Additional Desugarings}
\subsection{Typing}
\paragraph{Unicity}
The rules are structured so that if a term is well-typed, both its type and translation are unique.\todo{could move this whole thing to supplement if room needed}
\begin{theorem}[Unicity]
If $\vdash \Phi$ and $\vdash_\Phi \Upsilon \leadsto \Gamma$ and $\vdash_\Phi \st \leadsto \tau$ and $\vdash_\Phi \st' \leadsto \tau'$ and $\eanaX{e}{\st}{\iota}$ and $\eanaX{e}{\st'}{\iota'}$ then $\st = \st'$ and $\tau=\tau'$ and $\iota = \iota'$.
\end{theorem}


\subsection{Proof of Regular String Soundness Tycon Invariant}


\newpage
% We recommend abbrvnat bibliography style.

\bibliographystyle{abbrv}

% The bibliography should be embedded for final submission.

\bibliography{../research}
%\softraggedright
%P. Q. Smith, and X. Y. Jones. ...reference text...

\newpage
\appendix
\section{Appendix}
\begin{mathpar}
\small
\inferrule[s-ty-step]{
    \sstep{\st}{\argEnv}{\st'}
}{
    \sstep{\sty{c}{\st}}{\argEnv}{\sty{c}{\st'}}
}

\inferrule[s-ty-err]{
    \serr{\st}{\argEnv}
}{
    \serr{\sty{c}{\st}}{\argEnv}
}

\inferrule[s-ty-v]{
    \sval{\st}{\argEnv}
}{
    \sval{\sty{c}{\st}}{\argEnv}
}

\inferrule[s-otherty-v]{ }{
    \sval{\sotherty{m}{\tau}}{\argEnv}
}
\end{mathpar}

\begin{mathpar}
\small
\inferrule[s-tycase-step]{
    \sstep{\st}{\argEnv}{\st'}
}{
    \sstep{\stycase{c}{\st}{\sx}{\st_1}{\st_2}}{\argEnv}{\stycase{c}{\st'}{x}{\st_1}{\st_2}}
}

\inferrule[s-tycase-err]{
    \serr{\st}{\argEnv}
}{
    \serr{\stycase{c}{\st}{\sx}{\st_1}{\st_2}}{\argEnv}
}
\end{mathpar}
\begin{mathpar}\small
\inferrule[s-tycase-match]{
    \sval{\sty{c}{\st}}{\argEnv}
}{
    \sstep{\stycase{c}{\sty{c}{\st}}{\sx}{\st_1}{\st_2}}{\argEnv}{[\st/x]\st_1}
}

\inferrule[s-tycase-fail]{
    \st \neq \sty{c}{\st'}
}{
    \sstep{\stycase{c}{\st}{\sx}{\st_1}{\st_2}}{\argEnv}{\st_2}
}
\end{mathpar}
\begin{mathpar}
\small
\inferrule[keq-k]{\kvar \in \kDelta}{\keq{\kDelta}{\kvar}}

\inferrule[keq-ind]{\keq{\kDelta, \kvar}{\kappa}}{
  \keq{\kDelta}{\kmu{\kvar}{\kappa}}}

\inferrule[keq-unit]{ }{\keq{\kDelta}{\kunit}}
\end{mathpar}\begin{mathpar}
\small
\inferrule[keq-prod]{
  \keq{\kDelta}{\kappa_1}~~~~
  \keq{\kDelta}{\kappa_2}
}{
  \keq{\kDelta}{\kprod{\kappa_1}{\kappa_2}}
}

\inferrule[keq-sum]{
  \keq{\kDelta}{\kappa_1}~~~~
  \keq{\kDelta}{\kappa_2}
}{
  \keq{\kDelta}{\ksum{\kappa_1}{\kappa_2}}
}

\inferrule[keq-ty]{ }{\keq{\kDelta}{\kty}}
\end{mathpar}
 %We will discuss strategies for fragments that benefit from more sophisticated equivalences in Section \ref{examples}.\todo{do this}

\begin{mathpar}

\inferrule[k-ity-alpha]{ }{\sofkX{\sqity{\alpha}}{\kity}}
\end{mathpar}

\begin{mathpar}
\small
\inferrule[s-ity-lam-step-1]{
    \sstep{\sqity{\qity_1}}{\argEnv}{\sqity{\qity_1'}}%\sstep{\sqity{\qity_1}}{\argEnv}{\sstep{\sqity{\qity_1'}}}
}{
    \sstep{\sqity{\qity_1\times\qity_2}}{\argEnv}{\sqity{\qity_1' \times \qity_2}}
}

\inferrule[s-ity-lam-step-2]{
    \sval{\sqity{\qity_1}}{\argEnv}\\
    \sstep{\sqity{\qity_2}}{\argEnv}{\sqity{\qity_2'}}%\sstep{\sqity{\qity_1}}{\argEnv}{\sstep{\sqity{\qity_1'}}}
}{
    \sstep{\sqity{\qity_1\times\qity_2}}{\argEnv}{\sqity{\qity_1 \times \qity_2'}}
}

\inferrule[s-ity-lam-err-1]{
    \serr{\sqity{\qity_1}}{\argEnv}
}{
    \serr{\sqity{\qity_1\times\qity_2}}{\argEnv}
}

\inferrule[s-ity-lam-err-2]{
    \serr{\sqity{\qity_2}}{\argEnv}
}{
    \serr{\sqity{\qity_1\times\qity_2}}{\argEnv}
}

\inferrule[s-ity-lam-v]{
    \sval{\sqity{\qity_1}}{\argEnv}\\
    \sval{\sqity{\qity_2}}{\argEnv}
}{
    \sval{\sqity{\qity_1\times\qity_2}}{\argEnv}
}

\inferrule[s-ity-alpha-v]{ }{
    \sval{\sqity{\alpha}}{\argEnv}
}
\end{mathpar}
\begin{mathpar}
\small
\inferrule[k-tycase-parr]{
    \sofkX{\st}{\kty}\\
    \sofk{\kDelta}{\kGamma, \sx :: \kty \times \kty}{\Phi}{\st_1}{\kappa}\\
    \sofkX{\st_2}{\kappa}
}{
    \sofkX{\stycase{\rightharpoonup}{\st}{\sx}{\st_1}{\st_2}}{\kappa}
}
\end{mathpar}
\begin{mathpar}
\small
\inferrule[s-ity-unquote-step]{
    \sstep{\st}{\argEnv}{\st'}
}{
    \sstep{\sqity{\qtuq{\st}}}{\argEnv}{\sqity{\qtuq{\st'}}}
}

\inferrule[s-ity-unquote-err]{
    \serr{\st}{\argEnv}
}{
    \serr{\sqity{\qtuq{\st}}}{\argEnv}
}
\end{mathpar}
\begin{mathpar}
\inferrule[s-ity-trans-step]{
    \sstep{\st}{\argEnv}{\st'}
}{
    \sstep{\sqity{\srep{\st}}}{\argEnv}{\sqity{\srep{\st'}}}
}

\inferrule[s-ity-trans-err]{
    \serr{\st}{\argEnv}
}{
    \serr{\sqity{\srep{\st}}}{\argEnv}
}
\end{mathpar}
This judgement is defined by the following straightforward rules:
\begin{mathpar}\small
\inferrule[tstore-emp]{ }{\emptyset \leadsto \emptyset : \emptyset}

\inferrule[tstore-ext]{\memD \leadsto \delta : \Delta}{(\memD, \st \leftrightarrow \tau/\alpha) \leadsto (\delta, \tau/\alpha) : (\Delta, \alpha)}
\end{mathpar}

\[\small
\begin{array}{lll}
\textbf{Description} ~~~~& \textbf{Concrete Form}~~~~ & \textbf{Desugared Form}\\
\text{sequences} & (e_1, \ldots, e_n) ~\text{or}~ \texttt{[}e_1, \ldots, e_n\texttt{]} & \eintro{()}{e_1; \ldots; e_n}\\
\text{labeled sequences} & \{\mathtt{lbl}_1=e_1, \ldots, \mathtt{lbl}_n=e_n\} & \eintro{\texttt{[}\mathtt{lbl}_1, \ldots, \mathtt{lbl}_n\texttt{]}}{e_1; \ldots; e_n}\\
\text{label application} & \mathtt{lbl}\langle e_1, \ldots, e_n \rangle & \eintro{\mathtt{lbl}}{e_1, \ldots, e_n}\\
\text{numerals} & n & \eintro{n}{\cdot}\\
\text{labeled numerals} & n\texttt{lbl} & \eintro{(n, \texttt{lbl})}{\cdot}\\
\text{strings} & \texttt{"s"} & \eintro{\texttt{"s"}}{\cdot}
\end{array}
\]

\begin{mathpar}\small
\inferrule[s-itm-var-v]{ }{
    \sval{\sqitm{x}}{\argEnv}
}

\inferrule[s-itm-lam-step-1]{
    \sstep{\sqity{\qity}}{\argEnv}{\sqity{\qity'}}
}{
    \sstep{\sqitm{\ilam{\qity}{x}{\qitm}}}{\argEnv}{\sqitm{\ilam{\qity'}{x}{\qitm}}}
}

\inferrule[s-itm-lam-step-2]{
    \sval{\sqity{\qity}}{\argEnv}\\
    \sstep{\sqitm{\qitm}}{\argEnv}{\sqitm{\qitm'}}
}{
    \sstep{\sqitm{\ilam{\qity}{x}{\qitm}}}{\argEnv}{\sqitm{\ilam{\qity}{x}{\qitm'}}}
}
\end{mathpar}
\begin{mathpar}\small
\inferrule[s-itm-lam-err-1]{
    \serr{\sqity{\qity}}{\argEnv}
}{
    \serr{\sqitm{\ilam{\qity}{x}{\qitm}}}{\argEnv}
}

\inferrule[s-itm-lam-err-2]{
    \serr{\sqitm{\qitm}}{\argEnv}
}{
    \serr{\sqitm{\ilam{\qity}{x}{\qitm}}}{\argEnv}
}

\inferrule[s-itm-lam-v]{
    \sval{\sqity{\qity}}{\argEnv}\\
    \sval{\sqitm{\qitm}}{\argEnv}
}{
    \sval{\sqitm{\ilam{\qity}{x}{\qitm}}}{\argEnv}
}
\end{mathpar}
\begin{mathpar}\small
\inferrule[k-itm-unquote]{
    \sofkX{\st}{\kitm}
}{
    \sofkX{\sqitm{\quq{\st}}}{\kitm}
}
\end{mathpar}

\begin{mathpar}\small
\inferrule[s-itm-unquote-step]{
    \sstep{\st}{\argEnv}{\st'}
}{
    \sstep{\sqitm{\quq{\st}}}{\argEnv}{\sqitm{\quq{\st'}}}
}

\inferrule[s-itm-unquote-err]{
    \serr{\st}{\argEnv}
}{
    \serr{\sqitm{\quq{\st}}}{\argEnv}
}

\inferrule[s-itm-unquote-elim]{
    \sval{\sqitm{\qitm}}{\argEnv}
}{
    \sstep{\sqitm{\quq{\sqitm{\qitm}}}}{\argEnv}{\sqitm{\qitm}}
}
\end{mathpar}
\begin{mathpar}
\inferrule[s-ana-step]{
    \sstep{\st}{\argEnv}{\st'}
}{
    \sstep{\sana{n}{\st}}{\argEnv}{\sana{n}{\st'}}
}

\inferrule[s-ana-err]{
    \serr{\st}{\argEnv}
}{
    \serr{\sana{n}{\st}}{\argEnv}
}

\inferrule[s-itm-anatrans-step]{
    \sstep{\st}{\argEnv}{\st'}
}{
    \sstep{\sqitm{\anatrans{n}{\st}}}{\argEnv}{\sqitm{\anatrans{n}{\st'}}}
}

\inferrule[s-itm-anatrans-err]{
    \serr{\st}{\argEnv}
}{
    \serr{\sqitm{\anatrans{n}{\st}}}{\argEnv}
}
\end{mathpar}
, as specified by the judgement $\memG \leadsto \gamma : \Gamma$ defined by the following rules:
\begin{mathpar}
\inferrule[ttrs-emp]{ }{\emptyset \leadsto \emptyset : \emptyset}

\inferrule[ttrs-ext]{
    \memG \leadsto \gamma : \Gamma
}{
    (\memG, n : \st \leadsto \iota/x : \tau) \leadsto (\gamma, \iota/x) : (\Gamma, x : \tau)
}
\end{mathpar}


\begin{mathpar}\small
\inferrule[k-itm-lam]{
    \sofkX{\sqity{\qity}}{\kity}\\
    \sofkX{\sqitm{\qitm}}{\kitm}
}{
    \sofkX{\sqitm{\ilam{\qity}{x}{\qitm}}}{\kitm}
}
\end{mathpar} 
\begin{mathpar}\small
\inferrule[k-raise]{\kDelta \vdash \kappa}{\sofkX{\sraise{\kappa}}{\kappa}}

\inferrule[s-raise]{ }{\serr{\sraise{\kappa}}{\argEnv}}
\end{mathpar}

\begin{mathpar}\small
\inferrule[s-syn-success]{
    \keyw{nth}[n](\es) = e\\
    \esynX{e}{\st}{\iota}
}{
    \sstep{\ssyn{n}}{\es;\Upsilon;\Phi}{(\st, \sqitm{\syntrans{n}})}
}

\inferrule[s-syn-fail]{
    \keyw{nth}[n](\es) = e\\
    [\Upsilon \vdash_\Phi e \nRightarrow]
}{
    \serr{\ssyn{n}}{\es;\Upsilon;\Phi}
}
\end{mathpar}

\begin{mathpar}\small
\inferrule[k-itm-syntrans]{
    n' < n
}{
    \sofkX{\sqitm{\syntrans{n'}}}{\kitm}
}
\end{mathpar}
, e.g. for lambdas:
\begin{mathpar}\small
%\inferrule[abs-var]{ }{\edeabs{\tc}{\argEnv}{x}{\memD}{\memG}{x}{\memD}{\memG}}
%
\inferrule[abs-lam]{
    \tdeabs{\Phi}{\tc}{\qity}{\memD}{\ity}{\memD'}\\
    \edeabs{\tc}{\es;\Upsilon;\Phi}{\qitm}{\memD'}{\memG}{\itm}{\memD''}{\memG'}
}{
    \edeabs{\tc}{\es;\Upsilon;\Phi}{\ilam{\qity}{x}{\qitm}}{\memD}{\memG}{\ilam{\tau}{x}{\itm}}{\memG'}{\memD''}
}
\end{mathpar}
\begin{mathpar}\small
\inferrule[abs-anatrans-stored]{
    n : \st \leadsto \iota/x : \tau \in \memG
}{
    \edeabs{\tc}{\argEnv}{\anatrans{n}{\st}}{\memG}{\memD}{x}{\memG}{\memD}
}

\inferrule[abs-syntrans-stored]{
    n : \st \leadsto \iota/x : \tau \in \memG
}{
    \edeabs{\tc}{\argEnv}{\syntrans{n}}{\memG}{\memD}{x}{\memG}{\memD}
}\end{mathpar}
\begin{mathpar}\small
\inferrule[k-itm-anatrans]{
    n' < n\\
    \sofkX{\st}{\kty}
}{
    \sofkX{\sqitm{\anatrans{n'}{\st}}}{\kitm}
}
\end{mathpar}
\begin{mathpar}\small
\inferrule[abs-syntrans-new]{
    n \notin \text{dom}(\memG)\\
    \keyw{nth}[n](\es) = e\\
    \esynX{e}{\st}{\iota}\\
    \tdeabs{\Phi}{\tc}{\srep{\st}}{\memD}{\ity}{\memD'}\\
    (x~\text{fresh})
}{
    \edeabs{\tc}{\es;\Upsilon;\Phi}{\syntrans{n}}{\memG}{\memD}{x}{\memG, n : \st \leadsto \iota/x : \tau}{\memD'}
}
\end{mathpar}

\begin{mathpar}\small
\inferrule[etctx-emp]{ }{\vdash_\Phi \emptyset \leadsto \emptyset}

\inferrule[etctx-ext]{\vdash_\Phi \Upsilon \leadsto \Gamma\\\istype{\st}{\Phi}\\\vdash_\Phi \st \leadsto \tau}{\vdash_\Phi \Upsilon, x \Rightarrow \st \leadsto \Gamma, x : \tau}
\end{mathpar}

\[\small
\begin{array}{lll}
\textbf{Description}& \textbf{Concrete Form}~~~~ & \textbf{Desugared Form}\\
\text{index projection} & e_\text{targ}\texttt{\#}n & \etarg{\opname{idx}}{n}{e_\text{targ}}{\cdot}\\
\text{label projection} & e_\text{targ}\texttt{\#}\conclbl{lbl} & \etarg{\opname{prj}}{\conclbl{lbl}}{e_\text{targ}}{\cdot}\\
\text{explicit invocation}~~~~ & e_\text{targ}{\cdot}\opname{op}[\sttmidx](\es) & \etarg{\opname{op}}{\sttmidx}{e_\text{targ}}{\es}\\
& e_\text{targ}{\cdot}\opname{op}(\es) & \etarg{\opname{op}}{()}{e_\text{targ}}{\es}\\
& e_\text{targ}{\cdot}\opname{op}(\conclbl{lbl}_1=e_1, \ldots, \conclbl{lbl}_n=e_n) & \keyw{targ}[\opname{op}; \texttt{[}\conclbl{lbl}_1, \ldots, \conclbl{lbl}_n\texttt{]}](e_\text{targ}; \\
& & \quad e_1; \ldots; e_n)\\
\text{labeled case analysis} & e_\text{targ}{\cdot}\opname{case}\,\{ & \keyw{targ}[{\opname{case}}; {\texttt{[}\st_1, \ldots, \st_n\texttt{]}}]({e_\text{targ}}; \\
 & ~|~~\st_1\langle x_1, \ldots, x_k \rangle \Rightarrow e_1 & \quad \eanalam{x_1}{\ldots \eanalam{x_k}{e_1}};\\
 & ~|~~\ldots & \quad \ldots;\\ 
 & ~|~~\st_n\langle x_1, \ldots, x_k \rangle \Rightarrow e_n\} & \quad \eanalam{x_1}{\ldots \eanalam{x_k}{e_n}})
\end{array}
\]

 For example, 
\begin{mathpar}
\small
\inferrule[abs-prod]{
    \tdeabs{\Phi}{\tc}{\qity_1}{\memD}{\ity_1}{\memD'}\\
    \tdeabs{\Phi}{\tc}{\qity_2}{\memD'}{\ity_2}{\memD''}
}{
    \tdeabs{\Phi}{\tc}{\qity_1 \times \qity_2}{\memD}{\ity_1 \times \ity_2}{\memD''}
}
\end{mathpar}

The argument interfaces that populate the list provided to opcon definitions is derived from the argument list by the judgement $\keyw{args}(\es)=_n \stx{args}$, defined as follows:
\begin{mathpar}\small
\inferrule[args-z]{ }{
    \keyw{args}(\cdot)=_0 \skap{\karg}{\svar{nil}}
}

\inferrule[args-s]{
    \keyw{args}(\es)=_{n} \st
}{
    \keyw{args}(\es;e)=_{n+1} \skap{\karg}{\svar{rcons}}~\st~(\slam{\kty}{\svar{ty}}{\sana{n}{\svar{ty}}},\slam{\kunit}{\_}{\ssyn{n}})
}
\end{mathpar}
We assume that the definitions of the standard helper functions $\small\svar{nil} :: \kforall{\kalpha}{\klist{\kalpha}}$ and $\small\svar{rcons} :: \kforall{\kalpha}{\klist{\kalpha} \rightarrow \kalpha \rightarrow \klist{\kalpha}}$, which adds an item to the end of a list, have been substituted into these rules. The result is that the $n$th element of the argument interface list simply wraps the static terms $\sana{n}{\st}$ and $\ssyn{n}$.

\begin{lemma}
\small If $\sofkn{\kDelta}{\kGamma}{\Phi}{n'}{\st}{\kappa}$ and $n > n'
$ then $\sofkn{\kDelta}{\kGamma}{\Phi}{n}{\st}{\kappa}$.
\end{lemma}


\end{document}
